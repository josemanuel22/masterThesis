\documentclass[a4paper,11pt,spanish, twoside, leqno]{tfm-uam}

\usepackage[utf8]{inputenc}


\usepackage{amsfonts, amssymb, amsmath, amsthm, mathrsfs}
\usepackage{tasks}
\usepackage{enumerate}

\usepackage{mathtools}
\usepackage{sobolev} %https://ctan.math.illinois.edu/macros/latex/contrib/bosisio/sobolev.html
\usepackage{diffcoeff} %http://mirrors.ibiblio.org/CTAN/macros/latex/contrib/diffcoeff/diffcoeff.pdf
\usepackage{esint}
\usepackage{hyperref}
%\usepackage{natbib}
\usepackage{tfm-miscelaneos}

\usepackage{imakeidx}

\usepackage[nottoc,notlot,notlof]{tocbibind} % Incluimos en la tabla de contenidos el indice alfabetico y la bilbiografia

%% Indice de Simbolos

\makeindex[columns=2, title=Indice alfabético]

\usepackage[symbols,
	nogroupskip,
   record % using 'bib2gls'
]{glossaries-extra}

\usepackage{glossary-mcols}%multi-column styles
%\usepackage{glossary-longextra}

\newglossary[nlg]{notation}{not}{ntn}{Notation}

\GlsXtrLoadResources[
 src={set-symbols},
 type=symbols,
 group={set},% assign group label
 set-widest,% needed for 'alttree' styles
 selection=all,
 save-locations=false
]

\GlsXtrLoadResources[
 src={topo-symbols},
 type=symbols,
 group={topo},% assign group label
 set-widest,% needed for 'alttree' styles
 selection=all,
 save-locations=false
]

\GlsXtrLoadResources[
 src={numbers-symbols},
 type=symbols,
 group={numbers},% assign group label
 set-widest,% needed for 'alttree' styles
 selection=all,
 save-locations=false
]

\GlsXtrLoadResources[
 src={func-symbols},
 type=symbols,
 group={functions},% assign group label
 set-widest,% needed for 'alttree' styles
 selection=all,
 save-locations=false
]

\GlsXtrLoadResources[
 src={measure-symbols},
 type=symbols,
 group={measure},% assign group label
 set-widest,% needed for 'alttree' styles
 selection=all,
 save-locations=false
]

\GlsXtrLoadResources[
 src={simbolos-conjuntos},
 type=notation,
 group={conjuntos},% assign group label
 set-widest,% needed for 'alttree' styles
 selection=all,
 save-locations=false
]

\GlsXtrLoadResources[
 src={simbolos-topo},
 type=notation,
 group={topos},% assign group label
 set-widest,% needed for 'alttree' styles
 selection=all,
 save-locations=false
]

\GlsXtrLoadResources[
 src={simbolos-numero},
 type=notation,
 group={numeros},% assign group label
 set-widest,% needed for 'alttree' styles
 selection=all,
 save-locations=false
]

\GlsXtrLoadResources[
 src={simbolos-funciones},
 type=notation,
 group={funciones},% assign group label
 set-widest,% needed for 'alttree' styles
 selection=all,
 save-locations=false
]

\GlsXtrLoadResources[
 src={simbolos-medidas},
 type=notation,
 group={medidas},% assign group label
 set-widest,% needed for 'alttree' styles
 selection=all,
 save-locations=false
]

%Subseccionse de l indice de simbolos
\glsxtrsetgrouptitle{topo}{Topological and metric spaces}
\glsxtrsetgrouptitle{set}{Set-theoretic operation}
\glsxtrsetgrouptitle{numbers}{Number sets and vector spaces}
\glsxtrsetgrouptitle{functions}{Functions and function spaces: Let $f:X\to Y$.}\glsxtrsetgrouptitle{measure}{Measure theory}

\glsxtrsetgrouptitle{topos}{Espacios topológicos y métricos}
\glsxtrsetgrouptitle{conjuntos}{Operaciones de conjuntos}
\glsxtrsetgrouptitle{numeros}{Conjuntos de números y espacios vectoriales}
\glsxtrsetgrouptitle{funciones}{Funciones y espacio de funcones: Sea $f:X\to Y$.}\glsxtrsetgrouptitle{medidas}{Teoría de la medida}

%\glsxtrnewsymbol[description={position}]{x}{\ensuremath{x}}
%\glsxtrnewsymbol[description={velocity}]{v}{\ensuremath{v}}

%New Macros
\newtheorem{teo}{Teorema}[chapter]
\newtheorem{lema}[teo]{Lema}
\newtheorem{prop}[teo]{Proposición}
\newtheorem{cor}[teo]{Corolario}
\newtheorem{defi}[teo]{Definición}
\newtheorem{nota}[teo]{Nota}
\newtheorem{ejemplo}[teo]{Ejemplo}
\newtheorem{conjutura}[teo]{Conjetura}

%Informacion de portada
\title{Funciones de variación acotada, existencia y regularidad de las soluciones de problemas de discontinuidad libre y el funcional de Mumford-Shah.}
\author{Jose Manuel de Frutos Porras}
\tutor{Fernando Soria}
\curso{2019-2020}

%Metadatos
\hypersetup{
	pdfinfo={
	Title={Funciones de variacion acotada, existencia y regularidad de las soluciones de problemas de discontinuidad libre y el funcional Mumford-Shah.},
	Author={Jose Manuel de Frutos},
	Director1={Fernando.Soria},
	Tipo={TFM},
	Curso={2019-2020},
	MSC={26A45,49Q20,49-02,49J45},
	Palabrasclave={functions of bounded variation, BV,special functions of bounded variation,SBV,free discontinuity problems,Mumford-Shah image segmentation,Mumford-Shah functional,variational problems,abstract measure theory, geometric measure theory,lower semicontinuity, regularity},
	}
}

\begin{document}

\begin{abstract}[spanish]
C. Jordan introdujo las funciones de variación acotada en relación con el test de Dirichlet para la convergencia de las series de Fourier. Sin embargo, la definición moderna de funciones de variación acotada (en lo que seguirá funciones $BV$) se debe a los trabajos de E. De Giorgi, quien introdujo la clase de funciones cuyas derivadas distribucionales son medidas. Hoy día, la clase de funciones $BV$ tiene un papel crítico en varios problemas clásicos del cálculo de variaciones.

En el primer capítulo, revisaremos algunos resultados preliminares de la teoría de la medida abstracta y de la teoría geométrica de la medida. En el segundo capítulo nos centraremos  en el estudio de las funciones $BV$ y en la teoría estrechamente relacionada de los conjuntos de perímetro finito. Definiremos y estudiaremos las propiedades del espacio de las funciones $BV$ y examinaremos el vínculo entre conjuntos de perímetro finito y las funciones $BV$. El siguiente capítulo del trabajo estará dedicado a las funciones especiales de variación acotada (funciones $SBV$) y las propiedades de este espacio funcional. Gracias al trabajo preliminar, el último capítulo de la tesis estará orientado hacia el estudio de problemas variacionales específicos formulados en el marco previamente construido. E. De Giorgi, acuñó el término ``problemas de discontinuidad libre'' para indicar una clase de problemas de minimización en el que compiten energías volumétricas, concentradas en conjuntos $N$-dimensionales, y energías superficiales, concentradas en conjuntos $(N-1)$-dimensionales. Otra característica de estos problemas es el hecho de que los conjuntos soporte de la energía superficial no están fijados \textit{a priori}, y en muchos casos son la incógnita relevante del problema. El ejemplo más conocido de problema de discontinuidad libre es el propuesto por D. Mumford y J. Shah. En ese sentido, el objetivo final de este trabajo será demostrar que la formulación clásica del problema de Mumford-Shah y la formulación débil en el espacio $SBV$ son de hecho equivalentes en cualquier dimensión espacial.
\end{abstract}

\begin{keywords}{spanish}
Funciones de variación acotada, funciones especiales de de variación acotada, problemas de discontinuidad libre, segmentación de imágenes Mumford-Shah, funcional Mumford-Shah, teoría abstracta de la medida, teoría geométrica de la medida, regularidad.
\end{keywords}

\begin{classification}{spanish}
\begin{enumerate}[(a)]
\item 26A45: Funciones de variación acotada, generalización,
\item 49Q20: Problemas variacionales formulados en el marco de la teoría geométrica de la medida. 
\item 49J45: Problemas de control óptimos que involucran semicontinuidad y convergencia; relajación.
\end{enumerate}
\end{classification}


\begin{abstract}[english]
Functions of bounded variation were introduced by C. Jordan in connexion with Dirichlet's test for the convergence of Fourier series. However, the modern definition of functions of bounded variation ($BV$ functions in the sequel) is due to the works of E. De Giorgi, who introduced the class of functions whose distributional derivatives are measures. Today, the class of $BV$ functions presents a major role in several classical problems of the calculus of variations.

In the first chapter, we will review some preliminary results of the abstract measure theory and of the geometric measure theory. The second chapter is entirely devoted to the study of $BV$ functions and the closely related theory of sets of finite perimeter. In this regard, we will define and study fine properties of the space of $BV$ functions and will examine the link between sets of finite perimeter and $BV$ functions. The next chapter of the thesis is devoted to special functions of bounded variation ($SBV$ functions) and to the properties of this functional space. Thanks to the preliminary work, the last chapter of the thesis will be oriented towards the study of specific variational problems on the unitary framework built previously. E. De Giorgi, coined the term ``free discontinuity problems'' to refer to a class of minimum problems characterised by a competition between volume energies, concentrated on $N$-dimensional sets, and surface energies, concentrated on $(N-1)$-dimensional sets. Another feature of these problems is the fact that the supports of the surface energy are not fixed \textit{a priori}, and are, in many cases, the relevant unknow of the problem. The best-known example of free discontinuity problem was proposed by D. Mumford and J. Shah. The final goal of the thesis will be to prove that the classical formulation of the Mumford-Shah problem and the weak one in the $SBV$ are indeed equivalent in any space dimension. 
\end{abstract}
\begin{keywords}{english}
Functions of bounded variation, Special functions of bounded variation, Free discontinuity problems, Mumford-Shah image segmentation, Mumford-Shah functional, abstract measure theory, geometric measure theory, regularity.
\end{keywords}

\begin{classification}{english}
\begin{enumerate}[(a)]
\item 26A45: Functions of bounded variation, generalizations,
\item 49Q20: Variational problems in a geometric measure-theoretic setting,
\item 49J45: Optimal control problems involving semicontinuity and convergence; relaxation.
\end{enumerate}
\end{classification}

\begin{agradecimientos}

Recuerdo que cuando terminé mi trabajo de fin de carrera decidí en una especie de vana rebeldía no escribir un apartado de agradecimientos. Supongo que quería probar de este modo mi cabreo con el mundo que me rodeaba. Dos años después y siendo un poco más maduro si que me gustaría dedicar un espacio de este trabajo a dar las gracias a algunas personas, espero que no quede demasiado ``cursi''. Si es así perdonadme, no tengo mucha práctica.

Es evidente que nada de esto hubiese sido posible sin mi familia. Aunque se que tienen una gran ``cruz'' conmigo siempre han conseguido soportarme con mucha paciencia y eso debe ser muy complicado, ¡hasta a mi me cuesta! Por esto y por tantas otras cosas os quiero. 

La confección de este TFM ha sido un camino arduo y complicado, y empieza con un fracaso previo. Siguiendo el orden cronológico del asunto me gustaría agradecer primero a Jesús Medina y a María Eugenia Cornejo por haberme acogido con tanto cariño durante tres meses en su grupo de investigación y haber entendido que volviese a Madrid para terminar mi trabajo de fin de máster. Sinceramente os agradezco muchísimo todo lo que hicisteís y me hubiese gustado poder seguir, recuerdo con muchísimo cariño esos días en la UCA. Tras esto me gustaría agradecer a Fernando Soria, la verdad que no son suficientes las palabras para expresar todo mi reconocimiento. Con él hice mi trabajo de fin de carrera y con él he hecho mi trabajo de fin de master. En este tiempo he aprendido muchas matemáticas, pero no solo. Gracias Fernando por tener siempre en todas y cada una de las reuniones una sonrisa. Gracias por ayudarme a sacar este trabajo adelante, por confiar en mi y por nunca decir que no. ¡Gracias Fernando! A su vez me gustaría agradecer a Jezabel Curbelo por animarme y ayudarme a seguir con mis estudios aún cuando estaba cansado. Me has enseñado que los matemáticos también ``molan''. Espero que todo vaya bien allá por las antípodas.

Finalmente no podía terminar esto sin una mención a mis amigos. A Victor, que aunque nos veamos cada vez menos, siempre me he considerado afortunado de tenerte como amigo. Eres de las pocas personas que sobrelleva bien mi hiperactividad. Siempre he admirado tu temple, generosidad  y tu gran conocimiento de los fundamentos (hablo de baloncesto obviamente ``Mr Fundamental''). A Alberto, por nuestras tardes de discusiones de todo y sobre todo, ya sea de criptografía, trading, criptomonedas, política o nuestro deseo común de emigrar Liberland. ¡Si hubiesemos comprado bitcoins cuando dijimos ahora mismo seríamos ricos! A Enrique y a Luisma, porque sé que puedo confiar en vosotros para cualquier cosa, como ya me lo habeís demostrado en incontables ocasiones. Y a todos mis demás amigos que con el tiempo se han ido difuminando en el espacio y ya no nos vemos tanto pero se que están allí. A todos, ¡Muchas gracias!

\end{agradecimientos}



\makeindexes
\mainmatter

%\setglossarystyle{alttree}
%\printunsrtglossary[type=symbols, title={List of Symbols}]
\printunsrtglossary[type=notation, style= mcolalttreegroup, title={Lista de Símbolos}]
\printunsrtglossary[type=symbols, style= mcolalttreegroup, title={List of Symbols}]
%\printunsrtglossary[type=symbols, style=mcolindex, title={List of Symbols}]
%\printunsrtglossary[type=symbols, stylemods={mcolalttreegroup}, title={List of Symbols}]
\chapter{Introducción y preliminares}\label{cap:cap1}
\setcounter{page}{1}

En este capítulo presentamos algunos resultados de la teoría de la medida necesarios en el resto del trabajo. Para ello en la primera sección recordaremos nociones como medidas reales, vectoriales, equiintegrabilidad y  teorema de Radon-Nikod\'ym. En la segunda sección estudiaremos las medidas de Radon en espacios métricos (localmente compactos y separables) y sus propiedades de convergencia débil$^{*}$ así como la correspondencia entre medidas de Radon finitas y las funciones continuas con soporte compacto. Tras esto, en la tercera sección recordaremos algunos resultados sobre convoluciones, en particular estudiaremos las propiedades de la convolución entre una medida de Radon y una función. Finalmente, veremos las nociones de medidas tangentes, conjuntos rectificables y medidas rectificables. A lo largo de este capítulo enunciaremos numerosos resultados sin demostración. Las demostraciones de las mismas se pueden encontrar en \cite{ambrosio2000functions} en los capítulos $1$ y $2$, aunque en algunos casos daremos otras referencias clásicas.
\section{Teoría abstracta de la medida}
%En esta primera sección repasaremos la definición de medidas reales y vectoriales. Definiremos también el concepto de variación total de una medida y veremos que se trata de una medida positiva finita. Tras esto estudiaremos medidas originadas integrando una función con respecto a una medida positiva. Recordaremos también el concepto de equiitegrabilidad y criterios de la equiitegrabilidad. Finalmente se enunciará el teorema de Radon-Nikod\'ym.

Además de las medidas positivas, es posible definir a su vez medidas que toman valores vectoriales. Esta noción es esencial en este trabajo puesto que el gradiente de una función de variación acotada en el sentido de la teoría de distribuciones es una medida de este tipo. A continuación damos la definición abstracta de medidas reales y vectoriales.
\begin{defi}[Medidas reales y vectoriales] \index{medida!real}\index{medida!vectoriales} \index{variación!total de un medida} \index{medida!variación total}
Sean $X$ un conjunto y $\E$  una $\sigma$-álgebra de $\mathcal{P}(X)$, $m\in \N$ y $m\geq 1$.
\begin{enumerate}[(a)]
\item Se dice que $\mu: \E\to \R^{m}$ es una medida si $\mu(\emptyset)=0$ y si para cualquier sucesión $(E_{h})$ de elementos disjuntos dos a dos de $\E$ se tiene que:
\begin{align*}
\mu\left( \bigcup_{h=0}^{\infty}E_{h}\right)=\sum_{h=0}^{\infty}\mu(E_{h}),
\end{align*}
siempre que la serie converja para la función de conjuntos $\mu$.

Si $m=1$ decimos que $\mu$ es una medida real, si $m>1$ se dice $\mu$ es una medida vectorial. 
\item Si $\mu$ es una medida, definimos la variación total $\abs{\mu}$ para cada $E\in \E$ como sigue:
\begin{align*}
\abs{\mu}(E)=\sup\set*{\sum^{\infty}_{h=0}\abs{\mu(E_{h})}\given E_{h}\in \E\; \text{conjuntos disjuntos dos a dos},\, E=\bigcup_{h=0}^{\infty}E_{h}}.
\end{align*}
\end{enumerate}
\end{defi}.

Observamos que las medidas positivas no son un caso particular de medidas reales puesto que las medidas reales de acuerdo con la definición que acabamos de dar deben ser finitas.

Vemos a continuación que la variación total de una medida es a su vez una medida positiva finita.
\begin{teo}
Sea $\mu$ una medida en $(X,\E)$, entonces $\abs{\mu}$ es una medida positiva finita.
\end{teo}

Ahora introducimos las medidas inducidas por una distribución de masas sumable.
\begin{defi}\label{defi:fmu medida vectorial}
Sea el espacio de medida $(X,\E, \mu)$ donde $\mu$ es una medida positiva y sea $f\in [L^{1}(X,\mu)]^{m}$. Definimos la siguiente medida con valores en $\R^{m}$:
\begin{align*}
f\mu(B)=\int_{B}fd\mu,\quad \forall B\in \E.
\end{align*}
\end{defi}

Por las propiedades elementales de la integral es fácil comprobar que la fórmula anterior define una medida vectorial con valores en $\R^{m}$ y su variación total es dada por la siguiente proposición.
\begin{prop}\label{prop:variacion total de fmu medida vectorial} \index{variación!total de un medida} \index{medida!variación total}
Sea $f\mu$ la medida introducida en la definición anterior, entonces:
\begin{align*}
\abs{f\mu}(B)=\int_{B}\abs{f}d\mu,\quad \forall B\in \E.
\end{align*}
\end{prop}

Por la desigualdad de Chebyschev se sigue inmediatamente que si las medidas $f\mu$ y $g\mu$ coinciden, entonces $g=f$ en casi todo punto de $X$ respecto de la medida $\mu$. A continuación recordamos el concepto de equiintegrabilidad.
\begin{defi}[Equiintegrabilidad]\label{defi:equiintegrabilidad} \index{equiintegrabilidad}
Si $\mathcal{F}\subset L^{1}(X,\mu)$ decimos que $\mathcal{F}$ es una familia equiintegrable se se verifican las dos condiciones siguientes:
\begin{enumerate}[(a)]
\item Para cualquier $\epsilon>0$ existe un conjunto $A$ que es $\mu$-medible y con $\mu(A)<\infty$ tal que $\int_{X\setminus A}\abs{f}d\mu< \epsilon$ para cualquier $f\in \mathcal{F}$.
\item Para cualquier $\epsilon>0$ existe un $\delta>0$ tal que para cada conjunto $E$ que es $\mu$-medible, si $\mu(E)<\delta$ entonces se tiene que $\int_{E}\abs{f}d\mu<\epsilon$ para cada $f\in \mathcal{F}$.
\end{enumerate}
\end{defi}

Se observa que la primera condición se verifica trivialmente para medidas finitas, pues basta considerar  $A=X$. En la siguiente proposición se dan tres formulaciones equivalentes de la equiintegrabilidad.

\begin{prop}\label{teo:condiciones de equiitegrabilidad} \index{equiintegrabilidad!condiciones}
Sea $\mathcal{F}\subset L^{1}(X,\mu)$. Entonces $\mathcal{F}$ es equiintegrable si y solo si:
\begin{align*}
\begin{array}{ll}
\displaystyle
(E_{h})\subset \E, &\displaystyle E_{h}\to \emptyset \implies \lim_{h\to \infty}\sup_{f\in \mathcal{F}}\int_{E_{h}}\abs{f}d\mu =0.
\end{array}
\end{align*}
Si $\mu$ es una medida finita y $\mathcal{F}$ está acotada en $L^{1}(X,\mu)$, entonces $\mathcal{F}$ es equiintegrable si y solo si:
\begin{align*}
\mathcal{F}\subset \set*{f\in L^{1}(X,\mu)\given \int_{X}\varphi(\abs{f})d\mu\leq 1},
\end{align*}
para alguna función creciente $\varphi:[0,\infty)\to [0,\infty]$ que satisfaga $\varphi(t)/t\to\infty$ cuando $t\to \infty$ o equivalentemente si y solo si:
\begin{align*}
\lim_{t\to \infty}\sup_{f\in \mathcal{F}}\int_{\set{\abs{f}>t}}\abs{f}d\mu =0.
\end{align*}
\end{prop}

Es fácil ver que toda medida $f\mu$, con $f\in [L^{1}(X,\mu)]^{m}$, es absolutamente continua respecto a $\mu$ (escrito $f\mu \ll \mu$) en el sentido dado en \cite{folland2013real}. El teorema de Radon-Nikod\'ym nos dice entonces que bajo ciertas condiciones el recíproco es cierto. De hecho se tiene el siguiente teorema de Radon-Nikod\'ym-Lebesgue.
\begin{teo}[Radon-Nikod\'ym-Lebesgue] \index{teorema!de Radon-Nikod\'ym}
Sea $\mu$ una medida positiva y $\nu$ una medida real o vectorial en el espacio de medible $(X, \E)$, y supongamos además que $\mu$ es $\sigma$-finita. Entonces existe un único par de medidas $\nu^{a}, \nu^{s}$ con valores en $\R^{m}$ tales que $\nu^{a}\ll\mu$, $\nu^{s}\perp \mu$ y $\nu=\nu^{a}+\nu^{s}$. Además, existe una única función $f\in [L^{1}(X, \mu)]^{m}$ tal que $\nu^{a}=f\mu$. 
\end{teo}
A la función $f$ se le llama función densidad o derivada Radon-Nikod\'ym de $\nu$ con respecto $\mu$ y se denota por $\nu/\mu$. También recordamos que el símbolo $\perp$ indica que las medidas son singulares (ver de nuevo \cite{folland2013real}). Como cada medida vectorial $\mu$ es absolutamente continua respecto de $\abs{\mu}$, se sigue la siguiente descomposición aplicando directamente el teorema de Radon-Nikod\'ym.

\begin{cor}[Descomposición polar] \index{descomposición!polar}
Sea $\mu$ una medida vectorial con valores en $\R^{m}$ en el espacio de medible $(X,\E)$, entonces existe una única función $f\in [L^{1}(X, \abs{\mu})]^{m}$ con valores en la esfera unidad $\mathbb{S}^{m-1}$ tal que se tiene $\mu=f\abs{\mu}$.
\end{cor}

\section{Medidas de Radon y sus propiedades}

%En esta segunda sección veremos primero la noción de medidas de Radon y recordaremos algunos resultados clásicos como el teorema de representación de Riesz o la compacidad débil$^{*}$ del espacio de medidas de Radon. Estos resultados se utilizaran en el siguiente capítulo a la hora de demostrar la compacidad débil$^{*}$ del espacio de funciones de variación acotada.

En adelante trabajaremos con espacios métricos localmente compactos y separables. Utilizaremos la abreviatura l.c.s. en el resto del capítulo para referirnos a estos espacios. Se definen entonces las medidas de Radon como sigue.
\begin{defi}[Medidas de Borel y medidas de Radon]\index{medida!de Borel}\index{medida!de Radon} \index{espacio!de medidas de Radon}
Sea $X$ un espacio métrico l.c.s. y $\mathcal{B}(X)$ su $\sigma$-álgebra de Borel:
\begin{enumerate}[(a)]
\item Una medida de Borel en $X$ es simplemente una medida positiva definida sobre la $\sigma$-álgebra $\mathcal{B}(X)$. Si esta medida es a su vez finita en conjuntos compactos, se dirá entonces que es una medida de Radon positiva.
\item Una medida de Radon en $X$ en general es una función de conjuntos que toma valores en $\R^{m}$ y que además es una medida vectorial en $K$ definida sobre la $\sigma$-álgebra $\mathcal{B}(K)$ para cada conjunto compacto $K\subset X$. Diremos  que $\mu$ es una medida de Radon finita si es una medida vectorial en $X$ sobre la $\sigma$-álgebra $\mathcal{B}(X)$.
\end{enumerate}
\end{defi}
\DefaultSet{X}
Denotamos por $\Mloc[m]$ (respectivamente por $\M[m]$) el espacio de la medidas de Radon en el espacio métrico l.c.s. $X$ que toman valores en $\R^{m}$ (respectivamente las medidas de Radon finitas en $X$ que toman valores en $\R^{m}$).

\begin{nota}\label{nota:extension de medida de Radon a medida de Radon finita}
Notesé que si $\mu$ es una medida de Radon y si:
 $$\sup\set*{\abs{\mu}(K)\given K\subset X \;\text{compacto}}<\infty,$$ 
entonces se puede extender la medida a todo $\mathcal{B}(X)$ y la función resultante, que se seguirá denotando por $\mu$, es una medida de Radon finita.
\end{nota}

El siguiente resultado nos proporciona una fórmula para calcular la variación total de una medida. Esta proposición será de gran utilidad en el resto del trabajo a la hora de calcular la variación de una función de variación acotada.
\begin{prop}\label{prop:variacion total de la medida}
Sea un espacio métrico $X$  l.c.s. y $\mu$ una medida de Radon finita en $X$ que toma valores en $\R^{m}$. Entonces para cada conjunto abierto $A\subset X$ se tiene las siguiente igualdad:\DefaultSet{A}
\begin{align*}
\abs{\mu}(A)=\sup\set*{\sum^{m}_{i=1}\int_{X}u_{i}\,d\mu_{i}\given u\in \Cc[m]{}, \Norm{u}[\infty]\leq 1},
\end{align*}
donde $u_{i}$ y $\mu_{i}$ denotan las componentes de $u$ y $\mu$ respectivamente. 
\end{prop} \DefaultSet{X}
Recordemos que denotamos por $C_{c}(X)$ el espacio de todas las funciones continuas con soporte compacto y por $C_{0}(X)$ su cierre con respecto a la norma supremo. Enunciamos a continuación el teorema de representación de Riesz.
\begin{teo}[Riesz]\index{teorema!de Riesz}
Sea $X$ un espacio métrico l.c.s., supongamos que el funcional $L:[C_{0}(X)]^{m}\to \R$ es aditivo y acotado, i.e. satisface las siguientes condiciones:
\begin{align*}
\begin{array}{ll}
L(u+v)=L(u)+L(v), & \forall u,v\in [C_{0}(X)]^{m},
\end{array}
\end{align*} 
y,
\begin{align*}
\Norm{L}=\sup\set*{L(u)\given u\in [C_{0}(X)]^{m}, \abs{u}\leq 1}<\infty.
\end{align*}
Entonces, existe una única medida de Radon finita $\mu$ en $X$ que toma valores en $\R^{m}$ tal que:
\begin{align*}
\begin{array}{ll}
\displaystyle
L(u)=\sum_{i=1}^{m}\int_{X}u_{i}\,d\mu_{i}, & \forall u\in [C_{0}(X)]^{m}.
\end{array}
\end{align*}
Además, por la proposición \ref{prop:variacion total de la medida} se deduce que:
\begin{align*}
\Norm{L}=\abs{\mu}(X).
\end{align*}
\end{teo}

En el siguiente corolario se enuncia una versión local del teorema de Riesz cuya demostración es una consecuencia directa de la versión global.
\begin{cor}
Sea $X$ un espacio métrico l.c.s., supongamos que el funcional $L:\Cc[m]{} \to \R$ es lineal y continuo con respecto a la convergencia dada por la norma supremo. Entonces, existe una única medida de Radon $\mu$ en $X$ que toma valores en $\R^{m}$ tal que:
\begin{align*}
\begin{array}{ll}
\displaystyle
L(u)=\sum_{i=1}^{m}\int_{X}u_{i}\,d\mu_{i}, &\forall u\in \Cc[m]{}.
\end{array}
\end{align*}
\end{cor}

El teorema de Riesz se puede enunciar diciendo que el dual del espacio de Banach $[C_{0}(X)]^{m}$ es el espacio $\M[m]$ de las medidas de Radon finitas en $X$ que toman valores en $\R^{m}$, bajo el emparejamiento:
\begin{align*}
\Crochet{u}{\mu}=\sum_{i=1}^{m}\int_{X}u_{i}\,d\mu_{i}.
\end{align*}
La proposición \ref{prop:variacion total de la medida} nos dice que $\abs{\mu}(X)$ es la norma dual. Análogamente, $\Mloc[m]$ se puede identificar con el dual de un espacio localmente convexo $\Cc[m]{}$. En consecuencia, distinguimos dos nociones de convergencia débil$^{*}$ de medidas de Radon. 

\begin{defi}[Convergencia débil$^{*}$]\label{teo:convergencia débil*}\index{convergencia!débil*}
Sea $\mu\in \Mloc[m]$ y $(\mu_{h})$ una sucesión de medidas en $\Mloc[m]$, se dice que $(\mu_{h})$ converge localmente a $\mu$ en sentido débil$^{*}$ si:
\begin{align*}
\lim_{h\to \infty}\int_{X}u \,d\mu_{h} = \int_{X}u\,d\mu,
\end{align*}
para cualquier $u\in \Cc[m]{}$. Si $\mu$ y los $\mu_{h}$ son además finitas, se dice que $(\mu_{h})$ converge a $\mu$ en sentido débil$^{*}$ si:
\begin{align*}
\lim_{h\to \infty}\int_{X}u\,d\mu_{h}=\int_{X}u\,d\mu,
\end{align*}
para cualquier $u\in [C_{0}(X)]^{m}$.
\end{defi} 

Recoremos que dado $X$ un subconjunto de un espacio de Banach, se dice que una aplicación $I: X \to \R$ es semicontinua inferiormente en $X$ con respecto a la topología débil/débil$^{*}$ si para toda sucesión $(u_{h})\subset X$ convergente en sentido débil/débil$^{*}$ a $u\in X$ se tiene que $I(u)\leq \liminf_{h\to \infty} I(u_{h})$. Análogamente se dirá que $I$ es semicontinua superiormente si $\limsup_{h\to \infty} I(u_{h})\leq I(u)$. Recordemos que por el teorema de representación de Riesz el espacio de las medidas de Radon finitas en $X$ que toman valores en $\R^{m}$ es un espacio de Banach. El siguiente teorema afirma que el espacio de medidas de Radon finitas es compacto en la topología débil$^{*}$.
\begin{teo}[Compacidad débil$^{*}$]\label{teo:compacidad débil*}\index{compacidad!débil*}
Si $(\mu_{h})$ es una sucesión de medidas de Radon finitas en el espacio métrico l.c.s. $X$ con $\sup\set{\abs{\mu_{h}}(X)\given h\in \N}<\infty$, entonces existe una subsucesión que converge en el sentido débil$^{*}$. Además, la aplicación $\mu \mapsto \abs{\mu}(X)$ es semicontinua inferiormente con respecto a la convergencia débil$^{*}$.
\end{teo}

\begin{cor}
Sea $(\mu_{h})$ una sucesión de medidas de Radon en el espacio métrico $X$ l.c.s. tal que $\sup\set{\abs{\mu_{h}}(K)\given h\in \N}<\infty$ para cualquier compacto $K\subset X$, entonces existe una subsucesión que converge localmente en sentido débil$^{*}$. Además, para cada conjunto abierto $A$, la aplicación $\mu \mapsto \abs{\mu}(A)$ es semicontinua inferiormente con respecto a la convergencia local débil$^{*}$ .    
\end{cor}

En la siguiente proposición se discute el comportamiento de la aplicación $\mu\mapsto \mu(E)$ y $\mu\mapsto \int f d\mu$ bajo la convergencia débil$^{*}$.
\begin{prop}\label{prop:convergencia dominada de medidas}
Sea $(\mu_{h})$ una sucesión de medidas de Radon en el espacio métrico  l.c.s. $X$ que convergen localmente a $\mu$ en sentido débil$^{*}$. Entonces:
\begin{enumerate}[(a)]
\item Si las medidas $\mu_{h}$ son positivas, entonces para cada función semicontinua inferiormente $u:X\to [0,\infty]$ se tiene:
\begin{align*}
\liminf_{h\to \infty} \int_{X}u \, d\mu_{h}\geq \int_{X}u\,d\mu,
\end{align*} 
y para cada función semicontinua superiormente $v: X\to [0, \infty)$ con soporte compacto se tiene que:
\begin{align*}
\limsup_{h\to \infty}\int_{X}v\,d\mu_{h}\leq \int_{X}v\,d\mu.
\end{align*}\label{prop:convergencia dominada de medidas:a}
\item Si $\abs{\mu_{h}}$ converge localmente a $\lambda$ en sentido débil$^{*}$, entonces $\lambda\geq \abs{\mu}$. Además, si $E$ es un conjunto relativamente compacto $\mu$-medible tal que $\lambda(\partial E)=0$, entonces $\mu_{h}(E)\to \mu(E)$ cuando $h\to \infty$. Por lo general se tiene que:
\begin{align*}
\int_{X}u \,d\mu = \lim_{h\to \infty}\int_{X}u\, d\mu_{h},
\end{align*}
para cualquier función sobre conjuntos de Borel acotada $u:X\to \R$ con soporte compacto tal que el conjunto de puntos de discontinuidad tiene medida cero respecto de $\lambda$.\label{prop:convergencia dominada de medidas:b}
\end{enumerate}
\end{prop}

Introducimos las siguientes operaciones sobre las medidas. Estas operaciones definen nuevas medidas, la primera de ellas, en conexión con el orden natural entre medidas positivas, define la mayor medida (respectivamente la menor) menor o igual (respectivamente mayor) que todas las medidas de una familia de medidas positivas. 
\begin{defi}[m.c.m. y m.c.d. de medidas] \index{medida!m.c.m.} \index{medida!m.c.d.}
Sea $(X, \E)$ un espacio de medida y $(\mu_{\alpha})_{\alpha\in A}$ unas medidas positivas definidas en este espacio. Definimos entonces para cada $E\in \E$:
\begin{align*}
\bigvee_{\alpha\in A}\mu_{\alpha}(E)=\sup\set*{\sum_{\alpha\in A'}\mu_{\alpha}(E_{\alpha})\given E_{\alpha}\in \E \;\text{conjuntos disjuntos dos a dos},\; E=\bigcup_{\alpha\in A'}E_{\alpha}},
\end{align*}
y,
\begin{align*}
\bigwedge_{\alpha\in A}\mu_{\alpha}(E)=\inf\set*{\sum_{\alpha\in A'}\mu_{\alpha}(E_{\alpha})\given E_{\alpha}\in \E \;\text{conjuntos disjuntos dos a dos},\; E=\bigcup_{\alpha\in A'}E_{\alpha}},
\end{align*}
donde $A'$ toma como valores todos los posibles subconjuntos finitos o numerables de $A$.
\end{defi}
Es fácil comprobar que $\bigvee_{\alpha}\mu_{\alpha}$ y $\bigwedge_{\alpha}\mu_{\alpha}$ son medidas positivas, y respectivamente se corresponden con la medida más pequeña de todas las más grandes que $\sup_{\alpha} \mu_{\alpha}$ y con la medida más grande de todas las medidas más pequeñas que $\inf_{\alpha}\mu_{\alpha}$. Por lo tanto, la clase de medidas positivas es un retículo completo.

\begin{nota}\label{nota:sobre m.c.m y m.c.d medidas}
Supongamos que todas las medidas $\mu_{\alpha}$ en la definición anterior son absolutamente continuas con respecto a una medida fijada $\mu$, luego para cualquier $\alpha\in A$ se tiene $\mu_{\alpha}=f_{\alpha}\mu$ para funciones positivas $f_{\alpha}$ adecuadas. Entonces, si $A$ es un conjunto numerable se tiene la igualdad $\bigvee_{\alpha}\mu_{\alpha}=(\sup_{\alpha}f_{\alpha})\mu$.
\end{nota}

Finalmente, hacemos un repaso de resultados fundamentales debidos al estudio de las densidades de conjuntos en bolas de radio $\varrho$ centradas en $x$. Primero enunciamos el teorema de diferenciación de Besicovitch. Usaremos con frecuencia este teorema a la hora de representar la derivada distribucional de una función de variación acotada restringida a conjuntos rectificables.
\begin{teo}[Teorema de diferenciación de Besicovitch]\index{teorema!de diferenciación de Besicovitch}
Sea  $\mu$ una medida de Radon positiva sobre un conjunto abierto $\Omega\subset \R^{N}$ y $\nu$ una medida vectorial de Radon que toma  valores en $\R^{m}$. Entonces, para casi todo $x$ (respecto de la medida $\mu$) en el soporte de $\mu$, se tiene que el límite:
\begin{align}
f(x)=\lim_{\varrho\to 0}\dfrac{\nu(B_{\varrho}(x))}{\mu(B_{\varrho}(x))},
\end{align}
existe en $\R^{m}$ y además la descomposición de Radon-Nikod\'ym de $\nu$ es dada por $\nu=f\mu+\nu^{s}$, donde $\nu^{s}=\nu\Radot E$ es la restricción de $\nu$ al conjunto $E$, siendo $E$ es el conjunto de medida cero respecto de $\mu$ que verifica:
\begin{align*}
E=(\Omega\setminus \supp \mu)\cup \set*{x\in \supp \mu\given \lim_{\varrho\to 0}\dfrac{\abs{\nu}(B_{\varrho}(x))}{\mu(B_{\varrho}(x))}=\infty}.
\end{align*}
\end{teo}
El siguiente resultado es consecuencia directa del teorema anterior.
\begin{cor}[Puntos de Lebesgue]\index{punto!de Lebesgue}
Sea $\mu$ una medida de Radon sobre un conjunto abierto $\Omega\subset \R^{N}$, $f\in L^{1}(\Omega, \mu)$. Entonces, para casi todo $x\in \Omega$ respecto de la medida $\mu$ se tiene la siguiente igualdad:
\begin{align*}
\lim_{\varrho\to 0}	\dfrac{1}{\mu(B_{\varrho}(x))}\int_{B_{\varrho}(x)}\abs{f(y)-f(x)}d\mu=0.
\end{align*}
Cualquier punto $x\in\Omega$ donde la ecuación anterior es cierta se llama punto de Lebesgue.
\end{cor}
 
A menudo necesitaremos saber cuando una medida $\mu$ es representable en términos de la medida de Hausdorff, o por lo menos poder acotar la dimensión de Hausdorff de un conjunto donde $\mu$ está concentrado. Con el objetivo de comparar la medida $\mu$ con $\Hm{k}$ la idea natural consiste en estudiar el ratio $\mu(B_{\varrho}(x))/(\omega_{k}\varrho^{k})$ cuando $k\to 0$. Esto motiva la siguiente definición.
\begin{defi}[Densidades $k$-dimensionales]\index{densidad $k$-dimensional}
Sea $\mu$ una medida de Radon positiva sobre un conjunto $\DefaultSetWP\subset \R^{N}$ y $k\geq 0$. La densidad $k$-dimensionales superior e inferior de $\mu$ en $x$ son respectivamente:
\begin{align*}
\begin{array}{ll}
\displaystyle
\Theta_{k}^{*}(\mu, x)=\limsup_{\varrho\to 0} \dfrac{\mu(B_{\varrho}(x))}{\omega_{k}\varrho^{k}}, &\displaystyle \Theta_{*k}(\mu, x)=\lim_{\varrho\to 0}\inf\dfrac{\mu(B_{\varrho}(x))}{\omega_{k}\varrho^{k}}.
\end{array}
\end{align*}
Si $\Theta_{k}^{*}(\mu, x)= \Theta_{*k}(\mu, x)$ entonces al valor común se denotará por $\Theta_{k}(\mu, x)$.
\end{defi}

Usaremos esta notación también para medidas de Radon vectoriales $\mu$ con valores en $\R^{m}$ solo cuando las densidades de cada una de sus componentes $\mu_{i}$ estén definidas, i.e. el valor de la $i$-ésima componente de $\Theta_{k}(\mu, x)$ será $\Theta_{k}(\mu_{i}, x)$ para cualquier $i=1, \ldots, m$.

Análogamente, para cualquier conjunto de Borel $E\subset \Omega$ definimos:
\begin{align*}
\begin{array}{ll}
\displaystyle \Theta_{k}^{*}(E, x)=\limsup_{\varrho\to 0} \dfrac{\Hm{k}(E\cap B_{\varrho}(x))}{\omega_{k}\varrho^{k}}, & \displaystyle \Theta_{*k}(E, x)=\liminf_{\varrho\to 0}\dfrac{\Hm{k}(E\cap B_{\varrho}(x))}{\omega_{k}\varrho^{k}},
\end{array}
\end{align*}
y si la densidad superior e inferior coinciden denotamos al valor común por $\Theta_{k}(E,x)$. Está claro que $\Theta_{k}^{*}(E,x)=\Theta_{k}^{*}(\Hm{k}\Radot E, x)$ y que $\Theta_{*k}(E,x)=\Theta_{k*}(\Hm{k}\Radot E, x)$.

Ahora vemos que la densidad superior $\udens{k}{\mu}{x}$ puede utilizarse para acotar $\mu$ inferior y superiormente con respecto de $\Hm{k}$.
\begin{teo}\label{teo:resultados sobre densidades de hausdorff y medidas de radon}
Sea $\Omega\subset \R^{N}$ un conjunto abierto y $\mu$ una medida de Radon positiva en $\Omega$. Entonces, para cada $t\in (0,\infty)$ y cualquier conjunto de Borel $B\subset \Omega$ se tienen las siguientes implicaciones:
\begin{align}
&\udens{k}{\mu}{x}\geq t, \quad \forall x\in B \implies \mu \geq t\Hm{k}\Radot B,\\
&\udens{k}{\mu}{x}\leq t, \quad \forall x\in B \implies \mu \leq 2^{k}t\Hm{k}\Radot B.
\end{align}
\end{teo}

Dos consecuencias útiles del teorema anterior son:
\begin{align}\label{eq:1:hausdorff y medidas de radon}
\begin{array}{ll}
\udens{k}{\mu}{x}<\infty,& \text{para}\; \Hm{k}\text{-casi todo}\; x\in B,
\end{array}
\end{align}
y también,
\begin{align}\label{eq:2:hausdorff y medidas de radon}
\begin{array}{ll}
B\in \mathcal{B}(\DefaultSetWP), \mu(B)=0 \implies \dens{k}{\mu}{x}=0,& \text{para}\; \Hm{k}\text{-casi todo}\; x\in B. 
\end{array}
\end{align}
En efecto, para cualquier conjunto de Borel $B \subset\subset \set*{x\in \Omega\given \udens{k}{\mu}{x}=\infty}$, se tiene la desigualdad $\mu(B)\geq t\Hm{k}(B)$ para cualquier $t>0$, luego $\Hm{k}(B)=0$. Por la $\sigma$-subaditividad se prueba \ref{eq:1:hausdorff y medidas de radon}. Análogamente se demuestra \ref{eq:2:hausdorff y medidas de radon}.

\section{Algunos resultados sobre convoluciones}
\DefaultSet{\Omega}

En esta sección recogemos las principales propiedades de la convolución entre una medida de Radon y una función. Al igual que se define la convolución entre dos funciones, se puede definir la convolución entre funciones y medidas de Radon.
\begin{defi}[Convolución de una medida de Radon y una función] \index{convolución}
Sea $\mu$ una medida de Radon sobre un abierto $\DefaultSetWP\subset \R^{N}$ con valores en $\R^{m}$. Si $f$ es una función continua, llamamos convolución entre $f$ y $\mu$ a la función:
\begin{align*}
\mu*f(x)=\int_{\DefaultSetWP}f(x-y)d\mu(y),
\end{align*}
siempre y cuando la expresión tenga sentido.
\end{defi}

Para la convolución de funciones continuas con medidas se tiene el siguiente resultado de convergencia: 
\begin{prop}
Si $(\mu_{h})$ es una sucesión de medidas de Radon en $\R^{N}$ que convergen localmente en $\R^{N}$ a $\mu$ en sentido débil$^{*}$ y $f\in C_{c}(\R^{N})$, entonces $\mu_{h}*f\to \mu*f$ uniformemente en conjuntos compactos de $\R^{N}$.
\end{prop}

La principal aplicación de la definición anterior es cuando se considera una función $f=\rho_{\epsilon}$ con $(\rho_{\epsilon})_{\epsilon>0}$ una familia de mollifiers que cumplen $\rho_{\epsilon}(x)=\epsilon^{-N}\rho(x)$, con $\rho\in\Cc{\infty}$ que sastisface que $\rho(x)\geq 0$ y $\rho(-x)=\rho(x)$ para cualquier $x$, $\supp{\rho}\subset B_{1}$ y $\displaystyle\int_{\R{N}}\rho(x)dx=1$. Las convoluciones $\mu*\rho_{\epsilon}$ siguen siendo funciones regulares y se comportan bien respecto del límite de las medidas de conjuntos.
Vemos que:
\begin{align*}
\mu*\rho_{\epsilon}(x)=\int_{\DefaultSetWP}\rho_{\epsilon}(x-y)\,d\mu(y) = \epsilon^{-N}\int_{\DefaultSetWP}\rho\left( \dfrac{x-y}{\epsilon}\right)\, d\mu(y),
\end{align*}
está definido para cualquier $x\in \Omega_{\epsilon}$ donde $\Omega_{\epsilon}$ es el conjunto de puntos $x$ de $\Omega$ tales que $\dist{x}{\partial \Omega}>\epsilon$.

El siguiente teorema enuncia como se comporta la operación de convolución con mollifiers frente a la diferenciación y la convergencia débil$^{*}$.
\begin{teo}\label{teo:propiedades de molliers}
Sea $\Omega\subset\R^{N}$ un conjunto abierto, $\mu=(\mu_{1}, \ldots, \mu_{m})$ una medida de Radon en $\Omega$ y $(\rho_{\epsilon})_{\epsilon>0}$ una familia de mollifiers. Entonces:
\begin{enumerate}[(a)]
\item Las funciones $\mu*\rho_{\epsilon}$ pertenecen a $[C^{\infty}(\Omega_{\epsilon})]^{m}$ y $\nabla^{\alpha}(\mu*\rho_{\epsilon})=\mu*\nabla^{\alpha}\rho_{\epsilon}$ para cualquier $\alpha\in \N^{N}$.\label{teo:propiedades de molliers:a}
\item Las medidas $\mu_{\epsilon}=\mu* \rho_{\epsilon}$ convergen a $\mu$ en $\Omega$  en sentido débil$^{*}$ cuando $\epsilon\to 0$ y además se verifica la siguiente cota:
\begin{align*}
\int_{E}\abs{\mu*\rho_{\epsilon}}(x)\, dx\leq \abs{\mu}(I_{\epsilon}(E)),
\end{align*}
para cualquier conjunto de Borel $E\subset \Omega_{\epsilon}$.\label{teo:propiedades de molliers:b}
\item Las medidas $\abs{\mu_{\epsilon}}$ convergen a $\abs{\mu}$ en $\Omega$ en sentido débil$^{*}$ cuando $\epsilon\to 0$.\label{teo:propiedades de molliers:c}
\end{enumerate}
\end{teo}
Concluimos esta sección con la siguiente nota que será de gran utilidad a la hora de probar algunas propiedades de las funciones de variación acotada.
\begin{nota}
Usando Fubini y la simetría del kernel $\rho$ es fácil ver que:
\begin{align}
\int_{\DefaultSetWP}(\mu*\rho_{\epsilon})v\, dx = \int_{\DefaultSetWP}(v*\rho_{\epsilon})\, d\mu,
\end{align}
si $v\in \L1$ y o bien $\mu$ está concentrado en $\Omega_{\epsilon}$ o bien $v=0$ en casi todo punto fuera de $\Omega_{\epsilon}$.
\end{nota}

\section{Medidas tangentes y conjuntos rectificables}

En esta sección enunciaremos las nociones de conjunto y medida rectificable, para posteriormente introducir las nociones de medida tangente y de espacio de tangente aproximado. Finalmente, daremos algunos resultados respecto de la representación de las medidas de Radon sobre conjuntos rectificables y veremos la noción de espacio tangente aproximado a un conjunto.

En esta sección supondremos que $k\in[0,N]$ es un entero. Si $E$ es una conjunto $\Hm{k}$-medible con $\Hm{k}(E)<\infty$, en los casos extremos $k=0, k=N,$ sabemos que la densidad $\dens{k}{E}{x}$ existe y coincide con $\chi_{E}(x)$ en casi todo punto $x\in \R^{N}$ respecto de la medida $\Hm{k}$. Por otro lado, si $0<k<N$, el teorema \ref{teo:resultados sobre densidades de hausdorff y medidas de radon} solo nos permite concluir:
\begin{align}\label{eq:resultados de desnidades:1}
\begin{array}{ll}
\dens{k}{E}{x}=0, & \text{para c.t.p.}\;x\in \R^{N}\setminus E\; \text{respecto de}\;\Hm{k},
\end{array}
\end{align}
\begin{align}\label{eq:resultados de desnidades:2}
\begin{array}{ll}
\ 2^{-k}\leq \udens{k}{E}{x}\leq 1, & \text{para c.t.p.}\;x\in E\; \text{respecto de}\;\Hm{k}.
\end{array}
\end{align}
En efecto, la ecuación \ref{eq:resultados de desnidades:1} se sigue del resultado \ref{eq:2:hausdorff y medidas de radon} tomando $\mu=\Hm{k}\Radot E$. La cota inferior que proporciona la ecuación \ref{eq:resultados de desnidades:2} se sigue de que $\mu(B_{t})\leq 2^{k}t\mu(B_{t})$ con $B_{t}=\set*{x\in E\given \udens{k}{E}{x}\leq t}, t<2^{-k}$. Análogamente se obtiene la cota superior. El teorema \ref{teo:resultados sobre densidades de hausdorff y medidas de radon} no proporciona una cota inferior para $\ldens{k}{E}{x}$, luego no se tiene información sobre la existencia de $\dens{k}{E}{x}$ en $E$. Esta y otras propiedades están relacionadas con un noción débil de regularidad de $E$ conocida como $\Hm{k}$-rectificabilidad.

\begin{defi}[Conjuntos rectificables]\label{defi:conjuntos rectificables} \index{conjunto!rectificable}
Sea $E\subset \R^{N}$ un conjunto $\Hm{k}$-medible. Se dice que $E$ es numerablemente $k$-rectificable si existe una cantidad numerable de funciones Lipschitz $f_{i}:\R^{k}\to \R^{N}$ tal que se tiene:
\begin{align*}
E\subset \bigcup_{i=0}^{\infty}f_{i}(\R^{k}).
\end{align*}
Se dirá que $E$ es numerablemente $\Hm{k}$-rectificable si existe una cantidad numerable de funciones Lipschitz $f_{i}:\R^{k}\to \R^{N}$ tal que se tiene:
\begin{align*}
\Hm{k}\left( E\setminus \bigcup_{i=0}^{\infty}f_{i}(\R^{k})\right)=0.
\end{align*}
Finalmente, se dirá que $E$ es $\Hm{k}$-rectificable si $E$ es numerablemente $\Hm{k}$-rectificable y $\Hm{k}(E)<\infty$.
\end{defi} 

También introducimos la clase de medidas $k$-rectificables.
\begin{defi}[Medidas rectificables]\label{defi:medidas rectificables} \index{medida!rectificable}
Sea $\mu$ la medida de Radon sobre $\R^{N}$ que toma valores en $\R^{m}$. Se dice que $\mu$ es $k$-rectificabe si existe un conjunto $S$ que es $\Hm{k}$-rectificable y una función $\theta:S\to \R^{m}$ tal que $\mu=\theta \Hm{k}\Radot S$.
\end{defi}

A continuación estudiamos las medidas desde un punto de vista local. En efecto estamos interesados en el comportamiento asintótico de una medida $\mu$ cerca de un punto $x$ de su soporte. Esto conlleva a la introducción de las medidas rescaladas:
\begin{align*}
\mu_{x,\varrho}(B)=\mu(x+\varrho B),
\end{align*}
en un entorno de $x$ y al análisis de estas medidas normalizadas adecuadamente cuando $\varrho\to 0$. 
\begin{defi}[Medidas tangentes]\label{defi:medidas tangentes} \index{medida!tangente}
Se denotará por $\Tan{}{\mu}{x}$ el conjunto de todas las medidas finitas de Radon $\nu$ en $B_{1}$ que son límite débil$^{*}$ de:
\begin{align}\label{defi:medidas tangentes:eq:1}
\begin{array}{lll}
\dfrac{1}{c_{x, \varrho_{i}}}\mu_{x,\varrho_{i}}, & \text{con}, & c_{x, \varrho}=\abs{\mu}(B_{\varrho}(x)),
\end{array}
\end{align}
para alguna sucesión infinitesimal $(\varrho_{i})\subset (0,\infty)$. A los elementos de $\Tan{}{\mu}{x}$ se le llamará medidas tangentes de $\mu$ en $x$.
\end{defi}

La normalización en \ref{defi:medidas tangentes:eq:1} es conveniente ya que:
\begin{align*}
\begin{array}{ll}
\dfrac{1}{c_{x,\varrho}}\abs{\mu_{x,\varrho}}(B_1)=\dfrac{1}{c_{x,\varrho}}\abs{\mu}(B_{\varrho}(x))=1, & \forall \varrho\in (0, \dist{x}{\partial \DefaultSetWP}).
\end{array}
\end{align*}
Entonces por el teorema \ref{teo:compacidad débil*}, el conjunto $\Tan{}{\mu}{x}$ es no vacío. Gracias a la semicontinuidad inferior de $\mu\mapsto \abs{\mu}(B_1)$ con respecto a la convergencia débil$^{*}$ en $B_{1}$ se observa que:
\begin{align*}
\begin{array}{ll}
\abs{\nu}(B_1)\leq 1, & \forall \nu\in \Tan{}{\mu}{x}.
\end{array}
\end{align*}
El siguiente resultado asegura que $\Tan{}{\mu}{x}$ contiene medidas no-cero en casi todo punto de $\Omega$ respecto de la medida $\mu$.
\begin{teo}\label{teo:espacio medidas tanges contines medidas no cero}
Si $\mu$ es una medida positiva en $\Omega$ entonces para casi todo $x$ respecto de la medida $\mu$ se tiene que para cualquier $t\in (0,1)$ existe un $\nu\in \Tan{}{\mu}{x}$ tal que $\nu(\overline{B}_{t})\geq t^{N}$.
\end{teo} 

Se puede extender el análisis anterior a medidas con valores en $\R^{m}$ y ver que para casi todo $x$ respecto de $\abs{\mu}$ cualquier medida tangente de $\mu$  en $x$ es un múltiplo constante de una medida positiva. Es decir, si $f$ es la función con valores en $\mathbb{S}^{m-1}$ originada de la descomposición polar de $\mu=f\abs{\mu}$, entonces $\Tan{}{\mu}{x}$ consiste en todas las medidas con la forma $f(x)\sigma$, donde $\sigma\in \Tan{}{\abs{\mu}}{x}$.
\begin{teo}\label{teo: expresion del espacio de medidas tangentesen funciones del espacio de medidas tangentes de meddias radon positivas}
Sea $\mu$ una medida de Radon que toma valores en $\R^{m}$ en $\Omega$ y sea $\mu=f\abs{\mu}$ su descomposición polar. Para cualquier punto $x$ de Lebesgue de $f$ respecto de la medida $\mu$ se tiene la siguiente propiedad:
\begin{align*}
\nu=\lim_{i\to \infty}\dfrac{\mu_{x,\varrho_{i}}}{c_{x,\varrho_{i}}}\in \Tan{}{\mu}{x}\iff\abs{\nu}=\lim_{i\to \infty}\dfrac{\abs{\mu}_{x,\varrho_{i}}}{c_{x, \varrho_{i}}},
\end{align*}
y $\nu=f(x)\abs{\nu}$. En particular:
\begin{align*}
\Tan{}{\mu}{x}=f(x)\Tan{}{\abs{\mu}}{x}.
\end{align*}
\end{teo}

Para concluir damos unos resultados sobre las propiedades de la densidad de las medidas $k$-rectificables. Con el objetivo de estudiar el comportamiento de una medida de Radon $\mu$ en $\Omega$ alrededor de $x$, al igual que hicimos antes, consideramos las medidas rescaladas:
\begin{align*}
\begin{array}{ll}
\mu_{x,\varrho}(B)=\mu(x+\varrho B),& B\in \mathcal{B}(\R^{N}), B\subset\dfrac{\Omega -x}{\varrho},
\end{array}
\end{align*}
y estudiamos en este caso el comportamiento de $\varrho^{-k}\mu_{x,\varrho}$ cuando $\varrho\to 0$. A diferencia de la vez anterior, esta vez la constante de normalización $\varrho^{-k}$ está fija pues estamos interesados en comparar $\mu$ con $\Hm{k}$. Denotaremos por $\mathbf{G}_{k}$ el espacio métrico compacto de los $k$-planos (es decir, planos de dimensión $k$) no orientados en $\R^{N}$.
\begin{defi}[Espacio de tangente aproximada a una medida]\label{defi:espacio de tangente aproximada a una medida} \index{espacio!de tangente aproximada}
Sea $\mu$ una medida de Radon sobre un conjunto abierto $\Omega\subset\R^{N}$ que toma valores en $\R^{m}$ y $x\in \Omega$. Se dice que $\mu$ tiene espacio de tangente aproximada $\pi\in \mathbf{G}_{k}$ con multiplicidad $\theta\in\R^{m}$ en $x$, y escribimos:
\begin{align*}
\Tan{k}{\mu}{x}=\theta \Hm{k}\Radot \pi,
\end{align*}
si $\varrho^{-k}\mu_{x,\varrho}$ converge localmente a $\theta \Hm{k}\Radot \pi$ en $\R^{N}$ en sentido débil$^{*}$ cuando $\varrho\to 0$.
\end{defi}
En el siguiente teorema enunciamos las propiedades elementales de $\Tan{k}{\mu}{x}$.
\begin{teo}\label{teo:propiedades del espacio tangente aproximado a una medida}
Sea $\mu$ una medida de Radon en $\R^{N}$ que toma valores en $\R^{m}$, y supongamos que $\abs{\mu}(B_{\varrho}(x))/\varrho^{k}$ es acotada cuando $\varrho\to 0$. Entonces:
\begin{enumerate}[(a)]
\item $\nu=\Tan{k}{\mu}{x}$ si y solo si cualquier límite local en $\R^{N}$ de $\mu_{x,\varrho_{i}}/\varrho_{i}^{k}$ en sentido débil$^{*}$ coincide con $\nu$ para alguna sucesión infinitesimal $(\varrho_{i})\subset (0,\infty)$.\label{teo:propiedades del espacio tangente aproximado a una medida:a}
\item Si $\mu=f\abs{\mu}$ y $x$ es un punto de Lebesgue de $f$ relativo a $\abs{\mu}$, entonces $\nu=\Tan{k}{\mu}{x}$ si y solo si $\abs{\nu}=\Tan{k}{\abs{\mu}}{x}$.\label{teo:propiedades del espacio tangente aproximado a una medida:b}
\end{enumerate} 
\end{teo}

La rectificabilidad de conjuntos y medidas se puede caracterizar mediante la existencia de la densidad como puede verse para $\R^{2}$ en \cite{falconer1986geometry, falconer2004fractal} . Sin embargo para dimensiones mayores que dos las pruebas son por lo general muy complicadas (véase \cite{federer2014geometric}). Sin embargo, el siguiente teorema proporciona una caracterización más simple basada en la existencia de espacios tangentes aproximados. La demostración que no daremos por estar fuera de los objetivos del trabajo es asequible y se basa en el resultado (\ref{teo:propiedades del espacio tangente aproximado a una medida:b}) del teorema anterior.
\begin{teo}[Criterio de rectificabilidad para medidas]\label{teo:criterio de rectificabilidad para medidas} \index{medida!rectificable}
Sea $\mu$ una medida positiva de Radon en un conjunto abierto $\Omega\subset \R^{N}$.
\begin{enumerate}[(a)]
\item Si $\mu=\theta \Hm{k}\Radot S$ y $S$ es numerablemente $\Hm{k}$-rectificable, entonces $\mu$ admite un espacio tangente aproximado con multiplicidad $\theta(x)$ para casi todo punto $x\in S$ respecto de la medida $\Hm{k}$. En particular, $\theta(x)\in \Theta_{k}(\mu,x)$ para casi todo $x\in S$ con respecto a la medida $\Hm{k}$. \label{teo:criterio de rectificabilidad para medidas:a}
\item Si $\mu$ está concentrada en un conjunto de Borel $S$ y admite un espacio tangente aproximado con multiplicidad $\theta(x)>0$ para casi todo $x\in S$ respecto de la medida $\mu$, entonces $S$ es numerablemente $k$-rectificable y $\mu=\theta\Hm{k}\Radot S$. En particular,
\begin{align*}
\exists\, \Tan{k}{\mu}{x}\,  \text{para casi todo}\; x\in \Omega \implies \mu\, \text{es}\; k\text{-rectificable}.
\end{align*}\label{teo:criterio de rectificabilidad para medidas:b}
\end{enumerate}
\end{teo}
En la siguiente proposición señalaremos una propiedad de localidad de los espacios tangentes aproximados.
\begin{prop}[Localidad de $\Tan{k}{\mu}{x}$]\label{prop:localidad de espacio tangente aproximado}\index{localidad}
Sea $\mu_{i}=\theta_{i}\Hm{k}\Radot S_{i}$, $i=1,2,$ una medida positiva $k$-rectificable y $\pi_{i}$ un espacio tangente aproximado a $\mu_{i}$, definido para casi todo $x\in S_{i}$ respecto de $\Hm{k}$. Entonces:
\begin{align}\label{prop:localidad de espacio tangente aproximado:eq:1}
\displaystyle 
\begin{array}{ll}
\pi_{1}(x)=\pi_{2}(x),&\text{para c.t.p.}\; x\in S_{1}\cap S_{2}\; \text{respecto de}\, \Hm{k}. 
\end{array}
\end{align}
\end{prop}

Si consideramos en la proposición anterior que $S_{1}=S_{2}$ entonces concluimos que por lo general el espacio tangente aproximado a $\theta \Hm{k}\Radot S$ no depende de $\theta$ pero solo de $S$. La propiedad de localidad \ref{prop:localidad de espacio tangente aproximado:eq:1} sugiere la posibilidad de definir un espacio tangente aproximado $\Tan{k}{S}{x}$ a conjuntos $S$ numerablemente $\Hm{k}$-rectificables.
\begin{defi}[Espacio tangente aproximado a un conjunto] \index{espacio!tangente aproximado}
Sea $S\subset \R^{N}$ un conjunto $\Hm{k}$-rectificable y sea $(S_{i})$ una partición de $S$ en conjuntos $\Hm{k}$-rectificables. Se define $\Tan{k}{S}{x}$ como el espacio tangente aproximado a $\Hm{k}\Radot S_{i}$ en $x$ para cualquier $x\in S_{i}$ donde este último esté definido.
\end{defi}

Se observa que la medida $\Tan{k}{\mu}{x}$ está univocamente definida en cualquier punto $x$ donde existe, y de esta medida, tanto el espacio tangente aproximado como la multiplicidad en $x$ se pueden recuperar. Sin embargo, la definición de espacio tangente aproximado a un conjunto está bien planteada (es decir, es independiente de la partición $(S_{i})$ escogida) únicamente si entendemos $\Tan{k}{S}{x}$ como una aplicación que manda $S$ a la $\Hm{k}$ clase de equivalencia en $\mathbf{G}_{k}$. En efecto, por la ecuación \ref{prop:localidad de espacio tangente aproximado:eq:1}, dos particiones diferentes producen aplicaciones del espacio tangente de un conjunto que coinciden en casi todo punto $\Hm{k}$ en $S$ y que satisfacen la propiedad de localidad:
\begin{align}\label{eq:1:propiedad de localidad del espacio tangente aproximado a un conjunto}
\begin{array}{ll}
\Tan{k}{S}{x}=\Tan{k}{S'}{x},& \text{para c.t.p.}\; x\in S\cap S^{'}\, \text{respecto de}\, \Hm{k},
\end{array}
\end{align}
para cualquier par de conjuntos $S, S'$ $\Hm{k}$-rectificables y con la propiedad de consistencia:
\begin{align}\label{eq:1:propiedad de consistencia espacio tangente aproximado a un conjunto}
\begin{array}{ll}
\displaystyle
\supp\left[ \Tan{k}{\theta \Hm{k}\Radot S}{x}\right ]=\Tan{k}{S}{x},& \text{para c.t.p.}\; x\in S\, \text{respecto de}\, \Hm{k},
\end{array}
\end{align}
para cualquier función sobre conjuntos de Borel $\theta:S\to (0,\infty)$ localmente sumable con respecto a $\Hm{k}\Radot S$.

\chapter{Funciones de variación acotada}\label{cap:cap2}
\DefaultSet{\Omega} %\DefaultSet{\mathbb{R}^{n}} 
En este capítulo se abordan las funciones de variación acotada y los conjuntos de perímetro finito. El objetivo es introducir los resultados necesarios sobre funciones de variación acotada que se usarán posteriormente en la demostración de los teoremas de cierre y compacidad del espacio de funciones especiales de variación acotada (capítulo \ref{cap:cap3}). Como veremos en el capítulo \ref{cap:cap4}, las funciones especiales de variación acotada son el marco adecuado para los problemas variacionales en los que estamos interesados.

En la primera sección de este capítulo se introducirá el espacio de funciones de variación acotada. Veremos las dos principales caracterizaciones de este espacio: funciones cuya derivada distribucional es una medida o funciones que aparecen como límite en $\L1$ de sucesiones de funciones acotadas en $\W11$. Veremos también que el espacio $BV$ es un espacio de Banach en el que se puede definir dos tipos de convergencias: una convergencia débil$^{*}$ y una convergencia estricta. Finalmente, probaremos un resultado de compacidad en $BV$ respecto de la convergencia débil$^{*}$ definida antes. En la segunda sección se abarcará el caso particular de funciones de variación acotada en una variable. En una dimensión las propiedades de esta clase de funciones se pueden estudiar con herramientas elementales de la teoría de la medida. La sección está enfocada en encontrar ``buenos'' representantes en la clase de equivalencia a la que pertenece una función $BV$. Es decir, buscaremos representantes que posean propiedades óptimas de continuidad y diferenciabilidad. En esta segunda sección se observarán en una variable muchas de las propiedades que poseen a su vez la funciones de variación acotada con dos o más variable independientes.
Ya en la tercera sección se estudiarán las propiedades básicas de los conjuntos de perímetro finito. Destacaremos las propiedades de compacidad y estabilidad de esta clase de conjuntos. También veremos la conexión entre funciones de variación acotada  y conjuntos de perímetro finito. Siguiendo las nociones y propiedades introducidas en la sección anterior, en la cuarta sección probaremos la desigualdad isoperimétrica, teorema de inmersión y desigualdades del tipo Poincaré en el espacio $BV$. En la quinta sección de este capítulo, estudiaremos las propiedades de límite aproximado de las funciones de variación acotada empezando por las funciones características de los conjuntos de perímetro finito. Para estas funciones probaremos que la derivada distribucional  es representable por integración respecto de la medida de Hausdorff $\Hm{N-1}$ y que para casi todo punto respecto de la medida $\Hm{N-1}$ la densidad $N$-dimensional del conjunto será, $0,1/2$ o $1$. Con el objetivo de extender a dimensiones mayores las propiedades enunciadas en la segunda sección definiremos en la sección sexta las nociones de límite aproximado, salto aproximado y de diferenciabilidad aproximada. Ya en el séptimo apartado del capítulo estudiaremos propiedades sobre la existencia de límite aproximado, rectificabilidad o la relación entre punto de discontinuidad aproximado y punto de salto. En la sección octava veremos algunos resultados de descomposición del espacio $BV$ y la existencia de trazas en la frontera del dominio. 
Para finalizar, en la novena sección estudiaremos la de derivada distribucional de una función $BV$ en varias variables, y en la última sección probaremos resultados del tipo regla de la cadena para funciones $BV$.
\section{El espacio de funciones de variación acotada}\label{sec:el espacio de funciones de variación acotada}
A lo largo de este capítulo denotaremos por $\DefaultSetWP$ un conjunto abierto arbitrario en $\R^{N}$. Empezamos esta sección con la definición más común de función $\BV$. En el espacio Euclídeo, las funciones $\BV$ se definen como funciones integrables cuyas derivadas parciales débiles son medidas con signo de masa finita. Por lo tanto, forman una clase de funciones más general que las funciones del espacio Sobolev, cuyas derivadas parciales débiles son funciones integrables.

\begin{defi}\label{def:funcion de variacion acotada} \index{función!de variacion acotada}\index{espacio!$BV$}
Sea $u\in \L1$, se dice que $u$ es una función de variación acotada en $\DefaultSetWP$ si la derivada distribucional de $u$ es representable por una medida de Radon finita en $\DefaultSetWP$, i.e. si:
\begin{align}\label{def:funcion de variacion acotada:eq:1}
\begin{array}{lll}
\displaystyle
\int_{\DefaultSetWP} u \diffp \phi x_{i} dx= - \int_{\DefaultSetWP} \phi \,dD_{i}u, & \forall\phi\in \Cc{\infty}, & i=1,\ldots, N,
\end{array}
\end{align}
para alguna medida $Du=(D_{1}u,\ldots, D_{N}u)$ en $\Omega$ que toma valores en $\R^{N}$. El espacio vectorial de todas las funciones de variación acotada en $\DefaultSetWP$ se denota por $\BV$. 
\end{defi}

Siguiendo un argumento de smoothing se demuestra que las fórmulas de integración por partes \ref{def:funcion de variacion acotada:eq:1} siguen siendo cierta para cualquier $\phi\in \Cc{1}$. Las fórmulas \ref{def:funcion de variacion acotada:eq:1} siguen siendo válidas incluso si se consideran como funciones tests funciones Lipschitz con soporte compacto en $\DefaultSetWP$. 

El sistema de ecuaciones de \ref{def:funcion de variacion acotada:eq:1} se puede resusmir en una sola ecuación
\begin{align}
\begin{array}{ll}
\displaystyle
\int_{\DefaultSetWP} u \div \varphi \,dx = -\sum_{i=1}^{N}\int_{\DefaultSetWP}\varphi_{i} \, dD_{i}u, & \forall\varphi \in \Cc[N]{1}.
\end{array}
\end{align}
En adelante consideramos funciones de variación acotada vectoriales $u\in \BV[m]$ y usaremos la misma notación que para funciones de variación acotada. En el caso de una función $u$ de variación acotada vectorial, $Du$ es una matriz $m\times N$ de medidas $D_{i}u^{\alpha}$ en $\DefaultSetWP$ que satisface:
\begin{align}\label{eq:1:defi funcion variacion acotada}
\begin{array}{lll}
\displaystyle
\int_{\DefaultSetWP} u^{\alpha}\diffp \phi x_{i} dx = -\int_{\DefaultSetWP} \phi\, dD_{i}u^{\alpha}, & \forall \phi \in \Cc{1}, &  i=1,\ldots,N, \; \alpha=1, \ldots, m,
\end{array}
\end{align}
o equivalentemente:
\begin{align}\label{eq:2:defi funcion variacion acotada}
\begin{array}{ll}
\displaystyle
\sum^{m}_{\alpha=1}\int_{\DefaultSetWP}u^{\alpha}\div \phi^{\alpha} dx=-\sum^{m}_{\alpha=1}\sum^{N}_{i=1}\int_{\DefaultSetWP}\phi_{i}^{\alpha}\, dD_{i}u^{\alpha}, & \forall \phi\in \Cc[mN]{1}.
\end{array}
\end{align}

Se observa que $\W11$ está contenido en $\BV$. En efecto para cualquier $u\in \W11$ su derivada distribucional es $\nabla u \Lm{N}$. La inclusión sin embargo es estricta, basta observar que la derivada distribucional de la función de Heaviside $\chi_{(0,\infty)}$ es la medida de Dirac $\delta_{0}$. A continuación enunciamos algunas propiedades sobre la derivada distribucional que usaremos en el resto del capítulo.
\begin{prop}[Propiedades de $Du$]\label{prop:propiedades de Du}
Sea $u\in \BVloc[m]$.
\begin{enumerate}[(a)]
\item Para cualquier función Lipschitz local $\psi: \DefaultSetWP \to \R$ la función $u\psi$ pertenece a $\BVloc[m]$ y
\begin{align*}
D(u\psi)=\psi Du+ (u\otimes \nabla \psi)\Lm{N}.
\end{align*}\label{prop:propiedades de Du:a}
\item Sea $(\rho_{\epsilon})_{\epsilon >0}$ una familia de mollifiers como se definió en el capítulo \ref{cap:cap1}. Si $\DefaultSetWP_{\epsilon}=\set{x\in \DefaultSetWP \given \dist{x}{\partial\DefaultSetWP}>\epsilon}$, entonces
\begin{align*}
\begin{array}{ll}
\displaystyle
\nabla(u*\rho_{\epsilon})=Du*\rho_{\epsilon}, & \text{en}\; \DefaultSetWP_{\epsilon}.
\end{array}
\end{align*}\label{prop:propiedades de Du:b}
\item Si $Du=0$, $u$ es una constante en cualquier componente conexa de $\DefaultSetWP$.\label{prop:propiedades de Du:c}
\end{enumerate}
\end{prop} 
\begin{proof}
\begin{enumerate}[(a)]
\item \DefaultSet{\Omega} Claramente $u\psi\in \Lloc{1}$. Para cualquier $\rho\in\Cc{\infty}$ y cada $i=1,\ldots, n$, y cada $\alpha=1,\ldots, m,$ se tiene:
\begin{align*}
\int_{\DefaultSetWP}u^{\alpha}\psi\diffp{\rho}{x_{i}}dx=\int_{\DefaultSetWP}u^{\alpha}\diffp{\rho\psi}{x_{i}}dx-\int_{\DefaultSetWP}u^{\alpha}\diffp{\psi}{x_{i}}dx=-\int_{\DefaultSetWP}\rho\psi\, dD_{i}u^{\alpha}-\int_{\DefaultSetWP}u^{\alpha}\diffp{\psi}{x_{i}}\rho \, dx,
\end{align*}
pues $\psi\rho\in \Lipc$.
\item Para demostrar este apartado basta ver que el comportamiento de la derivada con respecto a la convolución y el teorema de Fubini dan: 
\begin{align*}
\int_{\DefaultSetWP}(u*\rho_{\epsilon})\nabla \psi \,dx =&\int_{\DefaultSetWP}u(\rho_{\epsilon}*\nabla \psi)\,dx=\int_{\DefaultSetWP}u\nabla(\psi*\rho_{\epsilon})\,dx\\
=&-\int_{\DefaultSetWP} (\psi*\rho_{\epsilon})\,dDu=-\int_{\DefaultSetWP}(Du*\rho_{\epsilon})\psi \, dx, \quad \forall \psi\in \DefaultSet{\Omega_{\epsilon}}\Cc{\infty}.
\end{align*}
\DefaultSet{\Omega}
\item Para cualquier $\epsilon>0$ se tiene que $u*\rho_{\epsilon}\in \Cc{\infty}$ y por (\ref{prop:propiedades de Du:b}), $\nabla(u*\rho_{\epsilon})=Du*\rho_{\epsilon}=0$. Luego $u*\rho_{\epsilon}$ es constante para todo $\epsilon>0$, como $u*\rho_{\epsilon}$ convergen a $u$ en $L^{1}(\Omega)$ cuando $\epsilon$ tiende a $0$, se tiene que $u$ es constante en cualquier componente conexa de $\DefaultSetWP$.
\end{enumerate}
\end{proof}

Introducimos ahora el concepto de variación $\V{u}$ de una función $u\in \Lloc[m]{1}$. La variación de una función puede ser infinita pero en caso de ser finita se denominará de variación acotada de acuerdo con la definición \ref{def:funcion de variacion acotada}.
\begin{defi}[Variación] \index{variación!de una función $BV$}
Sea $u\in \Lloc[m]{1}$. La variación $\V{u}$ de $u$ en $\DefaultSetWP$ se define como:
\begin{align*}
\V{u}=\sup \set*{\sum^{m}_{\alpha=1}\int_{\DefaultSetWP} u^{\alpha}\div \varphi^{\alpha} \, dx\given \varphi \in \Cc[mN]{1}, \Norm{\varphi}[\infty]\leq 1}.
\end{align*}
\end{defi}
Integrando por parte se ve que $\V{u}=\int_{\DefaultSetWP}\abs{\nabla u}dx$ si $u$ es continuamente diferenciable en $\DefaultSetWP$. Enumeramos otras propiedades de la variación en la siguiente nota.
\begin{nota}[Propiedades de la variación]\label{nota:propiedades de la variación} \index{variación}
\begin{enumerate}[(a)]
\item \textit{(Semicontinuidad inferior)} La aplicación $u\mapsto \V{u}\in [0,\infty]$ es semicontinua inferiormente en la topología $\Lloc[m]{1}$. Basta obsevar que $u\mapsto \sum_{\alpha=1}^{m}\int_{\DefaultSetWP}u^{\alpha}\div \varphi^{\alpha} dx$ es una aplicación continua en la topología $\Lloc[m]{1}$ para cualquier elección de $\varphi\in \Cc[mN]{1}$. \label{nota:propiedades de la variación:a}
\item \textit{(Aditividad)}\DefaultSet{A} $\V{u}$ \DefaultSet{\Omega} está definido para cualquier conjunto abierto $A\subset \DefaultSetWP$. Se puede probar entonces que:\DefaultSet{A}
\begin{align*}
\begin{array}{ll}
\tilde{V}(u, B)=\inf\set{\V{u}\given B\subset A, A\, \text{abierto}}, & B\in \mathcal{B}(\DefaultSetWP),
\end{array}
\end{align*}
extiende $V(u, \cdot)$ a una medida de Borel en $\DefaultSetWP$. \label{nota:propiedades de la variación:b}
\item \textit{(Localidad)} La aplicación \DefaultSet{A} $u\mapsto \V{u}$ es local, i.e. $\V{u}=\V{v}$ si $u$ coincide con $v$ en c.t.p. respecto de la medida de Lebesgue $\Lm{N}$ en \DefaultSet{\Omega} $A\subset\Omega$. \label{nota:propiedades de la variación:c}
\end{enumerate}
\end{nota}

La siguiente proposición, nos muestra que en caso de ser finita la variación, los conceptos de variación y de variación total de la derivada distribucional coinciden.
\begin{prop}[Variación de funciones $BV$]\label{prop:variacion de funciones bv}
Sea $u\in \L[m]{1}$. Entonces $u$ pertenece a $\BV[m]$ si y solo si $\V{u}<\infty$. Además $\V{u}$ coincide con $|Du|(\DefaultSetWP)$ para cualquier $u\in \BV[m]$ y $u\mapsto |Du|(\DefaultSetWP)$ es semicontinua inferiormente en $\BV[m]$ respecto de la topología $\Lloc[m]{1}$.
\end{prop}
\begin{proof}
Si $u\in \BV[m]$ podemos estimar el supremo de $\V{u}$ observando que:
\begin{align*}
\sum_{\alpha=1}^{m}\int_{\DefaultSetWP}u^{\alpha}\div \varphi^{\alpha} dx =- \sum_{i=1}^{N}\sum_{\alpha=1}^{m}\int_{\DefaultSetWP} \varphi_{i}^{\alpha}\, dD_{i}u^{\alpha},
\end{align*}
para cualquier $\varphi\in \Cc[mN]{1}$. Como en la definición de $\V{u}$ se requiere que $\Norm{\varphi}[\infty]\leq 1$, por la proposición \ref{prop:variacion total de la medida} se tiene que $\V{u}\leq \abs{Du}(\DefaultSetWP)<\infty$.

Por el contrario, si $\V{u}<\infty$ entonces por homogeneidad se tiene:
\begin{align*}
\begin{array}{ll}
\displaystyle
\abs*{\sum_{\alpha=1}^{m} u^{\alpha}\div\varphi^{\alpha}\, dx}\leq \V{u}\Norm{\varphi}[\infty], & \forall \varphi\in \Cc[mN]{1}.
\end{array}
\end{align*}
Como $\Cc[mN]{1}$ es denso en $[C_{0}(\DefaultSetWP)]^{mN}$, podemos encontrar un funcional lineal continuo $L$ en $[C_{0}(\DefaultSetWP)]^{mN}$ que coincide con:
\begin{align*}
\varphi\mapsto\sum_{\alpha=1}^{m}\int_{\DefaultSetWP}u^{\alpha}\div \varphi^{\alpha} dx,
\end{align*}
en $\Cc[mN]{1}$ y que satisface $\Norm{L}\leq \V{u}$. Por el teorema de Riesz, existe una medida de Radon finita $\mu=(\mu_{i}^{\alpha})$ con valores en $\R^{mN}$ tal que $\Norm{L}=\abs{\mu}(\DefaultSetWP)$ y:
\begin{align*}
\begin{array}{ll}
\displaystyle
L(\varphi)=\sum^{N}_{i=1}\sum_{\alpha=1}^{m}\int_{\DefaultSetWP}\varphi_{i}^{\alpha}d\mu_{i}^{\alpha},& \forall \varphi\in [C_{0}(\DefaultSetWP)]^{mN}.
\end{array}
\end{align*}
Por la ecuación \ref{eq:2:defi funcion variacion acotada} y la identidad:
\begin{align*}
\begin{array}{ll}
\displaystyle
\sum_{\alpha=1}^{m}\int_{\DefaultSetWP}u^{\alpha}\div\varphi^{\alpha}dx=\sum_{i=1}^{N}\sum_{\alpha=1}^{m}\int_{\DefaultSetWP}\varphi_{i}^{\alpha}d\mu_{i}^{\alpha},& \forall \varphi\in \Cc[mN]{1},
\end{array}
\end{align*}
se obtiene que $u\in \BV[m]$, $Du=-\mu$ y además
\begin{align*}
\abs{Du}(\DefaultSetWP)=\abs{\mu}(\DefaultSetWP)=\Norm{L}\leq \V{u}.
\end{align*}
Finalmente, la semicontinuidad inferior de $u\mapsto \abs{Du}(\DefaultSetWP)$ se deduce directamente de la nota (\ref{nota:propiedades de la variación}\ref{nota:propiedades de la variación:a}).
\end{proof}

Es fácil ver que el espacio $\BV[m]$ dotado de la norma: \index{espacio!$BV$}
\begin{align}
\Norm{u}[BV]=\int_{\DefaultSetWP} \abs{u} dx + \abs{Du}(\DefaultSetWP),
\end{align}
es un espacio de Banach. Se concluye rápidamente que si $u\in \W11$ entonces $\abs{Du}(\DefaultSetWP) = \Norm{\nabla u}[\L1]$. Con esta norma las funciones suaves no son densas en $\BV$ pues $\W11\subsetneq \BV$. Sin embargo, las funciones $\BV[m]$ se pueden aproximar en la topología de $\L[m]{1}$  por funciones suaves cuyo gradiente es acotado en norma $\L[m]{1}$. 

\begin{nota}\label{nota:desigualdades de normas bv}
Se puede remplazar en la definición dada de la norma en $BV$ el término $\abs{Du}(\DefaultSetWP)$ por $\sum_{\alpha}\abs{Du^{\alpha}}(\DefaultSetWP)$. Es decir, adoptar la normal matricial $\abs{A}_{1}=\sum_{\alpha}\abs{A^{\alpha}}$ para matrices $m\times N$ con filas $A^{1}, \ldots, A^{m}$. Por las desigualdades siguiente: 
\begin{align}\label{eq:1:desigualdades de normas bv}
\max_{1\leq \alpha\leq m}\abs{Du^{\alpha}}(\DefaultSetWP)\leq \abs{Du}(\DefaultSetWP)\leq \sum_{\alpha=1}^{m}\abs{Du^{\alpha}}(\DefaultSetWP),
\end{align}
se concluye que esta nueva norma es equivalente en $\BV[m]$ a la dada anteriormente. Sin embargo, como $\abs{\,\cdot\,}_{1}$ es una norma que no es estrictamente convexa surgen numerosos problemas en demostraciones que incluyen argumentos de smoothing.
\end{nota}

A continuación enunciamos un lema que será de gran utilidad a la hora de aproximar las funciones $BV$ en la norma $[L^{1}(\Omega)]^{m}$.
\begin{prop}\label{prop:aproximacion Du}
Sea $u\in\BV[m]$ y $U\subset\subset \DefaultSetWP$ tal que $\abs{Du}(\partial U)=0$. Entonces,
\begin{align*}
\lim_{\epsilon\to 0} \abs{Du_{\epsilon}}(U)=\abs{Du}(U).
\end{align*} 
\end{prop}
\begin{proof}
Por la semicontinuidad inferior de la variación, se tiene que $\abs{Du}(U)\leq \lim_{\epsilon}\abs{Du_{\epsilon}}(U)$. Por otro lado, denotando por $U_{\epsilon}$ la $\epsilon$-vecindad abierta de $U$, por el apartado (\ref{teo:propiedades de molliers:b}) del teorema \ref{teo:propiedades de molliers} se deduce que:
\begin{align*}
\limsup_{\epsilon\to 0} \abs{Du_{\epsilon}}(U)\leq \limsup_{\epsilon\to 0} \abs{Du}(U_{\epsilon})=\abs{Du}(\overline{U})=\abs{Du}(U).
\end{align*}
\end{proof}
El resultado global de la proposición anterior, es decir $\lim_{\epsilon\to 0}\abs{Du_{\epsilon}}(\R^{N})=\abs{Du}(\R^{N})$, también es cierto y la demostración es análoga. Ahora vemos que para dominios arbitrarios de $\DefaultSetWP$, la aproximación por funciones suaves con gradiente acotado en $L^{1}$ caracteriza a las funciones $BV$. 
\begin{teo}[Aproximación por funciones suaves] \label{teo:aproximación por funciones suaves}
Sea $u\in \L[m]{1}$. Entonces $u\in \BV[m]$ si y solo si existe una sucesión $(u_{h})\subset \C[m]{\infty}$ convergente a $u$ en $\L[m]1$ que satisface:
\begin{align}\label{teo:aproximación por funciones suaves:eq:1}
L=\lim_{h\to \infty}\int_{\DefaultSetWP} \abs{\nabla u_{h}}\,dx<\infty.
\end{align}
Además la constante $L$ en \ref{teo:aproximación por funciones suaves:eq:1} es $\abs{Du}(\DefaultSetWP)$.
\end{teo}
\begin{proof}
Probamos primero la implicación izquierda. Supongamos que $u$ se puede aproximar en $[L^{1}(\Omega)]$ por funciones suaves que satisfacen \ref{teo:aproximación por funciones suaves:eq:1}. Extrayendo quizás una subsucesión, entonces por la compacidad débil$^{*}$ del espacio de medidas de Radon finitas, se tiene que $\nabla{u}_{h}\Lm{N}$ converge en sentido débil$^{*}$ en $\Omega$ a alguna medida $\mu$ tal que $\abs{\mu}\leq L$. Tomando límite $h\to \infty$ en la fórmula de integración por partes se ve que $\mu=Du$, es decir, $u\in \BV[m]$. En particular, $\abs{Du}(\Omega)=\abs{\mu}(\Omega)\leq L$.

La prueba de la implicación derecha es análoga a la demostración de que $C^{\infty}(\Omega)\cap \W{1}{p}$ es denso en $\W{1}{p}$ (véase teorema de Meyers-Serrin en  \cite{evans1998partial}) remplazando la norma de $\W{1}{p}$ por la norma de $\BV$.
\end{proof}

\begin{nota}\label{nota:aproximación por funciones suaves}
Si $u$ es solamente localmente sumable en $\DefaultSetWP$, análogamente se puede obtener una sucesión $(u_{h})\subset C^{\infty}(\DefaultSetWP)$ convergente a $u$ en $\Lloc[m]{1}$ tal que $\V{u_{h}}$ converge a $\V{u}$.
\end{nota}

Ahora introducimos la noción de convergencia débil$^{*}$ y la convergencia estricta en $\BV$. La primera es útil por sus propiedades de compacidad mientras que la segunda se usa para probar numerosas identidades en $\BV$  mediante el uso de argumentos de smoothing.
\begin{defi}[Convergencia débil$^{*}$]
Sea $u,u_{h}\in \BV[m]$. Se dice que $(u_{h})$ converge débilmente$^{*}$ en $\BV[m]$ a $u$ si $(u_{h})$ converge a $u$ en $\L[m]{1}$ y $(Du_{h})$ converge débilmente$^{*}$ a $Du$ en $\DefaultSetWP$, i.e. :
\begin{align*}
\begin{array}{ll}
\displaystyle 
\lim_{h\to \infty}\int_{\DefaultSetWP} \phi\, dDu_{h}=\int_{\DefaultSetWP}\phi \, dDu, & \forall \phi\in C_{0}(\DefaultSetWP).
\end{array}
\end{align*}
\end{defi}

Un criterio simple para la convergencia débil$^{*}$ se establece en la siguiente proposición.
\begin{prop}\label{prop:criterio convergencia débi} \index{convergencia!débil*}
Sea $(u_{h})\subset \BV[m]$. Entonces $(u_{h})$ converge débilmente$^{*}$ a $u$ en $\BV[m]$ si y solo si $(u_{h})$ es acotada en $\BV[m]$ y converge a $u$ en $\L[m]{1}$.
\end{prop}
\begin{proof} \DefaultSet{\Omega}
 Supongamos que $(u_{h})$ está acotado en $BV$ y que converge en norma $L^{1}$ a $u$, basta probar entonces que $(Du_{h})$ converge débilmente$^{*}$ a $Du$ en $\DefaultSetWP$. Por la compacidad débil$^{*}$ de las medidas de Radon, $(Du_{h})$ es débilmente$^{*}$ relativamente compacto, por lo tanto basta probar que cualquier límite $\mu=\lim_{k}Du_{h(k)}$ coincide con $Du$. En efecto, tomando límites cuando $k\to \infty$ en:
\begin{align*}
\begin{array}{lll}
\displaystyle
\int_{\DefaultSetWP}u^{\alpha}_{h(k)}\diffp{\phi}{x_{i}}dx=-\int_{\DefaultSetWP}\phi\, dD_{i}u^{\alpha}_{h(k)},& \forall \phi\in\Cc{1}, & i=1,\ldots, N, \; \alpha=1,\ldots,m,
\end{array}
\end{align*}
obtenemos que,
\begin{align*}
\begin{array}{lll}
\displaystyle
\int_{\DefaultSetWP}u^{\alpha}\diffp{\phi}{x_{i}}dx=-\int_{\DefaultSetWP}\phi\, d\mu_{i}^{\alpha},& \forall \phi\in\Cc{1}, & i=1,\ldots, N, \; \alpha=1,\ldots,m,
\end{array}
\end{align*}
como queríamos probar.

La implicación opuesta se sigue del teorema de Banach-Steinhaus, puesto que la convergencia débil$^{*}$ de medidas de Radon finitas en $\DefaultSetWP$  se corresponde con convergencia débil$^{*}$  en el dual de $[C_{0}(\DefaultSetWP)]^{mN}$.
\end{proof} 

La convergencia estricta en el espacio de funciones de variación acotada se define como sigue.
\begin{defi}[Convergencia estricta] \index{convergencia!estricta}
Sea $u,u_{h}\in \BV[m]$. Se dice que $(u_{h})$ converge estrictamente en $\BV[m]$ a $u$ si $(u_{h})$ converge a $u$ en $\L[m]{1}$ y $\abs{Du_{h}}(\DefaultSetWP)$ converge a $\abs{Du}(\DefaultSetWP)$ cuando $h\to \infty$.
\end{defi}

Por el teorema \ref{teo:aproximación por funciones suaves} sabemos que para todo conjunto abierto $\DefaultSetWP$ el espacio $[C^{\infty}(\DefaultSetWP)]^{m}\cap \BV[m]$ es denso en el espacio $\BV[m]$ dotado de la topología inducida por la convergencia estricta. También sabemos por la proposición \ref{prop:criterio convergencia débi} que la convergencia estricta implica la convergencia débil$^{*}$. Sin embargo el recíproco no es cierto, \DefaultSet{0,2\pi} basta considerar las funciones $\sin(hx)/h$ que convergen débilmente$^{*}$ en $\BV$ a $0$, pero la convergencia no es estricta ya que $\abs{Du_{h}}((0,2\pi))=4$ para cualquier $h\geq 1$.

\DefaultSet{\Omega}
En ocasiones se ha de extender una función $u\in \BV[m]$ a una función \DefaultSet{\R^{N}}$\tilde{u}\in\BV[m]$. Esto permite deducir resultados globales en $\Omega$ de resultados locales en $\R^{N}$. Es por ello, que introducimos a continuación el concepto de extensión de dominio.
\begin{defi}[Extensión de dominio]\DefaultSet{\Omega}\index{extensión de dominio}
Se dice que un conjunto abierto $\DefaultSetWP\subset \R$ es una extensión de dominio si $\partial \DefaultSetWP$ está acotada y si para todo conjunto abierto $A\supset \overline{\DefaultSetWP}$ y para cualquier $m\geq 1$, existe un operador de extensión, lineal y continuo $T:\BV[m]\to [BV(\R^{N})]^{m}$ que satisface:
\begin{enumerate}[(a)]
\item $Tu=0$ c.t.p. en $\R^{N}\setminus A$ para cualquier $u\in \BV[m]$,
\item $\abs{DTu}(\partial \DefaultSetWP)=0$ para cualquier $u\in \BV[m]$,
\item para cualquier $p\in[0,\infty]$ la restricción de $T$ a $[\W{1}{p}]^{m}$ induce una aplicación lineal continua entre este espacio y \DefaultSet{\R^{N}}$[\W{1}{p}]^{m}$.
\end{enumerate}
\end{defi}
\DefaultSet{\Omega}
Para concluir esta sección, estudiamos el teorema compacidad \ref{teo:compacidad en BV} para funciones $BV$ que será de gran utilidad a la hora de trabajar con problemas variacionales. El espacio de Sobolev $W^{1,1}$ no posee semejante propiedad de compacidad y por lo tanto esto justifica la introducción de los espacios $BV$ en el cálculo de variaciones. Sin embargo antes de probar este resultado necesitaremos el siguiente lema técnico.
\begin{lema}\label{lema:compacidad en BV}\index{compacidad!$BV$}
Sea $u\in\BV[m]$ y $K\subset\DefaultSetWP$ un conjunto compacto. Entonces:
\begin{align*}
\begin{array}{ll}
\displaystyle
\int_{K}\abs{u*\rho_{\epsilon}-u} dx \leq \epsilon \abs{Du}(\DefaultSetWP), & \forall \epsilon \in (0, \dist{K}{\partial\DefaultSetWP}),
\end{array}
\end{align*}
para toda familia de mollifiers $(\rho_{\epsilon})_{\epsilon>0}$ como se definió en el capítulo anterior.
\end{lema}
\begin{proof}
Sin pérdida de generalidad por el teorema \ref{teo:aproximación por funciones suaves} suponemos que $u\in [C^{1}(\DefaultSetWP)]^{m}$.
El lema se deduce rápidamente al considerar la identidad:
\begin{align*}
\begin{array}{ll}
\displaystyle
u(x-\epsilon y)-u(x)=-\epsilon\int_{0}^{1}\Crochet{\nabla u(x-\epsilon t y)}{y} dt & x\in K, y\in B_{1},
\end{array}
\end{align*}
tomando normas a ambos lados, integrando y usando Fubini se obtiene la desigualdad:
\begin{align*}
\int_{K}\abs*{u(x-\epsilon t y)-u(x)}dx \leq \epsilon \int_{0}^{1}\int_{K}\abs*{\nabla u(x-\epsilon t y)}dxdt\leq\epsilon |Du|(\Omega).
\end{align*}
Multiplicando a ambos lados por $\rho(y)$ e integrando:
\begin{align*}
\int_{K}|u*\rho_{\epsilon}(x)-u(x)|dx=\int_{K}\left (\int_{\mathbb{R}^{n}}\abs*{u(x-\epsilon t y)-u(x)}\rho(y)dy \right )dx\leq \epsilon|Du|(\Omega).
\end{align*}
\end{proof}

Ya estamos en disposición de demostrar la compacidad del espacio de funciones $BV$ respecto de la convergencia débil$^{*}$.
\begin{teo}[Compacidad en $BV$]\label{teo:compacidad en BV}\index{compacidad!$BV$}
Cualquier sucesión $(u_{h})\subset \BVloc[m]$ que satisface :
\begin{align*}
\begin{array}{ll}
\displaystyle
\sup\set*{ \int_{A}\abs{u_{h}}dx+\abs{Du_{h}}(A)\given h\in \N}<\infty, & \forall A\subset\subset \DefaultSetWP\; \text{abierto},
\end{array}
\end{align*}
admite una subsucesión $(u_{h(k)})$ que converge en $\Lloc[m]{1}$ a $u\in \BVloc[m]$. Si $\DefaultSetWP$ es una extensión acotada del dominio y la sucesión está acotada en $\BV[m]$ entonces $u\in \BV[m]$ y la subsucesión converge débilmente$^{*}$ a $u$.
\end{teo}
\begin{proof}
Sea $\DefaultSetWP'\subset\subset \DefaultSetWP$ un conjunto abierto. Por un argumento diagonal basta probar la existencia de una subsucesión $(u_{h(k)})$ que converge en $[L^{1}(\DefaultSetWP')]^{m}$ a una función $u$.

Sea $\delta=\dist{\Omega'}{\partial\Omega}>0$, $U\subset \Omega$ una vecindad abierta $\delta/2$ de $\Omega$ y sea $u_{h,\epsilon}=u_{h}*\rho_{\epsilon}$. Si $\epsilon\in (0, \delta/2)$ las funciones $u_{h, \epsilon}$ son suaves en $\overline{\Omega'}$ y satisfacen:
\begin{align*}
\Norm{u_{h,\epsilon}}[C(\overline{\Omega'})]\leq \Norm{u_{h}}[L^{1}(U)]\Norm{\rho_{\epsilon}}[\infty],\quad \Norm{\nabla u_{h, \epsilon}}[C(\overline{\Omega'})]\leq \Norm{u_{h}}[L^{1}(U)]\Norm{\nabla\rho_{\epsilon}}[\infty].
\end{align*}
Por hipótesis sobre $(u_{h})$, la sucesión $(u_{h,\epsilon})$ es equiacotada y equicontinua para $\epsilon$ fijo. Esto significa que fijado un $\epsilon$, podemos encontrar una subsucesión de $(u_{h,\epsilon})$ convergente en $C(\overline{\DefaultSetWP'})$. Por un argumento diagonal podemos encontrar una subsucesión $(h(k))$ tal que $(u_{h(k),\epsilon})$ converge en $C(\overline{\Omega'})$ para cualquier $\epsilon=1/p$, con $p>2/\delta$ entero. Aplicando el lema anterior \ref{lema:compacidad en BV} vemos que:
\begin{align*}
\limsup_{k,k'\to \infty}&\int_{\Omega'}\abs*{u_{h(k)}-u_{h(k')}}dx\leq \limsup_{k,k'\to \infty}\int_{\Omega'} \abs*{u_{h(k),1/p}-u_{h(k'), 1/p}}dx\\
&+\limsup_{k,k'\to \infty}\int_{\Omega'}\left [\abs*{u_{h(k)}-u_{h(k), 1/p}}+ \abs*{u_{h(k'),1/p}-u_{h(k')}}\right]dx\leq \dfrac{2}{p}\sup_{h\in \N} \abs*{Du_{h}}(U).
\end{align*}
Eligiendo un $p$ arbitrariamente grande y como $L^{1}(\Omega')$ es un espacio de Banach, esto prueba que $(u_{h(k)})$ converge en $L^{1}(\Omega')$.

Finalmente, probamos la última parte del teorema. Si suponemos que $\Omega$ es una extensión acotada de dominio, podemos aplicar la primera parte del teorema a la extensión $Tu_{h}\in [BV(\R^{N})]^{m}$ para obtener la convergencia de una subsucesión $(Tu_{h(k)})$ a una función $u\in [BV(\R^{N})]^{m}$ en $[L^{1}_{loc}(\R^{N})]^{m}$. En particular, $(u_{h(k)})$ converge a $u$ en $[\L1]^{m}$ y $u\in \BV[m]$ por la proposición \ref{prop:criterio convergencia débi}. La convergencia débil se sigue de la proposición \ref{prop:criterio convergencia débi}.
\end{proof}

\section{Funciones $BV$ en una variable}\label{sec:Funciones BV en una variable}

En esta sección examinaremos el comportamiento puntual de las funciones de variación acotada en una variable. Para ello primero definiremos la variación puntual y veremos que en cada clase de equivalencia existe un representante con buenas propiedades de continuidad y diferenciabilidad. También estudiaremos la estructura de la derivada distribucional. Los resultados de esta sección nos permitirán hacernos una idea de la estructura de las funciones de variación acotada en varias variables ya que muchos conceptos y resultados se adaptan fácilmente al caso general.

Empezamos definiendo la variación puntual de una función de variación acotada en una variable.
\begin{defi}[Variación puntual]\index{variación!puntual}
Sea $a,b\in \overline{\R}$ con $a<b$ y $I=(a,b)$. Para cualquier función $u:I\to \R^{m}$ se define la variación puntual \DefaultSet{I} $\PV{u}$ de $u$ en $I$ como:
\begin{align*}
\PV{u}=\sup\set*{\sum_{i=1}^{n-1}\abs{u(t_{i+1})-u(t_{i})}\given n\geq 2, \quad a<t_{1}<\cdots<t_{n}<b}.
\end{align*}
Si \DefaultSet{\Omega}$\DefaultSetWP\subset \R$ es un abierto, se define la variación puntual $\PV{u}$ por \DefaultSet{I} $\sum_{I}\PV{u}$, donde la suma es sobre cada una de las componentes conexas de \DefaultSet{\Omega}$\DefaultSetWP$.
\end{defi}
Al tratarse del supremo de un funcional continuo la aplicación \DefaultSet{I}$u\mapsto \PV{u}$ es semincontinua inferiormente con respecto a la convergencia puntal en $I$. Por aditividad \DefaultSet{\Omega}$\PV{u}$ también es semicontinua inferiormente para cualquier conjunto abierto $\DefaultSetWP\subset\R$.

\begin{nota}
\begin{enumerate}[(a)]
\item Se sigue directamente de la definición que cualquier función $u$ con variación puntual finita en un intervalo $I\subset \R$ está acotada, pues sus oscilaciones se pueden estimar gracias a \DefaultSet{I} $\PV{u}$. Cualquier función $u$ con valores reales  en $I=(a,b)$ que además está acotada, tienes variación puntual finita, y esta es igual a la oscilación $\abs{u(b_{-})-u(a_{+})}$. 

\item Recordamos que un resultado clásico debido a Jordan es que las funciones de variación acotada en un intervalo se pueden representar como la diferencia de dos funciones monótonas acotadas. Para cualquier entero $h\geq 1$, denotamos por $\set{x_{i}^{h}}$ una colección posiblemente infinita de puntos tales que:
\begin{align*}
\begin{array}{llll}
\displaystyle
x^{h}_{-n_{h}}=a, & x_{n_{h}}^{h}=b, & \displaystyle 0<x_{i+1}^{h}-x_{i}^{h}\leq \frac{1}{h}, & \forall i \in \Z\cap [-n_{h}+1, n_{h}-2],
\end{array}
\end{align*}
con $x^{h}_{-n_{h}+1}\downarrow a$ y $x^{h}_{n_{h}-1}\uparrow b$, entonces la representación de las componentes de $u$ como diferencia de dos funciones monótonas acotadas implica que las funciones escalón: 
\begin{align}\label{eq:represesentacion funcion BV como suma funciones escalon}
u_{h}(t)=\sum^{n_{h}-1}_{i=-n_{h}}u(y^{h}_{i})\chi_{(x_{i}^{h},x_{i+1}^{h}]}(t),
\end{align}
convergen en $\Lloc[m]{1}$ a $u$ cuando $h\to \infty$ para cualquier elección de $y_{i}^{h}\in (x_{i}^{h},x_{i+1}^{h})$. 
\end{enumerate}
\end{nota}

Es evidente que la variación puntual $\PV{u}$ es sensible ante modificaciones puntuales de valores de $u$. Esto sugiere la necesidad de definir el concepto de variación esencial $\EV{u}$ en el que se busca minimizar la variación puntual en la clase de equivalencia.
\begin{defi}[Variación esencial]\label{def: variacion esencial}\index{variación!esencial}
Sea $\DefaultSetWP\subset \R$ un abierto. Para cualquier función $u:I\to \R$  se define la variación esencial de $u$ en $\DefaultSetWP$ como:
\begin{align}\label{def: variacion esencial:eq:1}
\EV{u}=\inf\set*{\PV{v}\given v=u\; \Lm{1}\text{-c.t.p. en}\; \DefaultSetWP}.
\end{align}
\end{defi}

Se observa que en el caso de que la función sea integrable localmente entonces la variación esencial coincide con la noción de variación vista en la sección anterior.
\begin{teo}\label{teo:variacion esencial equivalente a variacion}
Para cualquier $u\in \Lloc[m]{1}$ el ínfimo en \ref{def: variacion esencial:eq:1} se alcanza y la variación $\V{u}$ coincide con la variación esencial $\EV{u}$.
\end{teo}
\begin{proof}
Suponemos primero que $\Omega=I=(a,b)$ es un intervalo. Probamos entonces la \DefaultSet{I}desigualdad $\V{u}\leq \EV{u}$. Por la definición de la variación esencial, hay que probar que $\V{u}\leq \PV{v}$ para cualquier $v$ en la clase de equivalencia de $u$. Para $h\geq 1$, sean $v_{h}$ unas funciones escalón como en \ref{eq:represesentacion funcion BV como suma funciones escalon}, luego:
\begin{align*}
\V{v_{h}}=\sum_{i=-n_{h}}^{n_{h}-1}\abs{v(y^{h}_{i+1})-v(y_{i}^{h})}\leq \PV{v}.
\end{align*}
Suponiendo sin pérdida de generalidad que $\PV{v}$ es finita, podemos pasar al límite haciendo tender $h\to \infty$ y por la semicontinuidad inferior de la variación en la topología $[L^{1}_{loc}(I)]^{m}$ obtener:
\begin{align*}
\V{u}=\V{v}\leq \liminf_{h\to \infty} \V{v_{h}}\leq \PV{v}.
\end{align*}
Ahora probamos la desigualdad opuesta $\EV{u}\leq \V{u}$.  Por ello, no es restrictivo suponer que $\V{u}<\infty$, luego por la proposición \ref{prop:variacion de funciones bv}, $u\in \BVloc[m]$ y $\abs{Du}(J)=\V{u}$ para cualquier $J\subset\subset I$. Como $Du$ es una medida de Radon en $I$ y:
\begin{align*}
\sup_{J\subset\subset I}\abs{Du}(J)=\sup_{J\subset\subset I} \V{u}=\V{u}<\infty,
\end{align*}
por la nota \ref{nota:extension de medida de Radon a medida de Radon finita} se puede extender a una medida de Radon finita en $I$ que satisfaga $\abs{Du}(I)=\V{u}$. Sea $\mu=Du$, $w(t)=\mu((a,t))$, por Fubini se obtiene que $Dw=\mu$; como consecuencia la proposición \ref{prop:propiedades de Du}(\ref{prop:propiedades de Du:c}) proporciona un $c\in \R^{m}$ tal que $u(t)-w(t)=c$ para casi todo $t\in I$ respecto de la medida $\Lm{1}$, y esto prueba que $w+c$ pertenecen a la clase de equivalencia de $u$. Ahora observamos que:
\begin{align*}
\sum_{i=1}^{n-1}\abs*{w(t_{i+1})-w(t_{i})}=\sum_{i=1}^{n-1}\abs*{\mu\left([t_{i}, t_{i+1})\right )}\leq \sum_{i=1}^{n-1}\abs*{\mu}([t_{i},t_{i+1}))\leq \abs{\mu}(I),
\end{align*}
para cualquier colección de puntos $t_1<\cdots<t_{n}$ en $I$. En particular:
\begin{align*}
\EV{u}\leq \PV{w+c}=\PV{w}\leq \abs{\mu}(I)=\V{u}.
\end{align*}
Y vemos que como consecuencia de la demostración $w+c$ es un minimizador en \ref{def: variacion esencial:eq:1}. 
\DefaultSet{\Omega}
Finalmente, si $\Omega$ es un conjunto abierto, se observa que $\EV{u}=\DefaultSet{I}\sum_{I}\EV{u}$ pues $\Omega$ tiene como mucho una cantidad numerable de componentes conexas. Luego, por aditividad, $\V{u}$ y $\EV{u}$ coinciden. 
\end{proof}

Si $u\in \BV[m]$, por la proposición \ref{prop:variacion de funciones bv} sabemos que $\V{u}=\abs{Du}(\DefaultSetWP)<\infty$. Como $\V{u}=\EV{u}$, entonces existe una función $\overline{u}$ en la clase de equivalencia de $u$ tal que:
\begin{align}
\PV{\overline{u}}=\EV{u}=\V{u}.
\end{align} 
Cualquier representante de $u$ con esta propiedad se dirá que es un buen representante. El siguiente teorema proporciona las características y propiedades principales de los buenos representantes.
\begin{teo}[Buenos representantes]\label{teo:buenos representantes}\index{buen representante}
 Sea $I=(a,b)\subset \R$ un intervalo y sea \DefaultSet{I}$u\in\BV[m]$.\DefaultSet{\Omega} Denotemos por $A$ el conjunto de átomos de $Du$, i.e. $t\in A$ si y solo si $Du(\set{t})\not =0$. Entonces las siguientes  afirmaciones son ciertas:
\begin{enumerate}[(a)]
\item Existe un único $c\in \R^{m}$ tal que:
\begin{align*}
\begin{array}{lll}
u^{l}(t)=c+Du((a,t)),& u^{r}(t)=c+Du((a,t]),& t\in I,
\end{array}
\end{align*}
son buenos representantes de $u$. Cualquier otra función $\overline{u}:I\to \R^{m}$ es un buen representante de $u$ si y solo si:
\begin{align}\label{teo:buenos representantes:eq:1}
\begin{array}{ll}
\overline{u}(t)\in\set*{\theta u^{l}(t)+(1-\theta)u^{r}(t)\given \theta \in [0,1]}, & \forall t\in I.
\end{array}
\end{align}\label{teo:buenos representantes:a}
\item Cualquier buen representante $\overline{u}$ es continuo en $I\setminus A$ y tiene una discontinuidad de tipo salto en cualquier punto de $A$:
\begin{align*}
\begin{array}{lll}
\overline{u}(t_{-})=u^{l}(t)=u^{r}(t_{-}),\quad \overline{u}(t_{+})=u^{l}(t_{+})=u^{r}(t), & \forall t\in A.
\end{array}
\end{align*}\label{teo:buenos representantes:b}
%\item Cualquier buen representante $\overline{u}$ es diferenciable en $\Lm{1}$-c.t.p. de $I$. La derivada $\overline{u}'$ es la densidad de $Du$ con respecto a $\Lm{1}$.
\end{enumerate}\label{teo:buenos representantes:c}
\end{teo}
\begin{proof}
\begin{enumerate}[(a)]
\item Se ha visto en la demostración del teorema anterior que la función $u^{l}$ es un buen representante de $u$, para algún $c\in \R^{m}$. Análogamente se prueba que $u^{r}$ es un buen representante. 

Para probar que cualquier función $\tilde{u}$ que satisface \ref{teo:buenos representantes:eq:1} es un buen representante  y viceversa, primero acotamos la variación de $\tilde{u}$ por la variación de $u^{l}$ y la diferencia de entre $u^{l}$y $u^{r}$. Con esto se prueba la implicación derecha. Para probar la otra implicación se basta observa que $pV(\tilde{u}, \cdot)$ es superaditiva y regular.
Para la demostración completa véase teorema 3.28 de \cite{ambrosio2000functions}.
\item El resultado del apartado (\ref{teo:buenos representantes:b}) es consecuencia directa de (\ref{teo:buenos representantes:a}). En efecto por definición, $u^{l}$ y $u^{r}$ son continuos y coinciden en cualquier punto de $I\setminus A$. Por la ecuación \ref{teo:buenos representantes:eq:1} cualquier otro buen representante tiene las mismas características. Por esta razón, el límite izquierdo y derecho de un buen representante coincide con el de $u^{l}, u^{r}$ respectivamente.
\end{enumerate}
\end{proof}

Por lo general, cualquier medida $\mu$ sobre un conjunto abierto $\DefaultSetWP\subset \R$ se puede descomponer en tres partes, una parte absolutamente continua \index{parte!absolutamente continua} $\mu^{a}$, una parte puramente atómica \index{parte!puramente atómica} $\mu^{j}$, y una parte difusiva \index{parte!difusiva} (i.e. sin átomos) $\mu^{c}$. Para obtener esta descomposición vemos primero que gracias al teorema de Radon-Nikod\'ym-Lebesgue la medida $\mu$ se divide en una parte absolutamente continua $\mu^{\alpha}$ y una parte singular $\mu^{s}$. Al igual que antes denotamos por $A$ al conjunto de átomos de $\mu$. Si se definen $\mu^{j}=\mu^{s}\Radot A$ y $\mu^{c}=\mu^{s}\Radot (\Omega\setminus A)$ obtenemos entonces la descomposición siguiente:
\begin{align}\label{eq:descomposicion mu bv 1 dimension}
\mu=\mu^{a}+\mu^{s}=\mu^{a}+\mu^{j}+\mu^{c}.
\end{align}
Esta descomposición es única y como las medidas $\mu^{a}, \mu^{j}, \mu^{c}$ son mutuamente singulares se tiene que $|\mu|=|\mu^{a}|+|\mu^{j}|+|\mu^{c}|$.

De acuerdo con la descomposición anterior, se dirá que $u\in \BV$ es una función salto si $Du=D^{j}u$, i.e. si $Du$ es una medida puramente atómica. Si $Du=D^{c}u$, i.e. si $Du$ es una medida singular sin átomos, entonces en este caso se dirá que $u$ es una función Cantor. Por lo tanto, por \ref{eq:descomposicion mu bv 1 dimension} deducimos el siguiente resultado de representación para funciones $BV$ en un intervalo.
\begin{cor}[Descomposición de funciones $BV$]\label{cor:descomposición de funciones BV}
Sea $\DefaultSetWP=(a,b)\subset \R$ un intervalo acotado. Entonces, cualquier función $u\in \BV[m]$ se puede representar como la suma $u^{a}+u^{j}+u^{c}$, donde $u^{a}\in[\W{1}{1}]^{m}$, $u^{j}$ es una función salto y $u^{c}$ es una función Cantor. Las tres funciones están únicamente determinadas salvo la suma de una constante y se tiene: 
\begin{align*}
|Du|(\DefaultSetWP)&=|D^{a}u|(\DefaultSetWP)+|D^{j}u|(\DefaultSetWP)+|D^{c}u|(\DefaultSetWP).
\end{align*}
\end{cor}

\section{Conjuntos de perímetro finito}\label{sec:conjuntos de perímetro finito}

En esta sección estudiaremos un tipo de funciones de variación acotada que son las funciones características de los conjuntos de perímetro finito. Reformularemos muchas de las definiciones y resultados introducidos en la sección \ref{sec:el espacio de funciones de variación acotada} para este caso particular de funciones variación acotada.

Comenzamos dando la definición de conjunto de perímetro finito.

\begin{defi}[Conjunto de perímetro finito]\DefaultSet{\Omega}\label{def:definicion de conjunto de perímetro finito}\index{conjunto!de perímetro finito} \index{perímetro}
Sea $E\subset \R^{N}$ un conjunto $\Lm{N}$-medible. Se define el perímetro del conjunto $E$ en $\DefaultSetWP$ como la variación de $\chi_{E}$ en $\DefaultSetWP$, i.e.
\begin{align}\label{def:definicion de conjunto de perímetro finito:eq:1}
\Perimeter{E}=\sup\set*{\int_{E}\div{\varphi}\, dx\given \varphi\in \Cc[N]{1}, \Norm{\varphi}[\infty]\leq 1 }.
\end{align}
Se dice que $E$ es un conjunto de perímetro finito en $\DefaultSetWP$ si $\Perimeter{E}<\infty$.
\end{defi}

\begin{nota}
\begin{enumerate}[(a)]
\item La clase de conjuntos de perímetro finito en $\DefaultSetWP$ incluye todos los conjuntos $E$ con frontera $C^{1}$ en $\DefaultSetWP$ tales que $\Hm{N-1}(\DefaultSetWP\cap \partial E)<\infty$. En efecto, sea $E$ un conjunto con frontera $C^{1}$ en $\DefaultSetWP$, por el teorema de Gauss-Green se tiene que:\index{Gauss-Green!fórmula}
\begin{align}\label{eq:Gauss-Green conjuntos con frontera C1} 
\begin{array}{ll}
\displaystyle
\int_{E}\div\varphi\, dx= -\int_{\DefaultSetWP\cap \partial E} \Crochet{ \nu_{E}}{ \varphi} d\Hm{N-1}, & \forall \varphi\in \Cc[N]{1},
\end{array}
\end{align} 
donde $ \nu_{E}$ es la normal interior unitaria. Usando la fórmula anterior el supremo de \ref{def:definicion de conjunto de perímetro finito:eq:1} se puede calcular fácilmente y resulta ser igual a $\Perimeter{E}=\Hm{N-1}(\DefaultSetWP\cap \partial E)$.
\end{enumerate}
\end{nota}

La teoría de los conjuntos de perímetro finito está estrechamente ligada con la teoría de las funciones $BV$. Por ejemplo, si $\abs{E\cap \DefaultSetWP}<\infty$ entonces $\chi_{E}\in \L{1}$ y concluimos por la proposición \ref{prop:variacion de funciones bv}  que $E$ tiene perímetro finito en $\DefaultSetWP$ si y solo si $\chi_{E}\in \BV$ y en este caso $\Perimeter{E}$ coincide con $\abs{D\chi_{E}}(\DefaultSetWP)$, la variación total en $\DefaultSetWP$ de la derivada distribucional de $\chi_{E}$.  Por lo general solo se puede asegurar que la función característica de un conjunto de perímetro finito en $\DefaultSetWP$ pertenece $\BVloc$. Por otro lado, si $\chi_{E}\in \BVloc$ entonces $E$ tiene perímetro finito en cualquier conjunto abierto $\DefaultSetWP'\subset \subset \DefaultSetWP$. En este caso se dice que $E$ es un conjunto localmente de perímetro finito en $\DefaultSetWP$.

\begin{teo}\label{teo:generalizacion formula Gauss-Green} \index{Gauss-Green!fórmula}
Para cualquier conjunto $E$ de perímetro finito en $\DefaultSetWP$ la derivada distribucional $D\chi_{E}$ es una medida de Radon finita en $\DefaultSetWP$ que toma valores en $\R^{N}$. Además, $\Perimeter{E}=\abs{D\chi_{E}}(\DefaultSetWP)$ y se tiene la siguiente generalización de la fórmula de Gauss-Green:
\begin{align}\label{teo:generalizacion formula Gauss-Green:eq:1}
\begin{array}{ll}
\displaystyle
\int_{E}\div{\varphi}\, dx = - \int_{\DefaultSetWP}\Crochet{\nu_{E}}{\varphi}d\abs{D\chi_{E}},& \forall \varphi\in \Cc[N]{1},
\end{array}
\end{align}
donde $D\chi_{E}=\nu_{E}\abs{D\chi_{E}}$ es la descomposición polar de $D\chi_{E}$.
\end{teo}
\begin{proof}
Como $\chi_{E}\in L_{loc}^{1}(\Omega)$ y el perímetro es una función creciente de $\Omega$, por la proposición \ref{prop:variacion de funciones bv} se obtiene que $\chi_{E}\in BV_{loc}(\Omega)$. Como:
\begin{align*}
\begin{array}{ll}
\displaystyle
\abs{D\chi_{E}}(A)=P(E,A)\leq P(E, \Omega),& \forall A\subset\subset \Omega \, \text{abierto},
\end{array}
\end{align*}
por la nota \ref{nota:extension de medida de Radon a medida de Radon finita} se concluye que $D\chi_{E}$ es una medida de Radon finita en $\Omega$. 
\end{proof}

Como $\Perimeter{E}=\V{\chi_{E}}$ se deducen las siguentes propiedades del perímetro: localidad, semicontinuidad inferior con respecto al conjunto $E$ y aditividad en $\DefaultSetWP$.
\begin{prop}[Propiedades del perímetro]\label{prop:Propiedades del perímetro}\index{perímetro}
\begin{enumerate}[(a)]
\item La aplicación $\DefaultSetWP\mapsto \Perimeter{E}$ es la restricción a los conjuntos abiertos de una medida de Borel en $\R^{N}$.\label{prop:Propiedades del perímetro:a}
\item $E\mapsto \Perimeter{E}$ es semicontinua inferiormente con respecto a la convergencia en medida $\DefaultSetWP$.\label{prop:Propiedades del perímetro:b}
\item $E\mapsto \Perimeter{E}$  es local, i.e. $\Perimeter{E}=\Perimeter{F}$ cuando $\abs{\DefaultSetWP \cap (\setminuss{E}{F})}=0$. \label{prop:Propiedades del perímetro:c}
\item $\Perimeter{E}=\Perimeter{\R\setminus E}$ y se tiene:
\begin{align*}
\Perimeter{E\cup F}+\Perimeter{E\cap F}\leq \Perimeter{E}+\Perimeter{F}.
\end{align*}\label{prop:Propiedades del perímetro:d}
\end{enumerate}
\end{prop}
\begin{proof}
Todas las propiedades salvo (\ref{prop:Propiedades del perímetro:d}) son consecuencia directa de la identificación del perímetro con la variación. Por la nota \ref{nota:aproximación por funciones suaves} y truncado podemos encontrar sucesiones $(u_{h})$, $(v_{h})$ en $C^{\infty}$ que convergen a $\chi_{E}$ y a $\chi_{F}$ en $\Lloc{1}$ y tales que $0\leq u_{h}, v_{h}\leq 1$ y que:
\begin{align*}
\begin{array}{ll}
\displaystyle
\lim_{h\to \infty}\int_{\DefaultSetWP}\abs{\nabla u_{h}}\, dx=\Perimeter{E},& 
\displaystyle \lim_{h\to \infty}\int_{\DefaultSetWP}\abs{\nabla v_{h}}\, dx=\Perimeter{F}. 
\end{array}
\end{align*}
Por la convergencia de $(u_{h}v_{h})$ a $\chi_{E\cap F}$ y la convergencia de $(u_{h}+v_{h}-u_{h}v_{h})$ a $\chi_{E\cup F}$, tomando límite $h\to \infty$, obtenemos la siguiente desigualdad:
\begin{align*}
\int_{\DefaultSetWP} \abs{\nabla(u_{h}v_{h})}\, dx + \int_{\DefaultSetWP} \abs{\nabla(u_{h}v_{h}-u_{h}v_{h})}\,dx \leq \int_{\DefaultSetWP}\abs{\nabla u_{h}}\,dx+\int_{\DefaultSetWP}\abs{\nabla v_{h}}\,dx.
\end{align*}
\end{proof}

El teorema de compacidad \ref{teo:compacidad en BV} aplicado a funciones características nos muestra que la familia de conjuntos con perímetro localmente equiacotados son relativamente compactos con respecto a la convergencia local en medida.

\begin{teo}\label{teo:compacidad relativa de conjuntos de perimetro finito localmente equiacotados}
Cualquier sucesión de conjuntos $(E_{h})$ medibles con respecto a la medida $\Lm{N}$ tal que:
\begin{align*}\DefaultSet{A}
\begin{array}{ll}
\sup\set{\Perimeter{E_{h}}\given h\in \N}<\infty, & \forall A\subset\subset\Omega\; \text{abierto},
\end{array}
\end{align*}
admite una subsucesión $(E_{h(k)})$ localmente convergente en medida en $\DefaultSetWP$. Si $\abs{\DefaultSetWP}<\infty$ la subsucesión converge en medida en $\DefaultSetWP$. 
\end{teo}

Ya estamos en disposición de probar el teorema fundamental de esta sección. La fórmula de la coárea para funciones Lipschitz tiene una versión débil en el espacio $BV$, probada por W.H. Fleming y R. Rishel en \cite{fleming1960integral}, en la cual se remplaza $\int_{\DefaultSetWP}\abs{\nabla{u}}dx$ por $\V{u}$ y las medidas de Hausdorff de los conjuntos de nivel por el perímetro de los conjuntos de supernivel de $u$.

\begin{teo}[Fórmula de la coárea en $BV$]\label{teo:formula de la coárea en BV} \index{coárea en $BV$}
Para cualquier conjunto abierto $\DefaultSetWP\subset \R^{N}$ y $u\in \Lloc{1}$ se tiene:
\begin{align}\label{teo:formula de la coárea en BV:eq:1}
\V{u}=\int^{\infty}_{-\infty}\Perimeter{\set{x\in \DefaultSetWP \given u(x)>t}}dt.
\end{align}
\end{teo}
\begin{proof}
Como la afirmación es invariante bajo modificaciones de $u$ en conjuntos de medida cero respecto de $\Lm{N}$ podemos suponer sin pérdida de generalidad que $u$ es una función sobre conjuntos de Borel. Denotamos por $E_{t}(v)$ los  conjuntos de supernivel $\{ v> t\}$ de una función genérica $v$. 
Probamos primero el resultado para funciones $u\in C^{\infty}(\DefaultSetWP)$. Por la fórmula de la coárea para funciones Lipschitz sabemos que \ref{teo:formula de la coárea en BV:eq:1} es verdad para cualquier función $u\in C^{\infty}(\DefaultSetWP)$. En efecto, denotamos por $E$ el conjunto de puntos donde $\nabla u$ es cero entonces por el teorema de Sard sabemos que $E\cap u^{-1}(t)=\emptyset$ para casi todo $t\in\R$. Para estos valores de $t$ se tiene que la frontera topológica $\partial E_{t}=u^{-1}(t)$ es suave y por lo tanto:
\begin{align*}
\Hm{N-1}(\DefaultSetWP\cap u^{-1}(t))=\Hm{N-1}(\DefaultSetWP\cap \partial E_{t})=\Perimeter{E_{t}}.
\end{align*}
Luego:
\begin{align*}
\int_{\DefaultSetWP}\abs{\nabla u}dx=\int_{-\infty}^{\infty}\Perimeter{E_{t}}dt.
\end{align*}

Sea ahora $u\in \BV$. Probamos primero la desigualdad $\geq$ en \ref{teo:formula de la coárea en BV:eq:1}. Sin pérdida de generalidad suponemos que $\V{u}<\infty$. Por regularidad interior del perímetro podemos suponer que $u\in \L1$. Luego por la proposición \ref{prop:variacion de funciones bv} sabemos que $u\in \BV$ y $\abs{Du}(\DefaultSetWP)=\V{u}$. Considerando una sucesión $(u_{h})\subset C^{\infty}(\DefaultSetWP)$ que converge estrictamente a $u$ en $\BV$. Si $(u_{h(k)})$ es cualquier subsucesión convergente en casi todas partes a $u$ entonces $\chi_{E_{t}(u_{h(k)})}$  converge en casi todas partes a $\chi_{E_{t}(u)}$ para cualquier $t$ tal que $\{u=t\}$ es un conjunto de medida cero respecto de $\Lm{N}$. Por tanto, por la semicontinuidad inferior del perímetro:
\begin{align*}
\int_{-\infty}^{\infty}\Perimeter{E_{t}(u)}dt \leq \int_{-\infty}^{\infty}\liminf_{k\to \infty} \Perimeter{E_{t}\left (u_{h(k)}\right )}dt \\\leq \liminf_{k\to \infty} \int_{-\infty}^{\infty} \Perimeter{E_{t}\left (u_{h(k)}\right )}dt&=\lim_{k\to \infty} \abs{Du_{h(k)}}(\DefaultSetWP)=\abs{Du}(\DefaultSetWP).
\end{align*}
Probamos ahora la desigualdad $\leq$ en \ref{teo:formula de la coárea en BV:eq:1}. Para ello recordamos que $v(x)=\int_{0}^{\infty}\chi_{E_{t}(v)}(x)dt$ para cualquier función de Borel $v:\DefaultSetWP\to \R$ y esto proporciona la identidad:
\begin{align}\label{proof:formula de la coárea en BV:eq:1}
\begin{array}{ll}
\displaystyle
u(x)=\int_{0}^{\infty}\chi_{E_{t}(u)}(x)dt-\int_{-\infty}^{0}(1-\chi_{E_{t}(u)})(x)dt, & \forall x\in \DefaultSetWP.
\end{array}
\end{align}
Luego para cualquier $\varphi\in \Cc[N]{1}$ con $\Norm{\varphi}[\infty]\leq 1$, usamos el hecho que la integral de $\div \varphi$ es cero para obtener:
\begin{align*}
\int_{\DefaultSetWP}u(x)\div\varphi(x)dx &= \int_{\DefaultSetWP}\int_{-\infty}^{\infty}\chi_{E_{t}(u)}(x)\div \varphi(x)dtdx\\
&=\int_{-\infty}^{\infty}\int_{\DefaultSetWP}\chi_{E_{t}(u)}(x)\div \varphi(x)dxdt\leq \int_{-\infty}^{\infty}\Perimeter{E_{t}(u)}dt.
\end{align*}
Como $\varphi$ es arbitrario, la desigualdad $\leq$ se sigue inmediatamente de la definición de variación.
\end{proof}

\begin{cor}
Para cualquier abierto $\DefaultSetWP\subset\R^{N}$, si  $u\in \BV$ entonces el conjunto $\set{u>t}$ tiene perímetro finito en $\DefaultSetWP$ en casi todas partes respecto de la medida $\Lm{1}$ y:
\begin{align}\label{cor:formula de la coárea en BV:eq:1}
\begin{array}{ll}
\displaystyle
\abs{Du}(B)=\int^{\infty}_{-\infty}\abs{D\chi_{\set{u>t}}}(B) dt,&
\displaystyle Du(B)=\int^{\infty}_{-\infty}D\chi_{\set{u>t}}(B)dt,
\end{array}
\end{align}
para cualquier conjunto de Borel $B\subset \DefaultSetWP$.
\end{cor}
\begin{proof}
Sea $u\in \BV$. Si definimos $\mu(B)=\int_{-\infty}^{\infty}\abs{D\chi_{E_{t}(u)}}(B)dt$, para cualquier conjunto de Borel $B\subset \DefaultSetWP$. Es directo ver que $\mu$ es una medida de Borel positiva. Como $\abs{Du}$ y $\mu$ coinciden en subconjuntos abiertos de $\DefaultSetWP$, concluimos que coinciden.

Para probar la segunda identidad de \ref{cor:formula de la coárea en BV:eq:1} usamos la identidad \ref{proof:formula de la coárea en BV:eq:1} junto con  Fubini:
\begin{align*}
\int_{\DefaultSetWP}\phi dDu &=-\int_{\DefaultSetWP}u(x)\nabla \phi(x) dx
\\&=-\int_{\DefaultSetWP}\left (\int_{0}^{\infty}\chi_{E_{t}(u)}(x)dt \right ) \nabla \phi(x)dx + \int_{\DefaultSetWP}\left (\int_{-\infty}^{0} (1-\chi_{E_{t}(u)}(x)) dt\right)\nabla \phi(x) dx
\\&=-\int_{0}^{\infty}\left (\int_{\DefaultSetWP} \chi_{E_{t}(u)}(x)\nabla \phi(x)dx \right ) dt - \int_{-\infty}^{0}\left (\int_{\DefaultSetWP} \chi_{E_{t}(u)}(x) \nabla \phi(x) dx\right) dt
\\&= \int_{-\infty}^{\infty}\left ( \int_{\DefaultSetWP} \phi dD\chi_{E_{t}(u)}\right )dt,
\end{align*}
para cualquier función $\phi\in \Cc{\infty}$. Como ambos lados son medidas en $\DefaultSetWP$ no solo coinciden como distribuciones pero también como medidas. 
\end{proof}

\section{Teoremas de inmersion y desigualdades isoperimétricas} \label{sec:teoremas de inmersion y desigualdades isoperimétricas}
En esta sección estudiaremos como las oscilaciones en un sentido $L^{p}$ pueden ser controladas gracias la derivada distribucional. Es decir, probaremos desigualdades de tipo Poincaré. Para las funciones características de los conjuntos de perímetro finito $E$, las desigualdades tipo Poincaré nos dicen que en las bolas $B_{\varrho}(x)$ en las que $P(E, B_{\varrho}(x))\ll \varrho^{N-1}$ se tiene en el límite la siguiente dicotomía: O bien la intersección $E\cap B_{\varrho}(x)$ es $\emptyset$ o bien es todo $B_{\varrho}(x)$. Por un argumento de recubrimiento concluiremos propiedades globales, es decir, mayor integrabilidad de las funciones $BV$ hasta la potencia $N/(N-1)$, y desigualdades isoperimétricas para conjuntos de perímetro finito.

Introducimos la siguiente notación compacta para el valor medio $u_{\DefaultSetWP}$ de una función $u\in \L1$:
\begin{align}\label{eq:valor medio de una funcion}
u_{\DefaultSetWP}=\fint_{\DefaultSetWP}u(x)dx=\frac{1}{\abs{\DefaultSetWP}}\int_{\DefaultSetWP}u(x)dx.
\end{align}

Entonces se tiene el siguiente lema técnico:
\begin{lema}\label{lema:cota integral de la diferencia de la funcion menos el valor medio}
Sea $\DefaultSetWP\subset \R^{N}$ una extensión de dominio acotada. Entonces:
\begin{align}\label{lema:cota integral de la diferencia de la funcion menos el valor medio:eq:1}
\int_{\DefaultSetWP}\abs{u-u_{\DefaultSetWP}}dx\leq C\abs{Du}(\DefaultSetWP)\quad \forall u \in \BV
\end{align}
para algún constante real $C$ que depende únicamente de $\DefaultSetWP$.
\end{lema}
\begin{proof}
La demostración es análoga a la prueba de la desigualdad de Poincaré dada en \cite[Teorema 1, Cap. 5.8]{evans1998partial}, salvo que en este caso particular $p$ es $1$ y en vez de aplicar el teorema de Rellich-Kondrachov se usa su análogo en $BV$ (teorema de compacidad \ref{teo:compacidad en BV}).
\end{proof}

\begin{nota}
Si aplicamos el teorema anterior a bolas $B_{\varrho}(x)$ entonces en este caso la mejor constante $C$ no dependerá de $x$. Siguiendo un simple argumento de reescalado $(\text{usando la transformación } \hat{u}(y)=u(x+\varrho y)$ \text{ que envía } $BV(B_{\varrho}(x))$ \text{ en } $BV(B_{1}))$ vemos que $C(B_{\varrho}(x))=\gamma_{1}\varrho$, donde $\gamma_{1}$ es la constante relativa a la bola unidad. Esto prueba que:
\begin{align}\label{eq:1:lema cota integral de la diferencia de la funcion menos el valor medio aplicado a bolas}
\begin{array}{ll}
\displaystyle
\int_{B_{\varrho}(x)}\abs{u-u_{B_{\varrho}(x)}}dy\leq \gamma_{1}\varrho\abs{Du}(B_{\varrho}(x)),& \forall u \in BV(B_{\varrho}(x)),
\end{array}
\end{align}
para cualquier $x\in \R^{N}$ y cualquier $\varrho>0$. Si aplicamos la ecuación \ref{eq:1:lema cota integral de la diferencia de la funcion menos el valor medio aplicado a bolas} a funciones características $\chi_{E}$ de conjuntos de perímetro finito obtenemos:
\begin{align*}
2(\chi_{E})_{B_{\varrho}(x)}(1-(\chi_{E})_{B_{\varrho}(x)})\leq \gamma_{1} \dfrac{P(E,B_{\varrho}(x))}{\omega_{N}\varrho^{N-1}}.
\end{align*}
Como $(\chi_{E})_{B_{\varrho}(x)}\in [0,1]$ y $\min{\set{t,1-t}}\leq 2t(1-t)$ para cualquier $t\in [0,1]$ se tiene:
\begin{align}\label{eq:1:desigualdad isoperimetrica local}
\min\set*{(\chi_{E})_{B_{\varrho}(x)}, 1-(\chi_{E})_{B_{\varrho}(x)}}\leq \gamma_{1}\dfrac{P(E, B_{\varrho}(x))}{\omega_{N}\varrho^{N-1}}.
\end{align}
\end{nota}

Como se ve la ecuación \ref{eq:1:desigualdad isoperimetrica local} es un resultado local. A continuación deducimos el resultado global conocido como desigualdad isoperimétrica. En el teorema que probamos no estamos interesados en encontrar la constante óptima sin embargo esta constante se puede calcular como puede verse en \cite{talenti1976best},\cite{de1958sulla}.
\begin{teo}[Desigualdad isoperimétrica]\label{teo:desigualdad isoperimetrica} \index{desigualdad!isoperimétrica}
Sea $N>1$ un entero. Para cualquier conjunto $E$ de perímetro finito en $\R^{N}$ o bien $E$ o bien $\R^{N}\setminus E$ tienen medida de Lebesgue finita y se tiene: 
\begin{align*}
\min\set*{\abs{E}, \abs{\R^{N}\setminus E}}\leq \gamma_{2}\br*{P(E,\R^{N})}^{N/(N-1)},
\end{align*}
para alguna constante $\gamma_{2}$ que depende únicamente de la dimensión del espacio ambiente.
\end{teo}
\begin{proof}
Por la ecuación \ref{eq:1:desigualdad isoperimetrica local}, existe una constante $c$ tal que:
\begin{align}\label{proof:desigualdad isoperimetricas:eq:1}
\min\set*{\alpha_{\varrho}(x), 1-\alpha_{\varrho}(x)}\leq c\dfrac{P(E, Q_{\varrho}(x))}{\varrho^{N-1}},
\end{align}
para cualquier cubo abierto $Q_{\varrho}(x)\subset \R^{N}$ centrado en $x$ y de lado $2\varrho$, donde $\alpha_{\varrho}(x)$ es el valor medio de $\chi_{E}$ en $Q_{\varrho}(x)$, i.e. $\abs{Q_{\varrho}(x)\cap E}/(2\varrho)^{N}$. Tomando $\varrho=\br*{3cP(E, \R^{N})}^{1/(N-1)}$  se obtiene que $\alpha_{\alpha}(x)\in[0, 1/2)\cup (1/2, 1]$. Entonces por continuidad se tiene que o bien $\alpha_{\varrho}(x)\in [0, 1/2)$ para cualquier $x\in \R^{N}$, o bien $\alpha_{\varrho}(x)\in (1/2, 1]$ para cualquier $x\in \R^{N}$. Si se tiene que $\alpha_{\varrho}(x)\in [0, 1/2)$ entonces de la ecuación \ref{proof:desigualdad isoperimetricas:eq:1} deducimos que:
\begin{align*}
\begin{array}{ll}
\displaystyle \dfrac{\abs*{E\cap Q_{\varrho}(x)}}{(2\varrho)^{N}}=\alpha_{\varrho}(x)\leq c\dfrac{P(E, Q_{\varrho}(x))}{\varrho^{N-1}}, & \forall x\in \R^{N}.
\end{array}
\end{align*}
Cubriendo casi todas partes (respecto de la medida $\Lm{N}$) de $\R^{N}$ mediante una familia disjunta de cubos $\{Q_{\varrho}(x_{h})\}_{h\in\Z^{N}}$ se obtiene:
\begin{align*}
\abs*{E}=\sum_{h\in\Z^{N}}\abs*{Q_{\varrho}(x_{h})\cap E} \leq 2^{N}c\varrho\sum_{h\in\Z^{N}}P(E, Q_{\varrho}(x_{h}))\leq 2^{N}c\varrho P(E, \R^{N}),
\end{align*}
y el teorema se sigue de nuestra elección de $\varrho$ con $\gamma_{2}=2^{N}3^{1/(N-1)}c^{N/(N-1)}$. Si $\alpha_{\varrho}(x)\in (1/2, 1]$ para cualquier $x\in \R^{N}$ un argumento similar muestra que $\abs*{\R^{N}\setminus E}$ se puede acotar como antes.
\end{proof}
El teorema que acabamos de probar es falso en dimensión uno, pues por ejemplo la semirecta $E=(0,\infty)$ tiene perímetro finito en $\R$ pero tanto $E$ como $\R\setminus E$ tienen medida infinita. Usando la desigualdad isoperimétrica junto con la fórmula de la coárea se puede probar que se tiene una inmersión de $\BV$ en $\L{N/(N-1)}$. Sin embargo, antes de probar el teorema necesitamos el siguiente lema técnico.
\begin{lema}\label{lema:inyeccion BV en L1*}
Para cualquier $p\in[1,\infty)$ y cualquier función decreciente $g:(0,\infty)\to [0,\infty)$ se tiene:
\begin{align*}
\begin{array}{ll}
\displaystyle
p\int^{T}_{0}g(s)s^{p-1}ds \leq \left(\int_{0}^{T}g^{1/p}(s)ds \right)^{p}, & \forall T>0.
\end{array}
\end{align*}
\end{lema}
\begin{proof}
Este lema es un resultado clásico del análisis matemático. Su demostración se puede encontrara en \cite[Lema 3.48]{ambrosio2000functions}.
\end{proof}

A continuación probamos el teorema de inmersión de $\BV$ en $\L{N/(N-1)}$.
\DefaultSet{\R^{N}}
\begin{teo}\label{teo:inyeccion BV en L1*}
Para cualquier función $u\in \Lloc{1}$ que satisface $V(u, \R^{N})<\infty$ existe un $m\in \R$ tal que:
\begin{align*}
\Norm{u-m}[\L{1^{*}}]\leq \gamma_{3}V(u,\R^{N}).
\end{align*} 
Si $u\in \L{1}$ entonces $m=0$, $u\in \BV$ y por lo tanto $\Norm{u}[\L{1^{*}}]\leq \gamma_{3}\abs{Du}(\R^{N})$. En particular, la inmersión $ \inyec{\BV}{\L{1^{*}}}$ es continua.
\end{teo}
\begin{proof}
Si $N=1$, como se ha visto en la prueba del teorema \ref{teo:variacion esencial equivalente a variacion}, $u\in BV_{loc}(\R)$, $Du$ es una medida de Radon real en $\R$ de modo que existe una única constante real $c$ tal que se verifica que $c+Du((-\infty,t))$ está en la clase de equivalencia de $u$. Por lo tanto, eligiendo $m=c$ se obtiene:
\begin{align*}
\Norm{u-m}[\infty]=\sup_{t\in \R}\abs{Du((-\infty,t))}\leq \abs{Du}(\R).
\end{align*}

Supongamos ahora que $N>1$. Sea $E_{t}=\set*{x\in \R^{N}\given u(x)>t}$ y sea $T$ el conjunto de todos los $t\in \R$ tales que $P(E_{t}, \R^{N})<\infty$. Por la fórmula de la coárea en $BV$ este conjunto es denso en $\R$. Por la desigualdad isoperimétrica \ref{teo:desigualdad isoperimetrica} o bien $E_{t}$ o bien su complementario tiene medida finita. Definimos $m=\inf\set*{t\in T \given \abs*{E_{t}}<\infty}$. Si $t>m$ podemos encontrar un $\tau\in T\cap (m,t)$ de modo que $\abs*{E_{\tau}}<\infty$, por lo tanto, $\abs*{E_{t}}<\infty$. Análogamente, se prueba que $\abs*{\R^{N}\setminus E_{t}}<\infty$ para cualquier $t<m$. Ahora vemos que $m\in \R$. Supongamos lo contrario, es decir que $m=-\infty$. Entonces, $E_{t}$ tiene medida finita para cualquier $t\in \R$ y podemos aplicar la fórmula de la coárea y obtenemos:
\begin{align*}
\begin{array}{ll}
\displaystyle
\int_{-h-1}^{-h}P(E_{t}, \R^{N})dt\leq V(u, \R^{N}), & \forall h\in \N.
\end{array}
\end{align*}
Entonces podemos encontrar un $t_{h}\in (-h, -h-1)$ de modo que $P(E_{t_{h}}, \R^{N})$ está uniformemente acotado en $h$. Por el teorema \ref{teo:desigualdad isoperimetrica} esto significa que las medidas de Lebesgue de $E_{t_{h}}$ están uniformemente acotadas lo contradice el hecho que la unión de creciente de estos conjuntos sea $\R^{N}$. De forma análoga se puede probar que $m<\infty$.

Como $m\in \R$ y la cota que queremos probar es invariante bajo traslaciones en $u$ podemos suponer sin perdida de generalidad que $m=0$. Sea $v=u\vee 0$ entonces por el lema \ref{lema:inyeccion BV en L1*} con $p=N/(N-1)$ vemos que:
\begin{align}\label{proof:inyeccion BV en L1*:eq:1}
\int_{\R^{N}}v^{N/(N-1)}dx=\int_{0}^{\infty}\abs*{\set*{v^{N/(N-1)}>t}}dt&=\frac{N}{N-1}\int_{0}^{\infty}\abs{E_{s}}s^{1/(N-1)}ds \nonumber
\\&\leq \left ( \int_{0}^{\infty} \abs{E_{s}}^{(N-1)/N}ds\right )^{N/(N-1)}.
\end{align}
Como $E_{s}$ tiene medida de Lebesgue finita para cualquier $s\in (0,\infty)$ por la desigualdad isoperimétrica (teorema \ref{teo:desigualdad isoperimetrica}) deducimos:
\begin{align*}
\displaystyle
\begin{array}{ll}
\abs{E_{s}}^{(N-1)/N}\leq c\, P(\set*{u>s}, \R^{N}), & \forall s>0,
\end{array}
\end{align*}
para alguna constante $c$ que no depende la la dimensión del espacio ambiente. Por tanto podemos aplicar la fórmula de la coárea para obtener así:
\begin{align*}
\left ( \int_{\R^{N}}v^{N/(N-1)} \right )\leq c\int^{\infty}_{0} P(\set*{u>s}, \R^{N})ds\leq c\,\abs*{Du}(\R^{N}).
\end{align*} 
Análogamente, se puede estimar la parte negativa de $u$ teniendo en cuenta que $\abs*{\R^{N}\setminus E_{t}}<\infty$ para cualquier $t\in (-\infty, 0)$. Si $u\in L^{1}(\R^{N})$ entonces por la desigualdad de Chebyshev vemos que $m=0$.  
\end{proof}

\begin{cor}[Teorema de inmersión]\index{teorema!de inmersión}
Sea $\DefaultSetWP\subset \R^{N}$ una extensión de dominio acotada. Entonces, la inyección $\inyec{\BV}{\L{1^{*}}}$ es continua. Si $1\leq p<1^{*}$ entonces la inyección $\inyec{\BV}{\L{p}}$ es compacta.
\end{cor}
\begin{proof}\DefaultSet{\Omega}
La inyección $\inyec{\BV}{\L{1^{*}}}$ es consecuencia directa del teorema \ref{teo:inyeccion BV en L1*}. Conjuntos acotas en $\BV$ son acotados en $\L{1^{*}}$ y, por el teorema de compacidad \ref{teo:compacidad en BV} la inyección $\inyec{\BV}{\L{1}}$ es relativamente compacta en $\L{1}$. Por lo tanto, por la desigualdad de Hölder se ve que la inyección es compacta de $\BV$ en $\L{p}$ para cualquier $p\in [1,1^{*})$.
\end{proof}


\begin{nota}[Desigualdad de Poincaré]\DefaultSet{\Omega}\index{desigualdad!de Poincaré}
Si $\Omega$ es una extensión de dominio acotada, la continuidad de la inmersión de $\BV$ en $\L{1^{*}}$ y el teorema \ref{lema:cota integral de la diferencia de la funcion menos el valor medio} implican:
\begin{align}
\begin{array}{lll}
\displaystyle
\Norm{u-u_{\Omega}}[\L{p}]\leq C\abs{Du}(\Omega), & \forall u\in \BV, & 1\leq p\leq 1^{*},
\end{array}
\end{align}
para alguna constante $C$ que depende únicamente de $\Omega$. Para las bolas, rescalando al igual que hicimos para la obtención de la desigualdad isoperimétrica local se obtiene que:
\begin{align*}
\begin{array}{lll}
\displaystyle
\Norm{u-u_{B_{\varrho}(x)}}[L^{p}(B_{\varrho}(x))]\leq \gamma_{4}\varrho^{N/p}\dfrac{\abs{Du}(B_{\varrho}(x))}{\varrho^{N-1}},&\forall u \in BV(B_{\varrho}(x)), & 1\leq p \leq 1^{*}.
\end{array}  
\end{align*}
Tomando $p=1^{*}$ y siguiendo la misma argumentación que en la obtención de las ecuaciones \ref{eq:1:lema cota integral de la diferencia de la funcion menos el valor medio aplicado a bolas} y \ref{eq:1:desigualdad isoperimetrica local} obtenemos de la ecuación anterior, la desigualdad isoperimétrica relativa en bolas:
\begin{align}\label{eq:1:desigualdad isoperimetrica relativa}
\min\set*{\abs{B_{\varrho}(x)\cap E}^{(N-1)/N},\abs{B_{\varrho}(x)\setminus E}^{(N-1)/N}}\leq \gamma_{5} P(E,B_{\varrho}(x)).
\end{align} 
\end{nota}

Finalmente para terminar esta sección enunciamos el siguiente teorema.
\begin{teo}\label{teo:desigualdad tipo poincare Lp con mediana}
Para cualquier $u\in BV(B_{\varrho}(x))$ y cualquier mediana $m$ de $u$ en $B_{\varrho}(x)$ se tiene:
\begin{align}\label{teo:desigualdad tipo poincare Lp con mediana:eq:1}
\begin{array}{ll}
\displaystyle
\Norm{u-m}[L^{p}(B_{\varrho}(x))]\leq \gamma_{5}\varrho^{N/p}\dfrac{\abs{Du}(B_{\varrho}(x))}{\varrho^{N-1}}, & \forall p\in [1,1^{*}].
\end{array}
\end{align}
\end{teo}
\begin{proof}
Por invarianza bajo rescalado y traslación se ve que no es restrictivo suponer que $x=0, \varrho=1$ y $m=0$. Por la desigualdad de Hölder solo hay que probar \ref{teo:desigualdad tipo poincare Lp con mediana:eq:1} para $p=1^{*}$. Aplicando la desigualdad \ref{proof:inyeccion BV en L1*:eq:1} a la función $v=u^{+}\chi_{B_{\varrho}(x)}$ y teniendo en cuenta la desigualdad isoperimétrica \ref{teo:desigualdad tipo poincare Lp con mediana:eq:1} se concluye:
\begin{align*}
\left( \int_{B_{\varrho}(x)}(u^{+})^{1^{*}}\right)^{1/1^{*}}\leq \int_{0}^{\infty}\abs{\set{u>t}\cap B_{\varrho}(x)}^{1/1^{*}}dt\leq \gamma_{5}\int_{0}^{\infty}P(\set{u>t},B_{\varrho}(x))dt.
\end{align*}
Por un argumento similar también se puede acotar la integral de $u^{-}$. Sumando las dos desigualdades se obtiene por la fórmula de la coárea \ref{teo:desigualdad tipo poincare Lp con mediana:eq:1}.
\end{proof}

\section{Estructura de los conjuntos de perímetro finito}\label{sec:estructura de los conjuntos de perímetro finito}

\DefaultSet{\Omega}
En esta sección haremos un análisis detallado de las propiedades de los conjuntos de perímetro finito en $\DefaultSetWP$. Probaremos la existencia de un conjunto $\mathcal{F}E\subset \DefaultSetWP$ que es $\Hm{N-1}$-rectificable llamado frontera reducida tal que $\abs{D\chi_{E}}$ coincide con la medida $\Hm{N-1}\Radot \mathcal{F}E$. Esto permitirá obtener una nueva versión de la fórmula de Gauss-Green para conjuntos de perímetro finito en la que la frontera reducida está involucrada. Finalmente estudiaremos la densidad de estos conjuntos probando que converge a $0,1$ o $1/2$ para casi todo punto $x\in \DefaultSetWP$ con respecto a la medida $\Hm{N-1}$.

Vemos primero que la estructura de los conjuntos de perímetro finito en una dimensión es relativamente fácil de caracterizar.
\begin{prop}\label{prop:estructura conjuntos perimetro finito dim 1}\DefaultSet{{(a,b)}} \index{conjunto!de perímetro finito}
Si $E$ tiene perímetro finito en $(a,b)$ y $|E\cap (a,b)|>0$ entonces existe un entero $p\geq 1$ y $p$ intervalos disjuntos dos a dos $J_{i}=[a_{2i-1}, a_{2i}]\subset \R$ tales que $E\cap (a,b)$ es la unión de los $J_{i}$ y:
\begin{align}
\Perimeter{E}=\#(\set{i\in \set{1,\ldots,2p} \given a_{i}\in \DefaultSetWP}).
\end{align}
\end{prop}
\begin{proof}
Como el resultado es de naturaleza local, sin pérdida de generalidad podemos suponer que $(a,b)$ es un intervalo acotado. Sea $u=\chi_{E}\in BV((a,b))$, $u^{l}:(a,b)\to \R$, $A\subset (a,b)$ el conjunto de átomos de $Du$ y recordemos que $u^{l}$ tiene una discontinuidad de salto igual a $Du(\set{t})$ en cualquier punto de $A$. Como $u^{l}(x)\in \set{0,1}$ para casi todo $x\in (a,b)$, cualquier salto de $u^{l}$ es o bien $1$ o bien $-1$. En particular:
\begin{align*}
\#(A)=\abs{Du}(A)\leq \abs{Du}((a,b))=P(E,(a,b))<\infty.
\end{align*}
La función $u^{l}$ es continua en $(a,b)\setminus A$ e igual a $\chi_{E}$ en casi todo punto de $(a,b)$, por lo tanto es constante en cualquier componente conexa de $(a,b)\setminus A$. Luego, los conjuntos $J_{i}$ pueden tomarse como el cierre de los intervalos para los cuales $u^{l}$ es igual a $1$. Por construcción, $\#(A)=P(E,(a,b))$ es igual al número de puntos terminales $a_{i}$ en $(a,b)$.
\end{proof}

La estructura de los conjuntos de perímetro finito de dimensión $N>1$ es más compleja. Para poder estudiar la estructura de los conjuntos de perímetro finito de dimensión $N>1$ De Giorgi introduce el concepto de frontera reducida.
\begin{defi}[Frontera reducida]\label{defi:frontera reducida} \index{frontera!reducida}
Sea $E$ un subconjunto de $\R^{N}$ $\Lm{N}$-medible y $\DefaultSetWP$ el conjunto más grande tal que $E$ es localmente de perímetro finito en $\DefaultSetWP$. Se llama frontera reducida $\mathcal{F}E$ a la colección de todos los puntos $x\in \supp{\abs{D\chi_{E}}}\cap \DefaultSetWP$ tal que el límite: 
\begin{align*}
\nu_{E}(x)=\lim_{\varrho\to 0}\dfrac{D\chi_{E}(B_{\varrho}(x))}{\abs{D\chi_{E}}(B_{\varrho}(x))},
\end{align*}
existe en $\R^{N}$ y satisface $\abs{\nu_{E}(x)}=1$. La función $\nu_{E}:\mathcal{F}E\to \mathbb{S}^{N-1}$ se llama normal interior generalizada de $E$.\index{normal!interior}
\end{defi}
Es fácil comprobar que $\mathcal{F}E$ es un conjunto de Borel y que $\nu_{E}:\mathcal{F}E\to \mathbb{S}^{N-1}$ es una aplicación sobre conjuntos de Borel. Por el teorema de derivación de Besicovitch la medida $\abs{D\chi_{E}}$ está concentrada en $\mathcal{F}E$ y $D\chi_{E}=\nu_{E}\abs{D\chi_{E}}$. 
\begin{nota}\label{nota:frontera reducida punto de Lebesgue}\index{punto!de Lebesgue}
Se observa que cualquier punto $x\in \mathcal{F}E$ es un punto de Lebesgue de $\nu_{E}$ relativo a $\abs{D\chi_{E}}$ ya que: 
\begin{align*}
\dfrac{1}{2\abs{D\chi_{E}}(B_{\varrho}(x)}\int_{B_{\varrho}(x)}\abs{\nu_{E}(y)-\nu_{E}(x)}^{2}d\abs{D\chi_{E}}(y)=1-\Crochet{\nu_{E}(x)}{\dfrac{D\chi_{E}(B_{\varrho}(x))}{\abs{D\chi_{E}}(B_{\varrho}(x))}}.
\end{align*}
\end{nota}

Ahora veremos que $\mathcal{F}E$ es un conjunto numerablemente $(N-1)$-rectificable y que $\abs{D\chi_{E}}$ coincide con $\Hm{N-1}\Radot \mathcal{F}E$. Para ello primero necesitaremos los dos resultados que enunciamos y probamos a continuación. 
\DefaultSet{\Omega}
\begin{prop}[Localización]\label{prop:localizacion}\index{localidad}
Sea $E$ un conjunto de perímetro finito en $\DefaultSetWP$, $x_{0}\in \DefaultSetWP$ y $\delta=\dist{x_0}{\partial \DefaultSetWP}$. Entonces:
\begin{align}\label{prop:localizacion:eq:1}
\begin{array}{ll}
\displaystyle
P(E\cap B_{\varrho}(x_{0}), \R^{N})\leq P(E, \overline{B}_{\varrho}(x_{0}))+m_{+}'(x_{0}), & \forall \varrho\in (0,\delta),
\end{array}
\end{align}
donde $m(\varrho)=\abs{E\cap B_{\varrho}(x_{0})}$ y $m_{+}'$ es la derivada inferior derecha de $m$.
\end{prop}
\begin{proof}
No es restrictivo suponer que $x_{0}=0$ y probamos entonces la siguiente desigualdad más general:
\begin{align}\label{proof:localizacion:eq:1}
\begin{array}{ll}
\displaystyle
\abs*{Du_{\varrho}}(\R^{N})\leq \abs*{Du}(\overline{B}_{\varrho})+\left ( \int_{B_{\varrho}}\abs*{u(x)}\right )_{+}', &\forall\varrho\in (0, \delta),
\end{array}
\end{align}
para cualquier $u\in \BV$, donde $u_{\varrho}=u\chi_{B_{\varrho}}$. Dado cualquier $\sigma\in (0, \delta-\varrho)$ construimos un $u^{\sigma}\in BV(\R^{N})$ con soporte en $\overline{B}_{\varrho+\sigma}$ que coincide con $u$ en $B_{\varrho}$ y satisface:
\begin{align}\label{proof:localizacion:eq:2}
\abs*{Du^{\sigma}}(\R^{N})\leq \abs*{Du}(B_{\varrho+\sigma})+\sigma^{-1}\int_{B_{\varrho+\sigma}\setminus B_{\varrho}}\abs*{u(x)}dx.
\end{align} 
Para ello, definimos $u^{\sigma}(x)=u(x)\gamma_{\sigma}(\abs{x})$, donde:
\begin{align*}
\gamma_{\sigma}(x)=\left \{ \begin{array}{ll}
1, & \text{si}\; t\leq \varrho,\\
1+\frac{\varrho-t}{\sigma},& \text{si}\; \varrho\leq t\leq \varrho+\sigma,\\
0,& \text{si}\; t\geq \varrho+\sigma.
\end{array} \right .
\end{align*}
Por la proposición \ref{prop:propiedades de Du}(\ref{prop:propiedades de Du:b}) se obtiene $Du^{\sigma}=\gamma_{\sigma}(\abs{x})Du+u(x)\gamma_{\sigma}'(x)x/\abs{x}\Lm{N}$ de lo que inmediatamente se deduce \ref{proof:localizacion:eq:2}. Como $(u^{\sigma})$ converge a $u_{\varrho}$ en $L^{1}(\R^{N})$ cuando $\sigma\to 0$, \ref{proof:localizacion:eq:1} se sigue de \ref{proof:localizacion:eq:2} y de la semicontinuidad inferior de la variación.
\end{proof}

\begin{nota}
Si $P(E, \partial B_{\varrho}(x_{0}))=0$ sustrayendo $P(E, B_{\varrho}(x_{0}))$ a ambos lados de \ref{prop:localizacion:eq:1} y considerando la localidad del perímetro obtenemos:
\begin{align}\label{nota:localizacion:eq:2}
\begin{array}{ll}
P\left(E\cap B_{\varrho}(x_{0}), \partial B_{\varrho}(x_{0})\right)\leq m_{+}'(\varrho), & \forall \varrho \in (0,\delta).
\end{array}
\end{align}
\end{nota}

En lo que sigue necesitaremos algunas acotaciones sobre $\abs{E\cap B_{\varrho}(x_{0})}$ y sobre $P(E, B_{\varrho}(x_{0}))$ para puntos $x_{0}\in \mathcal{F}E$ y con $\varrho>0$ suficientemente pequeño.
\begin{lema}[Cota sobre el perímetro y volumen]\label{lema:cota sobre el perímetro y volumen} \index{cota!perímetro y volumen}
Sea $E$ un conjunto de perímetro finito en $\DefaultSetWP$ y sea $x_{0}\in \mathcal{F}E\cap\DefaultSetWP$. Entonces, existe un $\varrho_{0}\in (0,\dist{x_{0}}{\partial \DefaultSetWP})$ y unas constantes $\alpha, \beta>0$ que dependen únicamente de la dimensión del espacio ambiente tales que: 
\begin{align}
&P(E, B_{\varrho}(x_{0}))\leq \alpha \varrho^{N-1},\quad \forall \varrho \in (0, \varrho_{0}),\label{lema:cota sobre el perímetro y volumen:eq:cota perimetro}\\
&\min{\set{\abs{E\cap B_{\varrho}(x_{0})}, \abs{B_{\varrho}(x_{0})\setminus E}}}\geq \beta \varrho^{N},\quad \forall\varrho\in (0,\varrho_{0}).\label{lema:cota sobre el perímetro y volumen:eq:cota volumen}
\end{align}
\end{lema}
\begin{proof}
Por la proposición \ref{prop:estructura conjuntos perimetro finito dim 1} no es restrictivo suponer que $N>1$. Eligiendo $\varrho_{0}\in (0,\dist{x_{0}}{\partial \DefaultSetWP}/2)$ tal que:
\begin{align*}
\begin{array}{ll}
\displaystyle
P(E, B_{\varrho}(x_{0}))=\abs*{D\chi_{E}}(B_{\varrho}(x_{0}))\leq 2 \abs*{D\chi_{E}(B_{\varrho}(x_{0}))}, & \forall \varrho\in (0, 2\varrho_{0}),
\end{array}
\end{align*} 
Sea $E_{\varrho}=E\cap B_{\varrho}(x_{0})$ para cualquier $\varrho\in (0,2\varrho_{0})$ tal que $P(E, \partial B_{\varrho}(x_{0}))=0$, por la localidad del perímetro, la igualdad $D\chi_{E_{\varrho}}(\R^{N})=0$ y por la ecuación \ref{nota:localizacion:eq:2} deducimos:
\begin{align*}
\abs*{D\chi_{E}(B_{\varrho}(x_{0}))}=\abs*{D\chi_{E_{\varrho}}(B_{\varrho}(x_{0}))}=\abs*{D\chi_{E_{\varrho}}(\partial B_{\varrho}(x_{0}))}\leq P\left(E_{\varrho}, \partial B_{\varrho}(x_{0})\right)\leq m_{+}'(\varrho).
\end{align*}
Tomando en cuenta la elección de $\varrho_{0}$ obtenemos:
\begin{align}\label{proof:cota sobre el perímetro y volumen:eq:1}
\displaystyle 
\begin{array}{ll}
P(E, B_{\varrho}(x_{0}))\leq 2m'(\varrho), & \text{para casi todo}\; \varrho\in (0,2\varrho_{0})\; \text{respecto de}\; \Lm{1}.
\end{array}
\end{align}
Dado cualquier $\varrho\in (0,\varrho_{0})$, integrando entre $\varrho$ y $2\varrho$ se obtiene:
\begin{align*}
P(E,B_{\varrho}(x_{0}))\leq \varrho^{-1}\int_{\varrho}^{2\varrho}P(E, B_{t}(x_{0}))dt\leq \frac{2m(2\varrho)}{\varrho}\leq 2^{N+1}\omega_{N}\varrho^{N-1},
\end{align*}
y esto prueba la ecuación \ref{lema:cota sobre el perímetro y volumen:eq:cota perimetro}. Para probar la ecuación \ref{lema:cota sobre el perímetro y volumen:eq:cota volumen} vemos que por \ref{prop:localizacion:eq:1} y por \ref{proof:cota sobre el perímetro y volumen:eq:1}, $P(E_{\varrho}, \R^{N})$ se puede acotar por $3m'(\varrho)$ para casi todo $\varrho\in (0,\varrho_{0})$. Por la desigualdad isoperimétrica \ref{teo:desigualdad isoperimetrica} se obtiene entonces que:
\begin{align}
(m^{1/N})'(\varrho)=\frac{1}{N}m^{(1-N)/N}(\varrho)m'(\varrho)\geq\frac{1}{3N}m^{(1-N)/N}(\varrho)P(E_{\varrho}, \R^{N})\geq \gamma,
\end{align}
para casi todo $\varrho\in (0, \varrho_{0})$, con $\gamma=\gamma_{2}^{(1-N)/N}/(3N)$. Integrando respecto de $\gamma$ obtenemos $m(\varrho)\geq \gamma^{N}\varrho^{N}$ para cualquier $\varrho\in (0,\varrho_{0})$. El mismo argumento remplazando $E$ por $\R^{N}\setminus E$ proporciona finalmente \ref{lema:cota sobre el perímetro y volumen:eq:cota volumen}.
\end{proof}

Ya estamos en condiciones de probar el siguiente teorema estructural debido a De Giorgi.
\begin{teo}[Teorema estructural de De Giorgi]\label{teo:teorema estructural de De Giorgi}\index{teorema!estructural de De Giorgi}
Sea $E$ un subconjunto de $\R^{N}$ $\Lm{N}$-medible. Entonces $\mathcal{F}E$ es un conjunto numerablemente $(N-1)$-rectificable y $\abs{D\chi_{E}}=\Hm{N-1}\Radot \mathcal{F}E$. Además, para cualquier $x_{0}\in \mathcal{F}E$ se siguen las siguientes propiedades:
\begin{enumerate}[(a)]
\item Los conjuntos $(E-x_{0})/\varrho$ convergen localmente en medida en $\R^{N}$ cuando $\varrho\to 0$ al semiespacio $H$ ortogonal a $\nu_{E}$ que contiene a $\nu_{E}(x_{0})$.\label{teo:teorema estructural de De Giorgi:a}
\item $\Tan{N-1}{\Hm{N-1}\Radot \mathcal{F}E}{x_{0}}=\Hm{N-1}\Radot \nu^{\perp}_{E}(x_{0})$ y, en particular:
\begin{align}\label{teo:teorema estructural de De Giorgi:b:eq:1}
\lim_{\varrho\to 0}\dfrac{\Hm{N-1}(\mathcal{F}E\cap B_{\varrho}(x_{0}))}{\omega_{N-1}\varrho^{N-1}}=1.
\end{align}
\label{teo:teorema estructural de De Giorgi:b}
\end{enumerate}
\end{teo}
\begin{proof}
Sea $x_{0}\in \mathcal{F}E$, $\overline{v}=v_{E}(x_{0})$, $\varrho_{0}\in (0, \dist{x_{0}}{\partial\DefaultSetWP})$ y $E_{\varrho}=(E-x_{0})/\varrho$. Primero probamos la convergencia de $E_{\varrho}$ a $H$. Como $P(E_{\varrho}, B_{R})=P(E,B_{\varrho R}(x_{0}))/\varrho^{N-1}$, por \ref{lema:cota sobre el perímetro y volumen:eq:cota perimetro} y por la ecuación \ref{teo:compacidad en BV} se deduce que $(E_{\varrho})$ es relativamente compacto con respecto a la topología de la convergencia local en medida en $\R^{N}$. Por lo tanto, para probar la convergencia de $(E_{\varrho})$ a $H$ basta probar que para cualquier $F$, el límite de la sucesión $(E_{\varrho h})$ con $\varrho_{h}\to 0$ coincide con $H$. 

Como $D_{\chi_{E_{\varrho_{h}}}}$ converge débilmente$^{*}$ localmente en $\R^{N}$ a $D\chi_{F}$ cuando $h\to\infty$ y como por la nota \ref{nota:frontera reducida punto de Lebesgue}, $x_{0}$ es un punto de Lebesgue de $v_{E}$ relativo a $\abs*{D\chi_{E}}$ se tiene:
\begin{align}\label{proof:teorema estructural de De Giorgi:eq:1}
D\chi_{F}=\overline{v}\abs{D\chi_{F}},
\end{align}
con $\overline{v}=v(x_{0})$. En particular, $D\chi_{F}$ no tiene componente en dirección ortogonal a $\overline{v}$ y además $\Crochet{D\chi_{F}}{\overline{v}}\geq 0$. Como:
\begin{align*}
\nabla(\chi_{F}*\rho_{\epsilon})=(D\chi_{F})*\rho_{\epsilon}=(\abs*{D\chi_{F}}*\rho_{\epsilon})\overline{v},
\end{align*}
obtenemos que $\chi_{F}*\rho_{\epsilon}(x)$ se puede representar como $\gamma_{\epsilon}(\Crochet{x}{\overline{v}})$ para alguna función creciente $\gamma_{\epsilon}:\R\to [0,1]$. Tomando límite $\epsilon\to 0$ obtenemos:
\begin{align*}
\begin{array}{ll}
\displaystyle
\chi_{F}(x)=\gamma(\Crochet{x}{\overline{v}}),& \text{para casi todo-}\Lm{N}\; x\in \R^{N},
\end{array}
\end{align*}
para alguna función $\gamma:\R\to [0,1]$ tal que $D\gamma\geq 0$. Como $\chi_{F}\in \{0,1\}$ se sigue que $\gamma$ es una función característica $\chi_{L}$, y la proposición \ref{prop:estructura conjuntos perimetro finito dim 1} implica que $L$ debe ser $(c,\infty)$. Si $c$ es estrictamente positivo, entonces considerando $d=c\wedge 1$ se tiene:
\begin{align*}
\lim_{h\to \infty}\dfrac{\abs*{E\cap B_{d\varrho_{h}}}}{\varrho_{h}^{N}}=\lim_{h\to \infty}\abs*{E_{\varrho_{h}}\cap B_{d}}=\abs*{F\cap B_{d}}=0,
\end{align*} 
contradiciendo el hecho que $E$ tiene densidad inferior estrictamente positiva en el punto $x_{0}$, por tanto $c\leq 0$. Análogamente se puede ver que  $c$ no puede ser estrictamente negativo, luego $c=0$ y $F=H$. 

Como $(E_{\varrho})$ converge a $H$, $D\chi_{E_{\varrho}}$ converge débilmente$^{*}$ a $D\chi_{H}$ cuando $\varrho\to 0$, luego $\Tan{N-1}{D\chi_{E}}{x_{0}}=\overline{v}\Hm{N-1}\Radot \partial H$. Por el teorema \ref{teo:propiedades del espacio tangente aproximado a una medida}(\ref{teo:propiedades del espacio tangente aproximado a una medida:b}) se obtiene:
\begin{align}\label{proof:teorema estructural de De Giorgi:eq:2}
\Tan{N-1}{\abs{D\chi_{E}}}{x_{0}}=\Hm{N-1}\Radot \partial H.
\end{align} 
La rectificabilidad de $\mathcal{F}E$ y la coincidencia de $\abs{D\chi_{E}}$ y $\Hm{N-1}\Radot \mathcal{F}E$ se sigue del criterio de rectificabilidad enunciado en el teorema \ref{teo:criterio de rectificabilidad para medidas}(\ref{teo:criterio de rectificabilidad para medidas:b}). Remplazando $\abs{D\chi_{E}}$ por $\Hm{N-1}\Radot \mathcal{F}E$ en \ref{proof:teorema estructural de De Giorgi:eq:2} se sigue (\ref{teo:teorema estructural de De Giorgi:b}).
\end{proof}

Se puede ver rápidamente gracias al apartado (\ref{teo:teorema estructural de De Giorgi:b}) del teorema anterior  y a la ecuación \ref{eq:1:propiedad de consistencia espacio tangente aproximado a un conjunto} que:
\begin{align}\label{eq:1:conclusion sobre espacio tangente teorema estructural de De Giorgi}
\begin{array}{ll}
\Tan{N-1}{\mathcal{F}E}{x}=\nu^{\perp}_{E}(x) & \text{para c.t.p.}\; x\in \mathcal{F}E\;\text{respecto de}\, \Hm{N-1}.
\end{array}
\end{align}

Como consecuencia del teorema \ref{teo:teorema estructural de De Giorgi}  podemos reescribir la fórmula de Gauss-Green \ref{teo:generalizacion formula Gauss-Green:eq:1} para conjuntos de perímetro finito en $\DefaultSetWP$: \index{Gauss-Green!fórmula} \index{Gauss-Green!fórmula}
\begin{align}\label{eq:1:gauss-green frontera reducida}
\begin{array}{ll}
\displaystyle
\int_{E}\div \varphi \, dx = -\int_{\mathcal{F}E}\Crochet{\nu_{E}}{\varphi}d\Hm{N-1},& \forall \varphi\in \Cc[N]{1}.
\end{array}
\end{align}

Ahora examinamos las propiedades de densidad de los conjuntos de perímetro finito.
\begin{defi}[Puntos de densidad $t$ y frontera esencial] \index{punto!de densidad $t$} \index{frontera!esencial}
Para cada $t\in [0,1]$ y todo conjunto $E\subset \R^{N}$ $\Lm{N}$-medible denotamos por $E^{t}$ al conjunto:
\begin{align*}
\set*{x\in \R^{N} \given \lim_{\varrho\to 0}\dfrac{\abs{E\cap B_{\varrho}(x)}}{\abs{B_{\varrho}(x)}}=t},
\end{align*}
de todos los puntos de $E$ con densidad $t$. Denotamos por $\partial^{*}E$ la frontera esencial de $E$, i.e. el conjunto $\R^{N}\setminus (E^{0}\cup E^{1})$ de puntos donde la densidad no es ni $0$ ni $1$.
\end{defi}
Es fácil ver que todos los conjuntos $E^{t}$ son conjuntos de Borel. Los conjuntos $E^{1}$ y $E^{0}$ se pueden entender desde un punto de vista de la teoría de la medida como el interior y el exterior de $E$ respectivamente. Esto motiva la definición de la frontera esencial. El siguiente teorema debido a Federer expone la correspondencia entre la geometría de los conjuntos expuestos anteriormente y su densidad.
\begin{teo}[Teorema estructural de Federer]\label{teo:estrutural de Federer}\index{teorema!estrutural de Federer}
Sea $E$ un conjunto de perímetro finito en $\DefaultSetWP$. Entonces:
\begin{align*}
\begin{array}{lll}
\displaystyle
\mathcal{F}E\cap \DefaultSetWP \subset E^{1/2} \subset \partial^{*}E, & \text{y} & \Hm{N-1}(\DefaultSetWP\setminus (E^{0}\cup \mathcal{F}E \cup E^{1}))=0.
\end{array}
\end{align*}
En particular, $E$ tiene densidad o bien $0$, o bien $1/2$, o bien $1$ en c.t.p. respecto de la medida $\Hm{N-1}$ y casi todo punto $x\in \partial^{*}E \cap \DefaultSetWP$ respecto de la medida $\Hm{N-1}$ pertenece a $\mathcal{F}E$.
\end{teo}
\begin{proof}
La inclusión $\mathcal{F}E\subset E^{1/2}$ se sigue de la convergencia de $(E-x)/\varrho$ al semiespacio ortogonal a $\nu_{E}(x_{0})$ y que contiene a $\nu_{E}(x_{0})$. Ahora observamos que si $P(E, B_{\varrho}(x_{0}))=o(\varrho^{N-1})$ entonces por la desigualdad isoperimétrica relativa \ref{eq:1:desigualdad isoperimetrica local}, denotando por $\alpha(\varrho)$ a $\abs{E\cap B_{\varrho}(x_{0})}$, se sigue que $\set*{\alpha(\varrho),1-\alpha(\varrho)}$ es infinitesimal cuando $\varrho\to 0$. Por tanto, o bien $\alpha(\varrho)\to 0$ o bien $\alpha(\varrho)\to 1$. Luego $x_{0}\in E^{0}\cup E^{1}$. Como $P(E, B_{\varrho}(x_{0}))=\Hm{N-1}(\mathcal{F}E\cap B_{\varrho}(x_{0}))$ se deduce que $\partial^{*}E\cap \DefaultSetWP$ está contenido en un conjunto de densidad estrictamente positiva respecto de $\Hm{N-1}\Radot \mathcal{F}E$. Por otro lado, por \ref{eq:resultados de desnidades:1} sabemos que c.t.p. respecto de la medida $\Hm{N-1}$ el conjunto $\partial^{*}E\cap \DefaultSetWP$ pertenece a $\mathcal{F}E$.
\end{proof}

Vemos que por el teorema anterior, para conjuntos de perímetro finito tanto $\partial^{*}E$ como $E^{1/2}$ pueden usarse en vez de $\mathcal{F}E$ en la fórmula de Gauss-Green \ref{eq:1:gauss-green frontera reducida} y podemos calcular por tanto el perímetro de la siguiente manera:
\begin{align}
\Perimeter{E}=\Hm{N-1}(\DefaultSetWP\cap \partial^{*}E)=\Hm{N-1}(\DefaultSetWP\cap E^{1/2}).
\end{align}
Como consecuencia, podemos también reescribir la fórmula de la coárea usando la frontera esencial de los conjuntos de nivel:
\begin{align}\label{eq:1:formula de la coarea generalizada}
\begin{array}{ll}
\displaystyle
\abs{Du}(B)=\int_{-\infty}^{\infty}\Hm{N-1}(B\cap \partial^{*}\set{u>t})dt, & \forall B\in \mathcal{B}(\DefaultSetWP).
\end{array}
\end{align}

\section{Continuidad y diferenciabilidad aproximada}\label{sec:continuidad y diferenciabilidad aproximada}

Existen funciones $u$ de variación acotada en $\R^{N}$ con $N>1$ discontinuas en un conjunto con medida de Lebesgue estrictamente positiva (ver ejemplo $3.53$ de \cite{ambrosio2000functions}). Análogamente, también es posible encontrar ejemplos (siguiente el mismo argumento que en el ejemplo citado) de tales funciones en el espacio $W^{1,p}$, con $p\leq N$. Luego aparentemente, no hay esperanza de encontrar ``buenos representantes'' como si se podía en dimensión $1$. Esto nos hace ver la necesitada de definir una noción de límite y de diferenciabilidad débil adecuada para que pueda ser satisfecha por funciones en el espacio de Sobolev o en $BV$. Estas nociones se pueden introducir siguiendo la idea elemental de que no solo conjuntos con medida cero, pero también conjuntos con densidad cero pueden ser ignorados. A continuación definimos el límite aproximado. 

\begin{defi}[Límite aproximado]\label{defi:limite aproximado}\index{límite aproximado} \index{conjunto!de discontinuidad}\index{punto!de Lebesgue}
Sea $u\in \Lloc[m]{1}$, se dice que $u$ tiene límite aproximado en $x\in\DefaultSetWP$ si existe un $z\in \R^{m}$ tal que:
\begin{align}\label{defi:limite aproximado:eq:1}
\lim_{\varrho\to 0}\fint_{B_{\varrho}(x)}\abs{u(y)-z}dy=0.
\end{align}
El conjunto de puntos $S_{u}$ donde no se verifica esta propiedad se le llama conjunto aproximado de discontinuidad. Para cualquier $x\in \DefaultSetWP\setminus S_{u}$ al vector $z$, univocamente determinado por la ecuación \ref{defi:limite aproximado:eq:1}, se le llama límite aproximado de $u$ en $x$ y se denota por $\tilde{u}(x)$.
\end{defi}

Diremos que la función $u$ es aproximadamente continua en $x$ si $x\not \in S_{u}$ y si $\tilde{u}(x)=u(x)$, i.e. si $x$ es un punto de Lebesgue de $u$. Se observa que el conjunto de puntos donde el límite aproximado existe no depende del representante de la clase de equivalencia de $u$. Es decir, si $u=v$ $\Lm{N}$-c.t.p en $\DefaultSetWP$ entonces $x\not \in S_{u}$ si y solo si $x\not \in S_{v}$ y $\tilde{u}(x)=\tilde{v}(x)$. Por otro lado, la propiedad de ser aproximadamente continuo en un punto $x$ depende del valor de $u$ en el punto $x$, y este valor puede variar de un representante a otro de la clase de equivalencia.

\begin{prop}[Propiedades de los límites aproximados]\label{prop:Propiedades de los límites aproximados}
Sea $u$ una función en $\Lloc[m]{1}$.
\begin{enumerate}[(a)]
\item $S_{u}$ es un conjunto de Borel de medida cero respecto de la medida $\Lm{N}$ y, $\tilde{u}:\DefaultSetWP\setminus S_{u}\to \R^{m}$ es una función sobre conjuntos de Borel que coincide con $u$ en c.t.p de $\DefaultSetWP\setminus S_{u}$ con respecto a la medida $\Lm{N}$. \label{prop:Propiedades de los límites aproximados:a}
\item Si $x\in \DefaultSetWP\setminus S_{u}$, las funciones $u*\rho_{\epsilon}(x)$ convergen a $\tilde{u}$ cuando $\epsilon\to 0$. \label{prop:Propiedades de los límites aproximados:b}
\item Si $f:\R^{m}\to \R^{p}$ es una aplicación Lipschitz y $v=f\circ u$, entonces $S_{v}\subset S_{u}$ y $\tilde{v}(x)=f(\tilde{u}(x))$ para cualquier $x\in \DefaultSetWP\setminus S_{u}$. \label{prop:Propiedades de los límites aproximados:c}
\end{enumerate}
\end{prop}
\begin{proof}
\begin{enumerate}[(a)]
\item Como el complementario del conjunto de puntos de Lebesgue de $u$ tiene medida cero respecto de $\Lm{N}$, se deduce que $S_{u}$ tiene medida cero respecto de $\Lm{N}$ y que $\tilde{u}$ coincide en c.t.p. con $u$ respecto de la medida $\Lm{N}$. Se prueba que $S_{u}$ es un conjunto de Borel observando que
\begin{align*}
\DefaultSetWP\setminus S_{u} = \displaystyle \bigcap_{n=1}^{\infty}\bigcup_{q\in \Q^{m}} \set*{x\in \DefaultSetWP \given \limsup_{\varrho\to 0} \fint_{B_{\varrho}(x)}\abs*{u(y)-q}dy<\frac{1}{n}}. 
\end{align*}
En efecto la inclusión $\subset$ es trivial, por la densidad de $\Q^{m}$. Si $x$ pertenece al conjunto de la derecha entonces para cualquier entero $n\geq 1$ podemos encontrar un $q_{n}\in \Q^{m}$ tal que se tiene $\displaystyle\limsup_{\varrho\to 0} \fint_{B_{\varrho}(x)}\abs*{u(y)-q_{n}}dy<\frac{1}{n}$. Es fácil ver que $(q_{n})$ es una sucesión de Cauchy, cuyo límite $z$ satisface \ref{defi:limite aproximado:eq:1} y por lo tanto $x\not\in S_{u}$. Como consecuencia de \ref{defi:limite aproximado:eq:1}, para cualquier $x\in \DefaultSetWP \setminus S_{u}$ el valor medio $u_{B_{\varrho}(x)}$ de $u$ en la bola $B_{\varrho}(x)$ converge a $z=\tilde{u}(x)$ cuando $\varrho\to 0$. Luego, el hecho que $\tilde{u}$ es una función sobre Borel conjuntos de Borel en $\DefaultSetWP\setminus S_{u}$ se sigue directamente de la representación como límite puntual cuando $\varrho \to 0$ de las funciones continuas $x\mapsto u_{B_{\varrho}(x)}$.
\item Se puede demostrar este apartado observando que
\begin{align*}
\abs*{u*\rho_{\epsilon}(x)-\tilde{u}(x)}\leq \int_{\R^{N}}\abs*{u(x-\epsilon z)-\tilde{u}(x)}\rho(z)dz\leq \dfrac{\Norm{\rho}[\infty]}{\epsilon^{N}}\int_{B_{\epsilon}(x)}\abs*{u(y)-\tilde{u}(x)}dy.
\end{align*}
Esta última expresión tiende a $0$ cuando $\epsilon \to 0$.
\item Se comprueba (\ref{prop:Propiedades de los límites aproximados:c})  observando que: $\abs*{v(y)-f(\tilde{u}(x))}\leq \lip(f)\abs*{u(y)-\tilde{u}(x)}$.
\end{enumerate}
\end{proof}

\begin{nota}
H. Federer usa en \cite{federer2014geometric} una definición más débil de límite aproximado. Según Federer una función localmente sumable $u$ tiene límite aproximado en $x\in \DefaultSetWP$ y esté límite es $z$ si todos los conjuntos $E_{\epsilon}=\set*{y\in \DefaultSetWP \given \abs*{u(y)-z}>\epsilon}$ tienen densidad $0$. La definición de Federer solo toma en cuenta la geometría de los conjuntos de nivel sin importar lo grande que puede ser la función en conjuntos pequeños. En caso de trabajar con funciones localmente acotadas ambas definiciones son equivalentes (ver proposición 3.65 de \cite{ambrosio2000functions}), sin embargo la definición de Federer tiene sentido también para funciones que no son localmente sumables.  La motivación de usar una definición fuerte de límite aproximado se debe a que en el espacio $BV$ los límites aproximados existen en este sentido.
\end{nota}

Ahora estudiaremos de entre los puntos de discontinuidad de una función aquellos que se corresponden con una discontinuidad aproximada de salto entre dos valores $a$ y $b$ a lo largo de una dirección $\nu$. Antes de seguir introducimos la siguiente notación:
\begin{align}\label{eq:1:medias bolas contenidas en B determinadas por v}
\left \{ 
\begin{array}{ll}
B^{+}_{\varrho}(x,\nu)=\set*{y\in B_{\varrho}(x)\given \Crochet{y-x}{\nu}>0},\\
B^{-}_{\varrho}(x,\nu)=\set*{y\in B_{\varrho}(x)\given \Crochet{y-x}{\nu}<0}.
\end{array}
\right .
\end{align}
\begin{align}\label{eq:1:función que salta de a a b a lo largo del hiperplano ortogonal a v}
u_{a,b,\nu}(y)=\left\{ 
\begin{array}{ll}
a,& \text{si}\; \Crochet{y}{\nu}>0,\\
b,& \text{si}\; \Crochet{y}{\nu}<0.
\end{array}
\right .
\end{align}
para indicar las dos medias bolas contenidas en $B_{\varrho}(x)$ determinadas por $\nu$ y la función que salta entre $a$ y $b$ a lo largo del hiperplano ortogonal a $\nu$. Definimos ahora el concepto de punto aproximado de salto.
\begin{defi}[Punto de salto aproximado]\label{defi:punto aproximado de salto} \index{punto!de salto aproximado}
Sea $u\in\Lloc[m]{1}$ y $x\in \DefaultSetWP$. Se dice que $x$ es un punto de salto aproximado de $u$ si existen $a,b\in \R^{m}$ y $\nu \in \mathbb{S}^{N-1}$ tal que $a\not = b$ y:
\begin{align}\label{defi:punto aproximado de salto:eq:1}
\begin{array}{ll}
\displaystyle
\lim_{\varrho\to 0}\fint_{B^{+}_{\varrho}(x,\nu)}\abs{u(y)-a}dy=0, & \displaystyle \lim_{\varrho\to 0} \fint_{B^{-}_{\varrho}(x,\nu)}\abs{u(y)-b}dy=0.
\end{array}
\end{align}
\end{defi}
La terna $(a,b,\nu)$ queda únicamente determinada salvo permutación de $a$ con $b$ y cambio de signo de $\nu$. Para simplificar podemos considerar la siguiente relación de equivalencia. Se dirá que dos ternas $(a,b,\nu)$, $(a',b',\nu')$ son equivalentes si $(a,b,\nu)=(a',b',\nu')$, o bien $(a,b,\nu)=(b',a',-\nu')$. Para una función $u$ con punto de salto aproximado $x$ denotaremos a la terna de valores $a,b$ y $\nu$ por $(u^{+}(x), u^{-}(x), \nu_{u}(x))$. Al conjunto de puntos de salto aproximado lo denotaremos por $J_{u}$.

A continuación estudiamos las principales propiedades para el conjunto de puntos de salto aproximado.
\begin{prop} \label{prop:propiedades del conjunto de puntos de salto aproximado}
Sea $u\in \Lloc[m]{1}$.
\begin{enumerate}[(a)]
\item El conjunto $J_{u}$ es un subconjunto de Borel de $S_{u}$ y existen funciones sobre conjuntos de Borel:
\begin{align*}
(u^{+}(x),u^{-}(x), \nu_{u}(x)):J_{u}\to \R^{m}\times \R^{m}\times \mathbb{S}^{N-1},
\end{align*}
tales que las ecuaciones de \ref{defi:punto aproximado de salto:eq:1} se verifican para cualquier $x\in J_{u}$. \label{prop:propiedades del conjunto de puntos de salto aproximado:a}
\item Si $x\in J_{u}$, la función $u*\rho_{\epsilon}(x)$ converge a $[u^{+}(x)+u^{-}(x)]/2$ cuando $\epsilon\to 0$. \label{prop:propiedades del conjunto de puntos de salto aproximado:b}
\item Si $f:\R^{m}\to \R^{p}$ es una aplicación Lipschitz, $v=f\circ u$ y $x\in J_{u}$ entonces $x\in J_{v}$ si y solo si $f(u^{+}(x))\not = f(u^{-}(x))$, y en este caso:
\begin{align*}
(v^{+}(x), v^{-}(x), \nu_{v}(x))=(f(u^{+}(x)),f(u^{-}(x)),\nu_{u}(x)).
\end{align*}
En el resto de los casos, $x\not \in S_{v}$ y $\tilde{v}(x)=f(u^{+}(x))=f(u^{-}(x))$. \label{prop:propiedades del conjunto de puntos de salto aproximado:c}
\end{enumerate}
\end{prop}
\begin{proof}
Sea $D=\set{(a_{n}, b_{n}, \nu_{n})}$ un conjunto numerable y denso en $\R^{m}\times\R^{m}\times \mathbb{S}^{N-1}$ y sea $w_{n}(y)=u_{a_{n},b_{n},\nu_{n}}$ en consonancia con la definición \ref{eq:1:función que salta de a a b a lo largo del hiperplano ortogonal a v}. Entonces, por el mismo argumento que el dado en la demostración de la proposición \ref{prop:Propiedades de los límites aproximados}(\ref{prop:Propiedades de los límites aproximados:a}) se tiene que:
\begin{align*}
(\Omega\setminus S_{u})\cup J_{u} =\bigcap_{p=1}^{\infty}\bigcup_{n=0}^{\infty}\set*{x\in \Omega\given \limsup_{\varrho\to 0} \fint_{B_{\varrho}}\abs*{u(x+y)-w_{n}(y)}dy<\dfrac{1}{p}}.
\end{align*}
Como el lado derecho es un conjunto de Borel y $J_{u}\subset S_{u}$ deducimos que $J_{u}$ es un conjunto de Borel. Sea ahora para cualquier $x\in J_{u}$ la terna $(\bar{u}^{+}(x), \bar{u}^{-}(x), \bar{\nu}_{u}(x))$ que satisface las condiciones de la definición \ref{defi:punto aproximado de salto} y probemos que $x\mapsto \phi(x)=\left ( \bar{u}^{+}(x) - \bar{u}^{-}(x) \right )\otimes \bar{\nu}_{u}(x)$ es una aplicación sobre conjuntos de Borel en $J_{u}$, para ello definimos:
\begin{align*}
w_{x}(y)= \left \{ \begin{array}{ll} 
\bar{u}^{+}(x),& \text{si}\; \Crochet{y}{\bar{\nu}_{u}(x)}>0,\\
\bar{u}^{-}(x),& \text{si}\; \Crochet{y}{\bar{\nu}_{u}(x)}<0.
\end{array}
\right.
\end{align*}
y vemos que las funciones reescaladas $u^{x,\varrho}(y)=u(x+\varrho y)$ convergen en $[ L^{1}_{loc}(\R^{N})]^{m}$ a $w_{x}$ cuando $\varrho \to 0$. En particular:
\begin{align*}
x\mapsto \int_{B_{1}}w_{x}(y)\otimes \nabla\psi(y) dy = \lim_{\varrho \to 0}\varrho^{-N}\int_{B_{\varrho}(x)}u(y)\otimes\nabla \psi\left (\dfrac{y-x}{\varrho} \right ) dy,
\end{align*}
es una aplicación de Borel en $J_{u}$ para cualquier $\psi\in C^{\infty}_{c}(B_{1})$. Escogiendo una sucesión $(\psi_{h})\subset C^{\infty}_{c}(B_{1})$ que converge monotonamente a $\chi_{B_{1}}$ obtenemos:
\begin{align*}
\omega_{N-1}\phi(x)=Dw_{x}(B_{1})=\lim_{h\to \infty}\int_{B_{1}}\psi_{h}(y)dDw_{x}(y)=-\lim_{h\to \infty}\int_{B_1}w_{x}(y)\otimes \nabla\psi_{h}(y) dy,
\end{align*}
y esto prueba que $\psi$ es una aplicación de Borel. Para cualquier $\alpha\in \{1, \ldots, m\}$, sea $E_{\alpha}$ el conjunto de todos los $x\in J_{u}$ tales que $\alpha$ es el índice más grande tal que la $\alpha$-ésima fila de $\psi(x)$ es no nula. Como $\psi$ es una aplicación de Borel se puede ver fácilmente que $\{E_{\alpha}\}$ es una partición de Borel de $J_{u}$. En cualquier conjunto $E_{\alpha}$ se puede definir $\nu_{u}$ como $\phi^{\alpha}/\abs{\phi^{\alpha}}$, esto define una aplicación de Borel en $J_{u}$. En consecuencia, como $\nu_{u}$ y $\bar{\nu}_{u}$ son o bien iguales o bien opuestos en signo, podemos definir $(u^{+}(x), u^{-}(x))$ como igual a $(\bar{u}^{+}(x), \bar{u}^{-}(x))$ si $\nu_{u}(x)=\bar{\nu}_{u}(x)$ y igual a $(\bar{u}^{+}(x), \bar{u}^{-}(x))$ si $\nu_{u}(x)=-\bar{\nu}_{u}(x)$. Un argumento análogo al dado para probar \ref{prop:Propiedades de los límites aproximados}(\ref{prop:Propiedades de los límites aproximados:a}) pero sustituyendo $B_{\varrho}(x,\nu_{u}(x))$ por $B^{\pm}_{\varrho}(x,\nu_{u}(x))$ prueba que $u^{\pm}$ son aplicaciones de Borel en $J_{u}$.

La demostración de los apartados (\ref{prop:propiedades del conjunto de puntos de salto aproximado:b}) y (\ref{prop:propiedades del conjunto de puntos de salto aproximado:c}) es análoga a proporcionada en la proposición \ref{prop:Propiedades de los límites aproximados} pero dividiendo esta vez la región de integración en $B^{+}_{\varrho}(x, \nu)$ y $B^{-}_{\varrho}(x, \nu)$. 
\end{proof}

En lo que sigue de sección estudiaremos el valor medio de $\abs{u(y)-\tilde{u}(x)-L(y-x)}/\varrho$ en bolas de radio pequeño $B_{\varrho}(x)$ y veremos la necesidad de introducir la noción de diferenciabilidad aproximada. A continuación damos la definición de punto de diferenciabiliadad aproximada.
\begin{defi}[Punto de diferenciabiliadad aproximada]\label{defi:punto de diferenciabiliadad aproximada} \index{punto!de diferenciabiliadad}
Sea $u\in \Lloc[m]{1}$ y sea $x\in \DefaultSetWP\setminus S_{u}$, se dice que $u$ es diferenciable en un sentido aproximado en $x$ si existe una matriz $L$ de dimensiones $m\times N$ tal que:
\begin{align}\label{defi:Punto de diferenciabiliadad aproximada:eq:1}
\lim_{\varrho\to 0}\fint_{B_{\varrho}(x)}\dfrac{\abs{u(y)-\tilde{u}(x)-L(y-x)}}{\varrho} dy =0.
\end{align}
Si $u$ es diferenciable en sentido aproximado en $x$ entonces a la  matriz $L$, univocamente determinada por la identidad \ref{defi:Punto de diferenciabiliadad aproximada:eq:1}, se le llama diferencial aproximada de $u$ en $x$ y se denotará por $\nabla u(x)$.
\end{defi}
Al conjunto de puntos de diferenciabilidad en sentido aproximado lo denotaremos por $\mathcal{D}_{u}$. A continuación, exponemos algunas de las propiedades del conjunto de puntos diferenciables en sentido aproximado.
\begin{prop}\label{prop:propiedades del conjunto de puntos de diferenciabilidad aproximada}\index{regla de la cadena}
Sea $u\in \Lloc[m]{1}$.
\begin{enumerate}[(a)]
\item El conjunto de puntos $\mathcal{D}_{u}\subset \DefaultSetWP\setminus S_{u}$, donde $u$ es diferenciable en sentido aproximado es un conjunto de Borel y $\nabla u:\mathcal{D}_{u}\to \R^{mN}$ una aplicación sobre conjuntos de Borel.
\item Si $x\in \mathcal{D}_{u}$ y $f:\R^{m}\to \R^{p}$ es una función con crecimiento lineal en el infinito y diferenciable en $\tilde{u}(x)$, entonces $v=f\circ u$ es aproximadamente diferenciable en $x$ y se tiene: 
\begin{align*}
\nabla v(x)=\nabla f(\tilde{u}(x))\nabla u(x).
\end{align*}
\end{enumerate}
\end{prop}
\begin{proof}
\begin{enumerate}[(a)]
\item Denotando por $D=\set{L_{n}}$ un conjunto numerable y denso en el espacio de las matrices $m\times N$, entonces por un argumento análogo al de la proposición \ref{prop:Propiedades de los límites aproximados}(\ref{prop:Propiedades de los límites aproximados:a}) se prueba que $\mathcal{D}_{u}$ es representable por:
\begin{align*}
\bigcap_{p=1}^{\infty}\bigcup_{n=0}^{\infty}\set*{x\in \Omega\setminus S_{u}\given \limsup_{\varrho\to 0} \varrho^{-N-1}\int_{B_{\varrho}(x)}\abs{u(y)-\tilde{u}(x)-L_{n}(y-x)}dy<\dfrac{1}{p}},
\end{align*}
y por lo tanto es un conjunto de Borel. Para cualquier rectángulo $R\subset \R^{N}$ con interior no vacío se tiene que:
\begin{align*}
x\mapsto \dfrac{1}{\abs{R}}\int_{R}\nabla{u}(x)y\,dy=\dfrac{1}{\abs{R}}\lim_{\varrho\to 0}\varrho^{-N-1}\int_{x+\varrho R}(u(y)-\tilde{u}(x))dy,
\end{align*}
es una función de Borel en $\mathcal{D}_{u}$. Dado $\nu\in \mathbb{S}^{N-1}$ y tomando:
\begin{align*}
\begin{array}{ll}
\displaystyle
R_{p}=\set*{y\in \R^{N}\given 0\leq \Crochet{y}{v}\leq 1, \abs{y-\Crochet{y}{\nu}\nu}\leq \dfrac{1}{p}}, & p\in \N, p\geq 1,
\end{array}
\end{align*}
por un argumento de límite se prueba que $x\mapsto \nabla{u}(x)\nu$ es una aplicación de Borel.
\item Sea $G=\nabla{f(\tilde{u}(x))}, L=\nabla{u}(x)$. Por suposición sobre $f$ podemos encontrar una función $\omega:[0,\infty)\to[0,\infty)$ tal que $\omega(0)=0$ y:
\begin{align*}
\begin{array}{ll}
\displaystyle
\abs*{f(z)-f(\tilde{u}(x))-G(z-\tilde{u}(x))}\leq \omega(\abs{z-\tilde{u}(x)})\abs{z-\tilde{u}(x)},& \forall z\in \R^{m}.
\end{array}
\end{align*}
Sea $u_{\varrho}(z)=u(x+\varrho z)-\tilde{u}(x)$, y recordemos que $u_{\varrho}$ converge a $0$ en $L^{1}(B_{1})$. Teniendo en cuenta que $L$ es la diferencial aproximada de $u$ en $x$, entonces por el teorema de convergencia dominada de Vitali se ve que:
\begin{align*}
\limsup_{\varrho\to 0}& \varrho^{-N-1}\int_{B_{\varrho}(x)}\abs{f(u(y))-f(\tilde{u})-G[u(y)-\tilde{u}(x)]} dy\\
&\leq \limsup_{\varrho \to 0}\varrho^{-N-1} \int_{B_{\varrho}(x)}\omega(\abs{u(y)-\tilde{u}(x)})\abs{u(y)-\tilde{u}(x)}dy\\
&\leq \limsup_{\varrho \to 0}\varrho^{-N-1} \int_{B_{\varrho}(x)}\omega(\abs{u(y)-\tilde{u}(y)})\abs{L(y-x)}dy\\
&\leq \Norm{L}[\infty]\limsup_{\varrho \to 0}\int_{B_{1}}\omega(\abs{u_{\varrho}(z)})dz=0.
\end{align*}
Por otro lado:
\begin{align*}
\lim_{\varrho\to 0}\varrho^{-N-1}\int_{B_{\varrho}(x)}\abs{G(u(y)-\tilde{u}(x))-GL(y-x)}dy=0.
\end{align*}
Sumando las dos identidades anteriores se sigue el resultado.
\end{enumerate}
\end{proof}

Todos los conceptos introducidos hasta ahora se pueden reformular de una manera diferente reescalando en la variable independiente. 
Es decir, $u$ tiene límite aproximado $z$ en $x$ si y solo si la función reescalada $u^{x,\varrho}(y)=u(x+\varrho y)$ converge en $[L_{loc}^{1}(\R^{N})]^{m}$ a $z$ cuando $\varrho\to 0$. Análogamente, $x\in J_{u}$ y $(u^{+}(x), u^{-}(x), v_{u}(x))=(a,b,v)$ si y solo si $u^{x,\varrho}$ converge en $[L_{loc}^{1}(\R^{N})]^{m}$ a la función $u_{a,b,v}$ de \ref{eq:1:función que salta de a a b a lo largo del hiperplano ortogonal a v}. Finalmente reescalando en la variable dependiente podemos a su vez decir que $u$ e aproximadamente diferenciable en $x$ y $\nabla u(x)=L$ si y solo si las funciones $v_{\varrho}(y)=[u(x+\varrho y)-\tilde{u}(x)]/\varrho$ convergen en $[L_{loc}^{1}(\R^{N})]^{m}$ a la aplicación lineal $Ly$.

\section{Propiedades adicionales de las funciones de variación acotada}\label{sec:propiedades ajustadas de las funciones de variación acotada}

En esta sección estudiaremos más en detalle las propiedades del límite y de la diferenciabilidad aproximada de las funciones de variación acotada. En particular, extenderemos los resultados de la sección \ref{sec:estructura de los conjuntos de perímetro finito} a cualquier tipo de función de variación acotada. Como veremos cualquier función de variación acotada $u$ tiene trazas en conjuntos $\Hm{N-1}$-rectificables con respecto de la medida $\Hm{N-1}$. También veremos que casi todo punto $x\in S_{u}$ con respecto a la medida $\Hm{N-1}$ es un punto de salto aproximado con la dirección del salto ortogonal al espacio tangente aproximado. Con respecto a la diferenciabilidad aproximada veremos que cualquier función de variación acotada $u$ en su dominio es diferenciable en sentido aproximado en casi todas partes respecto de la medida $\Lm{N}$ y que $\nabla u$ es la densidad de la parte absolutamente continua de $Du$ con respecto de la medida $\Lm{N}$.

En primer lugar necesitaremos el siguiente resultado que nos dice que el valor medio de $\abs{u}^{1^{*}}$ en bolas $B_{\varrho}(x)$ está uniformemente acotado cuando $\varrho\to 0$ para casi todo punto $x$ respecto de $\Hm{N-1}$.
\begin{lema}\label{lema: valor medio 1* en bolas uniformemente acotado}
Para cualquier $u\in \BV[m]$ el conjunto:
\begin{align*}
L=\set*{x\in \Omega\given \limsup_{\varrho\to 0} \fint_{B_{\varrho}(x)}\abs{u(u)}^{1^{*}}dy=\infty},
\end{align*}
es de medida cero respecto de $\Hm{n-1}$.
\end{lema}
\begin{proof}
La demostración del lema es larga y técnica, a continuación, damos una idea de la demostración. En primer lugar, hay que acotar el tamaño de la intersección de una sucesión adecuada de conjuntos de perímetro finito. Se puede probar bajo ciertas suposiciones que esta intersección tiene medida cero respecto de $\Hm{N-1}$, si se consideran únicamente puntos donde la densidad es suficientemente grande. Este resultado corresponde al lema técnico $3.74$ de \cite{ambrosio2000functions}. Una vez probado esto se prueba que el conjunto $L$ está contenido en la unión dos conjuntos de medida cero, el conjunto de puntos $D=\set*{x\in \Omega \given \lim \sup_{\varrho\to 0} \abs{Du}(B_{\varrho}(x))/\varrho^{N-1}=\infty}$ y el conjunto de puntos de densidad superior o igual a $1$. Para probar esto último se usa la desigualdad de Poincaré, se reescala y traslada, y extrayendo quizás una subsucesión se concluye por contradicción que $L$ está contenido en la unión de estos dos conjuntos de medida cero debido a las propiedades de la densidad y por el resultado visto antes. La demostración de esta última parte corresponde con el lema $3.75$ de \cite{ambrosio2000functions}.
\end{proof}

En el lema siguiente se comparará la medida $\abs{Du}$ con la medida $\Hm{N-1}$, probando en particular que $\abs{Du}$ es absolutamente continua respecto de $\Hm{N-1}$, es decir, que se anula en cualquier conjunto de medida cero respecto de la medida de Hausdorff $(N-1)$-dimensional.
\begin{lema}\label{lema:trazas en el interior de conjuntos rectificables}\DefaultSet{\Omega}
Sea $u\in \BV[m]$. Entonces $\abs{Du}\geq \abs{u^{+}-u^{-}}\Hm{N-1}\Radot J_{u}$ y para cualquier conjunto de Borel $B\subset \DefaultSetWP$ se tienen las siguientes dos implicaciones:
\begin{align}
&\Hm{N-1}(B)=0\implies \abs{Du}(B)=0,\label{lema:trazas en el interior de conjuntos rectificables:eq:1}\\
&\Hm{N-1}(B)<\infty, B\cap S_{u}=\emptyset \implies \abs{Du}(B)=0.\label{lema:trazas en el interior de conjuntos rectificables:eq:2}
\end{align}
\end{lema}
\begin{proof}
Sea $x\in J_{u}$  entonces por la definición de $u^{\pm}(x)$ la familia de funciones rescaladas $u_{\varrho}(y)=u(x+\varrho y)$ convergen en $L_{loc}^{1}(\R^{N})$ a:
\begin{align*}
w_{x}(y)=\left \{\begin{array}{ll}
u^{+}(x),& \text{si}\; \Crochet{y}{\nu_{u}(x)}>0,\\
u^{-}(x),& \text{si}\; \Crochet{y}{\nu_{u}(x)}<0.
\end{array}
\right .
\end{align*}
Por la semicontinuidad inferior de la variación tenemos:
\begin{align*}
\lim_{\varrho\to 0}\inf \dfrac{\abs*{Du}(B_{\varrho}(x))}{\omega_{N-1}\varrho^{N-1}}=\lim_{\varrho\to 0}\inf \dfrac{\abs*{Du_{\varrho}}(B_{1})}{\omega_{N-1}}\geq \dfrac{\abs*{Dw_{x}(B_{1})}}{\omega_{N-1}}=\abs*{u^{+}(x)-u^{-}(x)},
\end{align*}
y por el teorema \ref{teo:resultados sobre densidades de hausdorff y medidas de radon} se tiene que $\abs*{Du}\geq \abs*{u^{+}-u^{-}}\Hm{N-1}\Radot J_{u}$.

Ahora probamos la ecuación \ref{lema:trazas en el interior de conjuntos rectificables:eq:1}. Si $m=1$ y $u=\chi_{E}$ es una función característica la implicación se sigue directamente de la representación $\abs*{Du}=\Hm{N-1}\Radot \mathcal{F}E$ dada en el teorema \ref{teo:teorema estructural de De Giorgi}. Si $u\in\BV$ gracias a la fórmula de la coárea \ref{eq:1:formula de la coarea generalizada} obtenemos la implicación buscada. Finalmente, si $u\in\BV[m]$ entonces podemos usar la desigualdad dada en \ref{eq:1:desigualdades de normas bv}.

Para probar la segunda ecuación \ref{lema:trazas en el interior de conjuntos rectificables:eq:2}, vemos que si $u\in \L1$ para cualquier punto $x$ donde el límite aproximado de $u$ existe, entonces hay como mucho un $t$, que denotamos por $\tilde{u}(x)$, tal que $x\in \partial^{*}\{u>t\}$. En efecto, si $t>\tilde{u}(x)$ se tiene:
\begin{align*}
\abs*{\{u>t\}\cap B_{\varrho}(x)}\leq \dfrac{1}{t-\tilde{u}(x)}\int_{B_{\varrho}(x)}\abs*{u(y)-\tilde{u}(x)} dy=o(\varrho^{N}),
\end{align*}
por tanto $\{u>t\}$ tiene densidad $0$ en $x$. Análogamente, si $t<\tilde{u}(x)$ podemos ver que $\{u>t\}$ tiene densidad $1$ en $x$. Sea $B\in \mathcal{B}(\DefaultSetWP)$ con $\Hm{N-1}$-medida finita y disjunto de $S_{u}$ y sea $u\in \BV$. Por \ref{eq:1:formula de la coarea generalizada} y por el teorema de Fubini se tiene:
\begin{align*}
\abs*{Du}(B)&=\int^{\infty}_{-\infty}\Hm{N-1}\Radot B(\partial^{*}\set{u>t})dt\\
&= \int_{B}\Lm{1}(\set*{t\in \R\given x\in \partial^{*}\set{u>t}})d\Hm{N-1}(x)=0.
\end{align*}
Finalmente, si $u\in \BV[m]$ observamos que $B$ y $S_{u^{\alpha}}$ siguen siendo disjuntos para cualquier componente $u^{\alpha}$ de $u$, por tanto $\abs*{Du^{\alpha}}(B)=0$ para cualquier $\alpha=1, \ldots, m$. Aplicando \ref{nota:desigualdades de normas bv} concluimos. 
\end{proof}

El primer teorema que enunciaremos en esta sección trata sobre la existencia de trazas de las funciones $BV$ en conjuntos $\Gamma$ numerablemente $\Hm{N-1}$-rectificables en el dominio. También veremos que las trazas proporcionan una representación de la medida $Du\Radot \Gamma$. Pero antes de enunciar el teorema introducimos la siguiente terminología. 
Se dice que $\Gamma$ es orientada por una aplicación $\nu:\Gamma\to \mathbb{S}^{N-1}$ sobre conjuntos de Borel, si $\Tan{N-1}{\Gamma}{x}=\nu^{\perp}(x)$ para casi todo $x\in \Gamma$ con respecto de la medida $\Hm{N-1}$. 

\begin{teo}[Teorema de trazas en el interior de conjuntos rectificables]\label{teo:teorema de trazas en el interior de conjuntos rectificables}\DefaultSet{\Omega}\index{teorema!de trazas}
Sea $u$ una función en $\BV[m]$ y sea $\Gamma\subset \DefaultSetWP$ un conjunto numerable $\Hm{N-1}$-rectificable orientado por $\nu$. Entonces, para casi todo $x\in \Gamma$ con respecto a la medida $\Hm{N-1}$ existen $u_{\Gamma}^{+}(x), u_{\Gamma}^{-}(x)$ en $\R^{m}$ tales que:
\begin{align}\label{teo:teorema de trazas en el interior de conjuntos rectificables:eq:1}
\left \{\begin{array}{ll}
 \displaystyle\lim_{\varrho\to 0}\fint_{B^{+}_{\varrho}(x,\nu(x))}\abs{u(y)-u_{\Gamma}^{+}(x)}dy=0,\\
\displaystyle\lim_{\varrho\to 0}\fint_{B^{-}_{\varrho}(x,\nu(x))}\abs{u(y)-u_{\Gamma}^{-}(x)}dy=0.
\end{array}
\right .
\end{align}
Además, $Du\Radot \Gamma = (u^{+}_{\Gamma}-u^{-}_{\Gamma})\otimes \nu\Hm{N-1}\Radot \Gamma$.
\end{teo}
\begin{proof}
Paso 1.(\textit{Existencia de trazas}) Si $u=\chi_{E}$ es la función característica de un conjunto de perímetro finito en $\DefaultSetWP$ se sabe por el teorema \ref{teo:estrutural de Federer} que casi todo $x\in\DefaultSetWP$ respecto de la medida $\Hm{N-1}$ pertenece a $(E^{0}\cup \mathcal{F}E\cup E^{1})$. Si $x\in E^{0}$ basta tomar $u_{\Gamma}^{\pm}(x)=0$ y si $x\in E^{1}$ se toma $u_{\Gamma}^{\pm}(x)=1$. Ahora consideremos puntos $x\in \mathcal{F}E$, y recordemos que el conjunto rescalado $E_{\varrho}=(E-x)/\varrho$ converge en $L_{loc}^{1}(\R^{N})$ cuando $\varrho\to 0$ a la función característica del semiespacio ortogonal a $\nu_{E}(x)$. Por la ecuación \ref{eq:1:conclusion sobre espacio tangente teorema estructural de De Giorgi} y por la propiedad de localidad de los espacios de tangente aproximada \ref{eq:1:propiedad de localidad del espacio tangente aproximado a un conjunto} se deduce que:
\begin{align*}
\begin{array}{ll}
\displaystyle
\nu_{E}(x)=\pm \nu(x),& \text{para}\; \Hm{N-1}-c.t.p. \;\; x\in \mathcal{F}E\cap \Gamma. 
\end{array}
\end{align*}
Si $\nu_{E}(x)=\nu(x)$ entonces \ref{teo:teorema de trazas en el interior de conjuntos rectificables:eq:1} se cumple tomando $u_{\Gamma}^{+}(x)=1$ y $u_{\Gamma}^{-}(x)=0$. Si $\nu_{E}(x)=-\nu(x)$ entonces \ref{teo:teorema de trazas en el interior de conjuntos rectificables:eq:1} se verifica si tomamos $u_{\Gamma}^{+}(x)=0$ y $u_{\Gamma}^{-}(x)=1$.

Por linealidad $u^{\pm}_{\Gamma}$ sigue existiendo en $\Hm{N-1}$-c.t.p. en $\Gamma$ si $u$ es una función simple $BV$, es decir si existe una cantidad finita de números $z_{1}, \ldots, z_{n}$ y conjuntos $E_{1}, \ldots, E_{n}$ de perímetro finito en $\DefaultSetWP$ tales que $u=\sum_{i}z_i \chi_{E_{i}}$. Por otro lado, por la fórmula de la coárea cualquier función acotada $u$ en $BV$ se puede aproximar por convergencia uniforme por una sucesión $(u_{h})$ de funciones simples $BV$. Luego $u^{\pm}_{\Gamma}(x)$ están definido en cualquier punto $x\in \Gamma$ donde todas las trazas $(u_{h})^{\pm}_{\Gamma}$ estén definidas.

Sea $u\in \BV$ y $u_{h}=h\wedge (u\vee -h)$ la función $u$ truncada. Es fácil ver por la definición de variación que las funciones $u_{h}\in\BV$. Consideremos el conjunto de medida cero $L$ del lema \ref{lema: valor medio 1* en bolas uniformemente acotado}  y sea $M_{h}\subset \Gamma$ los conjuntos de medida cero donde las trazas $(u_{h})_{\Gamma}^{\pm}$ no están definidas. Podemos ver entonces que $u_{\Gamma}^{\pm}$ existen en cualquier punto $x\in \Gamma\setminus (L\cup \bigcup_{h}M_{h})$. En efecto, las sucesiones $((u_{h})_{\Gamma}^{\pm}(x))$ están ambas acotadas pues $x\not \in L$ y $\abs{u_{h}}\leq \abs{u}$. Denotando por $((u_{h(k)})_{\Gamma}^{\pm}(x))$ las subsucesiones convergentes a los números reales $z^{\pm}$ podemos estimar:
\begin{align*}
\int_{B^{\pm}_{\varrho}(x,\nu(x))}&\abs{u(y)-z^{\pm}}dy\leq \int_{B^{\pm}_{\varrho}(x,\nu(x))}\abs{u(y)-u_{h(k)}(y)}dy\\
+&\int_{B^{\pm}_{\varrho}(x,\nu(x))}\abs{u_{h(k)}(y)-(u_{h(k)})_{\Gamma}^{\pm}(x)}dy+\frac{\omega_{N}\varrho^{N}}{2}\abs{(u_{h(k)})_{\Gamma}^{\pm}(x)-z^{\pm}}.
\end{align*}
Dividiendo a ambos lados por $\varrho^{N}$ y tomando límite cuando $\varrho\to 0$ se obtiene:
\begin{align*}
\limsup_{\varrho\to 0}& \varrho^{-N}\int_{B^{\pm}_{\varrho}(x,\nu(x))}\abs{u(y)-z^{\pm}}dy\\
\leq& 2 \limsup_{\varrho\to 0} \varrho^{-N}\int_{B^{\pm}_{\varrho}(x,\nu(x))\cap \set{\abs{u}>h(k)}}\abs{u(y)}dy+\frac{\omega_{N}}{2}\abs{(u_{h(k)})_{\Gamma}^{\pm}(x)-z^{\pm}}
\\
\leq &\dfrac{2}{h(k)^{1/(N-1)}}\limsup_{\varrho\to 0} \varrho^{-N}\int_{B_{\varrho}(x)}\abs{u(y)}^{1^{*}}dy+\frac{\omega_{N}}{2}\abs{(u_{h(k)})_{\Gamma}^{\pm}(x)-z^{\pm}}.
\end{align*}
Como $k$ es arbitrario, pasando al límite cuando $k\to \infty$ se concluye que $u_{\Gamma}^{\pm}(x)=z^{\pm}$ satisface \ref{teo:teorema de trazas en el interior de conjuntos rectificables:eq:1}.

Paso 2.(\textit{Representación de $Du\Radot \Gamma$}) Podemos suponer sin pérdida de generalidad que $\Hm{N-1}(\Gamma)<\infty$ y que ambas trazas $u_{\Gamma}^{+}(x)$,$u_{\Gamma}^{-}(x)$ están definidas en cualquier punto $x\in \Gamma$.
Sea $D_{1}u=Du\Radot \Gamma$ y $D_{2}u=Du-D_{1}u$. Por \ref{lema:trazas en el interior de conjuntos rectificables:eq:1} sabemos que $D_{1}u\ll \Hm{N-1}\Radot \Gamma$, por lo tanto por el teorema de derivación de Besicovitch solo hay que probar que:
\begin{align*}
\lim_{\varrho\to 0}\dfrac{D_{1}u(B_{\varrho}(x))}{\Hm{N-1}(\Gamma\cap B_{\varrho}(x))}=(u^{+}_{\Gamma}(x)-u^{-}_{\Gamma}(x))\nu(x),
\end{align*}
para casi todo $x\in \Gamma$ respecto de la medida $\Hm{N-1}$. Teniendo en cuenta que:
\begin{align*}
\begin{array}{ll}
\displaystyle 
\abs{D_{2}u}(B_{\varrho}(x))=o(\varrho^{N-1}), & \displaystyle \lim_{\varrho\to 0}\dfrac{\Hm{N-1}(\Gamma\cap B_{\varrho}(x))}{\omega_{N-1}\varrho^{N-1}}=1,
\end{array}
\end{align*}
para casi todo $x\in \Gamma$ respecto de la medida $\Hm{N-1}$ solo hace falta probar la existencia de una sucesión infinitesimal $(\varrho_{i})\subset (0,\infty)$ tal que:
\begin{align}\label{proof:teorema de trazas en el interior de conjuntos rectificables:eq:1}
\lim_{i\to \infty}\dfrac{Du(B_{\varrho_{i}}(x))}{\omega_{N-1}\varrho^{N-1}_{i}}=(u^{+}_{\Gamma}(x)-u^{-}_{\Gamma}(x))\nu(x),
\end{align}
para casi todo $x\in \Gamma$ con respecto a la medida $\Hm{N-1}$. Probamos que esto es cierto para cualquier $x\in \Gamma$ donde $\abs{Du}(B_{\varrho}(x))/\varrho^{N-1}$ está acotado cuando $\varrho\to 0$ (por \ref{eq:1:hausdorff y medidas de radon}, la condición falla solo en subconjuntos de medida cero de $\DefaultSetWP$). Sea $x\in \Gamma$ con esta propiedad, por la definición de $u^{\pm}_{\Gamma}$ la familia de funciones rescaladas $u^{\varrho}(y)=u(x+\varrho y)$ converge en $L^{1}_{loc}(\R^{N})$ a:
\begin{align*}
w_{x}(y)=\left \{ \begin{array}{ll}
u^{+}_{\Gamma}(x), \quad \text{si}\; \Crochet{y}{\nu(x)}>0,\\
u^{-}_{\Gamma}(x), \quad \text{si}\; \Crochet{y}{\nu(x)}<0.
\end{array}
\right.
\end{align*}
Como por suposición sobre $x$, $\abs{Du^{\varrho}}(B_{1})=\abs{Du}(B_{\varrho})/\varrho^{N-1}$ está acotado cuando $\varrho\to 0$ entonces por la proposición \ref{prop:criterio convergencia débi} se deduce que $(Du^{\varrho})$ converge débilmente$^{*}$  en $B_{1}$ a $Dw_{x}$ cuando $\varrho\to 0$. Sea $(\eta_{i})\subset (0,\infty)$ una sucesión infinitesimal tal que $\abs{Du^{\eta_{i}}}$ converge débilmente$^{*}$ a alguna medida $\sigma$ en $B_{1}$ y sea $t\in (0,1)$ tal que $\sigma(\partial B_{t})=0$ y $\varrho_{i}=t\eta_{i}$. Por \ref{prop:convergencia dominada de medidas}(\ref{prop:convergencia dominada de medidas:b}) se tiene:
\begin{align*}
\lim_{i\to 0}\dfrac{Du(B_{\varrho_{i}}(x))}{\omega_{N-1}\varrho_{i}^{N-1}}=\lim_{i\to \infty}\dfrac{Du^{\eta_{i}}(B_t)}{\omega_{N-1}t^{N-1}}=\dfrac{D\omega_{x}(B_{t})}{\omega_{N-1}t^{N-1}}.
\end{align*}
Como $D\omega_{x}=(u^{+}_{\Gamma}-u^{-}_{\Gamma})\nu(x)\Hm{N-1}\Radot \nu^{\perp}(x)$ se sigue \ref{proof:teorema de trazas en el interior de conjuntos rectificables:eq:1}.
\end{proof}

Sabemos por los teoremas \ref{teo:teorema estructural de De Giorgi} y \ref{teo:estrutural de Federer} que la frontera esencial $\partial^{*}E$ de un conjunto de perímetro finito $E$ en $\DefaultSetWP$ es $\Hm{N-1}$-rectificable y casi todo punto de $\partial^{*}E$ respecto de la medida $\Hm{N-1}$ pertenece a $\mathcal{F}E$. En particular, considerando $u=\chi_{E}$ esto significa que el conjunto de discontinuidades $S_{u}$ (i.e. $\partial^{*}E$) es $\Hm{N-1}$-rectificable y casi todo punto en $S_{u}$ respecto de la medida $\Hm{N-1}$ es un punto de salto aproximado. Esta afirmación también es cierta para funciones $\BV[m]$ como veremos en el siguiente teorema.
\begin{teo}[Teorema de Federer-Vol'pert]\label{teo:teorema de Federer-Vol'pert}\index{teorema!de Federer-Vol'pert}
Para cualquier función $u\in \BV[m]$ el conjunto de discontinuidades $S_{u}$ es numerablemente $\Hm{N-1}$-rectificable y $\Hm{N-1}(S_{u}\setminus J_{u})=0$. Además, $Du\Radot J_{u}=(u^{+}-u^{-})\otimes \nu_{u}\Hm{N-1}\Radot J_{u}$ y se tiene que:
\begin{align}\label{teo:teorema de Federer-Vol'pert:eq:1}
&\Tan{N-1}{J_{u}}{x}=\nu_{u}^{\perp}(x),\\ 
&\Tan{N-1}{\abs{Du}\Radot J_{u}}{x}=\abs{u^{+}(x)-u^{-}(x)}\Hm{N-1}\Radot \nu_{u}(x)^{\perp},\nonumber 
\end{align}
para casi todo $x\in J_{u}$ respecto de la medida $\Hm{N-1}$.
\end{teo}
\begin{proof}
Probamos primero que $S_{u}$ es un conjunto numerablemente $\Hm{N-1}$-rectificable. Como $S_{u}$ está contenido en $\bigcup_{\alpha}S_{u^{\alpha}}$, en la prueba de este hecho no es restrictivo suponer que $m=1$. Por la fórmula de la coárea en $BV$ podemos encontrar un conjunto numerable y denso $D\subset \R$ de números reales $t$ tales que $\set{u>t}$ tiene perímetro finito en $\DefaultSetWP$ para cualquier $t\in D$. Probamos que:
\begin{align*}
S_{u}\setminus L \subset \bigcup_{t\in D}\partial^{*}\set{u>t},
\end{align*}
donde $L$ es el conjunto de medida cero del lema \ref{lema: valor medio 1* en bolas uniformemente acotado}. Como las fronteras esenciales de conjuntos de perímetro finito son $\Hm{N-1}$-rectificables, esta inclusión prueba que $S_{u}$ es un conjunto rectificable.

Sea $x\not \in L$ y supongamos que $x$ no está en la unión de conjuntos de la ecuación anterior, i.e. o es un punto de densidad $0$ o es un punto de densidad $1$ de $\set{u>t}$ para cualquier $t\in D$. Como:
\begin{align*}
t\dfrac{\abs*{\set{u>t}\cap B_{\varrho}(x)}}{\abs{B_{\varrho}(x)}}\leq \dfrac{1}{\omega_{N}\varrho^{N}}\int_{B_{\varrho}(x)}\abs{u(y)}dy,
\end{align*} 
para cualquier $t\in D\cap (0,\infty)$. Para $t\in D$ suficientemente grande, $x$ es un punto de densidad $0$ de $\set{u>t}$. Análogamente, si $t\in D\cap (-\infty, 0)$ y $-t$ es suficientemente grande, $x$ es un punto de densidad $1$ para $\set{u>t}$. Esto prueba que $z=\sup\set*{t\in D\given \set{u>t}\; \text{tiene densidad 1}\;\text{en}\; x}$ es un número real. Por definición de $z$, $\set{u>t}$ tiene densidad $0$ en $x$ para cualquier $t\in D$, $t>z$, puesto que $D$ es denso en $\R$. Un argumento de comparación prueba que se tiene lo mismo para cualquier conjunto $\set{u>t}$ para cualquier $t> z$. Un argumento similar basado en el hecho que $x\not \in \partial^{*}\set{u>t}$ para cualquier $t\in D$, prueba que $\set{u<t}$ tiene densidad $0$ en $x$ para cualquier $t<z$.

Ahora probamos que $z=\tilde{u}(x)$, para ello definimos $E_{\epsilon}=\set{\abs{u-z}>\epsilon}$ y vemos que:
\begin{align*}
\int_{B_{\varrho}(x)}\abs{u-z}dy & \leq \epsilon \omega_{N}\varrho^{N} + \int_{E_{\epsilon}\cap B_{\varrho}(x)}\abs{u-z}dy\\&\leq \epsilon\omega_{N}\varrho^{N}+\abs{E_{\epsilon}\cap B_{\varrho}(x)}^{1/N}\left( \int_{B_{\varrho}(x)}\abs{u-z}^{1^{*}}\right )^{1/1^{*}}.
\end{align*}
Dividiendo ambos lados por $\omega_{N}\varrho^{N}$ y teniendo en cuenta que $\abs{E_{\epsilon}\cap B_{\varrho}(x)}=o(\varrho^{N})$ y que $x\not \in L$ se obtiene:
\begin{align*}
\limsup_{\varrho\to 0} \fint_{B_{\varrho}(x)}\abs{u(y)-z}dy\leq \epsilon.
\end{align*}
Como $\epsilon$ es arbitrario esto prueba que $x\not \in S_{u}$.

Volviendo al caso general $m\geq 1$, podemos probar que $\Hm{N-1}(S_{u}\setminus J_{u})=0$. Como $\Gamma=S_{u}$ es numerablemente $\Hm{N-1}$-rectificable se puede fijar la orientación $\overline{v}$ de $S_{u}$ para obtener por el teorema \ref{teo:teorema de trazas en el interior de conjuntos rectificables} los límites aproximados $u_{\Gamma}^{+}$ y $u_{\Gamma}^{-}$ definidos en casi todo $x\in \Gamma$ respecto de la medida $\Hm{N-1}$. Por la definición de $J_{u}$, cualquier punto donde ambos $u_{\Gamma}^{+}$ y $u_{\Gamma}^{-}$ existen, es un punto de salto aproximado con $(u^{+}(x), u^{-}(x), v_{u}(x))=(u_{\Gamma}^{+}(x), u_{\Gamma}^{-}(x))$. Luego $u_{\Gamma}^{+}(x)\not = u_{\Gamma}^{-}(x)$ ya que $x\in S_{u}$. Esto prueba que $\Hm{N-1}(S_{u}\setminus J_{u})=0$ y se tiene por lo tanto la identidad \ref{teo:teorema de Federer-Vol'pert:eq:1}, porque $v_{u}(x)=\overline{v}(x)$ es una orientación de $J_{u}$.

Probamos ahora la segunda identidad. Por el teorema \ref{teo:teorema de trazas en el interior de conjuntos rectificables} deducimos:
\begin{align*}
\abs{Du}\Radot J_{u}=\abs{u^{+}_{\Gamma}-u^{-}_{\Gamma}}\Hm{N-1}\Radot J_{u} =\abs{u^{+}-u^{-}}\Hm{N-1}\Radot J_{u}.
\end{align*}
Por los teoremas \ref{teo:teorema estructural de De Giorgi} y \ref{teo:estrutural de Federer} se obtiene que $\abs{Du}\Radot J_{u}$ tiene como espacio tangente aproximado $v_{u}(x)^{\perp}$ con multiplicidad $\abs{u^{+}(x)-u^{-}(x)}$ para casi todo $x\in J_{u}$ con respecto la medida $\Hm{N-1}$.
\end{proof}

El teorema de Federer-Vol'pert también puede ser usado junto con las proposiciones \ref{prop:Propiedades de los límites aproximados}(\ref{prop:Propiedades de los límites aproximados:b}) y \ref{prop:propiedades del conjunto de puntos de salto aproximado}(\ref{prop:propiedades del conjunto de puntos de salto aproximado:b}) para concluir la convergencia de las funciones $u*\rho_{\epsilon}$ fuera de $S_{u}\setminus J_{u}$, y por lo tanto en casi todas partes en el dominio de $u$ respecto de la medida $\Hm{N-1}$.

\begin{cor}[Convergencia al represente preciso]\label{cor:convergencia al represente preciso}\index{convergencia!al represente preciso}
Sea $u$ una función en $\BV[m]$ y definamos su representante preciso $u^{*}:\DefaultSetWP\setminus (S_{u}\setminus J_{u})\to \R^{m}$ como aquel representante igual a $\tilde{u}$ en $\DefaultSetWP\setminus S_{u}$ e igual a $[u^{+}+u^{-}]/2$ en $J_{u}$. Entonces, las funciones $u*\rho_{\epsilon}$ convergen puntualmente a $u^{*}$ en el dominio. 
\end{cor}

Notesé que en dimensión $1$ el representante preciso es un buen representante debido al teorema \ref{teo:buenos representantes}. Ahora examinamos las propiedades de diferenciabilidad en sentido aproximado de las funciones $BV$. Primero probamos la siguiente cota sobre el cociente de la diferencia. Posteriormente obtendremos la diferenciabilidad aproximada por un argumento de perturbación lineal.
\begin{lema}\label{lema:teorema de Calderón-Sygmund}
Sea $u\in [BV(B_{r}(x))]^{m}$ y supongamos que $u$ tiene límite aproximado en $x$. Entonces:
\begin{align*}
\int_{B_{r}(x)}\dfrac{\abs{u(y)-\tilde{u}(x)}}{\abs{y-x}}dy\leq \int_{0}^{1}\dfrac{\abs{Du}(B_{tr}(x))}{t^{N}}dt.
\end{align*}
\end{lema}
\begin{proof}
%La demostración del lema es sencilla es por ello que damos una idea de la demostración. Por traslación, sin pérdida de generalidad se puede suponer que $x=0$.Suponiendo que $u$ es una función regular por el teorema fundamental del cálculo para obtener la desigualdad $\abs{u(y)-u(\varrho y)}\leq \abs{y}\int_{\varrho}^{1}\abs{\nabla u}(ty)dt$ para cualquier $\varrho\in (0,1)$. Por Fubini se obtiene la desigualdad $ \int_{B_{r}(x)}\abs{u(y)-u(\varrho y)}dy\leq \abs{y-x}\int_{0}^{1}\abs{Du}(B_{tr}(x))t^{-N}dt$, para funciones regulares. Por el mismo argumento que la nota \ref{} la identidad anterior es cierta para cualquier función $u\in BV(B_{r})$. Finalmente, como $0\not \in S_{u}$, considerando una sucesión $(\varrho_{i})\subset (0,1)$ tal que $u(\varrho_{i}y)$ converge a $\tilde{u}(0)$ en casi todo punto, tomando límite $i\to \infty$  y por el lema de Fatou concluimos.
No es restrictivo suponer que $x=0$. Supongamos que $u$ es una función suave, entonces para cualquier $\varrho\in (0,1)$ se puede integrar la desigualdad:
\begin{align*}
\dfrac{\abs{u(y)-u(\varrho y)}}{\abs{y}}\leq \int_{\varrho}^{1}\abs{\nabla{u}}(ty)dt,
\end{align*}
y por Fubini se obtiene:
\begin{align*}
\int_{B_{r}}\dfrac{\abs{u(y)-u(\varrho y)}}{\abs{y}}dy\leq \int_{\varrho}^{1}\int_{B_{r}}\abs{\nabla{u}}(ty)dydt=\int_{\varrho}^{1}t^{-N}\abs{Du}(B_{tr})dt.
\end{align*}
Por un argumento de smoothing basado en el teorema \ref{prop:aproximacion Du}, esta desigualdad sigue siendo cierta para cualquier $u\in BV(B_{r})$. Como $0\not \in S_{u}$, se sigue que:
\begin{align*}
\lim_{\varrho\to 0}\int_{B_{r}}\abs{u(\varrho y)-\tilde{u}(0)}dy=\lim_{\varrho\to 0}\varrho^{-N}\int_{B_{\varrho r}}\abs{u(z)-\tilde{u}(0)}dz=0.
\end{align*}
En particular, podemos encontrar una sucesión infinitesimal $(\varrho_{i})\subset (0,1)$ tal que $u(\varrho_{i}y)$ que converge a $\tilde{u}(0)$ para cada $y\in B_{r}$ con respecto a $\Lm{N}$. Por lo tanto, escogiendo $\varrho=\varrho_{i}$ en la desigualdad anterior y pasando al límite $i\to \infty$, el resultado se sigue del lema de Fatou.
\end{proof}

A continuación probamos el siguiente teorema de diferenciación de las funciones de variación acotado debido a Calderón y Zygmund. En efecto Calderón y Zygmund prueban en \cite{calderon1962differentiability} que toda función de variación acotada $u:\R^{N} \to \mathbb{C}$ es diferenciable en casi todo punto en el sentido aproximado. De hecho prueban que es diferenciable no solo en sentido aproximado sino también en sentido $L^{q}$ con $q=N/(N-1)$. Para ello usan la definición de función de variación acotada dada por Tonelli en \cite{tonelli1936sulle} que es equivalente a la dada en \ref{def:funcion de variacion acotada}. Recordamos que una función $u$ se dice que es diferenciable en sentido $L^{q}$ si la métrica usada para calcular el límite en \ref{defi:punto de diferenciabiliadad aproximada} es la métrica $L^{q}$.
\begin{teo}[Teorema de Calderón-Zygmund]\label{teo:teorema de Calderón-Sygmund}\index{teorema!de Calderón-Sygmund}{}
Cualquier función $u\in \BV[m]$ es diferenciable en sentido aproximado en casi todo punto de $\DefaultSetWP$ con respecto de la medida $\Lm{N}$. Además, la diferencial en sentido aproximado $\nabla u$ es la densidad de la parte absolutamente continua de $Du$ con respecto a $\Lm{N}$.
\end{teo}
\begin{proof}
Sea $Du=D^{a}u+D^{s}u$ la descomposición de Radon-Nikod\'ym de $Du$ en su parte absolutamente continua y su parte singular con respecto a $\Lm{N}$, y sea $v\in [\L1]^{m}$ la densidad de $D^{a}u$ con respecto a $\Lm{N}$. Hay que probar que $u$ es diferenciable en sentido aproximado en cualquier punto $x_{0}\in \DefaultSetWP\setminus (S_{u}\cup S_{v})$ tal que $\abs{D^{s}u}(B_{\varrho}(x_{0}))=o(\varrho^{N})$ cuando $\varrho\to 0$. Por el teorema de derivación de Besicovitch, casi todo punto en $\DefaultSetWP$ verifica esta propiedad. Para probar entonces que para cualquiera de estos puntos $x_{0}$ la función $u$ es aproximadamente diferenciable basta aplicar el lema \ref{lema:teorema de Calderón-Sygmund} a la función:
\begin{align*}
w(x)=u(x)-\tilde{u}(x_{0})-\Crochet{\tilde{v}(x_{0})}{x-x_{0}},
\end{align*}
para obtener:
\begin{align*}
r^{-N}\int_{B_{r}(x_{0})}\dfrac{\abs{u(x)-\tilde{u}(x_{0})-\Crochet{\tilde{v}(x_{0})}{x-x_{0}}}}{\abs{x-x_{0}}}dx\leq \sup_{t\in(0,1)}\dfrac{\abs{Dw}(B_{tr}(x_{0}))}{(tr)^{N}}.
\end{align*}
Como $Dw=[v-\tilde{v}(x_{0})]\Lm{N}+D^{s}u$, debido a la elección de $x_{0}$ se tiene:
\begin{align*}
\abs{Dw}(B_{\varrho}(x_{0}))=\int_{B_{\varrho}(x_{0})}\abs{v(x)-\tilde{v}(x_{0})}dx+\abs{D^{s}u}(B_{\varrho}(x_{0}))=o(\varrho^{N}).
\end{align*}
Haciendo tender $r\to 0$ en la desigualdad anterior se obtiene la diferencial de $u$ en sentido aproximado en el punto $x_{0}$ y la igualdad $\nabla u(x_{0})=\tilde{v}(x_{0})$.
\end{proof}

\section{Descomposición de la derivada y propiedades sobre rango uno}\label{sec:descomposición de la derivada y propiedades sobre rango uno}

En esta sección nos centraremos más detenidamente en el análisis de la derivada distribucional de una función de variación acotada $u$. Al igual que en la sección \ref{sec:Funciones BV en una variable} para funciones de variación acotada en una variable, podemos descomponer la derivada distribucional $Du$ en su parte absolutamente continua \index{parte!absolutamente continua} $D^{a}u$ respecto a $\Lm{N}$ y en su parte singular $D^{s}u$ respecto de $\Lm{N}$. A su vez, podemos descomponer $D^{s}u$ en su parte de salto \index{parte!de salto} $D^{j}u$ y su parte de Cantor $D^{c}u$. Llamaremos parte difusiva a la suma de la parte absolutamente continua y la parte de Cantor y la denotaremos por $\tilde{D}u$. A continuación damos la definición formal de $D^{j}u$ y $D^{c}u$.
\begin{defi}[Parte de salto y de Cantor]\label{defi:parte de salto y de Cantor varias dimensiones}\DefaultSet{\Omega} \index{parte!de salto} \index{parte!de Cantor}
Para cualquier $u\in \BV[m]$ las medidas:
\begin{align*}
\begin{array}{ll}
D^{j}u=D^{s}u\Radot J_{u}, & D^{c}u=D^{s}u\Radot (\DefaultSetWP\setminus S_{u}),
\end{array}
\end{align*}
reciben el nombre de parte de salto y parte de Cantor de la derivada distribucional.
\end{defi}

Recordemos que $Du$ se anula en el conjunto de medida cero $S_{u}\setminus J_{u}$ y por la definición anterior \ref{defi:parte de salto y de Cantor varias dimensiones}, obtenemos la siguiente descomposición de $Du$:
\begin{align}\label{eq:1:descomposición de la derivada distribucional funcion BV}
Du=D^{a}u+D^{s}u=D^{a}u+D^{j}u+D^{c}u=\tilde{D}u+D^{j}u.
\end{align}

Además sabemos por el teorema \ref{teo:teorema de Calderón-Sygmund} que $D^{a}u=\nabla u \Lm{N}$, donde $\nabla u$ es la diferencial en sentido aproximado de $u$. Como $J_{u}$ es un conjunto numerablemente $\Hm{N-1}$-rectificable orientado por la dirección del salto $\nu_{u}$, del teorema \ref{teo:teorema de trazas en el interior de conjuntos rectificables} deducimos que $D^{j}u=Du\Radot J_{u}$ se puede calcular a partir de $\nu_{u}$  y de los límites aproximados $u^{\pm}$:
\begin{align}\label{eq:2:parte de salto de derivada distribucional funcion BV}
\begin{array}{ll}
\displaystyle
D^{j}u(B)=\int_{B\cap J_{u}}(u^{+}(x)-u^{-}(x))\otimes \nu_{u}(x) d\Hm{N-1}(x), & \forall B\in \mathcal{B}(\DefaultSetWP).
\end{array}
\end{align}
En lo que sigue listamos las principales propiedades de las tres componentes de $Du$.
\begin{prop}[Propiedades de $D^{a}u, D^{j}u, D^{c}u$]\label{prop:propiedades de Da u, Dj u, Dc u}\DefaultSet{\Omega}
Sea $u\in \BV[m]$. Entonces se tiene que:
\begin{enumerate}[(a)]
\item $D^{a}u=Du\Radot (\DefaultSetWP\setminus S)$ y $D^{s}u=Du\Radot S$, donde:
\begin{align*}
S=\set*{x\in \DefaultSetWP\given \lim_{\varrho\to 0}\varrho^{-N}\abs{Du}(B_{\varrho}(x))=\infty}.
\end{align*}
Si $E\subset \R^{m}$ es un conjunto de Borel de medida cero entonces $\nabla u$ se anula en $\Lm{N}$-c.t.p. en $u^{-1}(E)$.\label{prop:propiedades de Da u, Dj u, Dc u:a}
\item Sea $\Theta_{u}\subset S$ definido como:
\begin{align*}
\Theta_{u}=\set*{x\in \DefaultSetWP\given \lim_{\varrho\to 0}\inf \varrho^{1-N}\abs{Du}(B_{\varrho}(x))>0}.
\end{align*}
Entonces $J_{u} \subset \Theta_{u}$, $\Hm{N-1}(\Theta_{u}\setminus J_{u})=0$ y por lo tanto $D^{j}u=Du\Radot \Theta_{u}$. En general $D^{j}u=Du\Radot \Sigma$ para cualquier conjunto de Borel $\Sigma$ que contiene a $J_{u}$ y que es $\sigma$-finito con respecto a $\Hm{N-1}$.\label{prop:propiedades de Da u, Dj u, Dc u:b}
\item $D^{c}u=Du\Radot (S\setminus \Theta_{u})$. Además, $D^{c}u$ se anula en conjuntos que son $\sigma$-finitos con respecto a la medida $\Hm{N-1}$ y en conjuntos de la forma $\tilde{u}^{-1}(E)$ con $E\subset \R^{m}$, $\Hm{1}(E)=0.$ \label{prop:propiedades de Da u, Dj u, Dc u:c}
\end{enumerate}
\end{prop}
\begin{proof}
La primera parte de (\ref{prop:propiedades de Da u, Dj u, Dc u:a}) se sigue del teorema de diferenciación de Besicovitch. Respecto de (\ref{prop:propiedades de Da u, Dj u, Dc u:b}) ya se ha visto en el lema \ref{lema:trazas en el interior de conjuntos rectificables} que $J_{u}\subset \Theta_{u}$. Denotando por $L:\Theta_{u} \to (0,\infty]$ el límite inferior de la definición de $\Theta_{u}$, por el teorema \ref{teo:resultados sobre densidades de hausdorff y medidas de radon} sabemos que:
\begin{align}\label{proof:propiedades de Da u, Dj u, Dc u:eq:1}
\abs{Du}\Radot \set*{L\geq 1/p}\geq \dfrac{1}{p}\omega_{N-1}\Hm{N-1}\Radot\set*{L\geq 1/p},
\end{align}
para cualquier entero $p\geq 1$. En particular $\Hm{N-1}(\set*{L\geq 1/p})<\infty$ y por la ecuación \ref{lema:trazas en el interior de conjuntos rectificables:eq:2} se tiene también que $\abs{Du}(\set*{L\geq 1/p}\setminus S_{u})=0$. Por la ecuación \ref{proof:propiedades de Da u, Dj u, Dc u:eq:1} se concluye que $\set*{L\geq 1/p}\setminus S_{u}$ es un conjunto de medida cero respecto $\Hm{N-1}$. Teniendo en cuenta además que $(S_{u}\setminus J_{u})$ es de medida cero respecto de $\Hm{N-1}$ se obtiene que $\Hm{N-1}(\Theta_{u}\setminus J_{u})=0$ haciendo tender $p\to \infty$. Si $\Sigma\subset \Omega$ es cualquier conjunto de Borel de medida cero respecto de $\Lm{N}$ y que además contiene $J_{u}$ se ve entonces:
\begin{align*}
Du\Radot \Sigma &= Du\Radot J_{u}+Du \Radot (\Sigma\setminus J_{u})=D^{s}u\Radot J_{u}+Du\Radot (\Sigma\setminus S_{u})
\\&+Du\Radot (\Sigma \cap S_{u}\setminus J_{u})=D^{j}u+Du\Radot (\Sigma\setminus S_{u}),
\end{align*}
ya que $\Hm{N-1}(S_{u}\setminus J_{u})=0$. Si $\Sigma$ es $\sigma$-finito con respecto a la medida $\Hm{N-1}$ podemos concluir por \ref{lema:trazas en el interior de conjuntos rectificables:eq:2} que $Du\Radot \Sigma=D^{j}u$.

Finalmente probamos (\ref{prop:propiedades de Da u, Dj u, Dc u:c}). El resultado de representación de $D^{c}u$ se sigue directamente de la ecuación \ref{eq:1:descomposición de la derivada distribucional funcion BV} y de las fórmulas de representación de $D^{a}u$ y $D^{j}u$. Sea $B\subset \Omega$ un conjunto de Borel $\sigma$-finito con respecto a la medida $\Hm{N-1}$. De las ecuaciones \ref{lema:trazas en el interior de conjuntos rectificables:eq:1} y \ref{lema:trazas en el interior de conjuntos rectificables:eq:2} se obtiene que $\abs{Du}$ se anula en $B\setminus J_{u}$ y como $\tilde{D}u\Radot J_{u}=0$ obtenemos que $\tilde{D}u(B)=0$. Si $m=1$ y $B=\tilde{u}^{-1}(E)\subset \Omega\setminus S_{u}$, entonces la proposición $3.65$ de \cite{ambrosio2000functions} muestra que $\partial^{*}\set*{u>t}$ y $B$ son disjuntos si $t\not \in E$. Si $\Lm{1}(E)=0$ la fórmula de la coárea \ref{eq:1:formula de la coarea generalizada} nos proporciona:
\begin{align*}
\abs{Du}(B)=\int_{E}\Hm{N-1}(\partial^{*}(u>t)\cap B) dt=0.
\end{align*}
En el caso general $m>1$, denotando por:
\begin{align*}
\begin{array}{ll}
E_{\alpha}=\set*{t\in \R \given t=z_{\alpha}\, \text{para algún}\, z\in E}, & \alpha=1,\ldots,m,
\end{array}
\end{align*}
y observando que $\Lm{1}(E_{\alpha})\leq \Hm{1}(E)=0$, podemos usar \ref{eq:1:desigualdades de normas bv} para acotar $\abs{Du}(\tilde{u}^{-1}(E))$ por:
\begin{align*}
\sum^{m}_{\alpha=1}\abs{Du^{\alpha}}(\tilde{u}^{-1}(E))\leq \sum^{m}_{\alpha=1}\abs{Du^{\alpha}}(\tilde{u^{\alpha}}^{-1}(E_{\alpha}))=0.
\end{align*}
En particular, $\nabla{u}$ se anula en casi todas partes de $\tilde{u}^{-1}(E)$ con respecto de la medida $\Lm{N}$ y esto prueba la segunda parte de (\ref{prop:propiedades de Da u, Dj u, Dc u:a}) ya que $\tilde{u}$ y $u$ coinciden en casi todas partes en $\Omega$.
\end{proof}

\section{Regla de la cadena en BV}\label{sec:regla de la cadena en BV}
Finalmente, para terminar el capítulo veremos en esta sección como se puede generalizar el teorema de la cadena para funciones de variación acotada. Dado un conjunto abierto $\DefaultSetWP \subset \R^{N}$, $u\in \BV[m]$ y una función Lipschitz $f:\R^{m}\to \R^{p}$ se puede probar sin dificultad que $v=f\circ u$ pertenece a $\BV[p]$ y que además $\abs{Dv}\ll\abs{Du}$. Por lo tanto, un problema natural es buscar una especie de regla de la cadena para funciones $BV$ que relacione $Dv, Du$ y la derivada de $f$. Sin embargo, la parte difusiva y la parte de salto de la derivada tienen un comportamiento muy diferente como veremos. Suponiendo $m=p=1$, probaremos que $\tilde{D}v=f'(\tilde{u})\tilde{D}u$. Pero para la parte de salto se tendrá: 
\begin{align*}
D^{j}v=\dfrac{f(u^{+})-f(u^{-})}{u^{+}-u^{-}}D^{j}u.
\end{align*}
En el siguiente teorema supondremos que $f$ es continuamente diferenciable. Esta suposición será luego descarta en el teorema \ref{teo:Regla de la cadena en BV version 2}.

\begin{teo}[Regla de la cadena en $BV$]\label{teo:Regla de la cadena en BV version 1}\DefaultSet{\Omega}\index{regla de la cadena!en $BV$}
Sea $u\in \BV[m]$ y $f\in [C^{1}(\R^{m})]^{p}$ una función Lipschitz que satisface $f(0)=0$ si $\abs{\DefaultSetWP}=\infty$. Entonces, $v=f\circ u$ pertenece a $\BV[p]$ y se tiene además que:
\begin{align}\label{teo:Regla de la cadena en BV version 1:eq:1}
\left \{ \begin{array}{ll}
\displaystyle \tilde{D}v=\nabla f(u) \nabla u\, \Lm{N}+\nabla f(\tilde{u})D^{c}u=\nabla f(\tilde{u})\tilde{D}u,\\
\displaystyle D^{j}v=\left (f(u^{+})-f(u^{-})\right )\otimes \nu_{u}\Hm{N-1}\Radot J_{u}.
\end{array}
\right .
\end{align}
\end{teo}
\begin{proof}\DefaultSet{\Omega}
Probamos primero que $v\in \BV[p]$ y $\abs{Dv}\leq M\abs{Du}$, donde $M=\sup_{z}\abs{\nabla f(z)}_{\infty}$ es la constante de Lipschitz de $f$. Para ello, dado un conjunto abierto $A\subset \DefaultSetWP$ podemos aplicar el teorema \ref{teo:aproximación por funciones suaves} a $u$ para obtener una sucesión $(u_{h})\subset [C^{\infty}(A)]^{m}$ convergente en $[L^{1}(A)]^{m}$ a $u$ tal que $(\abs{Du_{h}}(A))$ converge a $\abs{Du}(A)$. Entonces, como $f$ es una función Lipschitz, $v_{h}=f(u_{h})$ converge en $[L^{1}(A)]^{p}$ a $v$ y:
\begin{align*}
\abs{Dv_{h}}(A)=\int_{A}\abs{\nabla v_{h}}dx=\int_{A}\abs{\nabla{f(u_{h})}\nabla u_{h}}dx\leq M\int_{A}\abs{\nabla{u_{h}}}dx=M\abs{Du_{h}}(A).
\end{align*}
Tomando límite cuando $h\to \infty$ y por la semicontinuidad inferior de la variación se concluye que $\abs{Dv}(A)\leq M \abs{Du}(A)$ y por lo tanto por regularidad exterior de la medida $\abs{Dv}\ll M\abs{Du}$.

Probamos ahora la fórmula de representación de $\tilde{D}v$. Para ello debemos probar la relación:
\begin{align*}
\lim_{\varrho\to 0}\dfrac{Dv(B_{\varrho}(x_{0}))}{\abs{Du}(B_{\varrho}(x_{0}))}=\nabla f(\tilde{u}(x_{0}))g(x_{0}),
\end{align*}
para casi todo punto $x_{0}\in \Omega\setminus S_{u}$ respecto a la medida $\abs{Du}$ y donde $Du=g\abs{Du}$ es una descomposición polar de $Du$. Gracias al corolario \ref{teo:espacio medidas tanges contines medidas no cero}, a la proposición \ref{prop:propiedades de Da u, Dj u, Dc u} y al teorema de diferenciación de Besicovitch se observa que dado $g$, casi todo punto $x_{0}\in \Omega\setminus S_{u}$ con respecto a la medida $\abs{Du}$ verifica las siguientes propiedades:
\begin{enumerate}[(a)]
\item $x_{0}$ es un punto de Lebesgue de $g$, relativo a $\abs{Du}$, $Tan(\abs{Du},x_{0})$ contiene una medida no cero y $\abs{Du}(B_{\varrho}(x_{0}))=o(\varrho^{N-1})$,\label{proof:Regla de la cadena en BV version 1:a}
\item $Dv(B_{\varrho}(x_{0}))/\abs{Du}(B_{\varrho}(x_{0}))$ converge a un cierto límite $\lambda(x_{0})$ cuando $\varrho\to 0$.\label{proof:Regla de la cadena en BV version 1:b}
\end{enumerate}
Probamos ahora que $\lambda(x_{0})=\nabla{f(\tilde{u}(x_{0}))}g(x_{0})$. En efecto, sea $c_{\varrho}=\abs{Du}(B_{\varrho}(x_{0}))$, $m_{\varrho}$ el valor medio de $u$ en $B_{\varrho}(x_{0})$ y:
\begin{align*}
\begin{array}{ll}
u^{\varrho}=\dfrac{u(x_{0}+\varrho y)-m_{\varrho}}{c_{\varrho}/\varrho^{N-1}},
\end{array} & v^{\varrho}(y)=\dfrac{v(x_{0}+\varrho y)-f(m_{\varrho})}{c_{\varrho}/\varrho^{N-1}}.
\end{align*}
Consideramos la siguiente normalización, $\abs{Du^{\varrho}}(B_1)=\abs{Du}(B_{\varrho}(x_{0}))/c_{\varrho}=1$. Denotamos por $I^{\varrho}(x)$ la aplicación de rescalado $(x-x_{0})/\varrho$, por la condición (\ref{proof:Regla de la cadena en BV version 1:a}) podemos encontrar una sucesión infinitesimal $(\varrho_{i})\subset (0,\infty)$ tal que $\abs{Du^{\varrho_{i}}}=I_{\#}^{\varrho_{i}}(\abs{Du})/c_{\varrho_{\varrho_{i}}}$ converge en sentido débil$^{*}$ en $B_1$ a una medida no-cero $\nu$ cuando $i\to \infty$. Como $x_{0}$ es un punto de Lebesgue de $g$, el teorema \ref{teo: expresion del espacio de medidas tangentesen funciones del espacio de medidas tangentes de meddias radon positivas} implica que $I_{\#}^{\varrho_{i}}(Du)/c_{\varrho_{\varrho_{i}}}$ converge en sentido débil$^{*}$ en $B_1$ a $g(x_0)\nu$ cuando $i\to \infty$. Por la desigualdad de Poincaré \ref{eq:1:lema cota integral de la diferencia de la funcion menos el valor medio aplicado a bolas} se deduce entonces:
\begin{align*}
\int_{B_1}\abs{u^{\varrho}(y)}dy=\dfrac{1}{\varrho c_{\varrho}}\int_{B_{\varrho}(x_0)}\abs{u(x)-m_{\varrho}}dx\leq \gamma_{1}.
\end{align*}
Como consecuencia, por el teorema de compacidad 	\ref{teo:compacidad en BV}, podemos suponer que $(u^{\varrho_{i}})$ converge en $[L^{1}(B_1)]^{m}$ y en casi todo punto respecto de $\Lm{N}$ a una función $u^{0}$ en $B_1$. Por rescalado se ve que $Du^{\varrho_{i}}=I_{\#}^{\varrho_{i}}(Du)/c_{\varrho_{i}}$ y por lo tanto por continuidad de la derivada distribucional se tiene que $Du^{0}=g(x_{0})\nu$. 

Ahora examinamos el comportamiento de $v^{\varrho_{i}}$. Como $\abs{Dv^{\varrho_i}}\leq M\abs{Du^{\varrho_i}}$, extrayendo otra subsucesión podemos suponer que $\abs{Dv^{\varrho_i}}$ converge en sentido débil$^{*}$ a una medida $\sigma\leq M \nu$ en $B_1$ cuando $i\to \infty$. Aplicando el teorema del valor medio a cada componente de $f^{\alpha}$ de $f$ se tiene que:
\begin{align*}
\abs*{\dfrac{f(z)-f(m_{\varrho})}{c_{\varrho}/\varrho^{N-1}}-\nabla{f(m_{\varrho})}\dfrac{z-m_{\varrho}}{c_{\varrho}/\varrho^{N-1}}}\leq m \,\omega(\abs{z-m_{\varrho}})\dfrac{\abs{z-m_{\varrho}}}{c_{\varrho}/\varrho^{N-1}},
\end{align*} 
donde $\omega$ es un módulo de continuidad de todas las funciones $\nabla f^{\alpha}$. Por lo tanto:
\begin{align*}
\abs*{v^{\varrho}(y)-\nabla{f(m_{\varrho})u^{\varrho}(y)}}\leq m\, \omega( \abs{u^{\varrho}(y)}c_{\varrho}/\varrho^{N-1} ) \abs{u^{\varrho}(y)},
\end{align*}
y la convergencia en casi todo punto de $u^{\varrho_{i}}$ a $u^{0}$ respecto de $\Lm{N}$ y la convergencia de $m_{\varrho}$ a $\tilde{u}(x_{0})$ cuando $\varrho\to 0$ implican que $v^{\varrho_{i}}$ converge en casi todo punto respecto de $\Lm{N}$ en $B_1$ a:
\begin{align}\label{proof:Regla de la cadena en BV version 1}
v^{0}=\nabla{f(\tilde{u}(x_{0}))}u^{0}.
\end{align}
Por otro lado, como $\abs*{v^{\varrho_{i}}}\leq M\abs*{u^{\varrho_{i}}}$ las funciones son también equiintegrables, por lo tanto por el teorema de Vitali, $(v^{\varrho_i})$ converge a $v^{0}$ en $[L^{1}(B_1)]^{m}$. En particular $(Dv^{\varrho_{i}})$ converge en sentido débil$^{*}$ a $Dv^{0}$ en $B_1$.

Eligiendo un $t\in (0,1)$ tal que $\nu(B_{t})>0$ y $\nu(\partial B_{t})=0$ por las propiedades de continuidad de las medidas de $B_{t}$ bajo convergencia débil$^{*}$ y por \ref{proof:Regla de la cadena en BV version 1} se obtiene:
\begin{align*}
\lim_{i\to \infty}\dfrac{Dv(B_{t\varrho_{i}}(x_0))}{\abs{Du}(B_{t\varrho_{i}}(x_0))}&=\lim_{i\to \infty}\dfrac{Dv(B_{t\varrho_{i}}(x_0))}{\abs{Du}(B_{\varrho_{i}}(x_0))}\dfrac{\abs{Du}(B_{\varrho_{i}}(x_0))}{\abs{Du}(B_{t\varrho_{i}}(x_0))}=\lim_{i\to \infty}\dfrac{Dv^{\varrho_{i}}(B_{t})}{\nu(B_t)}
\\&=\dfrac{Dv^{0}(B_{t})}{\nu(B_{t})}=\nabla{f(\tilde{u}(x_{0}))}\dfrac{Du^{0}(B_{t})}{\nu(B_{t})}=\nabla{f(\tilde{u}(x_{0}))}g(x_{0}).
\end{align*}
Por la condición (\ref{proof:Regla de la cadena en BV version 1:b}) concluimos que $\lambda(x_{0})=\nabla{f(\tilde{u}(x_{0}))}g(x_{0})$. Por lo tanto:
\begin{align*}
Dv\Radot (\DefaultSetWP\setminus S_{u})=\nabla{f(\tilde{u})}g\abs{Du}\Radot (\DefaultSetWP\setminus S_{u})=\nabla{f(\tilde{u})}Du\Radot (\DefaultSetWP\setminus S_{u})=\nabla{f(\tilde{u})}\tilde{D}u.
\end{align*}
Teniendo en cuenta que $D^{j}v\Radot (\DefaultSetWP\setminus S_{u})=0$ (ya que por la proposición \ref{prop:Propiedades de los límites aproximados}, $S_{v}$ es un subconjunto de $S_{u}$) se obtiene que $\tilde{D}v\Radot (\DefaultSetWP\setminus S_{u})=\nabla{f(\tilde{u})}\tilde{D}u$. Finalmente, observando que $\tilde{D}v$ se anula en los subconjuntos de Borel de $S_{u}$ concluimos la primera identidad de \ref{teo:Regla de la cadena en BV version 1:eq:1}.

La segunda identidad de \ref{teo:Regla de la cadena en BV version 1:eq:1} es verdad incluso si $f$ no es continuamente diferenciable. Se prueba observando que $J_{v}\subset S_{v}$ es un subconjunto de $S_{u}$ y por la proposición \ref{prop:propiedades del conjunto de puntos de salto aproximado}:
\begin{align*}
\begin{array}{ll}
(v^{+}(x), v^{-}(x), \nu_{v}(x))\sim (f(u^{+}(x)),f(u^{-}(x)), \nu_{u}(x)), & \forall x\in J_{u}\cap J_{v},
\end{array}
\end{align*}
mientras que $f(u^{+})=f(u^{-})$ en $J_{u}\setminus J_{v}$. Por el hecho que $(S_{u}\setminus J_{u})$ es un conjunto de medida cero respecto de $\Hm{N-1}$ se ve que:
\begin{align*}
D^{j}v&=Dv\Radot J_{v} =Dv\Radot (J_{v}\cap J_{u})+Dv\Radot (J_{v}\cap (S_{u}\setminus J_{u}))
\\&=(f(u^{+})-f(u^{-}))\otimes \nu_{u}\Hm{N-1}\Radot (J_{v}\cap J_{u})
\\&=(f(u^{+})-f(u^{-}))\otimes \nu_{u}\Hm{N-1}\Radot J_{u}.
\end{align*}
\end{proof}

Si suponemos que $f$ es una función Lipschitz, y que si $\abs{\DefaultSetWP}=\infty$ $f(0)=0$ entonces gracias al teorema que acabamos de probar sabemos que $v=f\circ u$ pertenece a $\BV[p]$. También acabamos de ver  que estas suposiciones son suficientes como para obtener la representación de la parte de salto de la derivada distribucional de $v$. Sin embargo si dejamos de lado la suposición de que $f$ es diferenciable en todo $\R^{m}$ es difícil dar siquiera una interpretación a la primera identidad de \ref{teo:Regla de la cadena en BV version 1:eq:1}, pues el rango de $u$ puede estar contenido en regiones donde $f$ no es ni siquiera diferenciable. Entonces de lo único de lo que disponemos es del teorema de Rademacher que afirma que una función Lipschitz es diferenciable en casi todas partes de $\R^{m}$ con respecto a la medida $\Lm{m}$. Como consecuencia, si $m=1$ entonces por la proposición \ref{prop:propiedades de Da u, Dj u, Dc u}(\ref{prop:propiedades de Da u, Dj u, Dc u:a}) vemos que $\nabla u =0$ en casi todas partes respecto de la medida $\Lm{N}$ en el conjunto de puntos donde $f'(u)$ no está definida. Por tanto, considerando $w=f'(u)\nabla u$, se puede ver que se trata de una aplicación bien definida si consideramos $w(x)=0$ cuando $f$ no es diferenciable en $u(x)$. Análogamente gracias a la proposición \ref{prop:propiedades de Da u, Dj u, Dc u}(\ref{prop:propiedades de Da u, Dj u, Dc u:c}) vemos que $f'(u)D^{c}u$ es una medida bien definida pues $f'(\tilde{u})$ no está definida en conjuntos de medida cero respecto de $\abs{D^{c}u}$. Por esto y siguiendo un argumento de smoothing se puede probar la siguiente regla de la cadena para funciones $BV$ a valores reales.
\begin{teo}\label{teo:Regla de la cadena en BV version 2}
Sea $u\in \BV$ y $f: \R\to \R$ una función Lipschitz que satisface además que $f(0)=0$ si $\abs{\DefaultSetWP}=\infty$. Entonces $v=f\circ u$ pertenece a $\BV$ y se tiene:
\begin{align*}
Dv=f'(u)\nabla u\Lm{N}+(f(u^{+})-f(u^{-}))\nu_{u}\Hm{N-1}\Radot J_{u}+f'(\tilde{u})D^{c}u. 
\end{align*}
\end{teo}
\begin{proof}\DefaultSet{\Omega}
Como la afirmación es local podemos suponer sin pérdida de generalidad que $\abs{\Omega}<\infty$, por lo tanto, las funciones constantes pertenecen a $\BV$. Sea la convolución $f_{\epsilon}=f*\rho_{\epsilon}$ y $v_{\epsilon}=f_{\epsilon}\circ u$. Por el teorema \ref{teo:Regla de la cadena en BV version 1} se deduce:
\begin{align}\label{proof:Regla de la cadena en BV version 2}
Dv_{\epsilon}=f'_{\epsilon}(u)\nabla{u}\Lm{N}+\left ( f(u^{+})-f(u^{-}))\right ) \nu_{u}\Hm{N-1}\Radot J_{u} + f'_{\epsilon}(\tilde{u})D^{c}u,
\end{align}
para cualquier $\epsilon>0$. Como $\abs{Dv_{\epsilon}}(\DefaultSetWP)$ son equiacotados y además como $(v_{\epsilon})$ converge uniformemente a $v$ en $\DefaultSetWP$ cuando $\epsilon\to 0$ se obtiene por la proposición \ref{prop:criterio convergencia débi} que $v\in \BV[p]$ y que las medidas $(Dv_{\epsilon})$ convergen en sentido débil$^{*}$ a $Dv$. Por otro lado, $f'_{\epsilon}(t)=f'*\rho_{\epsilon}(t)$ converge a $f'$ cuando $\epsilon\to 0$ en cualquier punto de Lebesgue de $f'$. Denotando por $F$ el conjunto de puntos de Lebesgue de $f'$, vemos que $\nabla{u}$ se anula en casi todo punto de $u^{-1}(\R\setminus F)$ respecto de la medida $\Lm{N}$. Por lo tanto, $f'_{\epsilon}(u)\nabla{u}$ converge a $f'(u)\nabla{u}$ en casi todo $\DefaultSetWP$ respecto de $\Lm{N}$. Análogamente, $f'_{\epsilon}(\tilde{u})$ converge a $f'(\tilde{u})$ en casi todas partes en $\DefaultSetWP$ respecto de $\abs{D^{c}u}$ pues $\abs{D^{c}}(\tilde{u}^{-1}(\R\setminus F))=0$. Concluimos tomando límite $\epsilon\to 0$ en \ref{proof:Regla de la cadena en BV version 2}  y por el teorema de convergencia dominada.
\end{proof}

Para cerrar esta sección conviene decir que existe una versión más general de la regla de la cadena para funciones de variación acotada probada por L. Ambrosio y G. Dal Maso en \cite{ambrosio1990general}. En esta versión $u$ pertenece al espacio $\BV[m]$ y $f$ es una función Lipschitz de $\R^{m}\to \R^{p}$. La demostración de este teorema se basa en la restricción de las funciones $BV$ al caso unidimensional y posteriormente usando el hecho de que la densidad de $Dv$ con respecto a $\abs{Du}$ es dada por el límite del ratio $Dv([t, t+\varrho))/\abs{Du}([t, t+\varrho))$ cuando $\varrho\to 0$. Sin embargo, no estudiaremos esta versión más general de la regla de la cadena ya que no se necesitará en el desarrollo de los próximos capítulos del trabajo.

\chapter{Funciones especiales de variación acotada}\label{cap:cap3}
En este capítulo estudiaremos el espacio $SBV$ de funciones especiales de variación acotada. Este espacio fue introducido por E. De Giorgi y L. Ambrosio en \cite{de1988nuovo} para estudiar problemas variacionales donde tanto el volumen como la energía superficial están involucrados. El objetivo del capítulo es definir el espacio $SBV$ y estudiar posteriormente sus propiedades de cierre y compacidad que serán de gran importancia a la hora de estudiar los problemas variacionales formulados en él. 

En la primera sección de este capítulo definiremos el espacio de funciones de variación acotada especiales. Después en la segunda sección probaremos los teorema de cierre y compacidad del espacio. Finalmente en la tercera y última sección probaremos una desigualdad tipo Poincarré para funciones especiales de variación acotada. Para ser más precisos se tiene la siguiente definición.

\section{El espacio $SBV$}
En esta sección introducimos el espacio de funciones de variación acotada especiales. Una función de variación acotada especial es una función $BV$ cuya derivada es la suma de una medida absolutamente continua con respecto a $\Lm{N}$ y de una medida concentrada en el conjunto de salto.

\begin{defi}[Función de variación acotada especial] \DefaultSet{\Omega} \index{función!de variación acotada especial} \index{espacio!$SBV$}
Se dice que $u\in \BV$ es una función de variación acotada especial en $\Omega$ y se escribe $u\in\SBV$, si la parte de Cantor de la derivada distribucional $\Dc{u}$ es cero.
\end{defi}
Por las ecuaciones \ref{eq:1:descomposición de la derivada distribucional funcion BV} y \ref{eq:2:parte de salto de derivada distribucional funcion BV} vemos que:
\begin{align}\label{eq:1:derivada distribucional SBV}
\begin{array}{ll}
\displaystyle
Du=\Da{u}+\Dj{u}=\nabla{u}\Lm{N}+(u^{+}-u^{-})\nu_{u}\Hm{N-1}\Radot J_{u},& \forall u\in \SBV.
\end{array}
\end{align}

Como por definición $\Dc{u}=D^{s}u\Radot (\Omega\setminus S_{u})$ se puede decir como apuntábamos al principio de la sección que una función $u$ pertenece a $\SBV$ si y solo si $\Dc{u}$ está concentrada en $S_{u}$. 

Observamos que el espacio de $\SBV$ es un subespacio propio de $\BV$. En efecto si $\Omega\subset \R$, la función de Cantor-Vitali \index{función!de Cantor-Vitali} es un ejemplo de función que pertenece a $BV((0,1))\setminus SBV((0,1))$, pues la parte absolutamente continua de su derivada es cero, el conjunto de discontinuidad es vacío pero sin embargo la derivada distribucional es una medida no nula cuyo soporte es el conjunto intermedio del conjunto de Cantor ternario (véase \cite{ambrosio2012calculus}). Considerando funciones como la de Cantor-Vitali que dependen únicamente de una variable se observa que $\SBV\subsetneq \BV$ para cualquier conjunto abierto $\Omega\subset \R^{N}$. También se puede ver fácilmente que $\W{1}{1}$ está contenido en $\SBV$ y por la ecuación \ref{eq:1:derivada distribucional SBV} y el teorema de Federer-Vol'pert \ref{teo:teorema de Federer-Vol'pert}, dado $u\in \SBV$ se tiene:
\begin{align}\label{eq:1:relacion espacio de sobolev vs sbv}
\begin{array}{l}
u\in \W{1}{1} \iff \Hm{N-1}(S_{u})=0.
\end{array}
\end{align}
Para ver que la inclusión es estricta basta ver que la función $u=\chi_{E}$ con $\abs{E}<\infty$ y con perímetro que verifica $0<P(E,\Omega)<\infty$ pertenece al espacio $SBV$ ya que $Du=\nu_{E}\Hm{N-1}\Radot \mathcal{F}E$, pero como sabemos $u$ no es una función del espacio de Sobolev $\W{1}{1}$.

Como por definición $D^{c}{u}=D^{s}{u}\Radot (\Omega\setminus S_{u})$, se puede decir que $u$ pertenece a $\SBV$ si y solo si $D^{s}u$ se concentra en $S_{u}$. Por lo general, se tiene el siguiente resultado.

\begin{prop}\label{prop:u es SBV si la parte singular se concentra en un  conjunto sigma-finito}
Cualquier $u\in \BV$ pertenece a $\SBV$ si y solo si $D^{s}u$ se concentra en un  conjunto $\sigma$-finito con respecto a $\Hm{N-1}$.
\end{prop}
\begin{proof}
Si $u\in \SBV$ entonces $D^{s}u=D^{j}u$ se concentra, por definición, en $J_{u}$. Por el teorema de Federer-Vol'pert, $J_{u}\subset S_{u}$ es numerablemente $\Hm{N-1}$-rectificable, luego $\sigma$-finito respecto de la medida $\Hm{N-1}$.

Para probar la implicación opuesta, recordamos que por la proposición \ref{prop:propiedades de Da u, Dj u, Dc u}(\ref{prop:propiedades de Da u, Dj u, Dc u:c}), la parte de Cantor de la derivada distribucional se anula en cualquier conjunto de Borel $\sigma$-finito con respecto $\Hm{N-1}$. Por lo tanto, si $D^{s}u$ se concentra en un conjunto de estas características entonces la parte de Cantor de la derivada distribucional debe ser cero.
\end{proof}

\begin{cor}
$\SBV$ es un subespacio cerrado de $\BV$.
\end{cor}
\begin{proof}
Si $I$ es finito o numerable, $u_{i}\in \SBV$ para cualquier $i\in I$ y $\sum_{i\in I}u_{i}$ converge a $u \in \BV$ en norma $BV$, entonces $Du=\sum_{i}Du_{i}$. Como $\sum_{i}D^{a}u_{i}$ es absolutamente continua con respecto a $\Lm{N}$ y $\sum_{i}D^{s}u_{i}$ es singular, se tiene:
\begin{align*}
\begin{array}{ll}
\displaystyle
D^{a}u=\sum_{i\in I}D^{a}u_{i},& \displaystyle D^{s}u=\sum_{i\in I}D^{s}u_{i},
\end{array}
\end{align*}
y puesto que $D^{s}u$ se concentra en $\bigcup_{i}S_{u_{i}}$, concluimos que $u\in \SBV$. Esto prueba que $\SBV$ es un subespacio cerrado de $\BV$.
\end{proof}

Se ha visto que $\W{1}{1}\subset \SBV \subset \BV$ y que ambas inclusiones son estrictas. En efecto, el espacio $\SBV$ contiene funciones ``Sobolev a trozos'' en un sentido débil como podemos ver en la siguiente proposición.
\begin{prop}\label{prop:funciones sobolev a trozos en SBV}
Sea $\Omega\subset \R^{N}$ un conjunto abierto, $K\subset\R^{N}$ un conjunto cerrado y supongamos que $\Hm{N-1}(K\cap\Omega)<\infty$. Entonces, cualquier función $u: \Omega \to \R$ que pertenece a $L^{\infty}(\Omega\setminus K)\cap W^{1,1}(\Omega\setminus K)$ pertenece también a $\SBV$ y satisface además $\Hm{N-1}(S_{u}\setminus K)=0$. 
\end{prop}
\begin{proof}
Para cada $h\in \N$, $h\geq 1$, existe una cantidad finita numerable de bolas $B_{i,h}$, $i=1, \ldots, n_{h},$ con radios $r_{i,h}<1/h$, tales que:
\begin{align*}
\begin{array}{ll}
\displaystyle
\Omega_{h}=\bigcup_{i=1}^{n_{h}}B_{i,h}\supset K\cap \Omega, &\displaystyle \sum_{i=1}^{n_{h}}\omega_{N-1}r_{i,h}^{N-1}\leq 2^{N-1}\left[ \Hm{N-1}(K\cap \Omega)+1\right ].
\end{array}
\end{align*}
Por subaditividad del perímetro se deduce que $P(\Omega_{h}, \Omega)\leq (N\omega_{N}2^{N-1}/\omega_{N-1})[\Hm{N-1}(K\cap \Omega)+1]$. Luego considerando:
\begin{align*}
u_{h}(x)=\left \{
\begin{array}{ll}
u(x),&\text{si}\; x\in \Omega\setminus \Omega_{h},\\
0, & \text{si}\; x\in \Omega\cap \Omega_{h},
\end{array} \right .
\end{align*}
y por el teorema $3.84$ de \cite{ambrosio2000functions} con $E=\Omega_{h}$ se obtiene que $u_{h}\in \BV$ y además que:
\begin{align*}
\abs{Du_{h}}(\Omega)\leq \int_{\Omega\setminus K}\abs{\nabla{u}}\, dx+\dfrac{N2^{N-1}\omega_{N}}{\omega_{N-1}}\Norm{u}[\infty]\left [ \Hm{N-1}(K\cap \Omega)+1\right ].
\end{align*}
La cota uniforme en $\sum_{i}r^{N-1}_{i,h}$ implica facilmente que $\abs{\Omega_{h}}\to 0$ cuando $h\to \infty$, luego $(u_{h})$ converge a $u$ en $L^{1}(\Omega)$, la semicontinuidad inferior de la variación implica que $u\in \BV$ y que:
\begin{align*}
\abs{Du}(\Omega)\leq \int_{\Omega\setminus K}\abs{\nabla{u}}\, dx+\dfrac{N2^{N-1}\omega_{N}}{\omega_{N}}\Norm{u}[\infty]\left [ \Hm{N-1}(K\cap \Omega)+1\right ].
\end{align*}
Finalmente, la parte singular de $Du$ tiene soporte en $K$ puesto que $u\in W^{1,1}(\Omega\setminus K)$, por lo tanto la proposición \ref{prop:u es SBV si la parte singular se concentra en un  conjunto sigma-finito} implica que $u\in \SBV$. Por la ecuación \ref{eq:1:relacion espacio de sobolev vs sbv} se obtiene que la intersección de $S_{u}$ con $\Omega\setminus K$ es despreciable respecto de la medida $\Hm{N-1}$.
\end{proof}

\section{Teoremas de cierre y compacidad en $SBV$}
En esta sección enunciaremos y probaremos los dos resultados fundamentales del capítulo, un resultado de cierre y otro de compacidad del espacio $SBV$. Para tener compacidad de sucesiones minimizadoras en el espacio $SBV$ es esencial conocer la topología para la cual el límite $u$ de una sucesión $(u_{h})\subset \SBV$ sigue perteneciendo al espacio $\SBV$ posiblemente bajo condiciones adicionales sobre la sucesión. Como el teorema \ref{teo:compacidad en BV} proporciona un resultado de compacidad con respecto a la topología débil$^{*}$, estamos interesados en buscar condiciones adicionales sobre la sucesión minimizadora de modo que su límite en el sentido débil$^{*}$ pertenezca al espacio $SBV$, pues por el teorema \ref{teo:aproximación por funciones suaves}, el cierre débil$^{*}$ del espacio $\W{1}{1}$ es todo $\BV$. Este hecho sugiere la necesidad de considerar cotas sobre las derivadas $\Da{u_{h}}$ y $\Dj{u_{h}}$. La suposición natural es que $\abs{\nabla u_{h}}$ sea equiintegrable, esto fuerza según el teorema de Dunford-Pettis a que cualquier límite de $\Da{u_{h}}$ y de $\abs{\Da{u_{h}}}$ sea absolutamente continuo respecto de $\Lm{N}$. Por la proposición \ref{teo:condiciones de equiitegrabilidad} esta suposición es equivalente a que $\int_{\Omega}\varphi(\abs{\nabla{u_{h}}})dx$ sea uniformemente acotado con respecto a $h$ para alguna función $\varphi(t)$ con crecimiento mayor que lineal cuando $t\to \infty$. La suposición sobre $\Dj{u_{h}}$ es análoga: $\int_{J_{u_{h}}}\theta(\abs{u^{+}_{h}-u^{-}_{h}})d\Hm{N-1}$ tiene que ser uniformemente acotado con respecto a $h$ para alguna función $\theta(t)$ que satisface $\theta(t)/t\to \infty$ cuando $t\to 0$.

A continuación enunciamos los dos resultados fundamentales del capítulo y luego procedemos a la demostración.

\begin{teo}[Cierre de $SBV$]\label{teo:cierre de SBV} \index{cierre en $SBV$}
Sean $\varphi:[0,\infty)\to [0,\infty]$ y $\theta:(0,\infty)\to (0,\infty]$ funciones crecientes semicontinuas inferiormente y supongamos además:
\begin{align}\label{teo:cierre de SBV:eq:1}
\begin{array}{ll}
\lim_{t\to \infty}\dfrac{\varphi(t)}{t}=\infty, &  \lim_{t\to 0}\dfrac{\theta(t)}{t}=\infty.
\end{array}
\end{align}
Sea $\Omega\subset \R^{N}$, un conjunto abierto y acotado, y sea $(u_{h})\subset \SBV$ un sucesión tal que:
\begin{align}\label{teo:cierre de SBV:eq:2}
\sup_{h}\set*{\int_{\Omega}\varphi(\abs{\nabla{u_{h}}})dx+\int_{J_{u_{h}}}\theta(\abs{u_{h}^{+}-u_{h}^{-}})d\Hm{N-1}}<\infty.
\end{align}
Si $(u_{h})$ converge a $u$ en sentido débil$^{*}$ en $\BV$, entonces $u\in \SBV$, el gradiente aproximado $\nabla{u_{h}}$ converge en sentido débil a $\nabla{u}$ en $[\L1]^{N}$, $\Dj{u_{h}}$ converge en sentido débil$^{*}$ a $\Dj{u}$ en $\Omega$ y:
\begin{align}
\begin{array}{ll}
\displaystyle
\int_{\Omega}\varphi(\abs{\nabla{u}})dx\leq \lim_{h\to \infty}\inf\int_{\Omega}\varphi(\abs{\nabla{u_{h}}})dx,& \text{si}\; \varphi\; \text{es convexa},\label{teo:cierre de SBV:eq:3}
\end{array}
\end{align}
\begin{align}
\begin{array}{l}
\displaystyle
\int_{J_{u}}\theta(\abs{u^{+}-u^{-}})d\Hm{N-1}\leq \lim_{h\to \infty}\inf \int_{J_{u_{h}}}\theta(\abs{u^{+}_{h}-u^{-}_{h}})d\Hm{N-1},\label{teo:cierre de SBV:eq:4}
\end{array}
\end{align}
si $\theta$ es cóncava.
\end{teo}

\begin{teo}[Compacidad en $SBV$] \label{teo:compacidad en SBV}\index{compacidad!en $SBV$}
Sean $\varphi, \theta$ y $\Omega\subset\R^{N}$ como en el teorema \ref{teo:cierre de SBV}. Además, sea $(u_{h})\subset \SBV$ una sucesión que satisface \ref{teo:cierre de SBV:eq:2} y supongamos que $\Norm{u_{h}}[\infty]$ es uniformemente acotado en $h$. Entonces existe una subsucesión $(u_{h(k)})$ que converge en sentido débil$^{*}$ en $\BV$ a $u\in \SBV$. 
\end{teo}

Una vez enunciado los dos teoremas pasamos a probarlos siguiendo esencialmente la prueba dada por G. Alberti y C. Mantegazza en \cite{alberti1997note} la cual es una simplificación de la dada por L. Ambrosio en \cite{ambrosio1995new}, que utiliza como principal herramienta la regla de la cadena en $BV$.
\DefaultSet{\R}
Dada cualquier función $\theta: (0,\infty)\to (0,\infty)$ se define entonces $\Norm{\psi}[\theta]$ como:
\begin{align*}
\begin{array}{ll}
\displaystyle
\Norm{\psi}[\theta]=\sup\set*{\dfrac{\abs*{\psi(s)-\psi(t)}}{\theta(\abs{t-s})}\given s,t\in \R, s\not = t},&  \forall \psi\in \Lip.
\end{array}
\end{align*}
Si $t/\theta(t)$ está acotado en $(0,1]$ y $\psi\in\W{1}{\infty}$, se puede estimar el supremo usando el hecho que $\psi$ está acotado si $\abs{s-t}\geq 1$ o que $\psi'$ está acotado si $\abs{s-t}<1$ para obtener:
\begin{align*}
\begin{array}{ll}
\displaystyle
\Norm{\psi}[\theta]\leq \dfrac{2}{\theta(1)}\Norm{\psi}[\infty]+\sup_{t\in (0,1)}\dfrac{t}{\theta(t)}\Norm{\psi'}[\infty], & \forall\psi\in\W{1}{\infty}.
\end{array}
\end{align*}
Por lo tanto, podemos estimar $\Norm{\,\cdot\,}[\theta]$ por $c(\theta)\Norm{\,\cdot\,}[\W{1}{\infty}]$. En el siguiente lema examinamos el valor de $\Norm{\,\cdot\,}[\theta]$ bajo el rescalado de la variable independiente.

\begin{lema}\label{lema:limite infitesimal prueba cierre SBV}
Sea $\theta:(0,\infty)\to (0,\infty)$ una función creciente que satisface $\theta(t)/t\to \infty$ cuando $t\to 0$ y sea $\gamma\in \W{1}{\infty}$. Para $\varrho>0$, sea $\psi_{\varrho}(t)=\gamma(t/\varrho)$. Entonces, $\varrho\Norm{\psi_{\varrho}}[\theta]$ es infinitesimal cuando $\varrho\to 0$.
\end{lema}
\begin{proof}
Por el cambio de variable $t'=t/\varrho$ y $s'=s/\varrho$ obtenemos:
\begin{align*}
\varrho\Norm{\psi_{\varrho}}[\theta]=\varrho\sup_{s\not = t}\dfrac{\abs{\psi_{\varrho}(s)-\psi_{\varrho}(t)}}{\theta(\abs{s-t})}=\sup_{s'\not = t'}\dfrac{\varrho\abs{\gamma(s')-\gamma(t')}}{\theta(\varrho\abs{s'-t'})}.
\end{align*}
La última expresión se puede acotar utilizando la cota de $\gamma$ si $\abs{s'-t'}\geq 1$ o la cota de $\gamma'$ si $\abs{s'-t'}<1$ para obtener así:
\begin{align*}
\varrho \Norm{\psi_{\varrho}}[\theta]\leq \dfrac{2\varrho}{\theta(\varrho)}\Norm{\gamma}[\infty]+\sup_{t\in (0,1)}\dfrac{\varrho\tau}{\theta(\varrho\tau)}\Norm{\gamma'}[\infty].
\end{align*}
Como $t/\theta\tosupremo 0$ cuando $t\to 0$ queda probado el lema.
\end{proof}

A continuación, vemos que para $\theta$ cóncava, se tiene que $\Norm{\,\cdot\,}[\theta]$ puede utilizarse para obtener por dualidad la siguiente fórmula de representación.
\begin{lema}\label{lema:representacion preuba cierre SBV}
Cualquier función cóncava creciente $\theta:(0,\infty)\to (0,\infty)$ es subaditiva y satisface:
\begin{align*}
\begin{array}{ll}
\displaystyle
\theta(\abs{s-t})=\sup_{\psi\in X}\dfrac{\abs{\psi(s)-\psi(t)}}{\Norm{\psi}[\theta]},& \forall s,t\in \R, s\not =t,
\end{array}
\end{align*}
donde $X=\set*{\psi\in C^{1}(\R)\given \psi'\in \Cc{},\, \psi\;\text{no constante} }.$
\end{lema} 
\begin{proof}
Suponemos primero que $\theta(0_{+})=0$ y usando la definición de concavidad se concluye sin dificultad la subaditividad.

Se puede comprobar fácilmente que $\Norm{\psi*\rho_{\epsilon}}[\theta]\leq \Norm{\psi}[\theta]$ para cualquier $\psi\in \Lip$, donde $(\rho_{\epsilon})_{\epsilon>0}$ es una familia de mollifiers como hemos visto en los capítulos anteriores. Además la aplicación $f\mapsto \Norm{f}[\theta]$ es semicontinua inferiormente con respecto a la convergencia puntual luego se concluye que $\Norm{\psi*\rho_{\epsilon}}[\theta]$ converge a $\Norm{\psi}[\theta]$ cuando $\epsilon\to 0$. Como consecuencia, el supremo no crece si consideramos las funciones no constantes $\psi\in \Lip$ tales que $\nabla{\psi}=0$ en casi todo punto con respecto a la medida $\Lm{1}$ fuera de un conjunto acotado. Además, un simple argumento de truncado prueba que el supremo es el mismo si consideramos todas las funciones Lipschitz no constantes, que denotaremos por $Lip^{*}(\R)$.

Por invarianza bajo traslaciones, solo hace falta probar:
\begin{align*}
\begin{array}{ll}
\displaystyle
\theta(t)=\sup\set*{\dfrac{\abs{\psi(t)}}{\Norm{\psi}[\theta]}\given \psi\in Lip^{*}(\R), \psi(0)=0 },& \forall t>0.
\end{array}
\end{align*}
La desigcarlualdad $\geq$ se sigue directamente de la definición de $\Norm{\,\cdot\,}[\theta]$. Para $t\geq 0$ y $\epsilon>0$, sea $\psi^{\epsilon}(t)=\theta(t+\epsilon)-\theta(\epsilon)$, entonces por la concavidad de $\theta$ se tiene:
\begin{align*}
\begin{array}{ll}
\displaystyle
\dfrac{\psi^{\epsilon}(t)-\psi^{\epsilon}(s)}{t-s}=\dfrac{\theta(t+\epsilon)-\theta(s+\epsilon)}{t-s}\leq \dfrac{\theta(\epsilon)-\theta(\epsilon/2)}{\epsilon/2},& 0\leq s< t<\infty,
\end{array}
\end{align*}
por lo tanto, $\psi^{\epsilon}$ es una función Lipschitz en $[0,\infty)$. La extensión par de $\psi^{\epsilon}$ a todo $\R$ se denotará también por $\psi^{\epsilon}$ y vemos que se trata de una función Lipschitz que satisface:
\begin{align*}
\abs{\psi^{\epsilon}(t)-\psi^{\epsilon}(s)}=\abs{\psi^{\epsilon}(\abs{t})-\psi^{\epsilon}(\abs{s})}=\abs{\theta(\abs{t}+\epsilon)-\theta(\abs{s}+\epsilon)}\leq \theta(\abs{\abs{t}-\abs{s}})\leq \theta(\abs{t-s}),
\end{align*}
y esto prueba que $\Norm{\psi^{\epsilon}}[\theta]\leq 1$. Como $\theta(0_{+})=0$ y $\theta(t)>0$ para cualquier $t>0$, $\psi^{\epsilon}\in Lip^{*}(\R)$ para $\epsilon$ suficientemente pequeño. En particular:
\begin{align*}
\sup\set*{\dfrac{\abs{\psi(t)}}{\Norm{\psi}[\theta]}\given \psi\in Lip^{*}(\R), \psi(0)=0}\geq \lim_{\epsilon\to 0}\abs{\psi^{\epsilon}(t)}=\theta(t).
\end{align*}
En el caso general aproximamos $\theta$ por debajo por una sucesión de funciones crecientes $\theta_{k}(t)=\theta(t)\wedge kt$ para obtener que todas estas funciones son subaditivas y que:
\begin{align*}
\sup_{\psi\in X}\dfrac{\abs{\psi(s)-\psi(t)}}{\Norm{\psi}[\theta]}\geq \sup_{\psi\in X}\dfrac{\abs{\psi(s)-\psi(t)}}{\Norm{\psi}[\theta_{k}]}=\theta_{k}(\abs{s-t}),
\end{align*}
puesto que $\Norm{\,\cdot\,}[\theta]\leq \Norm{\,\cdot\,}[\theta_{k}]$. El resultado se sigue haciendo tender $k$ a infinito.
\end{proof}

La principal dificultad a la hora de probar el resultado de cierre del espacio $SBV$ es ver que bajo las condiciones impuestas no aparecerá en el límite una parte de Cantor de la derivada. Para ello, la idea no es solo estudiar $Du$ pero también $D\psi(u)$, donde $\psi:\R\to \R$ es una función Lipschitz. Entonces, por la regla de la cadena en $BV$ la medida $\sigma_{u}=D\psi(u)-\psi'(u)\nabla{u}\Lm{N}$ se puede representar por:
\begin{align}
\sigma_{u}=D^{j}\psi(u)+D^{c}\psi(u)=(\psi(u^{+})-\psi(u^{-}))\nu_{u}\Hm{N-1}\Radot J_{u}+\psi'(\tilde{u})D^{c}u.
\end{align}
Por la definición de $\Norm{\psi}[\theta]$, si $u\in \SBV$ entonces la variación total de $\sigma_{u}$ se puede acotar por $\Norm{\psi}[\theta]\mu$ con $\mu=\theta(\abs{u^{+}-u^{-}})\Hm{N-1}\Radot J_{u}$. Por otro lado, si $u\not \in \SBV$, es decir si $\abs{D^{c}u}(\Omega)>0$ y además si $\theta(t)/t \to \infty$ cuando $t\to 0$ entonces no se puede tener una cota de este tipo. Por ejemplo si tomamos $\theta\equiv 1$ esto implicaría un cota puntual de $\psi'(\tilde{u})$ por las oscilaciones de $\psi$ en $\R$ y esto es imposible. 

Ya estamos en disposición de probar el siguiente teorema de caracterización de funciones en $SBV$ con el que posteriormente probaremos la convergencia débil del gradiente aproximado en el teorema \ref{teo:cierre de SBV}.

\begin{prop}[Caracterización de las funciones $SBV$]\label{prop:Caracterización de las funciones SBV}\DefaultSet{\Omega}
Sea $\Omega\subset\R^{N}$ un abierto acotado y sea $\theta:(0,\infty)\to (0,\infty)$ una función creciente que satisface $\theta(t)/t\to\infty$ cuando $t \to 0$. Sea $u\in \BV$ y supongamos que existe una función $a\in [\L1]^{N}$ y una medida positiva finita $\mu$ en $\Omega$ tal que:\DefaultSet{\R}
\begin{align}\label{prop:Caracterización de las funciones SBV:eq:1}
\begin{array}{ll}
\displaystyle
\abs*{D\psi(u)-\psi'(u)a\Lm{N}}\leq \Norm{\psi}[\theta]\mu,& \forall \psi\in \W{1}{\infty}\cap C^{1}(\R).
\end{array}
\end{align}\DefaultSet{\Omega}
Entonces $u\in \SBV$, $a=\nabla{u}$ en casi todo punto en $\Omega$ respecto de la medida $\Lm{N}$, y si $\theta$ es cóncava se tiene además que $\mu \geq \theta(\abs{u^{+}-u^{-}})\Hm{N-1}\Radot J_{u}$.

Por otro lado, si $\mu\in \SBV$ y si $\int_{J_{u}}\theta(\abs{u^{+}-u^{-}})d\Hm{N-1}<\infty$ entonces la ecuación \ref{prop:Caracterización de las funciones SBV:eq:1} se verifica con $a=\nabla{u}$ y $\mu=\theta(\abs{u^{+}-u^{-}})\Hm{N-1}\Radot J_{u}$.
\end{prop}
\begin{proof}
Por los comentarios anteriores se comprueba que la ecuación \ref{prop:Caracterización de las funciones SBV:eq:1} se cumple con $a=\nabla{u}$ y $\mu=\theta(\abs{u^{+}-u^{-}})\Hm{N-1}\Radot J_{u}$ si $u\in \SBV$. La implicación opuesta se prueba en tres pasos: suponiendo que la ecuación \ref{prop:Caracterización de las funciones SBV:eq:1} es cierta para algún $a$, $\mu$ primero probamos que $a=\nabla{u}$, luego el hecho de que la parte de Cantor de la derivada es nula y finalmente la cota inferior de $\mu$.

Paso 1. Afirmamos que $a(x_{0})=\nabla{u}(x_{0})$ para cualquier punto de Lebesgue $x_{0}$ de $a$ y $u$, donde $u$ es aproximadamente diferenciable y $\mu(B_{\varrho}(x_{0}))/\varrho^{N}$ está acotado cuando $\varrho\to 0$. Por \ref{eq:1:hausdorff y medidas de radon} y por el teorema de Calderón-Zygmund \ref{teo:teorema de Calderón-Sygmund}, casi todo $x_{0}\in \Omega$ verifica estas propiedades y por lo tanto $a=\nabla{u}$. Para probar la igualdad, sea $u_{0}(y)=\Crochet{\nabla{u}(x_{0})}{y}$ y fijemos una función $\gamma\in \Cc{1}$ que coincide con la identidad en $u_{0}(B_{1})$ y otra función no-cero positiva $\varphi\in C_c^{1}(B_1)$. Rescalando $\gamma$ y $\varphi$ consideramos:
\begin{align*}
\begin{array}{ll}
\psi_{\varrho}(t)=\gamma\left ( \dfrac{t-u(x_{0})}{\varrho}\right ),& \phi_{\varrho}(x)=\varphi\left ( \dfrac{x-x_{0}}{\varrho}\right).
\end{array}
\end{align*}
Primero vemos que:
\begin{align}\label{proof:caracterización de las funciones SBV:eq:1}
\lim_{\varrho\to 0}\dfrac{\Norm{\psi_{\varrho}}[\theta]\mu(B_{\varrho}(x_0))}{\varrho^{N-1}}=\lim_{\varrho\to 0}\varrho\Norm{\psi_{\varrho}}[\theta]\dfrac{\mu(B_{\varrho}(x_{0}))}{\varrho^{N}}=0,
\end{align}
debido al lema \ref{lema:limite infitesimal prueba cierre SBV}. Ahora haciendo un cambio de variable se tiene:
\begin{align*}
\int_{\Omega}\psi_{\varrho}(u)\nabla{\phi_{\varrho}}\,dx&=\dfrac{1}{\varrho}\int_{\Omega}\gamma\left ( \dfrac{u(x)-u(x_{0})}{\varrho}\right )\nabla\varphi\left ( \dfrac{x-x_{0}}{\varrho}\right )dx
\\&=\varrho^{N-1}\int_{B_1}\gamma(u_{\varrho}(y))\nabla\varphi(y)dy,
\end{align*}
donde $u_{\varrho}(y)=[u(x_{0}+\varrho y)-u(x_{0})]/\varrho$. Por otro lado:
\begin{align*}
\int_{\Omega}\psi_{\varrho}(u)\nabla{\phi_{\varrho}}\,dx&=-\int_{\Omega}\phi_{\varrho}(x)dD\psi_{\varrho}(u)\\
&=-\int_{\Omega}\phi_{\varrho}\, d[D\psi_{\varrho}(u)-a\psi_{\varrho}'(u)\Lm{N}]-\int_{\Omega}\phi_{\varrho}a\psi_{\varrho}'(u)dx\\
&=o(\varrho^{N-1})-\varrho^{N-1}\int_{B_1}\varphi(y)a(x_{0}+\varrho y)\gamma'(u_{\varrho}(y))dy,
\end{align*}
ya que \ref{prop:Caracterización de las funciones SBV:eq:1} y \ref{proof:caracterización de las funciones SBV:eq:1} implican:
\begin{align*}
\abs*{\int_{\Omega}\phi_{\varrho}\, d[D\psi_{\varrho}(u)-a\psi_{\varrho}'(u)\Lm{N}]}\leq \Norm{\varphi}[\infty]\Norm{\psi_{\varrho}}[\theta]\mu(B_{\varrho}(x_{0}))=o(\varrho^{N-1}).
\end{align*}
Comparando las dos expresiones de $\int_{\Omega}\psi_{\varrho}(u)\nabla\phi_{\varrho}dx$, vemos que:
\begin{align*}
\int_{B_1}\gamma(u_{\varrho}(y))\nabla\varphi(y)dy&=-\int_{B_1}\varphi(y)a(x_{0}+\varrho y)\gamma'(u_{\varrho}(y))dy+o(1)
\\&=-\int_{B_1}\varphi(y)a(x_0)\gamma'(u_{\varrho}(y))dy+o(1),
\end{align*}
ya que $\varphi$ y $\gamma'$ están acotadas y $x_{0}$ es un punto de Lebesgue de $a$. Ahora tomando límite cuando $\varrho\to 0$ y por la convergencia $L^{1}$ de $u_{\varrho}$ a $u_{0}$ se obtiene:
\begin{align*}
\int_{B_1}\gamma(u_{0}(y))\nabla{\varphi}(y)dy=-\int_{B_1}\varphi(y)a(x_{0})\gamma'(u_{0}(y))dy.
\end{align*}
Finalmente, por integración por partes se tiene que:
\begin{align*}
[\nabla{u}(x_0)-a(x_{0})]\int_{B_1}\gamma'(u_{0}(y))\varphi(y)dy=0.
\end{align*}
Como $\gamma'\equiv 1$ en el rango de $u_{0}$ concluimos que $\nabla{u}(x_0)=a(x_{0})$.

Paso 2. Por la ecuación \ref{teo:Regla de la cadena en BV version 1:eq:1}, la parte absolutamente continua $\Da{\psi(u)}$ de $D\psi(u)$ con respecto a $\Lm{N}$ es $\psi'(u)\nabla{u}\Lm{N}$. Por lo tanto, por \ref{proof:caracterización de las funciones SBV:eq:1} y por el paso 1 deducimos que:
\begin{align}\label{proof:caracterización de las funciones SBV:eq:2}
\begin{array}{ll}
\abs{D^{s}{\psi(u)}}\leq \Norm{\psi}[\theta]\mu,& \forall \psi  \in \W{1}{\infty}\cap C^{1}(\R).
\end{array}
\end{align}
Restringiendo ambas medidas al conjunto $E=\Omega\setminus S_{u}$ y por \ref{teo:Regla de la cadena en BV version 1:eq:1} obtenemos:
\begin{align*}
\begin{array}{ll}
\abs{\psi'(\tilde{u})}\abs{D^{c}u}\leq \Norm{\psi}[\theta]\mu \Radot E, & \forall \psi \in W^{1,\infty}(\R)\cap C^{1}(\R).
\end{array}
\end{align*}
Escogiendo $\psi^{1}_{\epsilon}(t)=\sin(t/\epsilon)$ y $\psi^{2}_{\epsilon}(t)=\cos(t/\epsilon)$ se obtiene que:
\begin{align*}
\begin{array}{ll}
\displaystyle
\dfrac{1}{\epsilon}\abs{\sin(\tilde{u}/\epsilon)}\abs{\Dc{u}}\leq \Norm{\psi^{2}_{\epsilon}}[\theta]\mu\Radot E,& \displaystyle \dfrac{1}{\epsilon}\abs{\cos(\tilde{u}/\epsilon)}\abs{\Dc{u}}\leq \Norm{\psi^{1}_{\epsilon}}[\theta]\mu\Radot E,
\end{array}
\end{align*}
y como $\abs{\sin t}+\abs{\cos t}\geq 1$ para cualquier $t\in \R$ se sigue que:
\begin{align*}
\abs{\Dc{u}}\leq \epsilon\left ( \Norm{\psi^{1}_{\epsilon}}[\theta]+\Norm{\psi^{2}_{\epsilon}}[\theta]\right) \mu \Radot E.
\end{align*}\DefaultSet{\Omega}
Tomando límite cuando $\epsilon\to 0$ y usando de nuevo el lema \ref{lema:limite infitesimal prueba cierre SBV} se obtiene $\abs{\Dc{u}}=0$, es decir, $u\in \SBV$.

\DefaultSet{\R}
Paso 3. Restringiendo ambas medidas en \ref{proof:caracterización de las funciones SBV:eq:2} al conjunto $J_{u}$ y por el teorema \ref{teo:Regla de la cadena en BV version 2} se tiene $\Dc{\psi(u)}=0$, por lo tanto:
\begin{align*}
\begin{array}{ll}
\abs{\psi(u^{+})-\psi(u^{-})}\Hm{N-1}\Radot J_{u} =\abs{\Dj{\psi(u)}}\leq \Norm{\psi}[\theta]\mu \Radot J_{u}, & \forall \psi \in \W{1}{\infty}\cap C^{1}(\R).
\end{array}
\end{align*}
Sea $X$ como en el lema \ref{lema:representacion preuba cierre SBV}, se ve que $X$ es un subespacio separable de $\W{1}{\infty}$, pues, por definición cualquier función en $X$ pertenece a $C_{c}(\R)$. Sea $D$ un subconjunto numerable denso de $X$, tomando en cuenta la nota \ref{nota:sobre m.c.m y m.c.d medidas} y el lema \ref{lema:representacion preuba cierre SBV} se obtiene:
\begin{align*}
\mu\geq& \sup_{\psi\in D}\left [\dfrac{\abs{\psi(u^{+})-\psi(u^{-})}}{\Norm{\psi}[\theta]}\Hm{N-1}\Radot J_{u} \right ]=\left [ \sup_{\psi\in D}\dfrac{\psi(u^{+})-\psi(u^{-})}{\Norm{\psi}[\theta]}\right ]\Hm{N-1}\Radot J_{u}
\\&=\left [ \sup_{\psi\in X}\dfrac{\abs{\psi(u^{+})-\psi(u^{-})}}{\Norm{\psi}[\theta]}\right]\Hm{N-1}\Radot J_{u}=\theta(\abs{u^{+}-u^{-}})\Hm{N-1}\Radot J_{u}.
\end{align*}
\end{proof}\DefaultSet{\Omega}
Ya estamos en disposición de probar los resultados de cierre (teorema \ref{teo:cierre de SBV}) y compacidad (teorema \ref{teo:compacidad en SBV}) del espacio $\SBV$.
\begin{proof}[Demostración teorema \ref{teo:cierre de SBV}]
Remplazando $\theta$ por $\theta\wedge p$ y haciendo tender $p\to \infty$ podemos suponer que $\theta(t)<\infty$ para cualquier $t\in (0,\infty)$.  Como $\varphi$ tiene un crecimiento mayor que lineal en el infinito, la proposición \ref{teo:condiciones de equiitegrabilidad} muestra que las funciones $\abs{\nabla{u_{h}}}$ son equiintegrables en $\Omega$. Por el teorema de Dunford-Pettis y por el teorema \ref{teo:convergencia débil*}, podemos asumir, extrayendo posiblemente una subsucesión, que $(\nabla{u_{h}})$ converge débilmente a una función $a$ en $[\L1]^{N}$ y que las medidas:
\begin{align*}
\mu_{h}=\theta(\abs{u^{+}_{h}-u^{-}_{h}})\Hm{N-1}\Radot J_{u_{h}},
\end{align*}
convergen en sentido débil$^{*}$ en $\Omega$ a alguna medida finita positiva $\mu$. Probemos que $u$ satisface la condición \ref{prop:Caracterización de las funciones SBV:eq:1} de la proposición \ref{prop:Caracterización de las funciones SBV}. Para ello primero probamos que $\psi'(u_{h})\nabla{u_{h}}$ converge débilmente a $\psi'(u)a$ en $[\L1]^{N}$ para cualquier función Lipschitz $\psi\in C^{1}(\R)$. Para comprobar esta propiedad basta escribir:
\begin{align*}
\psi'(u_{h})\nabla{u_{h}}=\left[ (\psi'(u_{h})-\psi'(u))\nabla{u_{h}}\right ]+\psi'(u)\nabla{u_{h}},
\end{align*}
y usar el teorema de convergencia dominada de Vitali para concluir que el término entre paréntesis tiende a $0$ en norma $L^{1}$, y por lo tanto:
\begin{align*}
\lim_{h\to \infty}\int_{\Omega}\varphi\psi'(u_{h})\nabla{u_{h}}\,dx=\lim_{h\to \infty}\int_{\Omega}\varphi\psi'(u)\nabla{u_{h}}\,dx=\int_{\Omega}\varphi\psi'(u)a\, dx,
\end{align*}
para cualquier $\varphi\in \L{\infty}$. También por el teorema \ref{teo:Regla de la cadena en BV version 1} las variaciones totales $\abs{D\psi(u_{h})}(\Omega)$ están equiacotadas, luego la convergencia $L^{1}$ de $\psi(u_{h})$ a $\psi(u)$ implica la convergencia débil$^{*}$ de $D\psi(u_{h})$ a $D\psi(u)$. En particular:
\begin{align}\label{proof:cierre de SBV:eq:1}
\lim_{h\to \infty}D\psi(u_{h})-\psi'(u_{h})\nabla{u_{h}}\Lm{N}=D\psi(u)-\psi'(u)a\Lm{N}.
\end{align}
Ya estamos en disposición de probar la condición \ref{prop:Caracterización de las funciones SBV:eq:1}. Sea $(h(k))$ una subsucesión tal que las medidas
\begin{align*}
\abs*{D\psi(u_{h(k)})-\psi'(u_{h(k)})\nabla{u_{h(k)}}\Lm{N}},
\end{align*}
convergen en sentido débil$^{*}$ a una medida positiva $\sigma$ en $\Omega$. Tomando en cuenta \ref{proof:cierre de SBV:eq:1} y la proposición \ref{prop:convergencia dominada de medidas}(\ref{prop:convergencia dominada de medidas:b}), pasando al límite cuando $k\to \infty$ en:
\begin{align*}
\abs{D\psi(u_{h(k)})-\psi'(u_{h(k)})\nabla{u_{h(k)}}\Lm{N}}\leq \Norm{\psi}[\theta]\mu_{h(k)},
\end{align*}
obtenemos:
\begin{align*}
\abs{D\psi(u)-\psi'(u)a\Lm{N}}\leq \sigma\leq \Norm{\psi}[\theta]\mu_{h(k)}. 
\end{align*}
Por la proposición \ref{prop:Caracterización de las funciones SBV} se obtiene que $u\in \SBV$, además como $a=\nabla{u}$, se sigue la convergencia débil del gradiente. En particular $\Da{u_{h}}$ converge en sentido débil$^{*}$ en $\Omega$ a $\Da{u}$, por lo tanto, $\Dj{u_{h}}=Du_{h}-\Da{u_{h}}$ converge en sentido débil$^{*}$ a $Du-\Da{u}=\Dj{u}$. Si $\varphi$ es una función convexa y creciente, entonces $w\mapsto \int_{\Omega}\varphi(\abs{w})dx$ es convexa en $[\L{1}]^{N}$. Como este funcional es fuertemente semicontinuo inferiomente (por el lema de Fatou) también es débilmente semicontinuo inferiormente y esto prueba \ref{teo:cierre de SBV:eq:3}.

Finalmente, \ref{teo:cierre de SBV:eq:4} es una consecuencia directa de la convergencia de $\mu_{h}$ a $\mu$ y de la desigualdad $\mu\geq \theta(\abs{u^{+}-u^{-}})\Hm{N-1}\Radot J_{u}$.
\end{proof}

\begin{proof}[Demostración teorema \ref{teo:compacidad en SBV}]
Sea $M$ el supremo de $\Norm{u_{h}}[\infty]$. Por \ref{teo:cierre de SBV:eq:1} podemos encontrar $\alpha\in \R$ y $\beta\in (0, \infty)$ tal que:
\begin{align*}
\begin{array}{lllll}
\varphi(t)\geq t+\alpha, &\forall t\in[0,\infty],&& \theta(t)\geq \beta t, & \forall t\in (0,2M].
\end{array}
\end{align*}
Estas desigualdades implican:
\begin{align*}
\abs{Du_{h}}(\Omega)&=\int_{\Omega}\abs{\nabla{u_{h}}}dx+\int_{J_{u_{h}}}\abs{u^{+}_{h}-u^{-}_{h}}d\Hm{N-1}
\\&\leq \int_{\Omega} \varphi(\abs{\nabla{u_{h}}})dx-\alpha\abs{\Omega}+\dfrac{1}{\beta}\int_{J_{u_{h}}}\theta(\abs{u^{+}_{h}-u^{-}_{h}})d\Hm{N-1}.
\end{align*}
Esto prueba que las variaciones totales $\abs{Du_{h}}(\Omega)$ están equiacotadas, luego por el teorema \ref{teo:compacidad en BV} existe una subsucesión $(u_{h(k)})$ que converge en $L^{1}_{loc}(\Omega)$ a $u\in BV_{loc}(\Omega)$. Como $\Omega$ está acotado y $\Norm{u_{h}}[\infty]+\abs{Du_{h}}(\Omega)$ son equiacotadas, se sigue que $u\in L^{\infty}(\Omega), u_{h(k)}\to u$ en $L^{1}(\Omega)$ y $\abs{Du}(\Omega)<\infty$. Por lo tanto, $(u_{h(k)})$ converge en sentido débil$^{*}$ a $u$ en $\Omega$ y por el teorema \ref{teo:cierre de SBV} concluimos que $u\in SBV(\Omega)$.
\end{proof}

\section{Teorema de Poincaré en $SBV$}\label{section:Teorema de Poincaré en SBV}

En esta última sección del capítulo estudiaremos el comportamiento de funciones $u$ de variación acotada especiales en bolas $B$ tales que la medida del conjunto de discontinuidades (i.e., $\Hm{N-1}(S_{u}\cap B)$) es pequeña respecto de la medida de la bola. En particular probaremos una desigualdad de Poincaré modificada para funciones en $SBV$. 

Por lo general no es posible acotar las oscilaciones de $u$ en norma $L^{q}$ únicamente con la norma $L^{p}$ del gradiente $\nabla{u}$, obviando la parte de salto de la derivada distribucional. Para $\epsilon\ll 1$, basta considerar una función idénticamente cero en $B_{1}\setminus B_{\epsilon}$ y  que tome un valor constante muy grande en $B_{\epsilon}$. Este caso muestra que no hay posibilidad de tener una desigualdad de tipo Poincaré que solo involucré $\nabla{u}$ por muy pequeño que sea el conjunto donde se produce el salto sin antes truncar la función para excluir el fenómeno que hemos descrito antes.

Sea $B\subset\R^{N}$ la bola y $u:B\to \R$ una función medible, se define entonces:
\begin{align*}
u_{*}(s,B)=\inf\set*{t\in [-\infty, \infty]\given \abs{\set{u<t}}\geq s},
\end{align*}
para cualquier $s\in [0,\abs{B}]$. Es fácil ver que $m=u_{*}(\abs{B}/2,B)$ es una mediana de $u$, es decir, $m$ verifica:
\begin{align*}
\begin{array}{lllll}
\abs{\set{u<t}}\leq \dfrac{\abs{B}}{2},& \forall t<m, &&\abs{\set{u>t}}\leq \dfrac{\abs{B}}{2},& \forall t>m,
\end{array}
\end{align*}
y es el número más pequeño con esta propiedad. Denotando por $\gamma_{5}$ la constante de la desigualdad isoperimétrica \ref{eq:1:desigualdad isoperimetrica relativa}, y suponiendo:
\begin{align}\label{eq:1:teorema de Poincaré en SBV}
\left(2\gamma_{5}\Hm{N-1}(S_{u}\cap B)\right)^{N/N-1}<\dfrac{\abs{B}}{2},
\end{align}
se define entonces:
\begin{align}\label{eq:1:taus}
\left\{ \begin{array}{l}
\tau^{-}(u,B)=u_{*}\left( [2	\gamma_{5}\Hm{N-1}(S_{u}\cap B)]^{N/(N-1)},B\right),\\
\tau^{+}(u,B)=u_{*}\left( \abs{B}-[2\gamma_{5}\Hm{N-1}(S_{u}\cap B)]^{N/(N-1)},B\right).\end{array} \right . 
\end{align}
Por la desigualdad \ref{eq:1:teorema de Poincaré en SBV} vemos que $\tau^{-}(u,B)\leq m\leq \tau^{+}(u,B)$ para cualquier mediana $m$ de $u$ en $B$. Seguidamente, enunciamos la siguiente desigualdad tipo Poincaré en $SBV$ probada por E. De Giorgi, M. Carriero y A. Leaci en \cite{de1989existence}.
\begin{teo}[Desigualdad de tipo Poincaré en $SBV$]\label{teo:Desigualdad de tipo Poincaré en SBV} \index{desigualdad!de Poincaré} \index{teorema!de Poincaré}
Sea $B$ una bola de $\R^{N}$, $u\in SBV(B)$ y $1\leq p<N$. Si se verifica la condición \ref{eq:1:teorema de Poincaré en SBV}, entonces la función $\overline{u}=\tau^{-}(u,B)\vee u\wedge \tau^{+}(u,B)$ satisface $\abs{D\overline{u}}(B)\leq 2 \int_{B}\abs{\nabla{u}}dx$ y además:
\begin{align}\label{teo:Desigualdad de tipo Poincaré en SBV:eq:1}
\left ( \int_{B} \abs{\overline{u}-m}^{p^{*}}\right )^{1/p^{*}}\leq \dfrac{2\gamma_{5}p(N-1)}{N-p}\left ( \int_{B}\abs{\nabla{u}}^{p}\right )^{1/p},
\end{align}
para cualquier mediana $m$ de $u$ en $B$.
\end{teo}
\begin{proof}
Sumando una constante a la función $u$ si es necesario podemos suponer que $m=0$. De ahora en adelante, denotaremos por $\tau^{+},\tau^{-}$ los números $\tau^{+}(u,B), \tau^{-}(u,B)$.
Por el teorema \ref{teo:Regla de la cadena en BV version 2} se sabe que $\overline{u}\in SBV(B)$, además el teorema \ref{prop:Propiedades de los límites aproximados}(\ref{prop:Propiedades de los límites aproximados:c}) implica que $S_{\overline{u}}\subset S_{u}$. Recordando también que por la propiedad local de la diferencial en sentido aproximado se tiene que $\nabla{\overline{u}}(x)=\nabla{u}(x)$ para casi todo $x\in \set*{u=\overline{u}}$ con respecto la medida $\Lm{N}$ mientras que $\nabla{\overline{u}}=0$ para casi todo $x\in \set*{u\not =\overline{u}}$ con respecto a la medida $\Lm{N}$, luego se tiene:
\begin{align}\label{proof:Desigualdad de tipo Poincaré en SBV:eq:1}
\abs{D\overline{u}}(B)&=\int_{B}\abs{\nabla{\overline{u}}}dx+\int_{S_{\overline{u}}}\abs{\overline{u}^{+}-\overline{u}^{-}}d\Hm{N-1}\nonumber
\\&\leq \int_{B}\abs{\nabla{u}}dx+(\tau^{+}-\tau^{-})\Hm{N-1}(S_{u}\cap B).
\end{align}
De la fórmula de la coárea para funciones $BV$ y de la desigualdad isoperimétrica \ref{teo:desigualdad isoperimetrica} se obtiene:
\begin{align*}
\abs{D\overline{u}}(B)&=\int^{\infty}_{-\infty}P(\set{\overline{u}>t},B)dt=\int^{\tau^{+}}_{\tau^{-}}P(\set{u>t},B)dt
\\&\geq \dfrac{1}{\gamma_{5}}\left [ \int_{\tau^{-}}^{0}\abs{\set{u\leq t}}^{(N-1)/N}dt+\int^{\tau^{+}}_{0}\abs{\set{u>t}}^{(N-1)/N}dt\right ].
\end{align*}
Por definición de $\tau^{\pm}$, se tiene entonces que:
\begin{align*}
\left \{ \begin{array}{ll}
\abs{\set{u\leq t}}^{(N-1)/N}\geq 2\gamma_{5}\Hm{N-1}(S_{u}\cap B),& \forall t\in (\tau^{-},0),\\
\abs{\set{u> t}}^{(N-1)/N}\geq 2\gamma_{5}\Hm{N-1}(S_{u}\cap B),&\forall t\in (0,\tau^{+}),
\end{array}\right.
\end{align*}
por lo tanto $2(\tau^{+}-\tau^{-})\Hm{N-1}(S_{u}\cap B)\leq \abs{D\overline{u}}(B)$, luego de \ref{proof:Desigualdad de tipo Poincaré en SBV:eq:1} se deduce:
\begin{align*}
(\tau^{+}-\tau^{-})\Hm{N-1}(S_{u}\cap B)\leq \int_{B}\abs{\nabla{u}}dx.
\end{align*}
Usando de nuevo \ref{proof:Desigualdad de tipo Poincaré en SBV:eq:1} se puede acotar $\abs{D\overline{u}}(B)$ por $2\int_{B}\abs{\nabla{u}}\,dx$. Por la desigualdad de Poincaré \ref{teo:desigualdad tipo poincare Lp con mediana:eq:1} se concluye:
\begin{align}\label{proof:Desigualdad de tipo Poincaré en SBV:eq:2}
\left ( \int_{B}\abs{\overline{u}}^{1^{*}}dx\right )^{1/1^{*}}\leq \gamma_{5}\abs{D\overline{u}}(B)\leq 2\gamma_{5}\int_{B}\abs{\nabla{u}}dx.
\end{align}
Esto prueba el teorema para $p=1$. Si $1<p<N$, suponemos que $u$ es acotado y consideramos $v=\abs{u}^{q-1}u$ con $q=p(N-1)/(N-p)$, observando que $0$ es una mediana de $v$:
\begin{align*}
\begin{array}{ll}
\tau^{+}(v,B)=(\tau^{+}(u,B))^{q},& \tau^{-}(u,B)=-(-\tau^{-}(u,B))^{q},
\end{array}
\end{align*}
usando la ecuación \ref{proof:Desigualdad de tipo Poincaré en SBV:eq:2} para $v$ y por la desigualdad de Hölder se obtiene \ref{teo:Desigualdad de tipo Poincaré en SBV:eq:1}. El caso general se deduce por un argumento de truncamiento. 
\end{proof}


\chapter{Existencia y regularidad de las soluciones de problemas de discontinuidad libre} \label{cap:cap4}

En este último capítulo estudiaremos una clase de problemas variacionales conocidos como problemas de discontinuidad libre. La terminología de ``problemas de discontinuidad libre'' fue acuñada por E. De Giorgi en \cite{de1991free} para indicar la clase de problemas variacionales que consisten en la minimización de un funcional que involucra tanto una energía volumétrica como una energía superficial y que depende de un conjunto cerrado $K$ y de una función $u$ generalmente suave fuera de $K$. El conjunto $K$ no está fijado a priori y por lo general tampoco es una frontera. Por lo tanto, esta clase de problemas es diferente de la de los problemas de frontera libre, en consecuencia, se necesita un nuevo enfoque. Las funciones de variación acotada especiales han sido de gran utilidad para estos problemas ya que como veremos son el marco adecuado para su formulación débil y para la regularización de las soluciones del problema ``relajado''.

En la primera sección introduciremos informalmente algunos de los problemas de discontinuidad libre más conocidos. En particular, veremos el problema de los conjuntos con curvatura media prescrita, el de partición óptima y el de la segmentación de imágenes Mumford-Shah. En la sección dos y tres del capítulo probaremos la existencia y la unicidad de las soluciones de una clase particular de problemas de discontinuidad libre cuyo modelo es el funcional de Mumford-Shah:\index{funcional!de Mumford-Shah}
\begin{align*}
J(K,u)=\int_{\Omega\setminus K}\left [ \abs{\nabla{u}}^{2}+\alpha (u-g)^{2}\right ]dx + \beta \Hm{N-1}(K\cap\Omega),
\end{align*}
donde $\Omega\subset \R^{N}$ es un conjunto abierto, $K\subset \R^{N}$ un conjunto cerrado, $g\in \L{2}\cap \L{\infty}$ y $\alpha, \beta$ son parámetros estrictamente positivos. El problema de Mumford-Shah consiste en minimizar el funcional anterior entre todos los pares los pares $(K,u)$ con $K$ un subconjunto cerrado de $\R^{N}$ y $u\in C^{1}(\Omega\setminus K)$. Aunque para el problema en dos dimensiones existe una prueba directa de la existencia de un minimizador (véase \cite{morel2012variational}), sin embargo, los métodos directos habituales del cálculo de variaciones no se aplican fácilmente para este problema. Esto es debido a que no existe una topología sobre los conjuntos cerrados que garantice la compacidad de la sucesión minimizante y la semicontinuidad inferior de la medida $\Hm{N-1}$. Es por ello que introduciremos el funcional ``relajado'':
\begin{align}\label{eq:1:problema variacional}
\begin{array}{ll}
\displaystyle
\mathcal{F}(u)=\int_{\Omega\setminus S_{u}}\left [ \abs{\nabla{u}}^{2}+\alpha(u-g)^{2}\right]dx+\beta \Hm{N-1}(S_{u}),& u\in \SBV,
\end{array}
\end{align}
para el cual, por los resultados de cierre y compacidad en $SBV$ del capítulo anterior, se obtiene fácilmente la existencia de un minimizador. El punto clave de estas dos secciones y también del capítulo, consistirá en probar que si $u$ es una minimizador entonces para cualquier $x\in S_{u}$ y para cualquier bola $B_{\varrho}(x)\subset \Omega$, con $\varrho$ suficientemente pequeño, se tendrá la siguiente cota inferior de densidad:
\begin{align}
\Hm{N-1}(S_{u}\cap B_{\varrho}(x))\geq \vartheta_{0}\varrho^{N-1},
\end{align}
donde $\vartheta_{0}=\vartheta_{0}(N)$ es una constante estrictamente positiva que únicamente depende de la dimensión del espacio ambiente. Como consecuencia inmediata de esta cota se deducirá que si $u\in \SBV$ y si se verifica la condición anterior entonces:
\begin{align*}
\Hm{N-1}(\Omega\cap \overline{S}_{u}\setminus S_{u})=0.
\end{align*}
Con este último resultado probaremos que $u$ tiene un representante $\tilde{u}\in C^{1}(\Omega\setminus \overline{S}_{u})$ y que el par $(\overline{S}_{u}, \tilde{u})$ es en efecto un par minimizante del funcional $J$. Una vez obtenido un minimizador $u$, es natural investigar qué regularidad se puede esperar para el conjunto de discontinuidades $S_{u}$. Para ello, en la última sección demostraremos la fórmula de la primera variación del área y recordaremos algunos hechos básicos sobre la curvatura y finalmente se establecerán las ecuaciones de Euler-Lagrange y se probarán propiedades de regularidad de $u$ en $\overline{S}_{u}$.

\section{Introducción a los problemas de discontinuidad libre}

A continuación, en esta sección damos tres ejemplos de problema de discontinuidad libre, que son: el problema de conjuntos con curvatura media prescrita, el problema de partición óptima y el problema de segmentación de imágenes de Mumford-Shah.

\subsection{Conjuntos con curvatura media prescrita} \index{problema!de curvatura media prescrita}
El problema más sencillo donde energía volumétrica \index{energía!volumétrica} y superficial  \index{energía!superficial} compiten es el problema con curvatura media prescrita:
\begin{align*}
\min_{E}\set*{\int_{E}g(x)dx+\Hm{N-1}(\partial E)\given E\subset \R^{N}},
\end{align*}
donde $g\in L^{1}(\R^{N})$ es dado. En este problema, si $g<0$ en alguna región $F$, entonces los dos términos tendrán signo opuesto, y por lo tanto, si $F$ no es demasiado irregular,  el conjunto $E$ solución contendrá a $F$. La terminología del problema se puede entender a través de la primera variación: si $g$ es una función continua en un punto regular $x$ de $\partial E$ y $E$ minimiza el funcional, entonces se tiene la ecuación:
\begin{align*}
\mathbf{H}(x)=g(x)\nu_{E}(x),
\end{align*} 
donde $\mathbf{H}$ es la curvatura media de $\partial E$ y $\nu_{E}$ es la normal exterior de $E$.

\subsection{Partición óptima}\index{problema!de partición óptima}
Una generalización del problema anterior es el problema de partición óptima. Dado $\Omega\subset \R^{N}$ y $g\in \L{\infty}$, estudiamos el problema:
\begin{align*}
\min_{K,u}\set*{\Hm{N-1}(K\cap \Omega)+\alpha\int_{\Omega\setminus K}\abs{u-g}^{2}dx},
\end{align*}
donde $K\subset\R^{N}$ es un conjunto cerrado y $u$ es una función constantes en las componentes conexas de $\Omega\setminus K$. La solución de este problema de minimización corresponde a la función constante a trozos que mejor aproxima $g$, con un control del área total del conjunto de discontinuidades $K$. Es evidente que dado $K$ el valor de $u$ en cada componente conexa será el valor medio de $g$ en $\Omega\setminus K$, y viceversa, conocido $u$ el conjunto $K$ será el conjunto de discontinuidades de $u$. Luego las variables $K,u$ pueden reducirse a una sola. Por lo tanto, reformulando el problema en el espacio $\SBV$ queda:
\begin{align*}
\min_{u}\set*{\Hm{N-1}(S_{u})+\alpha\int_{\Omega}\abs{u-g}^{2}dx},
\end{align*}
para cualquier función constante a trozos $u\in \SBV$.

\subsection{Problema de segmentación de imágenes de Mumford-Shah}\index{problema!de Mumford-Shah} 
El problema de partición óptima es el caso límite del problema quizás más popular de discontinuidad libre ques consiste en minimizar el conocido funcional de Mumford-Shah:
\begin{align*}
\inf\set*{J(K,u)\given K\subset \overline{\Omega}\;\text{cerrado}, u\in C^{1}(\Omega\setminus K)},
\end{align*}
donde $J$ está definido por: \index{funcional!de Mumford-Shah}
\begin{align*}
J(K,u)=\int_{\Omega\setminus K}\left [ \abs{\nabla{u}}^{2}+\alpha(u-g)^{2}\right ] dx+\beta \Hm{N-1}(K\cap \Omega),
\end{align*}
donde $\Omega$ es un conjunto acotado en $\R^{N}$,  $\alpha, \beta>0$ están fijos y donde $g\in \L{\infty}$. Este problema de discontinuidad es en cierto sentido canónico pues involucra dos objetos clásicos en matemáticas: la integral de Dirichlet y el funcional de área. Haciendo tender $\beta\to \infty$ el problema converge (en el sentido variacional de $\Gamma$-convergencia \index{$\Gamma$-convergencia}, ver \cite{ambrosio2000functions}) a un problema de partición óptima.

Al igual que en el problema  de partición óptima, se puede reducir las variables $K$ y $u$ a una sola. Por ejemplo, si conocemos el conjunto $K$ entonces $u$ será la solución del problema variacional en el espacio de Sobolev $W^{1,2}(\Omega\setminus K)$:
\begin{align*}
\min_{u}\set*{\int_{\Omega\setminus K}\left [ \abs{\nabla{u}}^{2}+\alpha(u-g)^{2}\right ]dx \given u\in W^{1,2}(\Omega\setminus K)}.
\end{align*}
Sin embargo, no es fácil minimizar $J$ por el método directo del cálculo de variaciones pues no existe una topología en los conjuntos cerrados que asegure la compacidad de las sucesiones minimizantes y la semicontinuidad inferior de las medidas de Hausdorff. Por otro lado, se podría pensar en $u$ como definido en todo $\Omega$ y permitir que sea discontinua a lo largo de conjuntos $(N-1)$-dimensionales, es decir $u\in \BV \cap W^{1,\infty}(\Omega\setminus K)$, con $K=\overline{S}_{u}$. Sin embargo $\BV$ es un espacio demasiado grande y existen funciones que son densas en $\L2$ tales que $\nabla{u}=0$ en casi todas partes respecto de la medida $\Lm{N}$ y $\Hm{N-1}(S_{u})=0$. Por lo tanto, $J(\overline{S}_{u}, u)$ se reducirá al término $\alpha\int_{\Omega}\abs{u-g}^{2}dx$ para estas funciones, y este funcional puede ser arbitrariamente pequeño sin importar el $g$ fijado. En lugar de esto, es posible dar una formulación débil significativa en $SBV$ del funcional $J$. Considerando $u\in \SBV$, entonces tenemos la siguiente formulación débil del funcional $J$:
\begin{align*}
\mathcal{F}(u)=\int_{\Omega\setminus S_{u}}\left [ \abs{\nabla{u}}^{2}+\alpha(u-g)^{2}\right ]dx+\beta \Hm{N-1}(S_{u}).
\end{align*}
Aunque la existencia de un minimizador de $\mathcal{F}$ en $SBV$ puede probarse directamente usando  los teoremas de compacidad y semicontinuidad inferior desarrollados en el capítulo anterior, esto no proporciona necesariamente un par  minimizador de $J$ pues $S_{u}$ no es un conjunto cerrado para una función $SBV$ genérica y su cierre puede ser incluso todo $\Omega$. Es por ello que en las siguientes secciones probaremos que el cierre de $S_{u}$ no es mucho más grande que el propio $S_{u}$.

\section{Comportamiento asintótico de una sucesión en $SBV$}
En esta sección estudiaremos el comportamiento de sucesiones $(u_{h})$ que aproximan al minimizante de una generalización del funcional descrito anteriormente:
\begin{align*}
\mathcal{F}(u)=\int_{\Omega\setminus S_{u}}\left [ \abs{\nabla{u}}^{2}+\alpha(u-g)^{2}\right ]dx+\beta \Hm{N-1}(S_{u}),
\end{align*}
de forma que el término de área $\Hm{N-1}(S_{u_{h}})$ se hace cero en el límite. Para ello, primero estudiamos dos lemas sobre acotación de energía que usaremos a lo largo de la sección para estudiar el comportamiento de las sucesiones minimizadoras. Posteriormente, daremos dos resultados que describen el límite del comportamiento de sucesiones que minimizan el funcional en $SBV$ tal que el límite del término de área es cero. Finalmente, veremos condiciones suficientes para la existencia de límites aproximados en un punto dado. 

A continuación, introducimos la notación que usaremos a lo largo del capítulo y damos los lemas correspondientes para la acotación de la energía. Sea $f:\R^{N}\to \R$ una función convexa tal que:
\begin{align*}
\begin{array}{ll}
L^{-1}\abs{z}^{p}\leq f(z)\leq L\abs{z}^{p}, & \forall z\in \R^{N},
\end{array}
\end{align*}
para algún $L>0$ y $p>1$. Si $u\in SBV_{loc}(\Omega)$ y $c>0$, consideramos para cada conjunto de Borel $E\subset \Omega$ el funcional:
\begin{align*}
F(u,c,E)=\int_{E}f(\nabla{u})dx+c\Hm{N-1}(S_{u}\cap E).
\end{align*}
Para el caso $c=1$ simplificaremos la notación: $F(u,E)=F(u,1,E)$.
\begin{defi}[Minimizadores locales]\label{defi:minimizadores locales} \index{minimizador!local}
Se dice que $u\in\SBVloc$ es un minimizador local de $F(u,c,E)$ en $\Omega$ si:
\begin{align}\label{defi:minimizadores locales:eq:1}
\begin{array}{ll}
F(u,c,A)<\infty, & \forall A\subset \subset \Omega,
\end{array}
\end{align}
y si $F(u,c,A)\leq F(v,c,A)$ para cualquier $v\in \SBVloc$ tal que
\begin{align*}
\set*{v\not = u}\subset\subset A\subset\subset \Omega.
\end{align*}

Análogamente, se dirá que $u\in W^{1,p}_{loc}(\Omega)$ es un minimizador en $\Omega$ del funcional:
\begin{align*}
F_{0}(u, E)=\int_{E}f(\nabla{u})dx,
\end{align*}
si $F_{0}(u, A)\leq F_{0}(v,A)$ para cualquier $v\in W^{1,p}_{loc}(\Omega)$ tal que $\set{v\not = u}\subset\subset A\subset\subset\Omega$. 
\end{defi}
La siguiente definición proporciona un modo de acotar lo lejos que $u$ está de ser un mínimo.
\begin{defi}[Desviación con respecto del mínimo]\label{defi:desviación con respecto del mínimo} \index{desviación}
Sea $c>0$. La desviación con respecto del mínimo $\Dev{u,c}$ de una función $u\in \SBVloc$ que satisface la ecuación \ref{defi:minimizadores locales:eq:1} se define como el mínimo $\lambda\in [0,\infty]$ tal que:
\begin{align*}
\int_{A}f(\nabla{u})dx+c\Hm{N-1}(S_{u}\cap A)\leq \int_{A} f(\nabla{v})dx+c\Hm{N-1}(S_{v}\cap A)+\lambda,
\end{align*}
para cualquier $v\in \SBVloc$ que satisface $\set{v\not = u}\subset\subset A\subset\subset \Omega$.
\end{defi}
Al igual que antes cuando $c=1$ escogeremos la siguiente notación: $\Dev{u}=\Dev{u,1}$. Claramente $\Dev{u,c}=0$ si y solo si $u$ es un minimizador local de $F(u,c,\Omega)$ en $\Omega$.

En el siguiente lema compararemos la energía de $u$ en la bola de radio $\varrho'$ ($0<\varrho<\varrho'$) con la energía de $z=v\chi_{\varrho}+u\chi_{\varrho'\setminus \varrho}$. En la comparación de la energía del salto, el área de $\set{\tilde{v}\not =\tilde{u}}\cap \partial B_{\varrho}$ aparecerá debido a que $z$ no es aproximadamente continuo en este conjunto.
\begin{lema}\label{lema:estimacion de energia 1}
Sea $u,v\in SBV_{loc}(B_{r})$, $\varrho<\varrho'<r$. Si $\Hm{N-1}(S_{v}\cap \partial B_{\varrho})=0$ y $F(u,c, B_{\varrho'})<\infty$, $F(v,c, B_{\varrho'})<\infty$ entonces:
\begin{align}
F(u,c,B_{\varrho})&\leq F(v,c,B_{\varrho})+c\Hm{N-1}(\set{\tilde{u}\not =\tilde{v}}\cap \partial B_{\varrho})+Dev(u,c, B_{\varrho'}),\label{lema:estimacion de energia 1:eq:1}\\
Dev(v,c,B_{\varrho})&\leq F(v, c, B_{\varrho})-F(u, c, B_{\varrho})+c\Hm{N-1}(\set{\tilde{u}\not = \tilde{v}}\cap \partial B_{\varrho})\label{lema:estimacion de energia 1:eq:2}\\
&+Dev(u, c, B_{\varrho'}).\nonumber
\end{align}
\end{lema}
\begin{proof}
Sea $0<\varrho<\varrho'$ y $z=v\chi_{\varrho}+u\chi_{\varrho'\setminus \varrho}$, entonces se puede comprobar que $S_{z}\cap \partial B_{\varrho}\subset (S_{u}\cup S_{v}\cup\set{\tilde{u}\not =\tilde{v}})\cap \partial B_{\varrho}$, por lo tanto por hipótesis y por la definición de desviación se obtiene que:
\begin{align*}
F(u,c, B_{\varrho'})&\leq F(z,c,B_{\varrho'})+Dev(u,c,B_{\varrho'})\leq F(v, c,B_{\varrho})+F(u,c, B_{\varrho'}\setminus \overline{B}_{\varrho})
\\&+c\Hm{N-1}(S_{z}\cap \partial B_{\varrho})+Dev(u,c,B_{\varrho'})\leq F(v,c,B_{\varrho})+F(u,c,B_{\varrho'}\setminus B_{\varrho})
\\&+c\Hm{N-1}(\set{\tilde{u}\not = \tilde{v}}\cap \partial B_{\varrho})+Dev(u,c,B_{\varrho'}),
\end{align*}
y por lo tanto se sigue \ref{lema:estimacion de energia 1:eq:1}. Consideremos ahora cualquier $w\in SBV(B_{\varrho'})$ tal que $\set*{w\not = v}\subset\subset B_{\varrho}$ y el conjunto:
\begin{align*}
w'=w\chi_{B_{\varrho}}+u\chi_{B_{\varrho'}\setminus B_{\varrho}}.
\end{align*}
Como antes:
\begin{align*}
F(v, c, B_{\varrho})&\leq F(v, c, B_{\varrho})+F(w', c, B_{\varrho'})+Dev(u,c,B_{\varrho'})-F(u,c,B_{\varrho'})
\\&\leq F(w, c, B_{\varrho})+[F(v, c, B_{\varrho})-F(u,c,B_{\varrho})+Dev(u,c,B_{\varrho'})\\&+c\Hm{N-1}(\set*{\tilde{u}\not=\tilde{v}}\cap \partial B_{\varrho})],
\end{align*}
y por la definición de desviación con respecto del mínimo se sigue \ref{lema:estimacion de energia 1:eq:2}.
\end{proof}

En el siguiente lema nuevamente damos una cota para la energía. Esta vez se comparará $u$ con la función $\varphi v+(1-\varphi)u$, donde $\varphi$ es una función cut-off en $B_{\varrho'}$, es decir, $\chi_{B_{\varrho'}}\le \varphi \le \chi_{B_{\varrho}}$. En este caso, la diferencia de energía se obtendrá usando la norma $L^{p}$ de $u-v$ en vez de la medida $\Hm{N-1}$ de $\set{\tilde{u}\not = \tilde{v}}\cap \partial B_{\varrho}$. 
\begin{lema}\label{lema:estimacion de energia 2}
Existe una constante $\gamma(p,L)$ tal que si $u,v\in SBV_{loc}(B_{r})$, $\varrho<\varrho'<r$ y si $F(u,c,B_{\varrho'})<\infty$ entonces:
\begin{align*}
F(u,c,B_{\varrho'})&\leq F(v,c,B_{\varrho})+\gamma [F(u,c,B_{\varrho'}\setminus B_{\varrho})+F(v,c,B_{\varrho'}\setminus B_{\varrho})]
\\& +\dfrac{\gamma}{(\varrho'-\varrho)^{p}}\int_{B_{\varrho'}\setminus B_{\varrho}}\abs{u-v}^{p}dx+Dev(u,c,B_{\varrho'}).
\end{align*}
\begin{proof}
Sea $\eta\in C_{c}^{1}(B_{\varrho'})$ una función tal que $\eta=1$ en $B_{\varrho}$, $0\leq \eta\leq 1$, $\abs{\nabla{\eta}}\leq 2/(\varrho'-\varrho)$. Si $w=\eta v+(1-\eta)u$, entonces gracias a la hipótesis sobre el crecimiento de $f$ se obtiene:
\begin{align*}
F(u, c, B_{\varrho'})&\leq F(w, c, B_{\varrho'})+Dev(u,c,B_{\varrho'})
\\&\leq F(v,c,B_{\varrho})+\gamma(p,L)\int_{B_{\varrho'}\setminus B_{\varrho}}\left [ \abs{\nabla{u}}^{p}+\abs{\nabla{v}}^{p}+\dfrac{\abs{u-v}^{p}}{(\varrho'-\varrho)^{p}}\right ]dx
\\&+c\left [ \Hm{N-1}(S_{u}\cap B_{\varrho'}\setminus B_{\varrho} )+ \Hm{N-1}(S_{v}\cap B_{\varrho'}\setminus B_{\varrho})\right]+Dev(u,c,B_{\varrho'}).
\end{align*}
De esta desigualdad se sigue inmediatamente el lema.
\end{proof}
\end{lema}

La siguiente proposición es consecuencia de la desigualdad de Poincaré y del teorema de compacidad para funciones en $SBV$. Recordamos que si $B$ es una bola y $u\in SBV(B)$ entonces denotamos por $\overline{u}$ la función acotada $(u\wedge\tau^{+}(u,B))\vee \tau^{-}(u,B)$, donde $\tau^{-}(u,B)$, $\tau^{+}(u,B)$ están definidos en \ref{eq:1:taus}.
\begin{prop}\label{teo:comportamiento asintótico de una sucesión en SBV}
Sea $B\subset \R^{N}$ una bola y sea $(u_{h})\subset SBV(B)$ una sucesión tal que:
\begin{align*}
\begin{array}{ll}
\displaystyle
\sup_{h\in \N}\int_{B}f(\nabla{u_{h}})dx<\infty, &  \displaystyle \lim_{h\to \infty}\Hm{N-1}(S_{u_{h}})=0,
\end{array}
\end{align*}
y $m_{h}$ las medianas de $u_{h}$ en $B$. Entonces existe una subsucesión $(u_{h(j)})$ y una función $u\in W^{1,p}(B)$ tal que las funciones $\overline{u}_{h(j)}-m_{h(j)}$ convergen a $u$ en $L^{p}(B)$ y:
\begin{align}\label{teo:comportamiento asintótico de una sucesión en SBV:1}
\int_{B}f(\nabla{u})dx\leq \lim_{j\to \infty}\inf\int_{B}f(\nabla{\overline{u}_{h(j)}})dx.
\end{align}
\end{prop}
\begin{proof}
Por simplicidad supongamos que $1<p<N$. Por la desigualdad de Poincaré del teorema \ref{teo:Desigualdad de tipo Poincaré en SBV} y por hipótesis, se tiene para $h$ suficientemente grande:
\begin{align*}
\left( \int_{B}\abs{\overline{u}_{h}-m_{h}}^{p^{*}}dx\right)^{1/p^{*}}\leq c(N,p)\left (\int_{B}\abs{\nabla{u_{h}}}^{p}dx \right  )^{1/p}\leq c\left (\int_{B} f(\nabla{u_{h}}) dx\right )^{1/p},
\end{align*}
y que $\abs{D\overline{u}_{h}}(B)\leq 2\int_{B}\abs{\nabla{u_{h}}}dx$. Luego por el teorema de compacidad para funciones $BV$ se tiene que existe una subsucesión $v_{j}=\overline{u}_{h(j)}-m_{h(j)}$ que converge a $u\in BV(B)$ fuertemente en $L^{p}(B)$. Para cualquier $M>0$ y para cualquier función medible $v$ consideremos $v^{M}=(-M)\vee u \wedge M$. Por el teorema \ref{teo:compacidad en SBV} sabemos que $u^{M}\in SBV(B)$. Además el teorema \ref{teo:cierre de SBV} implica:
\begin{align}\label{proof:comportamiento asintótico de una sucesión en SBV:eq:1}
&\int_{B}f(\nabla{u^{M}})dx\leq \lim_{j\to \infty}\inf\int_{B}f(\nabla{v_{j}^{M}})dx\leq \lim_{j\to \infty}\inf\int_{B} f(\nabla{\overline{u}_{h(j)}})dx,\\
&\Hm{N-1}(S_{u^{M}})\leq \lim_{j\to \infty}\inf\Hm{N-1}(S_{v_{j}^{M}})\leq \lim_{j\to \infty}\Hm{N-1}(S_{u_{h(j)}})=0.\nonumber
\end{align}
Por la ecuación \ref{eq:1:relacion espacio de sobolev vs sbv} se deduce que $u^{M}\in W^{1,1}(B)$ para cualquier $M>0$ y por \ref{proof:comportamiento asintótico de una sucesión en SBV:eq:1}, $(\nabla{u^{M}})$ es equiacotado en $L^{p}$. Luego tomando límite $M\to\infty$ se obtiene que $u\in W^{1,p}(B)$ y se concluye \ref{teo:comportamiento asintótico de una sucesión en SBV:1}.
\end{proof}

El siguiente resultado describe el comportamiento asintótico de una sucesión $(u_{h})$ en $SBV$ cuando la desviación con respecto del mínimo tiende a cero y el término de área se anula en el límite. 
\begin{teo}\label{teo:comportamiento asintótico 2 de una sucesión en SBV}
Sea $(u_{h})\subset SBV(B_{r})$, $m_{h}$ una mediana  de $u_{h}$ en $B_{r}$, $(c_{h})\subset (0,\infty)$. Supongamos que:
\begin{enumerate}[(a)]\label{teo:comportamiento asintótico 2 de una sucesión en SBV:hipetesis}
\item $\lim_{h\to\infty}\Hm{N-1}(S_{u_{h}})=0$, \label{teo:comportamiento asintótico 2 de una sucesión en SBV:hipetesis:a}
\item $\sup_{h\in \N} F(u_{h},c_{h}, B_{r})<\infty$,\label{teo:comportamiento asintótico 2 de una sucesión en SBV:hipetesis:b}
\item $\lim_{h\to \infty}Dev(u_{h}, c_{h}, B_{r})=0$,\label{teo:comportamiento asintótico 2 de una sucesión en SBV:hipetesis:c}
\item $\lim_{h\to \infty}(u_{h}-m_{h})=u\in W^{1,p}(B_{r})$ en c.t.p. de $B_{r}$ respecto de $\Lm{N}$.\label{teo:comportamiento asintótico 2 de una sucesión en SBV:hipetesis:d}
\end{enumerate}
Entonces $u$ es un minimizador local del funcional $v\mapsto \int_{B_{r}}f(\nabla{v})dx$ en $W^{1,p}(B_{r})$ y:
\begin{align*}
\begin{array}{ll}
\displaystyle
\lim_{h\to \infty} F(u_{h}, c_{h}, B_{\varrho})=\int_{B_{\varrho}}f(\nabla{u})dx,& \forall \varrho\in (0,r).
\end{array}
\end{align*}
\end{teo}
\begin{proof}
Como las funciones $\varrho\mapsto F(u_{h}, c_{h}, B_{\varrho})$ son crecientes e equiacotadas podemos asumir, extrayendo quizás una subsucesión que (criterio de selección de Helly):
\begin{align*}
\begin{array}{ll}
\displaystyle \alpha(\varrho)=\lim_{h\to \infty}F(u_{h}, c_{h}, B_{\varrho}), & \text{existe para casi todo}\; \varrho\in(0,r)\;\text{respecto de}\; \Lm{1},
\end{array}
\end{align*}
para alguna función creciente a valores reales $\alpha$, y que $c_{\infty}=\lim_{h}c_{h}$ existe en $[0,\infty]$.

De las hipótesis (\ref{teo:comportamiento asintótico 2 de una sucesión en SBV:hipetesis:a}), (\ref{teo:comportamiento asintótico 2 de una sucesión en SBV:hipetesis:b}) y de la proposición anterior se deduce que $\overline{u}_{h}-m_{h}$ converge a $u$ en $L^{p}(B_{r})$ y que:
\begin{align}\label{proof:comportamiento asintótico 2 de una sucesión en SBV:eq:1}
\begin{array}{ll}
\displaystyle
\int_{B_{\varrho}}f(\nabla{u})dx\leq \lim_{h\to \infty}\inf F(\overline{u}_{h},c_{h},B_{\varrho}),& \forall\varrho\in (0,r).
\end{array}
\end{align}
Integrando respecto de $\varrho$ y como $\set{u\not = \tilde{u}}=\set{u>\tau^{+}(u,B)}\cup \set{u<\tau^{-}(u,B)}$, por la definición de $\tau^{\pm}$ entonces $\set*{u\not = \tilde{u}}\leq 2\left (2\gamma_{5}\Hm{N-1}(S_{u}\cap B) \right )^{N/(N-1)}$, luego se tiene que:
\begin{align*}
c_{h}\int^{r}_{0}\Hm{N-1}\left ( \set*{\tilde{u}_{h}\not = \tilde{\overline{u_{h}}}}\cap \partial B_{\varrho}\right ) d\varrho&=c_{h}\abs*{\set*{u_{h}\not = \overline{u}_{h}}\cap B_{r}}
\\&\leq 2c_{h}(2\gamma_{5}\Hm{N-1}(S_{u_{h}}\cap B_{r}))^{1^{*}}.
\end{align*}
Se puede ver fácilmente que $\lim_{h\to \infty}2c_{h}(2\gamma_{5}\Hm{N-1}(S_{u_{h}}\cap B_{r}))^{1^{*}}=0$. Esto es inmediato de la hipótesis (\ref{teo:comportamiento asintótico 2 de una sucesión en SBV:hipetesis:a}) si $c_{\infty}<\infty$, mientras que si $c_{\infty}=\infty$ es consecuencia de la hipótesis (\ref{teo:comportamiento asintótico 2 de una sucesión en SBV:hipetesis:b}), pues $2c_{h}(2\gamma_{5}\Hm{N-1}(S_{u_{h}}\cap B_{r}))^{1^{*}}\leq 2c_{h}^{-1/(N-1)}(2\gamma_{5}M)^{1^{*}}$, donde $M$ es el supremo en (\ref{teo:comportamiento asintótico 2 de una sucesión en SBV:hipetesis:b}). Por lo tanto, considerando quizás otra subsucesión se tiene:
\begin{align*}
\lim_{h\to\infty}c_{h}\Hm{N-1}\left ( \set*{\tilde{u}_{h}\not = \tilde{\overline{u}}_{h}}\cap \partial B_{\varrho}\right )=0,
\end{align*}
para casi todo $\varrho\in (0,r)$ con respecto $\Lm{N-1}$. Como para cualquier $h\in N$ y cualquier $\varrho\in (0,r)$ respecto de la medida $\Lm{1}$ se tiene $\Hm{N-1}(S_{\overline{u}_{h}}\cap \partial B_{\varrho})=0$, por el lema \ref{lema:estimacion de energia 1} se deduce:
\begin{align*}
F(\overline{u}_{h},c_{h},B_{\varrho})&\leq F(u_{h}, c_{h}, B_{\varrho})
\\&\leq F(\overline{u}_{h},c_{h},B_{\varrho})+c_{h}\Hm{N-1}\left( \set*{\tilde{u}_{h}\not =\tilde{\overline{u}}_{h}}\cap \partial B_{\varrho} \right ) + Dev(u_{h}, c_{h}, B_{r}).
\end{align*}
Por lo tanto por la hipótesis (\ref{teo:comportamiento asintótico 2 de una sucesión en SBV:hipetesis:c}) se concluye que para casi todo $\varrho\in (0, r)$ con respecto $\Lm{1}$ se tiene que:
\begin{align}\label{proof:comportamiento asintótico 2 de una sucesión en SBV:eq:2}
\lim_{h\to \infty} F(\overline{u}_{h}, c_{h}, B_{\varrho})=\alpha(\varrho).
\end{align}
Por esto y por \ref{lema:estimacion de energia 1:eq:2} se concluye también que:
\begin{align*}
\begin{array}{ll}
\lim_{h\to \infty}Dev(\overline{u}_{h}, c_{h}, B_{\varrho})=0, &\forall \varrho\in (0,r).
\end{array}
\end{align*}
Ahora se puede probar que $u$ es un minimizador local. Sea $v\in  W^{1,p}(B_{r})$ una función tal que $\set*{v\not = u}\subset\subset B_{r}$ y sea $\varrho'\in (0,r)$ tal que \ref{proof:comportamiento asintótico 2 de una sucesión en SBV:eq:2} es cierto, $\alpha$ es continua en $\varrho'$ y $\set*{v\not= u}\subset\subset B_{\varrho'}$. Escogiendo $\varrho<\varrho'$ de modo que \ref{proof:comportamiento asintótico 2 de una sucesión en SBV:eq:2} sea cierto, por el lema \ref{lema:estimacion de energia 1} se tiene:
\begin{align*}
F(\overline{u}_{h},c_{h}, B_{\varrho})&\leq \int_{B_{\varrho}}f(\nabla{v})dx+Dev(\overline{u}_{h}, c_{h}, B_{\varrho'})
\\&+\gamma\left [ F(\overline{u}_{h}, c_{h}, B_{\varrho'}\setminus B_{\varrho})+F_{0}(v,B_{\varrho'}\setminus B_{\varrho})\right]+\gamma \int_{B_{\varrho'}\setminus B_{\varrho}}\dfrac{\abs{\overline{u}_{h}-m_{h}-v}^{p}}{(\varrho'-\varrho)^{p}}dx.
\end{align*}
Por lo tanto, haciendo tender $h\to \infty$ se obtiene:
\begin{align*}
\alpha(\varrho)\leq &\int_{B_{\varrho}}f(\nabla{v})dx+\gamma \left [ \alpha(\varrho')-\alpha(\varrho)+\int_{B_{\varrho'}\setminus B_{\varrho}}f(\nabla{v})dx\right]\\&+\dfrac{\gamma}{(\varrho'-\varrho)^{p}}\int_{B_{\varrho'}\setminus B_{\varrho}}\abs{u-v}^{p}dx.
\end{align*}
A partir de esta desigualdad haciendo tender $\varrho\to \varrho'$ y observando que $u=v$ en la corona $B_{\varrho'}\setminus B_{\varrho}$ si $\varrho$ es suficientemente cercano a $\varrho'$, se tiene que:
\begin{align*}
\alpha(\varrho')\leq \int_{B_{\varrho'}}f(\nabla{v})dx.
\end{align*}
Escogiendo $v=u$ en la desigualdad anterior y teniendo en cuenta \ref{proof:comportamiento asintótico 2 de una sucesión en SBV:eq:1} vemos que $\alpha(\varrho')$ coincide con $\int_{\varrho'}f(\nabla{u})dx$. En particular, la desigualdad anterior dice que $u$ es un mínimo local.

Finalmente, si $\varrho\in (0,r)$ no es un punto de continuidad de $\alpha$, por monotonicidad se puede acotar $\limsup_{h\to \infty} F(u_{h}, c_{h}, B_{\varrho})\leq \int_{B_{\varrho}}f(\nabla{u})dx$, para cualquier punto de continuidad $\varrho'\in (\varrho,r)$ de $\alpha$. Haciendo tender $\varrho'\to \varrho$ se ve entonces que:
\begin{align*}
\limsup_{h\to \infty} F(u_{h}, c_{h}, B_{\varrho})\leq \int_{B_{\varrho}}f(\nabla{u})dx,
\end{align*}
lo que en conjunción con \ref{proof:comportamiento asintótico 2 de una sucesión en SBV:eq:1} permite concluir.
\end{proof}

La aplicación de la desigualdad de Poincaré para funciones $SBV$ da lugar a una condición suficiente para la existencia del límite aproximado en un punto.
\begin{teo}\label{teo:condicion existencia de limite aproximado}
Sea $q>1$, $u\in \SBVloc$ y $x\in \Omega$. Si:
\begin{align*}
\lim_{\varrho\to 0}\varrho^{-(N-1)}\left [ \int_{B_{\varrho}(x)}\abs{\nabla{u}}^{p}dy+\Hm{N-1}\left (S_{u}\cap B_{\varrho}(x)\right)\right]=0,
\end{align*} 
y si se tiene además:
\begin{align}\label{teo:condicion existencia de limite aproximado:eq:1}
\limsup_{\varrho\to 0}\fint_{B_{\varrho}(x)}\abs{u(y)}^{q}dy<\infty,
\end{align}
entonces $x\not \in S_{u}$.
\end{teo}
\begin{proof}
No es restrictivo suponer que $p<N$, $x=0$. Por hipótesis existe un $\varrho_{0}>0$ tal que para cualquier $\varrho\leq \varrho_{0}$ se verifica:
\begin{align}\label{proof:condicion existencia de limite aproximado:eq:1}
\int_{B_{\varrho}}\abs{\nabla{u}}^{p}dy+\Hm{N-1}(S_{u}\cap B_{\varrho})<\dfrac{1}{2\gamma_{5}}\left ( \dfrac{1}{4}\abs{B_{\varrho}}\right )^{1/1^{*}}.
\end{align}
Siguiendo la notación de la sección \ref{section:Teorema de Poincaré en SBV}, $m_{\varrho}$ será una mediana de $u$ en $B_{\varrho}$ y $\overline{u}_{\varrho}=(u\wedge \tau^{+}(u, B_{\varrho}))\vee \tau^{-}(u,B_{\varrho})$. Probamos ahora que $m_{\varrho}$ tiene un límite real $z$ cuando $\varrho\to 0$. Para probar esto fijamos un $\alpha\in \R$ tal que $1/2<\alpha^{N}<1$ y los radios $s,r$ tales que $\alpha r\leq s<\alpha_{0}$. Entonces se tendrá:
\begin{align}\label{proof:condicion existencia de limite aproximado:eq:2}
\begin{array}{ll}
\tau^{-}(u,B_{s})\leq m_{r}\leq \tau^{+}(u, B_{s}),&\tau^{-}(u,B_{r})\leq m_{s}\leq \tau^{+}(u, B_{r}).
\end{array}
\end{align}
A continuación, solo probamos que  $\tau^{-}(u,B_{r})\leq m_{s}$ pues las otras desigualdades se demuestran de la misma manera. Supongamos que existe un $t$ tal que $m_{s}<t<\tau^{-}(u,B_{r})$, luego por la ecuación \ref{proof:condicion existencia de limite aproximado:eq:1} y por la definición de $\tau^{-}$ se tiene que:
\begin{align*}
\abs{\set{u<t}\cap B_{s}}\leq \abs{\set{u<t}\cap B_{r}}\leq (2\gamma_{5}\Hm{N-1}(S_{u}\cap B_{r}))^{1^{*}}<\frac{1}{4}\omega_{N}r^{N}<\frac{1}{2}\omega_{N}s^{N},
\end{align*}
lo que contradice la definición de mediana de $u$ en $B_{s}$. Definamos ahora para cualquier $y\in B_{s}$:
\begin{align*}
\overline{\overline{u}}(y)=(u(y)\wedge \tau^{+}(u,B_{r})\wedge\tau^{+}(u,B_{s}))\vee (\tau^{-}(u,B_{r})\vee \tau^{-}(u,B_{s})),
\end{align*}
y observemos que por \ref{proof:condicion existencia de limite aproximado:eq:2} se tiene:
\begin{align*}
\begin{array}{ll}
\abs{\overline{\overline{u}}-m_{s}}\leq \abs{\overline{u}_{s}-m_{s}},& \abs{\overline{\overline{u}}-m_{r}}\leq \abs{\overline{u}_{r}-m_{r}}.
\end{array}
\end{align*}
Por lo tanto, por la desigualdad de Poincaré para funciones en $SBV$ y por \ref{proof:condicion existencia de limite aproximado:eq:1} se obtiene:
\begin{align*}
\abs{m_{r}-m_{s}}&\leq \abs{B_{s}}^{-1/p^{*}}\left [ \left ( \int_{B_{s}}\abs{\overline{\overline{u}}-m_{s}}^{p^{*}}dy \right )^{1/p^{*}}+\left ( \int_{B_{s}} \abs{\overline{\overline{u}}-m_{r}}^{p^{*}}dy\right )^{1/p^{*}}  \right ]
\\&\leq  \abs{B_{s}}^{-1/p^{*}} \left [\left ( \int_{B_{s}}\abs{\overline{u}_{s}-m_{s}}^{p^{*}}dy\right )^{1/p^{*}}+\left ( \int_{B_{r}} \abs{\overline{u}_{r}-m_{r}}^{p^{*}}dy\right )^{1/p^{*}}  \right ]
\\&\leq c(N,p)r^{-N/p^{*}}\left [ \left( \int_{B_{s}}\abs{\nabla{u}}^{p}dy\right )^{1/p}+\left ( \int_{B_{r}}\abs{\nabla{u}}^{p}dy\right)^{1/p}\right ]
\\&\leq cr^{1-1/p}.
\end{align*}
Si $\varrho<r<\varrho_{0}$ y $k\in \N$ es tal que $\alpha^{k+1}\leq \varrho\leq \alpha^{k}r$, la cota anterior proporciona:
\begin{align*}
\abs{m_{\varrho}-m_{r}}\leq \abs{m_{\varrho}-m_{\alpha^{k}r}}+\sum_{i=0}^{k-1}\abs{m_{\alpha^{i+1}r}-m_{\alpha^{i}r}}\leq c\sum^{k}_{i=0}(\alpha^{i}r)^{1-1/p}\leq cr^{1-1/p},
\end{align*}
con $c$ que depende de $\alpha,N,p$. Esta desigualdad implica que $m_{\varrho}$ converge a un número real $z$ cuando $\varrho\to 0$.

Probamos ahora que $z$ es el límite aproximado de $u$ en $x$. Primero vemos que por la desigualdad de Poincaré en $SBV$ se tiene que: 
\begin{align*}
\int_{B_{\varrho}}\abs{\overline{u}_{\varrho}-m_{\varrho}}^{p^{*}}dx\leq c\left ( \int_{B_{\varrho}}\abs{\nabla{u}}^{p} dx\right )^{p^{*}/p},
\end{align*}
para $\varrho>0$ suficientemente pequeño, luego:
\begin{align}\label{proof:condicion existencia de limite aproximado:eq:3}
\int_{B_{\varrho}}\abs{\overline{u}_{\varrho}-m_{\varrho}}^{p^{*}}dx=o(\varrho^{N}),
\end{align} 
puesto que $mp^{*}/p>N$. Como $\Hm{N-1}(S_{u}\cap B_{\varrho})$ es infinitesimal, y además como $\set*{u\not = \tilde{u}}=\set*{u>\tau^{+}(u,B)}\cup \set*{u<\tau^{-}(u, B)}$ por definición se tiene $\abs{\set*{u\not = \tilde{u}}}\leq 2\left ( 2\gamma_{5}\Hm{N-1}(S_{u}\cap B)\right )^{N/(N-1)}$, luego:
\begin{align*}
\abs*{\set*{\overline{u}_{\varrho}\not = u}\cap B_{\varrho}}=o(\varrho^{N}),
\end{align*}
y la ecuación \ref{teo:condicion existencia de limite aproximado:eq:1} da:
\begin{align}\label{proof:condicion existencia de limite aproximado:eq:4}
\int_{B_{\varrho}}\abs*{u-\overline{u}_{\varrho}}dx=o(\varrho^{N}).
\end{align}
Ahora por \ref{proof:condicion existencia de limite aproximado:eq:3}, \ref{proof:condicion existencia de limite aproximado:eq:4} y la desigualdad de Hölder se sigue que:
\begin{align*}
\int_{B_{\varrho}}\abs{u-z}dx &\leq \int_{B_{\varrho}}\abs{u-m_{\varrho}}dx+\abs{m_{\varrho}-z}\omega_{N}\varrho^{N}
\\& \leq \int_{B_{\varrho}}\abs{\overline{u}_{\varrho}-m_{\varrho}}dx+\int_{B_{\varrho}}\abs{u-\overline{u}_{\varrho}}dx+o(\varrho^{N})=o(\varrho^{N}),
\end{align*}
y esto prueba que $z$ es el límite aproximado de $u$ en $x$.
\end{proof}

\section{Cota inferior para la densidad}

El punto fundamental de la teoría de existencia y regularidad de los problemas de discontinuidad libre lo desarrollan E. De Giorgi, M. Carriero y A. Leaci en \cite{de1989existence}. De Giorgi, Carriero y Leaci prueban que si $u$ es un minimizador del problema variacional:
\begin{align*}
\begin{array}{ll}
\displaystyle
\mathcal{F}(u)=\int_{\Omega\setminus S_{u}}\left [ \abs{\nabla{u}}^{2}+\alpha(u-g)^{2}\right]dx+\beta \Hm{N-1}(S_{u}),& u\in \SBV,
\end{array}
\end{align*}
entonces para cualquier $x\in S_{u}$ y para cualquier bola $B_{\varrho}(x)\subset \Omega$, con $\varrho$ suficientemente pequeño, se tendrá la siguiente cota inferior para la densidad:
\begin{align*}
\Hm{N-1}(S_{u}\cap B_{\varrho}(x))\geq \vartheta_{0}\varrho^{N-1},
\end{align*}
donde $\vartheta_{0}=\vartheta_{0}(N)$ es una constante estrictamente positiva que depende únicamente de la dimensión del espacio ambiente. En esta sección probaremos esta desigualdad mediante un argumento clásico de rescaldo. Una consecuencia inmediata de esta cota es que si $u\in SBV(\Omega)$ entonces gracias al resultado anterior se concluye:
\begin{align*}
\Hm{N-1}(\Omega \cap \overline{S}_{u}\setminus S_{u})=0.
\end{align*}
Finalmente tras esto probaremos en esta sección que $u$ tiene un representante $\tilde{u}\in C^{1}(\Omega\setminus \overline{S}_{u})$ y tal que el par $(\overline{S}_{u}, \tilde{u})$ es un minimizador del funcional de Mumford-Shah $J$.

Como hemos dicho en la introducción, en esta sección probamos la existencia de un minimizador para el funcional:
\begin{align*}
J(K,u)=\int_{\Omega\setminus K}f(\nabla{u})dx+\alpha \int_{\Omega\setminus K}\abs{u-g}^{q}dx+\beta \Hm{N-1}(K\cap \Omega),
\end{align*}
donde $\alpha,\beta>0,q\geq 1, g\in \L{q}\cap \L{\infty}$ y $f:\R^{N}\to \R$ es una función convexa tal que para todo $z\in \R^{N}$ y $\varphi\in \Cc{1}$ se verifican las siguientes tres condiciones:
\begin{enumerate}[(H1)]\label{eq:hipotesis sobre funcional}
\item $\begin{array}{ll}
L^{-1}\abs{z}^{p}\leq f(z)\leq L\abs{z}^{p}, & p>1,L\geq 1.
\end{array}$\label{eq:hipotesis sobre funcional:H1}
\item $\begin{array}{ll}
f(tz)=t^{p}f(z),& \forall t>0,
\end{array}$\label{eq:hipotesis sobre funcional:H2}
\item $\begin{array}{ll} \displaystyle \int_{\Omega}f(z+\nabla{\varphi})dx \geq \int_{\Omega}\left [ f(z)+\nu(\abs{z}^{2}+\abs{\nabla{\varphi}}^{2})^{(p-2)/2}\abs{\nabla{\varphi}}^{2}\right ]dx,& \nu>0.\end{array}$\label{eq:hipotesis sobre funcional:H3}
\end{enumerate}

\begin{nota}[Condición de elipticidad]\index{condición de elipticidad}
La hipótesis (H\ref{eq:hipotesis sobre funcional:H3}) se satisface también si $f$ es una función $C^{2}(\R^{N})$ y si verifica además la condición de elipticidad clásica:
\begin{align}
\begin{array}{ll}
\displaystyle
\sum_{i,j=1}^{N}\diffp{f(z)}{z_{i},z_{j}}\xi_{i}\xi_{j}\geq \nu \abs{z}^{p-2}\abs{\xi}^{2}, & p\geq 2,
\end{array}
\end{align}
para cualquier $z,\xi\in \R^{N}$.
Sin embargo, la condición (H\ref{eq:hipotesis sobre funcional:H3}) que implica que $f$ es estrictamente convexa, no requiere que $f$ sea diferenciable en todas partes.
\end{nota}

\begin{teo}\label{teo:regularity propierties functional f}
Si $f$ satisface la hipótesis (H\ref{eq:hipotesis sobre funcional:H1}), (H\ref{eq:hipotesis sobre funcional:H3}) y $u\in W_{loc}^{1,p}(\Omega)$ es un minimizador local del funcional:
\begin{align*}
w\mapsto \int_{\Omega}f(\nabla{w})dx,
\end{align*}
entonces $u$ es localmente Lipschitz en $\Omega$ y se tiene:
\begin{align}\label{teo:regularity propierties functional f:eq:1}
\sup_{x\in B_{r/2}(x_{0})}\abs{\nabla{u}}^{p}\leq C_{0}\fint_{B_{r}(x_0)}\abs{\nabla{u}}^{p}dx,
\end{align}
para cualquier bola $B_{r}(x_{0})\subset \Omega$, donde $C_{0}$ depende únicamente de $N,p,L$ y $\nu$.
\end{teo}
\begin{proof}
La prueba del teorema supera los objetivos del trabajo, la demostración se puede encontrar en el artículo \cite{fonseca1997regularity}.
\end{proof}

\begin{nota}[Rescalado]\index{rescaldado}
Es fácil comprobar que $u\in SBV, B_{\varrho}(x_{0})\subset \Omega$ y que:
\begin{align*}
u_{\varrho}(y)=\varrho^{(1-p)/p}u(x_{0}+\varrho y),
\end{align*}
entonces $u_{\varrho}\in SBV(\Omega_{\varrho})$, donde $\Omega_{\varrho}=\varrho^{-1}(\Omega -x_{0})$, y:
\begin{align*}
\Hm{N-1}(S_{u_{\varrho}}\cap B_{\sigma})=\varrho^{-(N-1)}\Hm{N-1}(S_{u}\cap B_{\sigma\varrho}(x_{0})),
\end{align*}
para todo $0<\sigma\leq 1$. Además, si $f$ satisface la hipótesis (H\ref{eq:hipotesis sobre funcional:H2}) tendremos también:
\begin{align*}
\int_{B_{\sigma}}f(\nabla{u_{\varrho}})dy=\varrho^{-(N-1)}\int_{B_{\sigma\varrho}(x_{0})}f(\nabla{u})dx,
\end{align*}
y por lo tanto:
\begin{align*}
F(u_{\varrho},c,B_{\sigma})&=\varrho^{-(N-1)}F(u_{\varrho},c,B_{\sigma\varrho}(x_{0})),\\
Dev(u_{\varrho},c,B_{\sigma})&=\varrho^{-(N-1)}Dev(u_{\varrho},c,B_{\sigma\varrho}(x_{0})).
\end{align*}
\end{nota}

A la hora de probar la existencia de un minimizador del funcional $J$ podemos suponer que $f$ es una función homogénea, es decir, se verifica la condición (H\ref{eq:hipotesis sobre funcional:H2}). Entonces, remplazando $u$ por $\beta^{1/p}u$, podemos suponer siempre que el parámetro $\beta$ que aparece en $J$ es siempre igual a $1$. Por este motivo, de ahora en adelante consideraremos el problema:
\begin{align*}
J(K,u)=\int_{\Omega\setminus K}f(\nabla u) dx + \alpha \int_{\Omega\setminus K}\abs{u-g}^{q}dx+\Hm{N-1}(K\cap \Omega),
\end{align*}
para cualquier conjunto cerrado $K\subset \R^{N}$ y cualquier función $u\in W_{loc}^{1,p}(\Omega\setminus K)$.


Las propiedades de rescalado del funcional $F$ y de la desviación con respecto del mínimo nos permiten acotar superiormente el decaimiento de $F$ en bolas de pequeño tamaño.
\begin{lema}[Decaimiento]\label{lema:decaimiento}\index{decaimiento}
Sea $f:\R^{N}\to \R$ una función convexa que satisface (H\ref{eq:hipotesis sobre funcional:H1}), (H\ref{eq:hipotesis sobre funcional:H2}), (H\ref{eq:hipotesis sobre funcional:H3}). Si existe una constante $C_{1}(N,p,L,\nu)$ con la propiedad de que para cada $0<\tau<1$, existen $\epsilon(\tau),\vartheta(\tau)$ tales que si $u\in \SBV$, $\nabla{u}\in \L{p}$, $B_{\varrho}(x)\subset \Omega$ y:
\begin{align*}
\begin{array}{ll}
\Hm{N-1}(S_{u}\cap B_{\varrho}(x))\leq \epsilon \varrho^{N-1},& Dev(u, B_{\varrho})\leq \vartheta F(u, B_{\varrho}(x)),
\end{array}
\end{align*}
entonces:
\begin{align*}
F(u,B_{\tau\varrho}(x))\leq C_{1}\tau^{N} F(u, B_{\varrho}(x)).
\end{align*} 
\end{lema}
\begin{proof}
Fijamos primero $0<\tau<1/2$. Probamos  por reducción al absurdo el enunciado dado $C_{1}>L^{2}C_{0}$, donde $C_{0}$ es la constante que aparece en \ref{teo:regularity propierties functional f:eq:1}. Si la propiedad de decaimiento no fuera verdadera entonces existirían dos sucesiones $(\epsilon_{h}), (\vartheta_{h})$, con $\lim_{h}\epsilon_{h}=\lim_{h}\vartheta_{h}=0$, funciones $u_{h}\in \SBV$ con $\abs{\nabla{u_{h}}}\in \L{p}$ y bolas $B_{\varrho_{h}}(x_{h})\subset\Omega$, tales que:
\begin{align*}
\begin{array}{ll}
\Hm{N-1}(S_{u_{h}}\cap B_{\varrho_{h}}(x_{h}))=\epsilon_{h}\varrho^{N-1}_{h},& Dev(u_{h}, B_{\varrho_{h}}(x_{h}))=\vartheta_{h}F(u_{h},B_{\varrho_{h}}(x_{h})),
\end{array}
\end{align*}
y:
\begin{align*}
F(u_{h},B_{\tau\varrho_{h}}(x_{h}))>C_{1}\tau^{N}F(u_{h}, B_{\varrho_{h}}(x_{h})).
\end{align*}
Imponiendo:
\begin{align*}
\begin{array}{lll}
v_{h}(y)=\varrho_{h}^{(1-p)/p}c_{h}^{1/p}u_{h}(x_{h}+\varrho_{h}y),& h\in \N, & y\in B_{1},
\end{array}
\end{align*}
con $c_{h}=\varrho^{N-1}_{h}\left[ F(u_{h}, B_{\varrho_{h}}(x_{h}))\right]^{-1}$, por las propiedades de rescalamiento se tiene:
\begin{align*}
\begin{array}{lll}
F(v_{h}, c_{h}, B_{1})=1, & Dev(v_{h}, c_{h}, B_{1})=\vartheta_{h}, & \Hm{N-1}(S_{v_{h}}\cap B_{1})=\epsilon_{h},
\end{array}
\end{align*}
y además:
\begin{align}\label{proof:decaimiento:eq:1}
F(v_{h}, c_{h}, B_{\tau})>C_{1}\tau^{N}.
\end{align}
Como $\lim_{h}\Hm{N-1}(S_{v_{h}}\cap B_{1})=0$, por la proposición \ref{teo:comportamiento asintótico de una sucesión en SBV} se sigue que se puede extraer una subsucesión que volvemos a denotar por $(v_{h})$, y una función $v\in W^{1,p}(B_1)$ tal que $(v_{h}-m_{h})$, donde $m_{h}$ es una mediana de $v_{h}$, converge a $v$ en $B_{1}$ en c.t.p. respecto de $\Lm{N}$ y además:
\begin{align*}
\int_{B_{1}}f(\nabla{v})dy\leq \lim_{h\to \infty} \inf \int_{B_{1}}f(\nabla{v_{h}})dy\leq 1.
\end{align*}
Como $\lim_{h} Dev(v_{h},c_{h},B_{1})=0$, aplicando el teorema \ref{teo:comportamiento asintótico 2 de una sucesión en SBV} concluimos que $v$ es un mínimo local del funcional $w\mapsto \int_{B_{1}}f(\nabla{w})dy$ en $W^{1,p}(B_{1})$ y que:
\begin{align*}
\begin{array}{ll}
\displaystyle
\lim_{h\to \infty}F(v_{h}, c_{h}, B_{\varrho})=\int_{B_{\varrho}}f(\nabla{v})dy, & \forall \varrho\in (0,1).
\end{array}
\end{align*}
Luego por el teorema \ref{teo:regularity propierties functional f}, $v$ es localmente Lipschitz en $B_{1}$ y:
\begin{align*}
\sup_{y\in B_{1/2}}\abs{\nabla{v(y)}}^{p}\leq C_{0}\fint_{B_{1}} \abs{\nabla{v}}^{p}dy\leq LC_{0}\fint_{B_{1}}f(\nabla{v})dy\leq \dfrac{LC_{0}}{\omega_{N}}.
\end{align*}
Por lo tanto:
\begin{align*}
\lim_{h\to\infty}F(v_{h}, c_{h}, B_{\tau})=\int_{B_{\tau}}f(\nabla{v})dy \leq L\int_{B_{\tau}}\abs{\nabla{v}}^{p}dy\leq L^{2}C_{0}\tau^{N},
\end{align*}
lo que contradice \ref{proof:decaimiento:eq:1}, pues $C_{1}>L^{2}C_{0}$. Esto prueba el enunciado para $0<\tau<1/2$. La prueba del caso $1/2\leq \tau<1$ es inmediata considerando $C_{1}>2^{N}$.
\end{proof}
Para probar la existencia de minimizadores del funcional $J$, introducimos el siguiente funcional ``relajado'' para cada $u\in \SBVloc$:
\begin{align}\label{eq:1:problema variacional relajado}
\mathcal{F}(u, \Omega)=\int_{\Omega}f(\nabla{u})dx+\alpha\int_{\Omega}\abs{u-g}^{q}dx+\Hm{N-1}(S_{u}\cap \Omega).
\end{align}
Vemos que si se trunca $u$ definiendo:
\begin{align}
u^{M}=-M\vee (u\wedge M),
\end{align}
con $M=\Norm{g}[\infty]$, entonces $\mathcal{F}(u^{M}, \Omega)\leq \mathcal{F}(u, \Omega)$. De hecho $u^{M}\in \SBVloc$, $S_{u^{M}}\subset S_{u}$ y por el teorema \ref{teo:Regla de la cadena en BV version 2}, $\nabla{u^{M}}=\nabla{u}\chi_{\set*{\abs{u}<M}}$ en c.t.p. en $\Omega$ respecto de $\Lm{N}$. Por la desigualdad anterior vemos que para minimizar el funcional $\mathcal{F}$ hay que restringirse a las funciones tales que $\Norm{u}[\infty]\leq \Norm{g}[\infty]$. Por lo tanto, se sigue inmediatamente la existencia del mínimo gracias al resultado de compacidad  y de cierre en $SBV$ (teoremas \ref{teo:compacidad en SBV} y \ref{teo:cierre de SBV}).

\begin{teo}[Existencia de minimizadores en $SBV$]\index{minimizador}
Sea $\alpha>0$, $q\geq 1$, $g\in \L{q}\cap \L{\infty}$ y $f$ una función convexa que satisface (H\ref{eq:hipotesis sobre funcional:H1}). Entonces, existe un minimizador $u\in \SBVloc$ del funcional $\mathcal{F}$. Además, cualquier elemento minimizador es tal que verifica $\Norm{u}[\infty]\leq \Norm{g}[\infty]$.
\end{teo}

\begin{nota}\label{nota:deviacion minimizadores}
Si $u$ es un minimizador de $\mathcal{F}$, $B_{\varrho}(x)\subset \Omega$ y $v\in SBV(B_{\varrho}(x))$ es tal que $\set*{v\not =u}\subset\subset B_{\varrho}(x)$ entonces $\mathcal{F}(u,B_{\varrho}(x))\leq \mathcal{F}(v^{M},B_{\varrho}(x))$, pues $\set*{v^{M}\not = u}\subset\subset \Omega$. Como consecuencia se tiene:
\begin{align*}
\int_{B_{\varrho}(x)}f(\nabla{u})dy+\Hm{N-1}(S_{u}\cap B_{\varrho}(x))\leq \int_{B_{\varrho}(x)}&f(\nabla{v})dy+\Hm{N-1}\left (S_{v}\cap B_{\varrho}(x)\right)
\\&+2^{q}\alpha\omega_{N}\Norm{g}[\infty]^{q}\varrho^{N}.
\end{align*}
En otras palabras, si $u$ es un minimizador de $\mathcal{F}$ entonces $Dev(u, B_{\varrho}(x))\leq 2^{q}\alpha\omega_{N}\Norm{g}[\infty]^{q}\varrho^{N}$ para todas las bolas $B_{\varrho}(x)\subset \Omega$.
\end{nota}

Una característica importante de la teoría que presentamos en este capítulo es que los resultados de regularidad se mantienen no solo para los minimizadores de $\mathcal{F}$, sino también para todas las funciones $u$ de variación acotada especiales cuya desviación con respecto del mínimo $Dev(u, B_{\varrho}(x))$ decaiga como $\varrho^{s}$, para algún $s>N-1$ cuando $\varrho$ tiende a cero. Sin embargo, para simplificar restringiremos la exposición al caso de decaimiento de orden $\varrho^{N}$. Para el estudio del caso general se remite a \cite{ASNSP_1997_4_24_1_1_0, ASNSP_1997_4_24_1_39_0}. 

En correspondencia con esto, a continuación, vemos la definición de cuasi-minimizador.
\begin{defi}[Cuasi-minimizadores]\label{defi:cuasi-minimizadores}\index{cuasi-minimizadores}
Se dice que una función $u\in \SBVloc$ es un cuasi-minimizador del funcional $F$ en $\Omega$ si existe una constante $\omega\geq 0$ tal que para todas las bolas $B_{\varrho}(x)\subset \Omega$ se tiene que:
\begin{align}\label{defi:cuasi-minimizadores:eq:1}
Dev(u,B_{\varrho}(x))\leq \omega\varrho^{N}.
\end{align}
Denotaremos por $\mathcal{M}_{\omega}(\Omega)$ al conjunto de todos los cuasi-minimizadores que satisfacen la condición \ref{defi:cuasi-minimizadores:eq:1}.
\end{defi}

Como hemos visto en la nota anterior cualquier minimizador de $\mathcal{F}$ pertenecerá a $\mathcal{M}_{\omega}(\Omega)$, con $\omega=2^{q}\alpha\omega_{N}\Norm{g}[\infty]^{q}$.

En los dos próximos resultados probamos la regularidad de Ahlfors de los conjuntos de discontinuidades de los cuasi-minimizadores. Este tipo de regularidad nos dice que existe una cota superior e inferior para $\Hm{N-1}(S_{u}\cap B_{\varrho}(x))$, donde $S_{u}$ es el conjunto de discontinuidad de un cuasi-minimizador $u$ y $B_{\varrho}(x)$ es cualquier bola de radio $\varrho$ centrada en $x\in S_{u}$. La obtención de la cota superior es directa como puede observarse en la demostración del siguiente lema.
\begin{lema}[Cota superior de energía]\label{lema:cota superior de energía}\index{cota!superior de energía}
Si $f$ verifica (H\ref{eq:hipotesis sobre funcional:H1}) y $u\in \mathcal{M}_{\omega}(\Omega)$ entonces para cualquier bola $B_{\varrho}(x)\subset \Omega$ se tiene que:
\begin{align}\label{lema:cota superior de energía:eq:1}
\int_{B_{\varrho}(x)}f(\nabla{u})dy+\Hm{N-1}(S_{u}\cap B_{\varrho}(x))\leq N\omega_{N}\varrho^{N-1}+\omega\varrho^{N}.
\end{align}
\end{lema}
\begin{proof}
Fijemos $\varrho'<\varrho$ y sea la función $w(y)=u\chi_{B_{\varrho}(x)\setminus B_{\varrho'}(x)}$. Entonces por la casi-minimalidad de $u$ se tiene:
\begin{align*}
\int_{B_{\varrho'}(x)}f(\nabla{u})dy+\Hm{N-1}(S_{u}\cap \overline{B}_{\varrho'}(x))&\leq \Hm{N-1}(S_{w}\cap \overline{B}_{\varrho'}(x))+Dev(u,B_{\varrho}(x))
\\&\leq N\omega_{N}\varrho^{N-1}+\omega\varrho^{N}.
\end{align*}
Haciendo tender $\varrho'\to \varrho$ obtenemos el resultado.
\end{proof}

Notesé que en la demostración de la cota superior \ref{lema:cota superior de energía:eq:1} no se menciona que la bola $B_{\varrho}(x)$ esté centrada en $S_{u}$. Por el contrario a la hora de calcular la cota inferior, si es necesario que $B_{\varrho}(x)$ tenga su centro en $S_{u}$ pues es necesario que contenga una porción importante del conjunto de puntos de discontinuidad. Antes de probar la cota inferior para $\Hm{N-1}(S_{u}\cap B_{\varrho}(x))$ hacemos la siguiente observación.

\begin{nota}\label{nota:holder continuidad de los cuasi minimizadores}
Sea $u\in \mathcal{M}_{\omega}(\Omega)\cap W^{1,p}_{loc}(\Omega)$. Si $B_{\varrho}(x)\subset \Omega$, entonces por la cota superior de energía y por la desigualdad de Poincaré para funciones de Sobolev se tiene:
\begin{align*}
\int_{B_{\varrho}(x)}\abs{u(y)-u_{x,\varrho}}^{p}dy\leq c\varrho^{p}\int_{B_{\varrho}(x)}\abs{\nabla{u}}^{p}dy\leq c\varrho^{N-1+p},
\end{align*}
donde $u_{x,\varrho}$ es la media integral de $u$ en $B_{\varrho}(x)$. Por lo tanto, por el teorema clásico debido a Campanato \cite[teorema 7.51]{ambrosio2000functions} se sigue que $u\in C^{0,\gamma}(\Omega)$, con $\gamma=(p-1)/p$.
\end{nota}

\begin{teo}[Cota inferior de densidad]\label{teo:Cota inferior de densidad} \index{cota!inferior de densidad}
Sea $f$ una función convexa que satisface (H\ref{eq:hipotesis sobre funcional:H1}), (H\ref{eq:hipotesis sobre funcional:H2}) y (H\ref{eq:hipotesis sobre funcional:H3}). Existen $\vartheta_{0}$ y $\varrho_{0}$, que dependen únicamente de $N,p,L$ y de $\nu$ con las propiedades de que si $u\in \mathcal{M}_{\omega}(\Omega)$ entonces:
\begin{align}\label{teo:Cota inferior de densidad:eq:1}
\Hm{N-1}\left (S_{u}\cap B_{\varrho}(x)\right )>\vartheta_{0}\varrho^{N-1},
\end{align}
para cualquier bola $B_{\varrho}(x)\subset \Omega$ con centro $x\in \overline{S}_{u}$ y radio $\varrho<\varrho_{\omega}=\varrho_{0}/\omega$.
\end{teo}
\begin{proof}
Fijemos $0<\tau<1$ tal que $C_{1}\tau^{N}\leq \tau^{N-1/2}$ y sea $\epsilon_{0}=\epsilon(\tau)$, donde $C_{1}$ y $\epsilon(\tau)$ son como los del lema de decaimiento \ref{lema:decaimiento}. Siguiendo la misma notación que la de ese lema, sea entonces $0<\sigma<1$ tal que:
\begin{align*}
C_{1}\sigma (N\omega_{N}+1)\leq \epsilon_{0},
\end{align*}
y definamos:
\begin{align*}
\varrho_{0}=\min \set*{1, \epsilon_{0}\tau^{N}\vartheta(\tau), \epsilon_{0}\sigma^{N-1}\vartheta(\sigma)}.
\end{align*}
Sin pérdida de generalidad, asumamos que el punto $x$ en $S_{u}$ es el origen.

\textit{Paso 1.} Afirmamos que si $\varrho<\varrho_{0}/\omega$ y $B_{\varrho}\subset \Omega$ entonces la desigualdad:
\begin{align}\label{proof:Cota inferior de densidad:eq:1}
\Hm{N-1}(S_{u}\cap B_{\varrho})\leq \epsilon(\sigma)\varrho^{N-1},
\end{align}
implica que:
\begin{align}\label{proof:Cota inferior de densidad:eq:2}
\begin{array}{ll}
\displaystyle
F(u,B_{\sigma\tau^{h}\varrho})\leq \epsilon_{0}\tau^{h/2}(\sigma\tau^{h}\varrho)^{N-1}, & \forall h\in \N.
\end{array}
\end{align}
En primer lugar, probemos la ecuación \ref{proof:Cota inferior de densidad:eq:2} para $h=0$. Si se tiene que \ref{proof:Cota inferior de densidad:eq:1} es cierto y además:
\begin{align}\label{proof:Cota inferior de densidad:eq:3}
Dev(u,B_{\varrho})\leq \vartheta(\sigma)F(u, B_{\varrho}),
\end{align}
entonces por el lema de decaimiento y por la cota superior de energía \ref{lema:cota superior de energía:eq:1} se deduce que:
\begin{align*}
F(u,B_{\sigma\varrho})&\leq C_{1}\sigma^{N}F(u, B_{\varrho})\leq C_{1}\sigma^{N}(N\omega_{N}\varrho^{N-1}+\omega\varrho^{N})\\
&\leq (\sigma\varrho)^{N-1}C_{1}\sigma (N\omega_{N}+1)\leq \epsilon_{0}(\sigma\varrho)^{N-1}.
\end{align*}
Por otro lado, si \ref{proof:Cota inferior de densidad:eq:3} no es verdad, entonces por la quasi-minimalidad de $u$ y por la definición de $\varrho_{0}$ se tiene:
\begin{align*}
F(u, B_{\sigma\varrho})\leq F(u,B_{\varrho})\leq \dfrac{1}{\vartheta(\sigma)}Dev(u, B_{\varrho})\leq \dfrac{\omega \varrho^{N}}{\vartheta(\sigma)}\leq \epsilon_{0}(\sigma\varrho)^{N-1}.
\end{align*}
Esto prueba \ref{proof:Cota inferior de densidad:eq:2} para $h=0$. Si \ref{proof:Cota inferior de densidad:eq:2} es verdad para un $h\geq 0$ y:
\begin{align}\label{proof:Cota inferior de densidad:eq:4}
Dev(u,B_{\sigma\tau^{h}\varrho})\leq \vartheta(\tau)F(u,B_{\sigma\tau^{h}\varrho}),
\end{align}
entonces el lema de decaimiento implica esta vez que:
\begin{align*}
F(u,B_{\sigma \tau^{h+1}\varrho})&\leq F(u, B_{\sigma\tau^{h}\varrho})\leq \dfrac{1}{\vartheta(\tau)}Dev(u, B_{\sigma \tau^{h}\varrho})\\
&\leq \dfrac{\omega}{\vartheta(\tau)}(\sigma\tau^{h}\varrho)^{N}\leq \epsilon_{0}\tau^{(h+1)/2}(\sigma\tau^{h+1}\varrho)^{N-1}.
\end{align*}
Luego, recordando la elección de $\tau$, obtenemos \ref{proof:Cota inferior de densidad:eq:2} para $h+1$. Finalmente, si verifica la ecuación \ref{proof:Cota inferior de densidad:eq:2} para $h$, pero no se cumple \ref{proof:Cota inferior de densidad:eq:4}, entonces al igual que antes obtenemos que:
\begin{align*}
F(u, B_{\sigma\tau^{h+1}\varrho})&\leq F(u, B_{\sigma\tau^{h}\varrho})\leq \dfrac{1}{\vartheta(\tau)}Dev(u, B_{\sigma\tau^{h}\varrho})\\
&\leq \dfrac{\omega}{\vartheta(\tau)}(\sigma \tau^{h}\varrho)^{N}\leq \epsilon_{0}\tau^{(h+1)/2}(\sigma\tau^{h+1}\varrho)^{N-1}.
\end{align*}
\textit{Paso 2.} Supongamos que:
\begin{align*}
\Hm{N-1}(S_{u}\cap B_{\varrho})\leq \epsilon(\sigma)\varrho^{N-1},
\end{align*}
para alguna bola $B_{\varrho}\subset \Omega$ con $\varrho<\varrho_{\omega}$. Por \ref{proof:Cota inferior de densidad:eq:2} se deduce inmediatamente que $F(u,B_{r})=o(r^{N-1})$ cuando $r\to 0$ y por el teorema \ref{teo:condicion existencia de limite aproximado} con $q=1^{*}$, implica que el origen $0\in I$, donde:
\begin{align*}
I=\set*{x\in \Omega \given \limsup_{\varrho\to 0} \fint_{B_{\varrho}(x)}\abs{u(y)}^{1^{*}}dy=\infty}.
\end{align*}
Esto prueba \ref{teo:Cota inferior de densidad:eq:1} para cualquier $x\in S_{u}\setminus I$. Por un argumento de densidad, la desigualdad sigue siendo cierta para bolas centradas en puntos de $\overline{S_{u}\setminus I}$. Por lo tanto, la prueba se completa si probamos que $\overline{S_{u}\setminus I}=\overline{S_{u}}$.

Sea $x\not \in \overline{S_{u}\setminus I}$, como por el lema \ref{lema: valor medio 1* en bolas uniformemente acotado}, $I$ tiene medida cero respecto de $\Hm{N-1}$, podemos encontrar una vecindad $U$ de $x$ tal que $\Hm{N-1}(U\cap S_{u})=0$. Por \ref{eq:1:relacion espacio de sobolev vs sbv} deducimos que $u\in W^{1,p}(U)$, luego por la nota \ref{nota:holder continuidad de los cuasi minimizadores} concluimos que $u$ es una función Hölder continua en $U$. Como consecuencia, $x\not \in \overline{S_{u}}$.
\end{proof}

Observamos que, a partir de la cota inferior de densidad y de la propiedad de densidad \ref{eq:resultados de desnidades:1}, se tiene:
\begin{align}\label{eq:1:consecuncia cota inferior de densidad}
\Hm{N-1}(\Omega\cap \overline{S}_{u}\setminus S_{u})=0.
\end{align}
De la ecuación \ref{eq:1:consecuncia cota inferior de densidad} se obtiene inmediatamente la existencia de un minimizador de $J$ a lo largo de todos los pares $(K,u)$ con $K\subset \overline{\Omega}$ cerrado y $u\in W^{1,p}_{loc}(\Omega\setminus K)$.
\begin{teo}[Existencia de un par minimizante $(K,u)$] \index{minimizador}
Sea $f$ una función convexa que satisface (H\ref{eq:hipotesis sobre funcional:H1}), (H\ref{eq:hipotesis sobre funcional:H2}) y (H\ref{eq:hipotesis sobre funcional:H3}), $\alpha>0, q\geq 1, g\in \L{\infty}\cap \L{q}$. Si $u\in \SBVloc$ es un minimizador de $\mathcal{F}(u,\Omega)$ entonces el par $(\overline{S}_{u},u)$ es un minimizador de $J$, i.e.
\begin{align*}
J(\overline{S}_{u},u)\leq J(K,v),
\end{align*} 
para cualquier conjunto cerrado $K\subset\overline{\Omega}$ y cualquier $v\in W^{1,p}_{loc}(\Omega\setminus K)$.
\end{teo}
\begin{proof}
Primero se ve que si $u$ es un minimizador de $\mathcal{F}$ en $\Omega$ entonces $\mathcal{F}(u, \Omega)\leq \mathcal{F}(0, \Omega)=\Norm{g}_{q}^{q}<\infty$. Por lo tanto $\nabla{u}\in \L{p}$, luego $u\in W^{1,p}_{loc}(\Omega\setminus \overline{S}_{u})$. Sea ahora $v$ una función que pertenece a $W^{1,p}_{loc}(\Omega\setminus K)$ y tal que $J(K,v)<\infty$. No es restrictivo suponer que $v$ está acotado. Por la proposición \ref{prop:funciones sobolev a trozos en SBV}, $v\in \SBVloc$ y $\Hm{N-1}(S_{v}\setminus K)=0$, por lo tanto como $u$ es un elemento minimal y además por \ref{eq:1:consecuncia cota inferior de densidad} se sigue que:
\begin{align*}
J(\overline{S}_{u},u)=\mathcal{F}(u, \Omega)\leq \mathcal{F}(v, \Omega)\leq J(K,v),
\end{align*}
con lo que probamos el teorema.
\end{proof}

\section{Primera variación del área, curvatura media y ecuación de Euler-Lagrange}

En esta sección probaremos resultados sobre la regularidad de los minimizadores del funcional $\mathcal{F}$ o más concretamente de los cuasi-minimizadores de $F$. Para ello primero recordamos algunas nociones básicas de geometría como la de gradiente de una variedad o la de curvatura media, y estableceremos notación. En esta sección trabajaremos siempre con variedades de codimensión $1$. Tras esto estudiaremos la regularidad de los minimizadores del funcional $\mathcal{F}$, simplificado de la siguiente modo:
\begin{align*}
\begin{array}{ll}
\displaystyle \mathcal{F}(u,\Omega)=\int_{\Omega}\abs{\nabla{u}}^{2}dx+\alpha\int_{\Omega}\abs{u-g}^{2}dx+\Hm{N-1}(S_{u}\cap \Omega),& \forall u\in SBV_{loc}(\Omega).
\end{array}
\end{align*}
Gran parte de lo que probaremos para este caso particular puede probarse análogamente en el caso general, siempre y cuando la función $f$ sea $C^{2}$ convexa y satisfaga las hipótesis (H\ref{eq:hipotesis sobre funcional:H1}), (H\ref{eq:hipotesis sobre funcional:H2}), (H\ref{eq:hipotesis sobre funcional:H3}) de la sección anterior.

A continuación damos la definición de gradiente a una variedad:

\begin{defi}[Gradiente a una variedad] \index{gradiente a una variedad}
Sea $M$ una variedad de clase $C^{1}$ de dimensión $(N-1)$ y sea $\varphi\in C^{1}(\Omega)$. Si $x\in M\cap \Omega$ y $h$ es un vector que pertenece al espacio tangente $T_{x}M$ de $M$ en $x$ se define la derivada direccional $\nabla_{h}{\varphi(x)}$ como:
\begin{align*}
\nabla_{h}\varphi(x)=\Crochet{\nabla{\varphi(x)}}{h},
\end{align*}
y el gradiente tangencial de $\varphi$ en $x$, denotado por $\nabla^{M}\varphi(x)$, se define a su vez como:
\begin{align*}
\nabla^{M}\varphi(x)=\sum_{i=1}^{N-1}\nabla_{\tau_{i}}\varphi(x)\tau_{i},
\end{align*}
donde $\tau_{1},\ldots,\tau_{N-1},$ es una base ortonormal de $T_{x}M$.
\end{defi}

Vemos que el gradiente de $\varphi$ en $x$ no es más que la proyección ortogonal de $\nabla{\varphi(x)}$ en el espacio tangente $T_{x}M$. Análogamente, si $\varphi\in [C^{1}(\Omega)]^{p}$, la derivada direccional $\nabla_{h}\varphi(x)$ se define de la siguiente manera:
\begin{align*}
\nabla_{h}\varphi(x)=\sum^{p}_{k=1}\Crochet{\nabla{\varphi_{k}(x)}}{h}e_{k},
\end{align*}
donde $\varphi_{1},\ldots, \varphi_{p},$ son las componentes de $\varphi$ y $e_{1}, \ldots,e_{p},$ es la base canónica de $\R^{p}$. La aplicación lineal inducida $d^{M}\varphi_{x}:T_{x}M\to \R^{p}$ es dada por:
\begin{align*}
\begin{array}{ll}
d^{M}\varphi_{x}(h)=\nabla_{h}\varphi(x),& h\in T_{x}M,
\end{array}
\end{align*}
es la diferencial de $\varphi$ en $x$.

A continuación definimos la divergencia de una función en una variedad.
\begin{defi}[Divergencia]\label{defi:Divergencia}\index{divergencia} 
Si $\varphi\in [C^{1}(M)]^{N}$, entonces la divergencia de $\varphi$ en $M$ se define como:
\begin{align}\label{defi:Divergencia:eq:1}
\begin{array}{ll}
\displaystyle\div^{M}\varphi(x)=\sum^{N}_{k=1}\Crochet{\nabla^{M}\varphi_{k}(x)}{e_{k}},& \forall x\in M.
\end{array}
\end{align}
\end{defi}

Si denotamos por $\tau_{ik}=\Crochet{\tau_{i}}{e_{k}}$ para cada $i=1,\ldots,m,$ y cada $k=1,\ldots, N,$ por la definición anterior se tiene:
\begin{align*}
\displaystyle\div^{M}\varphi(x)=\sum_{k=1}^{N}\sum^{N-1}_{i=1}\Crochet{\nabla{\varphi_{k}(x)}}{\tau_{i}}\tau_{ik}=\sum_{i=1}^{N-1}\Crochet{\nabla_{\tau_{i}}\varphi(x)}{\tau_{i}}. 
\end{align*}
Se puede obtener otra expresión para la divergencia en la variedad $M$ usando la matriz $(\pi_{ij})$ de proyección ortogonal \index{matriz de proyección ortogonal} en el plano tangente de $M$ en $x$. En efecto se tiene que:
\begin{align*}
\begin{array}{ll}
\displaystyle
\nabla^{M}\varphi_{k}(x)=\sum^{N}_{i,j=1}\pi_{ij}\diffp{\varphi_{k}}{x_{j}}(x),& \forall k=1, \ldots,N,
\end{array}
\end{align*}
y de la ecuación \ref{defi:Divergencia:eq:1} se sigue que:
\begin{align}\label{eq:1:ecuacion divergencia}
\div^{M}\varphi(x)=\sum^{N}_{i,j=1}\pi_{ij}\diffp{\varphi_{i}}{x_{j}}(x).
\end{align}
Como todas  las nociones anteriores son locales, podemos siempre ver una variedad $M$ como el gráfico de una función $C^{1}$. A continuación fijamos la notación que usaremos posteriormente en la sección. Consideramos $\phi:D\subset \R^{m}\to \R$ una función $C^{1}$ sobre un conjunto abierto $D$ y $M=\set*{x=(z,\phi(z))\given z\in D}$. Entonces, la normal exterior $\nu$ \index{normal!exterior} en un punto $(z,\phi(z))$ de $M$ vendrá dada por:
\begin{align*}
\nu(z)=\dfrac{1}{\sqrt{1+\abs{\nabla{\phi(z)}}^{2}}}(-\nabla{\phi}(z),1).
\end{align*}
Como la matriz de la proyección ortogonal en el espacio tangente tiene coeficientes: $\pi_{ij}=\delta_{ij}-\nu_{i}\nu_{j}$ para cada $i,j=1, \ldots,N$, de la ecuación \ref{eq:1:ecuacion divergencia} se sigue que la divergencia en $M$ de un campo vectorial $\varphi$ que es $C^{1}$ viene dada por la ecuación:
\begin{align}
\div^{M}\varphi(z,\phi(z))=\div\varphi(z,\phi(z))-\sum_{i,j=1}^{N}\nu_{i}(z)\nu_{j}(z)\diffp{\varphi_{i}}{x_{j}}(z,\phi(z)).
\end{align}
\begin{ejemplo}
Si $M$ es el gráfico de una función $C^{2}(D)$, entonces la normal exterior $\nu$ define un campo vectorial $\nu(z,t)=\nu(z)$ que es $C^{1}$ para cada $(z,t)\in D\times \R$ y cuyo gradiente en $M$ es dado por:
\begin{align*}
\div^{M}\nu=\div \nu -\sum^{N}_{i,j=1}\nu_{i}\nu_{j}\diffp{\nu_{i}}{x_{j}}.
\end{align*}
Como $\abs{\nu}^{2}=1$, entonces $\displaystyle \sum_{i}\nu_{i}\diffp{\nu_{i}}{x_{j}}=0$ para cualquier $j=1, \ldots,N$. Por lo tanto:
\begin{align}\label{ejemplo:1:eq:1:ecuacion divergencia}
\div^{M}\nu=\sum^{N-1}_{i=1}\diffp{\nu_{i}}{z_{i}}=-\sum^{N-1}_{i=1}\diffp*{\left ( \dfrac{\diffp{\phi}/z_{i}}{\sqrt{1+\abs{\nabla{\phi}}^{2}}}\right )}{z_{i}}.
\end{align}
\end{ejemplo}

Sea $\Omega\subset\R^{N}$ un conjunto abierto y $M\subset \Omega$ un conjunto $\Hm{N-1}$-rectificable. Consideremos $\Phi:\Omega\to \R^{N}$ una aplicación $C^{1}$ con la propiedad de que la restricción de $\Phi$ a $M$ es biyectiva. Recordamos que por la fórmula de área generalizada para cualquier conjunto de Borel $E\subset M$ se tiene:
\begin{align*}
\Hm{N-1}(\Phi(E))=\int_{E}\mathbf{J}_{N-1}(d^{M}\Phi_{x})d\Hm{N-1},
\end{align*}
donde por $\mathbf{J}_{N-1}$ denotamos al jacobiano de $d^{M}\Phi_{x}$ en $T_{x}M$.

Gracias a las nociones introducidas hasta ahora, ya podemos probar la fórmula de la primera variación del área. 

\begin{teo}[Primera variación del área]\label{teo:primera variación del área} \index{primera variación del área}
Sea $\Omega\subset \R^{N}$ un abierto y $M\subset \Omega$ un conjunto $\Hm{N-1}$-rectificable. Si $\eta\in \Cc[N]{1}$ y $\Phi_{\epsilon}(x)=x+\epsilon\eta (x)$ entonces:
\begin{align*}
\diff*{\Hm{N-1}(\Phi_{\epsilon}(M\cap \Omega))}{\epsilon}\mid_{\epsilon=0}=\int_{M\cap \Omega}\div^{M}{\eta}\, d\Hm{N-1}.
\end{align*}
\end{teo}
\begin{proof}
Como $\nabla{\Phi_{\epsilon}(x)}=I+\epsilon\nabla{\eta(x)}$, para $\abs{\epsilon}$ pequeño se tiene que $\Phi_{\epsilon}$ es un difeomorfismo de $\Omega$ en si mismo. Calculemos ahora el jacobiano de $\Phi_{\epsilon}$, la matriz $C$ que representa la aplicación diferencial $d^{M}(\Phi_{\epsilon})_{x}$ con respecto a la base ortonormal $\tau_1,\ldots\tau_{N-1},$ de $T_{x}M$ y la base canónica $e_{1},\ldots,e_{N},$ de $\R^{N}$ tiene como coeficientes:
\begin{align*}
c_{ki}=\Crochet{\tau_{i}}{e_{k}}+\epsilon\nabla_{\tau_{i}}\eta_{k},
\end{align*}
donde $k=1, \ldots,N$, $i=1,\ldots,N-1$. Por lo tanto, denotando por $A=(a_{ij})$ la matriz que representa $(d^{M}\Phi_{x})^{*}\circ d^{M}\Phi_{x}$ se tiene que:
\begin{align*}
a_{ij}&=\sum^{N}_{k=1}c_{ki}c_{kj}\\
&=\Crochet{\tau_{i}}{\tau_{j}}+\epsilon\left [ \Crochet{\nabla_{\tau_{i}}\eta(x)}{\tau_{j}}+\Crochet{\nabla_{\tau_{j}}\eta(x)}{\tau_{i}}\right ]+\epsilon^{2}\Crochet{\nabla_{\tau_{i}}\eta(x)}{\nabla_{\tau_{j}}\eta(x)},
\end{align*}
para cada $i,j=1,\ldots,N-1$. Como $\Crochet{\tau_{i}}{\tau_{j}}=\delta_{ij}$ se obtiene fácilmente:
\begin{align*}
\det(a_{ij})=1+2\epsilon\sum_{i=1}^{N-1}\Crochet{\nabla_{\tau_{i}}\eta(x)}{\tau_{i}}+o(\epsilon)=1+2\epsilon\div^{M}\eta(x)+o(\epsilon).
\end{align*}
Por lo tanto, como $\sqrt{1+t}=1+t/2+o(t)$ se tiene que: $\mathbf{J}_{m}(d^{M}(\Phi_{\epsilon})_{x})=1+\epsilon\div^{M}\eta(x)+o(\epsilon)$ y por lo tanto, teniendo en cuenta que la ecuación anterior es uniforme en $x$, de la fórmula del área concluimos que:
\begin{align}\label{proof:primera variación del área:eq:1}
\Hm{N-1}(\Phi_{\epsilon}(M\cap \Omega))-\Hm{N-1}(M\cap \Omega)=\epsilon\int_{M\cap \Omega}\div^{M}\eta\, d\Hm{N-1}+o(\epsilon).
\end{align} 
De esta última ecuación se sigue directamente el resultado.
\end{proof}

Se observa que si $M$ es de clase $C^{2}$ siempre se puede definir en $M$ (localmente) un campo vectorial unitario $\nu$ que sea $C^{1}$. La divergencia de $\nu$ en $M$ juega un papel fundamental a la hora de describir la geometría de la variedad y además nos proporciona una nueva versión de la fórmula de integración por partes. Antes de enunciar la formulación de la integración por partes en un variedad recordamos la definición de vector de curvatura. 
\begin{defi}[Vector de curvatura media]\label{defi:vector de curvatura media} \index{vector de curvatura media} \index{curvatura media escalar}
Sea $M$ una variedad de clase $C^{2}$ y de dimensión $N-1$. Sea $x\in M$, $A$ un conjunto abierto y $\nu:M\cap A \to \mathbb{S}^{N-1}$ un campo vectorial normal de clase $C^{1}$, i.e. $\nu(x)$ es ortogonal al espacio tangente $T_{x}M$ en cada punto $x\in M\cap A$. Entonces, el vector de curvatura media $\mathbf{H}$ se define como:
\begin{align}\label{defi:vector de curvatura media:eq:1}
\begin{array}{ll}
\mathbf{H}(x)=-\left( \div^{M}\nu(x)\right)\nu(x), & \forall x\in M\cap A,
\end{array}
\end{align}
y la curvatura media escalar con respecto a $\nu$ se define en $M\cap A$ como $H=-\div^{M}\nu$, luego $\mathbf{H}=H\nu$. 
\end{defi}
\begin{nota}
El vector de curvatura media definido en \ref{defi:vector de curvatura media:eq:1} no depende de la orientación de $\nu$. Si $M$ es un gráfico $C^{2}$ y:
\begin{align}
M=\set*{x=(z,\phi(z))\given z\in D},
\end{align}
donde $\phi:D\subset \R^{N-1}\to \R$ es una función sobre un conjunto abierto $D$, entonces de la ecuación \ref{ejemplo:1:eq:1:ecuacion divergencia} se sigue que la curvatura media escalar $H$ con respecto a la normal exterior vendrá dada por:
\begin{align}\label{nota:eq:1:curvatura media escalar respecto a la normal exterior}
\div\left( \dfrac{\nabla{\phi}}{\sqrt{1+\abs{\nabla{\phi}}^{2}}}\right)=H.
\end{align} 
\end{nota}

Finalmente recordamos el teorema de la divergencia en variedades.
\begin{teo}[Teorema de divergencia en variedades]\label{teo:teorema de divergencia en variedades} \index{teorema!de divergencia en variedades}
Sea $M\subset \Omega$ una variedad de clase $C^{2}$ sin frontera en $\Omega$, i.e. $\partial M\cap \Omega =\emptyset$. Entonces:
\begin{align*}
\begin{array}{ll}
\displaystyle
\int_{M}\div^{M}\eta\, d\Hm{N-1}=-\int_{M}\Crochet{\eta}{\mathbf{H}}d\Hm{N-1},& \forall \eta\in \Cc[N]{1}.
\end{array}
\end{align*}
\end{teo}

A continuación estudiamos la regularidad del funcional \ref{eq:1:problema variacional relajado} pero considerando las siguientes simplificaciones sobre el término volumétrico.
\begin{align*}
\begin{array}{ll}
\displaystyle 
\mathcal{F}(u,\Omega)=\int_{\Omega}\abs{\nabla{u}}^{2}dx+\alpha\int_{\Omega}\abs{u-g}^{2}dx+\Hm{N-1}(S_{u}\cap \Omega),&\forall u\in \SBVloc .
\end{array}
\end{align*}
En lo sucesivo supondremos que $u\in \SBVloc$ es un minimizador local de $\mathcal{F}$, i.e. que $\mathcal{F}(u, \Omega')<\infty$ para cualquier conjunto abierto $\Omega'\subset\subset \Omega$ y que para cualquier función $v\in \SBVloc$, con $\set{u\not = v}\subset\subset A\subset\subset \Omega$ se tiene que:
\begin{align*}
\mathcal{F}(u,A)\leq \mathcal{F}(v,A).
\end{align*}
Notesé que si $u$ es un minimizador local de $\mathcal{F}$ entonces por la nota \ref{nota:deviacion minimizadores} y por la ecuación \ref{eq:1:consecuncia cota inferior de densidad} se sigue que $\Hm{N-1}(\Omega\cap \overline{S}_{u}\setminus S_{u})=0$.
\begin{teo}[Ecuación de Euler-Lagrange]\label{teo:ecuación de Euler-Lagrange}\index{ecuación!de Euler-Lagrange}
Sea $u\in \SBVloc$ un minimizador local de $\mathcal{F}$ y $g\in C^{1}(\Omega)$. Entonces para cualquier campo vectorial $\eta\in \Cc[N]{1}$ se tiene la siguiente ecuación:
\begin{align*}
\int_{\Omega}\set*{ [\abs{\nabla{u}}^{2}+\alpha (u-g)^{2}]\div \eta -2\alpha (u-g)\Crochet{\nabla{g}}{\eta}-2\Crochet{\nabla{u}}{\nabla{u}\cdot\nabla{\eta}}}dx
\\+\int_{S_{u}}\div^{S_{u}}\eta\, d\Hm{N-1}=0.
\end{align*}
\end{teo}
\begin{proof}
Sea $\eta\in \Cc[N]{1}$ un campo vectorial y $\epsilon\not = 0$ tal que la aplicación $\Phi_{\epsilon}=x+\epsilon \eta(x)$ es un difeomorfismo de $\Omega$ en si mismo. Si tomamos $u_{\epsilon}(y)=u(\Phi_{\epsilon}^{-1}(y))$ por la minimalidad de $u$ se obtiene:
\begin{align}\label{proof:Ecuación de Euler-Lagrange:eq:1}
\int_{\Omega}&\abs{\nabla{u_{\epsilon}}}^{2}dy-\int_{\Omega}\abs{\nabla{u}}^{2}dx+\alpha\int_{\Omega}\abs{u_{\epsilon}-g}^{2}dy-\alpha\int_{\Omega}\abs{u-g}dx\nonumber\\
&+\Hm{N-1}(S_{u_{\epsilon}})-\Hm{N-1}(S_{u})\geq 0.
\end{align}
Haciendo cambio de variable se ve que:
\begin{align*}
\int_{\Omega}\abs{\nabla{u_{\epsilon}(y)}}^{2}dy=\int_{\Omega}\abs{\nabla{u}\cdot \nabla{\Phi_{\epsilon}^{-1}(\Phi_{\epsilon}(x))}}^{2}\abs{\det\nabla{\Phi_{\epsilon}(x)}}\,dx,
\end{align*}
y como también:
\begin{align*}
\nabla{\Phi_{\epsilon}^{-1}(\Phi_{\epsilon}(x))}=[I+\epsilon\nabla\eta(x)]^{-1}=I-\epsilon\nabla{\eta(x)}+o(\epsilon),\\
\det\nabla{\Phi_{\epsilon}(x)}=\det(I+\epsilon\nabla{\eta(x)})=1+\epsilon\div\eta(x)+o(\epsilon),
\end{align*}
por un simple cálculo se tiene que:
\begin{align}
\int_{\Omega}&\abs{\nabla u_{\epsilon}}^{2}dy-\int_{\Omega}\abs{\nabla{u}}^{2}dx\nonumber\\
&=\int_{\Omega}\left[ \abs{\nabla{u}(x)-\epsilon\nabla{u}(x)\cdot\nabla{\eta(x)}}^{2}(1+\epsilon\div\eta)-\abs{\nabla{u(x)}}^{2}\right]dx+o(\epsilon)\nonumber\\
&=\int_{\Omega}\left[\abs{\nabla{u}}^{2}\div \eta - 2\Crochet{\nabla{u}}{\nabla{u}\cdot\nabla{\eta}} \right]dx+o(\epsilon).
\end{align}
Análogamente, como $g(\Phi_{\epsilon}(x))-g(x)=\epsilon\Crochet{\nabla{g(x)}}{\eta(x)}+o(\epsilon)$, se tiene que:
\begin{align*}
\int_{\Omega}&\abs{u_{\epsilon}-g}^{2}dy-\int_{\Omega}\abs{u-g}^{2}dx\\
&=\int_{\Omega}\abs{u(x)-g(\Phi_{\epsilon}(x))}^{2}\abs{\det \nabla{\Phi_{\epsilon}(x)}}dx-\int_{\Omega}\abs{u(x)-g(x)}^{2}dx\\
&=\int_{\Omega}\left \{ \abs{u-g-\epsilon\Crochet{\nabla{g}}{\eta}}^{2}(1+\epsilon\div\eta )-\abs{u-g}^{2}\right \} dx+o(\epsilon)\\
&=\epsilon\int_{\Omega}\left [ \abs{u-g}^{2}\div \eta - 2(u-g)\Crochet{\nabla{g}}{\eta}\right ]dx + o(\epsilon).
\end{align*}
Finalmente, observando que $S_{u_{\epsilon}}=\Phi_{\epsilon}(S_{u})$ y por \ref{proof:primera variación del área:eq:1} se tiene:
\begin{align}
\Hm{N-1}(S_{u_{\epsilon}})-\Hm{N-1}(S_{u})=\epsilon\int_{S_{u}}\div^{S_{u}}\eta\, d\Hm{N-1}+o(\epsilon).
\end{align}
Dividiendo \ref{proof:Ecuación de Euler-Lagrange:eq:1} por $\epsilon$ y haciendo tender $\epsilon\to 0$ se obtiene el resultado enunciado.
\end{proof}

En el capítulo octavo de \cite{ambrosio2000functions} se prueba que si $u$ es un minimizador de $\mathcal{F}$ entonces existe un conjunto cerrado $S\subset \overline{S}_{u}$ tal que $\Hm{N-1}(S)=0$ y $\overline{S}_{u}\setminus S$ es una variedad $C^{1,1/4}$.  Sin embargo, en este trabajo daremos por hecho la regularidad parcial del conjunto de discontinuidades y estudiaremos qué información adicional puede obtenerse a partir de las ecuaciones de Euler-Lagrange cerca de una porción regular de $\overline{S}_{u}$.

Por lo tanto, si $u\in \SBVloc$ es un minimizador local del funcional $\mathcal{F}$ y $A\subset\subset \Omega$ es un conjunto abierto tal que la intersección $\overline{S}_{u}\cap A$ es un gráfico, entonces salvo rotación podemos suponer que:
\begin{align*}
\overline{S}_{u}\cap A =\set*{x=(z,\phi(z))\given z\in D},
\end{align*}
para algún conjunto abierto $D\subset \R^{N-1}$ y para algún $\phi:D\to \R$, y además,
\begin{align*}
A=A^{+}\cup A^{-}	\cup (\overline{S}_{u}\cap A),
\end{align*}
donde,
\begin{align*}
\begin{array}{ll}
A^{+}=\set*{(z,t)\in A \given t>\phi(z)},& A^{-}=\set*{(z,t)\in A \given t<\phi(z)}.
\end{array}
\end{align*}
Si $\varphi\in C^{1}(\overline{A})$ es una función que se anula en una vecindad de $\partial A^{+}\setminus \overline{S}_{u}$, comparando $u$ con la función $v$ tal que $v=u+\epsilon \varphi$ en $A^{+}$ y $v=u$ en $A^{-}$, entonces por la minimalidad de $u$ se puede ver fácilmente que:
\begin{align*}
\int_{A^{+}}[\Crochet{\nabla{u}}{\nabla{\varphi}}+\alpha (u-g)\varphi]dx=0.
\end{align*}
Esto significa que $u$ es una solución débil del problema:
\begin{align}\label{eq:1:problema variacional formulacion fuerte conjunto de discontinuidades regular}
\begin{array}{ll}
\displaystyle
\Delta u =\alpha (u-g),& \text{en}\; A^{+},\\
\displaystyle
\diffp{u}{\nu}=0,& \text{en}\; \partial A^{+}\cap \overline{S}_{u}.
\end{array}
\end{align}
Análogamente, $u$ resuelve un problema similar en $A^{-}$.

\begin{nota}
Vemos que si $g$ es una función acotada en $\Omega$ y $u$ minimiza el funcional $\mathcal{F}$ entonces $u$ está acotada y por lo tanto por un teorema clásico de ecuaciones en derivadas parciales elípticas (\cite[cap 6.3, teorema 2]{evans1998partial}) se deduce que $u\in W^{2,p}_{loc}(A\setminus \overline{S}_{u})$ para todos los $p\in [1,\infty)$, luego $u\in C^{1,\sigma}(A\setminus \overline{S}_{u})$ para cualquier $\sigma\in (0,1)$.
\end{nota}
\begin{teo}\label{teo:regularidad minimizador de F}
Si $u\in \SBVloc$ es un minimizador local de $\mathcal{F}, g\in L^{\infty}_{loc}(\Omega)$ y $\overline{S}_{u}\cap A$ es el gráfico de una función $C^{1,\gamma}$, $\gamma<1$, entonces $u$ tiene una extensión $C^{1,\sigma}$ a cada lado de $\overline{S}_{u}\cap A$ para algún $\sigma\leq \gamma$ ($\sigma=\gamma$ si $N=2$).
\end{teo}
\begin{proof}
Como $u$ es una solución débil de la ecuación \ref{eq:1:problema variacional formulacion fuerte conjunto de discontinuidades regular}, el resultado es consecuencia inmediata de los teoremas de regularidad para los problemas de Neumann (ver  \cite[cap.7 teoremas 7.52 y 7.49]{ambrosio2000functions}).  
\end{proof}
Suponiendo que $\overline{S}_{u}\cap A$ es el gráfico de una función $C^{1,\gamma}$ entonces se puede reescribir la ecuación de Euler-Lagrange anterior. Con este objetivo, si $w:A\setminus \overline{S}_{u}\to \R$ es una función que tiene una extensión continua a cada lado de $\overline{S}_{u}\cap A$, denotamos por $w^{+}$ (respectivamente por $w^{-}$) la traza superior (inferior) de $w$ en $\overline{S}_{u}\cap A$ y por $[w]^{\pm}=w^{+}-w^{-}$, el salto a través de $\overline{S}_{u}$.
\begin{teo}[Segunda forma de la ecuación de Euler-Lagrange]\label{teo:segunda forma de la ecuación de Euler-Lagrange} \index{ecuación!de Euler-Lagrange}
Sea $u$ un minimizador local de $\mathcal{F}, g\in C^{1}(\Omega)$ y $\overline{S}_{u}\cap A$ el gráfico de una función $C^{1, \gamma}$. Entonces, para cualquier campo vectorial $\eta\in [C_{c}^{1}(A)]^{N}$ se tiene la siguiente ecuación:
\begin{align}\label{teo:segunda forma de la ecuación de Euler-Lagrange:eq:1}
\int_{S_{u}\cap A}\left [ \abs{\nabla{u}}^{2}+\alpha (u-g)^{2}\right ]^{\pm}\Crochet{\eta}{\nu}d\Hm{N-1}=\int_{S_{u}\cap A}\div^{S_{u}}\eta\, d\Hm{N-1},
\end{align}
donde $\nu$ es la normal exterior de $\overline{S}_{u}\cap A$.
\end{teo}
\begin{proof}
Probamos el enunciado para el caso más sencillo en el que $u\in W^{2,2}(A^{+})\cap W^{2,2}(A^{-})$. Dado $\eta$, integrando por partes la parte izquierda de la ecuación de Euler-Lagrange se tiene que:
\begin{align*}
\int_{A^{+}}& \left [\abs{\nabla{u}}^{2}+\alpha (u-g)^{2}\right ]\div \eta\, dx\\
=&-2\int_{A^{+}}\left [ \Crochet{\eta}{\nabla^{2}u\cdot \nabla{u}}+\alpha (u-g) \Crochet{\eta}{\nabla{u}-\nabla{g}}\right]dx\\
&- \int_{S_{u}\cap A} \left [\abs{\nabla{u}}^{2}+\alpha (u-g)^{2} \right ]^{+}\Crochet{\eta}{\nu}d\Hm{N-1},
\end{align*}
donde $\nabla^{2}u$ denota la matriz de segundas derivadas de $u$. Por el teorema \ref{teo:regularidad minimizador de F}, $u$ y $\nabla{u}$ tienen una extensión Hölder continua en $\overline{S}_{u}\cap A$. Entonces, por la ecuación \ref{eq:1:problema variacional formulacion fuerte conjunto de discontinuidades regular} y por el hecho que la derivada normal de $u$ en $\partial A^{+}\cap \overline{S}_{u}$ es cero, se obtiene que:
\begin{align*}
2\int_{A^{+}}\Crochet{\nabla{u}}{\nabla{u}\cdot\nabla{\eta}}dx=-2\int_{A^{+}}\left [\Delta{u}\Crochet{\eta}{\nabla{u}}+\Crochet{\eta}{\nabla^{2}u\cdot \nabla{u}} \right ]dx\\
=-2\int_{A^{+}}\left [\alpha(u-g)\Crochet{\eta}{\nabla{u}}+\Crochet{\eta}{\nabla^{2}u\cdot\nabla{u}}\right ]dx.
\end{align*}
Por lo tanto:
\begin{align*}
\int_{A^{+}}&\set*{\left [ \abs{\nabla{u}}^{2}+\alpha(u-g)^{2}\right]}\div \eta -2\alpha(u-g)\Crochet{\nabla{g}}{\eta}-2\Crochet{\nabla{u}}{\nabla{u}\cdot \nabla{\eta}} dx\\
&=-\int_{S_{u}\cap A}\left [ \abs{\nabla{u}}^{2}+\alpha (u-g)^{2}\right]^{+} \Crochet{\eta}{\nu} d\Hm{N-1},
\end{align*}
y análogamente:
\begin{align*}
\int_{A^{-}}&\set*{\left [ \abs*{\nabla{u}}^{2}+\alpha (u-g)^{2}\right ]\div\eta -2\alpha(u-g)\Crochet{\nabla{g}}{\eta}-2\Crochet{\nabla{u}}{\nabla{u}\cdot\nabla{\eta}}}dx\\
&=\int_{S_{u}\cap A}\left [ \abs{\nabla{u}}^{2}+\alpha (u-g)^{2}\right ]^{-}\Crochet{\eta}{\nu} d\Hm{N-1}.
\end{align*}
Luego por las dos ecuaciones anteriores y por Euler-Lagrange se sigue la afirmación cuando $W^{2,2}(A^{+})\cap W^{2,2}(A^{-})$. 

Para el caso general esbozamos la idea de la demostración. Si $u\in W_{loc}^{2,2}(A\setminus \overline{S}_{u})$ entonces levantando (o tirando hacia abajo) un poco $\overline{S}_{u}\cap A$ se puede obtener una fórmula de integración por partes parecida a la anterior, en la cuál los términos adicionales desaparecen en el límite, ya que tanto $u$ como $\nabla{u}$ están acotados y $\partial u/\partial \nu =0$ en $\overline{S}_{u}\cap A$.
\end{proof}

Se observa que si $\overline{S}_{u}\cap A$ es el gráfico de una función $C^{2}$, entonces por el teorema \ref{teo:teorema de divergencia en variedades} y por la ecuación \ref{teo:segunda forma de la ecuación de Euler-Lagrange:eq:1} se sigue que la curvatura escalar \index{curvatura escalar} $H$ de $\overline{S}_{u}\cap A$ con respecto a la normal exterior $\nu$ es dada por $-\left [\abs{\nabla{u}}^{2}+\alpha (u-g)^{2}\right ]^{\pm}$. Por lo tanto, por \ref{nota:eq:1:curvatura media escalar respecto a la normal exterior} se tiene que:\index{ecuación!de curvatura media}
\begin{align}\label{eq:1:ecuacion de curvatura media}
-\div\left( \dfrac{\nabla{\phi}}{\sqrt{1+\abs{\nabla{\phi}}^{2}}}\right ) = \left [ \abs{\nabla{u}}^{2}+\alpha (u-g)^{2}\right ]^{\pm},
\end{align}
donde $\phi$ es la función cuyo gráfico es $\overline{S}_{u}\cap A$. 

Veamos que en efecto, la ecuación \ref{eq:1:ecuacion de curvatura media} sigue verificándose pero en sentido débil si únicamente se sabe que $\phi$ es de clase $C^{1,\gamma}(D)$. Antes de probar este resultado hacemos las siguientes simplificaciones del problema. Supondremos que $A=D\times (-1,1)$ y además que:
\begin{align*}
\Norm{\phi}[\infty]=\tau<1.
\end{align*}
Las pruebas de los teoremas \ref{teo:ecuación de Euler-Lagrange} y \ref{teo:segunda forma de la ecuación de Euler-Lagrange} dependen de la suposición de que $g\in C^{1}(\Omega)$. Sin embargo, incluso cuando $g$ es un función acotada, $\phi$ sigue satisfaciendo la ecuación de curvatura media.
\begin{prop}\label{prop:ecuacion de curvatura media para phi, g acotada}
Si $u$ es un minimizador local de $\mathcal{F},g\in L^{\infty}(\Omega)$ y $\overline{S}_{u}\cap A$ es el gráfico de una función $\phi$ $C^{1,\gamma}(D)$, entonces existe una función $H\in L^{\infty}(D)$ tal que la ecuación:
\begin{align}\label{prop:ecuacion de curvatura media para phi, g acotada:eq:1}
-\div\left( \dfrac{\nabla{\phi}}{\sqrt{1+\abs{\nabla{\phi}}^{2}}}\right)=H,
\end{align}
se verifica débilmente en $D$. Además, $\phi\in C^{1,1}(D)$ si $N=2$ y $\phi\in W^{2,2}_{loc}(D)\cap C^{1,\sigma}(D)$ para cualquier $\sigma\in (0,1)$ si $N>2$. 
\end{prop}
\begin{proof}
Por el teorema \ref{teo:regularidad minimizador de F} podemos suponer que tanto $u$ y $\nabla{u}$ están acotados en $D$.

\textit{Paso 1}. Afirmamos que existe un $\Lambda>0$ que depende únicamente de $\phi$ tal que si $\psi\in C^{1}_{c}(D)$ y $\Norm{\psi}[C^{1}(D)]<1/\Lambda$ entonces:
\begin{align}\label{proof:ecuacion de curvatura media para phi, g acotada:eq:1}
\int_{D}\sqrt{1+\abs{\nabla{\phi}}^{2}}dz\leq \int_{D}\sqrt{1+\abs{\nabla{\phi}+\nabla{\psi}}^{2}}dz+\lambda\int_{D}\abs{\psi(z)}dz,
\end{align}
donde $\lambda$ es una constante positiva que depende únicamente de $\phi$ y de las cotas de $u$ y $\nabla{u}$. En otras palabras $\phi$ minimiza el funcional:
\begin{align}\label{proof:ecuacion de curvatura media para phi, g acotada:eq:2}
\chi\mapsto \int_{D}\sqrt{1+\abs{\nabla{\chi}}^{2}}dz+\lambda \int_{D}\abs{\chi-\phi}dz,
\end{align}  
con respecto a la familia de funciones $\chi\in \phi+C^{1}_{c}(D)$ suficientemente cercanas a $\phi$ en norma $C^{1}$. Tomemos $\Lambda$ tal que para cualquier $\psi$ perteneciente a $C^{1}_{c}(D)$ y $\Norm{\psi}[C^{1}(D)]<1/\Lambda$ entonces $\Norm{\phi}[\infty]+2\Norm{\psi}[\infty]<1$ y $\Norm{\nabla{\psi}}[\infty]<1$. Denotemos ahora por $\Phi:A\to A$ la aplicación $\Phi(z,t)=(z,L_{z}(t))$ donde para cada $z\in D$, $L_{z}(t):(-1,1)\to (-1,1)$ es una función Lipschitz definida:
\begin{align*}
L_{z}(t)=\left \{ \begin{array}{ll}
t,& \text{si}\,  \phi(z)+2\abs{\psi(z)}\leq t,\\
\text{lineal}, & \text{si}\, \phi(z)\leq t \leq \phi(z)+2\abs{\psi(z)},\\
\phi(z)+\psi(z), &\text{si}\, t=\phi(z),\\
\text{lineal}, & \text{si}\, \phi(z)-2\abs{\psi(z)}\leq t\leq \phi(z),\\
t, & \text{si}\, t\leq \phi(z)-2\abs{\psi(z)}.\\
\end{array} \right .
\end{align*}
Es fácil ver que $\Phi$ se reduce a la identidad en una vecindad del borde de $\partial A$, que $\Phi$ es invertible y que $\Phi(\overline{S}_{u}\cap A)$ es igual al gráfico $\Gamma$ de las funciones $\phi+\psi$. Además como las derivadas $\partial L_{z}/\partial z_{i}$ están acotadas por las constantes Lipschitz de $\phi$ y $\psi$, y $1/2\leq L^{'}_{z}(t)\leq 3/2$, entonces tanto $\Phi$ y $\Phi^{-1}$ están acotadas. Finalmente:
\begin{align*}
\abs{\set*{x\in A\given \Phi(x)\not = x}}\leq \abs{\set*{(z,t)\given z\in D, \abs{t-\phi(z)}\leq 2 \abs{\psi(z)}}}=4\int_{D}\abs{\psi}dz.
\end{align*}
Si comparamos $\mathcal{F}(u, A)$ y $\mathcal{F}(v, A)$, donde $v=u\circ \Phi^{-1}$, teniendo en cuenta la desigualdad anterior se tiene que:
\begin{align*}
\Hm{N-1}(\overline{S}_{u}\cap A)\leq \Hm{N-1}(\Gamma)+\lambda \int_{D}\abs{\psi(z)}dz,
\end{align*}
para un $\lambda$ suficientemente grande que depende únicamente de las cotas de $g,u,\nabla{u}$. Por lo tanto hemos probado \ref{proof:ecuacion de curvatura media para phi, g acotada:eq:1}. 

\textit{Paso 2.} Veamos ahora que \ref{prop:ecuacion de curvatura media para phi, g acotada:eq:1} es de hecho la ecuación de Euler-Lagrange del funcional \ref{proof:ecuacion de curvatura media para phi, g acotada:eq:1}. En efecto si en la ecuación \ref{proof:ecuacion de curvatura media para phi, g acotada:eq:2} tomamos $\psi=-\epsilon \eta$, donde $\eta\in C^{1}_{c}(D)$ y $\epsilon>0$, diferenciando respecto de $\epsilon$ se obtiene:
\begin{align*}
\int_{D}\dfrac{\Crochet{\nabla{\phi}}{\nabla{\eta}}}{\sqrt{1+\abs{\nabla{\phi}}^{2}}}dz\leq \lambda \Norm{\eta}[L^{1}(D)].
\end{align*}
Por lo tanto, la parte izquierda de la desigualdad define un funcional lineal continuo en $L^{1}(D)$ y esto significa que $\phi$ satisface la ecuación \ref{prop:ecuacion de curvatura media para phi, g acotada:eq:1} para alguna función acotada $H$ cuya norma $L^{\infty}$ no excede $\lambda$.  
\end{proof}

Concluimos esta sección viendo que si el conjunto $\overline{S}_{u}\cap A$ es un gráfico $C^{1,\gamma}$, podemos obtener tanta regularidad de la solución $u$ como deseada suponiendo suficiente regularidad en $g$.

\begin{teo}\label{teo:mayor regularidad}
Sea $u$ un minimizador local de $\mathcal{F}$ y $\overline{S}_{u}\cap A$ el gráfico  $C^{1, \gamma}(D)$ de una función $\phi$. Si $g\in C^{k,\beta}(A)$ para algún $k\geq 1, \beta\leq 1$, entonces existe un $\sigma$ que depende únicamente de $N,\beta$ y $\phi$ tal que $\phi\in C^{k+2, \sigma}(D)$ y $u$ tiene una extensión $C^{k+2,\sigma}$ a cada lado de $\overline{S}_{u}\cap A$.
\end{teo}
\begin{proof}
Supongamos que $k=1$. Por la proposición \ref{prop:ecuacion de curvatura media para phi, g acotada} se sigue que $\phi\in C^{1,\delta}(D)$ para cualquier $\delta<1$ y $\phi$ tiene segunda derivada en $L^{2}_{loc}(D)$. Por lo tanto $\phi$ satisface la ecuación \ref{eq:1:ecuacion de curvatura media} puntualmente en casi todo punto de $D$ respecto de la medida $\Lm{N-1}$. Si se expande la divergencia del lado izquierdo de la ecuación \ref{eq:1:ecuacion de curvatura media} obtenemos que $\phi$ resuelve la ecuación elíptica:
\begin{align}\label{proof:mayor regularidad}
\sum_{i,j=1}^{N-1}A_{i,j}\diffp{\phi}{z_{i},z_{j}}=-\left [ \abs{\nabla{u}}^{2}+\alpha (u-g)^{2}\right]^{\pm}(z,\phi(z)),
\end{align}
donde los coeficientes son localmente $\delta$-Hölder continuos para cualquier $\delta<1$ y el lado derecho es localmente $\sigma$-Hölder continuo para algún $\sigma>0$. Por lo tanto, por el teorema clásico de acotación de Schauder \cite[Teorema 9.19]{gilbarg2015elliptic}, $\phi$ tiene localmente una derivada segunda Hölder continua. Esto implica usando la ecuación \ref{eq:1:problema variacional formulacion fuerte conjunto de discontinuidades regular} y el teorema clásico de acotaciones de Schauder para el problema de Neumann (\cite[Teorema 6.31]{gilbarg2015elliptic}), que $u$ tiene localmente una derivada segunda Hölder continua hasta $\overline{S}_{u}\cap A$. Luego podemos concluir que los coeficientes y el lado derecho de la ecuación \ref{proof:mayor regularidad} pertenecen a $C^{1,\sigma}$ para algún $\sigma>0$ y esto a su vez nos proporciona que $\phi\in C^{3,\sigma}(D)$ y que $u$ tiene un extensión $C^{3,\sigma}$ a cada lado de $\overline{S}_{u}\cap A$. Esto prueba el resultado cuando $k=1$. Si $k>1$ la mayor regularidad de $u$ y $\phi$ se sigue por un argumento de bootstrap, basado una vez más en las ecuaciones \ref{proof:mayor regularidad} y \ref{eq:1:problema variacional formulacion fuerte conjunto de discontinuidades regular}.
\end{proof}

Como consecuencia del teorema \ref{teo:mayor regularidad}, si $\phi\in C^{1,\gamma}(D)$ y si $g\in C^{\infty}(A)$ entonces $\phi\in C^{\infty}(D)$ y $u$ tiene una extensión $C^{\infty}$ a cada lado de $\overline{S}_{u}\cap A$. A continuación finalizamos el trabajo enunciando la siguiente conjetura dada por De Giorgi.

\begin{conjutura}
Si $(K,u)$ es un minimizador local del funcional $\mathcal{F}$ y $K\cap A$ es una variedad $C^{1,\gamma}$ para algún conjunto abierto $A$, entonces $K\cap A$ es analítico.
\end{conjutura}

%Apendices
%\begin{appendix}
%\chapter*{Apéndices}
%\addcontentsline{toc}{chapter}{Apéndices}
%\chapter{}\label{anexo:anexo1}
%\end{appendix}

\printindex

\nocite{*} % Anadimos toda la bibliografia lista aunque no esté citada directamente.
%\bibliographystyle{alpha}
\bibliographystyle{abbrv}
\bibliography{ref_1}

\cleardoublepage
\end{document}
